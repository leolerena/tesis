\documentclass[tesis.tex]{subfiles}

\begin{document}
	
\section{Grafos de Cayley.} \label{seccion_treewidth}

Un grafo en esta sección va a estar definido por su conjunto de vértices $V(X)$ y por uno de aristas $E(X) \subseteq V(X) \times V(X)$.
En particular las aristas no van a estar dirigidas y no tendremos bucles.
Esto es que $(v,v) \notin E(X)$ para ningún $v \in V(X)$.

\begin{deff}
	Sea $G$ un \fg y $A$ un conjunto de generadores como grupo tal que $1 \notin A$.
	Definimos el \blue{grafo de Cayley} $\Gamma = \text{Cay}(G,A)$ como el grafo que tiene como vértices $V(\Gamma) = G$ y aristas $(g,h) \in E(\Gamma)$ sii $h=ga$ para ciertos $g,h \in G$ y $a \in A \cup A^{-1}$. 
\end{deff}

Para una arista $(g,ga)$ llamaremos la \emph{etiqueta} al generador $a$.
Si $B = A \cup A^{-1}$ el conjunto simétrico de generadores entonces extendemos esta definición a caminos sobre el grafo para obtener una palabra en el monoide $B^*$.
Este grafo por como lo definimos no tiene aristas múltiples ni tampoco tiene bucles.
Dado que $A$ es un conjunto de generadores tenemos que el grafo de Cayley $\Gamma$ es conexo.

Todo grafo lo podemos ver como un espacio métrico justamente si hacemos que todas las aristas sean isométricas a $[0,1]$. 
Esto es que $d(g,ga) = 1$ para todo $g \in G, a \in B$. 
Un camino $\alpha = g_0 \dots g_n$ es una \emph{geodésica} si $d(g_i,g_j) = j-i$ para todo $i,j \in [0,n]$.


Un \emph{árbol} es un grafo conexo y sin ciclos. 
En particular los árboles vistos como espacios métricos resultan ser \emph{únicamente geodésicos}. 
Esto es que si $T$ es un árbol, entonces dados $v,w$ dos vértices existe una única geodésica $\gamma:[0,1] \to T$ tal que $\gamma(0)=v$ y $\gamma(1)=w$.

%\begin{ej}
%	Sea $\mathbb F_2 = \langle a, b \rangle$ el grupo libre en $a,b$. Su grafo de Cayley para estos generadores tiene la siguiente pinta,
%	\bigskip
%	\begin{center}
	%	\begin{tikzpicture}
		%	\draw l-system [l-system={cayley, axiom=[A] [+A] [-A] [++A], step=2.35cm, order=3}];
		%	\end{tikzpicture}
	%	\end{center}
%	
%	
%	 Notemos que este grafo resulta ser un árbol y esto es algo que vale en general para todos los grupos libres cuando miramos respecto a sus generadores canónicos.
%\end{ej}

Dado que el grafo de Cayley de todo grupo libre se puede tomar para que sea un árbol es razonable pensar que todo grupo virtualmente libre es tal que su grafo de Cayley se parezca a un árbol. 

La primera noción que podemos tomar para modelizar esto es el de tener un treewidth finito.

\begin{deff}\label{desc-arbol}
	Una \blue{descomposición en un árbol} de un grafo $X$ es un árbol $T$ y un mapa 
	\[
	X: V(T) \to {\cal P}(V(X))
	\]
	que denotaremos $X_t$ para cada vértice $t \in V(T)$. 
	Le vamos a pedir que cumpla las siguientes condiciones:
	\begin{enumerate}
		\item[\textbf{T1.}] Para todo vértice $v \in V(X)$ debe existir $t \in V(T)$ tal que $x \in X_t$. 
		\item[\textbf{T2.}] Para toda arista $e$ entre dos vértices $v,w \in V(X)$ debe existir $t \in V(T)$ tal que $v,w \in X_t$.
		\item[\textbf{T3.}] Si $v \in V(X)$ es tal que $v \in X_t \cap X_s$ luego $v \in X_r$ para todo $r$ en la geodésica que va desde $s$ a $t$ dentro de $T$. En otras palabras esto es que $\{ t \in V(T) :  v \in X_t \}$ forma un subárbol. 
	\end{enumerate} 
\end{deff}
\smallskip

La idea de la descomposición es que los \emph{bolsones} $X_t \in {\cal P}(V(T))$ no tengan muchos vértices si es que queremos modelizar que el grafo se parezca a un árbol. Esto nos conduce a la siguiente definición.

\begin{deff}
	El \blue{bagsize} de una descomposición en un árbol $T$ de un grafo $X$ es el siguiente valor:
	\begin{equation*}
		bs(X,T) = \{ \sup |X_t|, \ t \in V(T) : T \ \text{descomposición de} \ X  \} - 1
	\end{equation*}
	Un grafo $X$ tiene \blue{treewidth finito} si existe una descomposición en un árbol de bagsize finito.	
\end{deff}
\begin{ej}\label{desc-arbol-arbol}
	En particular si el grafo $X$ es un árbol notemos que tiene treewidth exactamente igual a 1.
	
	Una manera de probarlo es tomar $T$ la subdivisión baricéntrica de $X$. 
	Dada una arista $(x,y)$ consideramos el baricentro como la siguiente suma formal
	\[
		b(x,y)  = \frac{x+y}{2}.
	\]
	

	Con estas definiciones podemos definir la subdivisión baricéntrica del árbol $X$.
	Va a ser un árbol $T$ tal que tiene como vértices a 
	\[
	V(T) = \{ b(x,y) \ | \ (x,y) \in E(X) \} \cup \{  x \ | \ x \in V(X)  \}.
	\]
	Las aristas están formadas por 
	\[
	E(T) = \{  (x,b(x,y)) \ | \ (x,y) \in E(X) \}.
	\]
	Consideramos la siguiente función $X: V(T) \to {\cal P}(V(X))$  
	\[
		X_t = 
		\begin{cases}
			\{ x \} \ & \text{si} \ t = x 				\\
			\{ (x,y)  \} \ &\text{si} \ t = b(x,y).
		\end{cases}
	\]
	
	Por como los definimos es evidente que $|X_t| \le 2$ para todo $t \in V(T)$.
	De esta manera si vemos que se trata de una descomposición en un árbol tendremos probado que $bs(X,T) = 1$ tal como queríamos ver.
	
	Finalmente veamos que se trata de una descomposición. Corroboremos que cumple las tres condiciones necesarias.
	\begin{enumerate}
		\item[\textbf{T1.}] 
		Sea $x \in V(X)$ luego consideremos $X_x = \{ x \}$ dado que $v \in V(T)$ por construcción de la subdivisión baricéntrica.
		
		\item[\textbf{T2.}] 
		Dado $(x,y) \in E(X)$ consideramos $X_{b(x,y)} = \{ x,y \} $ de manera que tanto $x$ como $y$ están en un mismo bolsón.
		
		\item[\textbf{T3.}] 
		Sea $x \in V(X)$, queremos ver que $\{ t \in V(T) :  v \in X_t \}$ resulta ser un subárbol de $T$.		
		Para empezar tenemos que $x \in X_t$ si y solo sí $t = b(x,y)$ para cierta arista $(x,y) \in E(X)$ o bien $t = x$.
		De esta manera el subárbol que nos queda es conexo porque tenemos $(x,b(x,y))$ para toda arista $(x,y) \in E(X)$ y no tiene ciclos porque no hay más aristas entre estos vértices de la subdivisión baricéntrica del árbol $X$.
%		Sea $x \in X_t \cap X_s$. 
%		Por como armamos los bolsones, uno de estos dos tiene que corresponderse con una arista de $X$.
%		Supongamos que $X_t$ tiene esa pinta entonces $X_t = \{ x,y\}$ para cierto vértice $y \in V(X)$.
%		
%		En el caso que $X_s = \{ x \}$ la geodésica que podemos tomar entonces es $st$ puesto que $s$ se corresponde con el vértice $x$ y sabemos que existe una arista $st \in E(T)$ porque justamente $t$ se corresponde con una arista en $X$ que contiene a $x$.
%		
%		En el caso que $s \in V(T)$ sea tal que el bolsón $X_s$ se corresponda con los vértices que aparecen en una arista debería ser del tipo $X_s = \{x,z\}$ para cierto $z \in V(X)$. 
%		Si miramos la geodésica sobre $T$ entre ambos vértices no es más que el siguiente camino $tvs$ donde $v \in V(T)$ es el vértice que está tanto en el árbol original $X$ como en la subdivisión baricéntrica $T$.
%		Esto es que $X_v = \{ x \}$.
%		De esta manera $x$ está en todos los bolsones correspondientes a la geodésica sobre $T$ tal como queríamos ver.	
	\end{enumerate}
\end{ej}


% [Creo que está bastante mal esto. Si lo tengo que usar más en adelante repensarlo bien.]
%	\begin{obs}
	%	Si un grafo $\Gamma$ tiene treewidth finito podemos tomarnos otra descomposición tal que siga siendo de treewidth finito pero los bolsones de la descomposición sean conexos. 
	%	Si algún bolsón no es conexo agregamos todos los vértices del grafo que estén en la geodésica que une las componentes conexas. 
	%	Haciendo esto la descomposición sigue siendo finita aunque el tamaño de los bolsones puede aumentar considerablemente.
	%	\end{obs}

Veamos ahora una propiedad que tienen los caminos en los grafos que tienen una descomposición en un árbol.

\begin{prop}\label{prop-camino-desc}
	Sea $\Gamma$ grafo tal que tiene una descomposición en un árbol $T$.
	Sean $X,Y,Z$ bolsones tales que el vértice correspondiente a $Z$ en el árbol $T$ está en la geodésica que va de $X$ a $Y$. 
	Sean $x \in X, y \in Y$ tales que $x = x_0 \dots x_n=y$ es algún camino en $\Gamma$ conectándolos.
	Entonces debe existir algún $ 0 \le i \le n$ tal que $x_i \in Z$. 
\end{prop}

\begin{proof}	
	
	Vamos a demostrarlo haciendo inducción en la longitud $n$ del camino. 
	El caso base es que $n = 0$ por lo tanto $x=y$. 
	En este caso por ser una descomposición en un árbol tenemos que usando la tercer propiedad que $x \in Z$ también.
	
	Para el paso inductivo tomemos $X'$ bolsón que contiene tanto a $x$ como a $x_1$ que sabemos existe por ser una de las propiedades de las descomposiciones en árboles.
	Si este bolsón $X'$ es $Z$ ya está porque nos alcanza con tomar $i=1$ de manera que $x_1 \in Z$.
	En el caso que esto no ocurra miramos el camino de longitud $n-1$ dado por $x_1 x_2 \dots x_n = y$ y usando la hipótesis inductiva llegamos al resultado.	
\end{proof}


\begin{deff}
	Sea $\Gamma$ un grafo y $C \subseteq V(\Gamma)$ un conjunto de vértices entonces definimos los \blue{vecinos de $C$} por medio de 
	\[
	N(C) = \{ v \in V(\Gamma) : \exists w \in C, \ vw \in E(\Gamma) \}.
	\]
	De esta manera podemos definir recursivamente los l-ésimos vecinos por medio de $N^l(C) = N(N^{l-1}(C))$.
\end{deff}

\begin{obs}
	Esta definición la podemos escribir de manera más concisa como $N^l (C) = \{ v \in V(\Gamma) : \ \exists w \in C, \  d(v,w) \le l  \}$.
\end{obs}

\begin{prop}\label{prop-vecinos-desc}
	Si $(X_t)_{t \in V(T)}$ es una descomposición en un árbol $T$ para un grafo $\Gamma$ entonces si tomamos como bolsones a $N^l(X_t)$ también tenemos una descomposición en un árbol.
\end{prop}
\begin{proof}
	Equivalentemente queremos ver que el mapa $Y: V(T) \to {\cal P}(V(X))$ dado por $Y_t = N^l(X_t)$ nos da otra descomposición en un árbol del grafo $\Gamma$.
	
	Probemos este resultado haciendo inducción en $l$.
	
	Consideremos el caso base $l=1$.
	En este caso, las dos primeras condiciones de la descomposición en un árbol \textbf{T1} y \textbf{T2} se siguen cumpliendo porque no hicimos más que agrandar los bolsones. 
	Esto es que si $X_t$ era un bolsón luego $X_t \subseteq Y_t$.
	
	Debemos ver que cumple \textbf{T3}.
	Partamos de un vértice $x \in V(X)$ tal que $x \in Y_s \cap Y_t$ y veamos que si $r \in V(T)$ está en la geodésica de $t$ a $s$ entonces $x \in Y_r$.
	Notemos que en este caso debe existir algún $y \in Y_s$ tal que $(x,y) \in E(X)$.
	Análogamente existe $z \in Y_t$ tal que $(x,z) \in E(X)$.
	Si no fuera así tendríamos que $x \in X_s \cap X_t$ y esta condición ya se cumpliría por ser los bolsones de una descomposición.
	Entonces si usamos la proposición \ref{prop-camino-desc} tomando el camino $yxz$ tenemos que alguno de estos tres vértices debe estar en $X_r$.
	De esta manera $x \in Y_r$ tal como queríamos ver.
	
	El paso inductivo se sigue directamente del caso base porque si sabemos que $N^{l-1}(X_t)$ forma una descomposición de un árbol usamos el caso base para ver que los vecinos de esta descomposición lo siguen siendo y con esto terminamos de probarlo porque $N^l(X_t) = N (N^{l-1})(X_t)$.
\end{proof}
\medskip

\begin{deff}
	Dado un grafo $X$, el \blue{borde} de un conjunto de vértices $C \subseteq V(X)$ se define como
	\[
	\beta C = \{ u \in V(X) : \exists v \in V(X), \ (u,v) \in E(X), \  \ \ (u \in C \wedge v \in \overline C) \ \ \lor  (u \in \overline C \wedge v \in  C)   \}.
	\] 
\end{deff}

\begin{obs}
Otra manera de definir el borde de un conjunto de vértices $C$ puede ser 
\[
	\beta C = N(C) \cap N(\ol{ C}).
\]
\end{obs}

\begin{ej}\label{desc-grafo-cayley}%[Descomposición válida para todo grupo finitamente generado].
	
	Construyamos una descomposición en un árbol que podemos hacer en general para todos los grafos de Cayley de grupos finitamente generados. 
	Consideremos el grafo $\Gamma = \text{Cay}(G,A)$ para cierto conjunto de generadores $A$ finito.
	
	
	Sea $V_n = \Gamma \setminus B_n(1) $ tal que $V_0 = \Gamma \setminus \{1\}$. 
	Los vértices del árbol $T$ van a estar dados por los siguientes conjuntos,
	\[
	V(T) = \{  \beta C : C \subseteq V_n \ \text{componente conexa}   \} \cup \{ 1 \}
	\]
	esto es todas las fronteras de las componentes conexas que nos quedan cuando consideramos algún $V_n$. 
	Las aristas entonces van a estar dadas por lo siguiente,
	\[
	E(T) = \{ (\beta C , \beta D) : C \subseteq D \subseteq V_n, \ C \subset V_{n+1}   \} \cup \{  (1, \beta C) : C \subseteq V_0  \}
	\]
	esto es que $C$ es una componente conexa del grafo que queda de sacarla la bola de radio un número mayor que el que está $D$.
	Notemos en particular que por como definimos este grafo $T$ resulta que si $C$ es una componente conexa de $V_n$ entonces solo existe una única arista $(\beta C, \beta D)$ tal que $D \subseteq V_{n-1}$. 
	Esto se debe a que al ser $C$ conexo entonces como $V_{n-1} \subseteq V_{n}$ en particular $C \subseteq V_{n}$ y debe estar contenida en una sola componente conexa $D \subseteq V_{n-1}.$
	
	
	Podemos ver que así como lo construimos tenemos que el grafo $T$ resulta ser un árbol tal que su raíz es $1$. 
	
	Para ver que es un árbol veamos primero que es conexo. Dado un vértice $\beta C \in V(T)$ vamos a armar un camino que lo conecte con la raíz del árbol $1$. 
	Usemos inducción en el $n$ tal que $C$ es una componente conexa de $V_n$. 
	El caso base es algún $C$ tal que es una componente conexa de $V_1$ y queremos armarnos un camino que lo una con $B_1(\Gamma)$. 
	Por construcción de las aristas necesariamente tenemos que $(1, \beta C) \in E(T)$.
		
	Para el paso inductivo supongamos para cualquier borde de una componente conexa de $V_{n-1}$ tenemos un camino que lo une con la raíz $1$ y veamos de construirnos un camino con cualquier borde de una componente conexa $C \subseteq V_n$.
	De esta manera debe existir $D$ componente conexa de $V_{n-1}$ tal que $C \subseteq D$. Esto es porque $V_{n-1} \subseteq V_{n}$ y al ser $C$ un conexo si interseca alguna de las componentes conexas de $V_{n-1}$ debe estar contenida en ella. 
	Necesariamente debe intersecar a alguna de estas componentes conexas porque particionan al espacio.
	Así vimos que existe una arista $(\beta C, \beta D)$ y ahora usando la hipótesis inductiva obtenemos un camino de $1$ a $\beta C$.
	Concluimos que el grafo $T$ es conexo.   
	
	% Otra manera que se me ocurrió es tomar un ciclo de longitud mínimo y llegar a un absurdo achicandolo. Creo que es más corto pero no me convence.
	Para terminar de ver que es un árbol veamos que es acíclico. 
	Dado un camino cerrado $\sigma$ en $T$ probemos que necesariamente tiene que repetirse algún vértice por lo tanto ningún camino cerrado puede ser un ciclo. Vamos a probarlo usando inducción en el máximo $n \in \NN$ tal que $C \in V_n$ y $\beta C$ es uno de los vértices en el camino. 
	%Notemos que está bien definido porque los caminos cerrados son compactos.
	Para el caso base notemos que si $\sigma$ es un camino cerrado tal que el máximo $n$ que aparece es $n=1$ entonces necesariamente $\sigma$ es idéntico al camino constante fijo en $1$.
	%reescribir esto!
	Para el paso inductivo supongamos que para todo camino cerrado con máximo $m \in \NN$ tal que algún vértice que aparece en este camino es una componente conexa de $V_m$ resulta ser $n-1$ entonces este camino no es un ciclo. Veamos que si $m=n$ entonces este camino cerrado también resulta ser un ciclo.
	
	Partamos de un camino tal que pasa por $\beta C$ con $C \subseteq V_n$. 
	Como no existe $(\beta C, \beta C') \in E(T)$ tal que $C,C'$ sean componentes conexas de un mismo $V_n$ por la definición que dimos de las aristas necesariamente tiene que haber una arista con alguna componente conexa $D \in V_{n-1}$ dado que $n$ es máximo en este camino cerrado.   
	Como vimos anteriormente solamente hay una única arista que una a $\beta C$ con alguna $\beta D$ con $D \subseteq V_{n-1}$. 
	Esto nos dice que si el camino cerrado pasa por la arista $(\beta C, \beta D)$ necesariamente debe volver a pasar por la arista $(\beta C, \beta D)$ por lo tanto no es un ciclo. 
	Concluimos que el grafo $T$ es acíclico y por lo tanto como es conexo también resulta que $T$ es un árbol.
	
	Para ver que es una descomposición de árboles debemos ver que cumple las tres condiciones \ref{desc-arbol} de la definición. 
	\begin{enumerate}
		\item[\textbf{T1.}] La primera condición la cumple por como lo definimos porque cualquier vértice del grafo $g \in V(\Gamma)$ 
		es tal que existe $n \in \NN$ de manera que $d(1,g)=n$, por lo que está en alguna componente conexa $C \subseteq V_{n-1}$. 
		En particular como la distancia es exactamente $n$ tiene que existir una arista $(g,g') \in E(\Gamma)$ con un vértice $g' \in V_{n-1}$ y por lo tanto $g \in \beta C$.
		
		\item[\textbf{T2.}] La segunda condición partimos de una arista $(g,h) \in E(\Gamma)$ luego miramos las distancias que hay al $1$ de ambos vértices para conseguirnos el bolsón adecuado. 
		Supongamos que ambas están a la misma distancia del $1$. En tal caso sea $n$ tal que 
		\[
			d(g,1)=n = d(h,1)
		\] 
		luego si miramos $V_{n-1}$ notemos que tienen que estar en la misma componente conexa porque existe una arista entre ambos vértices. Sea esta componente $C$ luego como están a distancia exactamente $n$ ambas están en el borde, esto es que $g,h \in \beta C$ tal como queríamos ver. 
		El otro caso es que 
		\[
			d(g,1)=n < n+1 = d(h,1)
		\] 
		y en este caso como están conectadas resulta que están en $\beta C$ si $C$ es la componente conexa que contiene a $h$ en $V_n$ por definición del borde de un conjunto de vértices de un grafo.
		
		\item[\textbf{T3.}] Para la tercera condición supongamos que hay $g \in V(\Gamma)$ tal que está en la intersección de dos bolsones $\beta C \cap \beta D$. 
		Queremos ver que está en todos los bolsones que aparecen en la geodésica de $\beta C$ a $\beta D$ en el árbol T. 
		Si $d(g,1) = n$ luego solo puede estar en la frontera de alguna componente conexa $C$ tal que esté en $V_{i}$ para $i\in \{n-1, n, n+1\}$. 
		Supongamos que $D \subseteq V_{n}$, en tal caso vale que $D \nsubseteq V_{n}$ porque sino tendríamos que resulta ser la misma componente conexa. 
		Sin pérdida de generalidad supongamos que $C \subset V_n$ mientras que $D \subset V_{n+1}$. 
		Como ambas son componentes conexas luego si hay una intersección tiene que haber una contención, esto es que $D \subseteq C$. 
		De esta manera vemos que $(\beta C, \beta D) \in E(T)$ por lo tanto como están unidos por una arista la geodésica es justamente esta arista.
		
	\end{enumerate}
\end{ej}



\section{Cuasisometrías.}


\begin{deff}
	Sean $(X,d_X),(Y,d_Y)$ espacios métricos. 
	Una \blue{cuasisometría} es una función $f:X \to Y$ tal que:
	\begin{itemize}
		\item[\textbf{Q1.}] Existe constante $A > 0$ tal que para todo par de puntos $x_1,x_2 \in X$ hace valer la siguientes desigualdades
		\[
		\frac{1}{A} d_X(x_1,x_2) - A \le d_Y(f(x_1),f(x_2)) \le A d_X(x_1,x_2) + A
		\]
		\item[\textbf{Q2.}] Existe una constante $C \ge 0$ tal que para todo punto $y \in Y$ debe existir $x \in X$ de manera que 
		\[
		d(y,f(x)) \le C
		\]
	\end{itemize}
\end{deff}


\begin{prop}
	Si existe $f:X \to Y$ cuasisometría entonces también debe haber $g:Y \to X$ cuasisometría.
\end{prop}
\begin{proof}
	Resultado estándar. Ver \cite{bridson2013metric}.
\end{proof}

\begin{deff}
	Dos espacios métricos que se dicen \emph{cuasisométricos} si existe una cuasisometría entre ellos.
\end{deff}


\begin{obs}
	Si nos restringimos a que los espacios métricos sean grafos podemos redefinir una cuasisometría para que sea una función de los vértices de un grafo a los del otro.
	Esto es si tenemos dos grafos $\Gamma_1$ y $\Gamma_2$ entonces son cuasisométricos como espacios métricos si y solo si existe $f:V(\Gamma_1) \to V(\Gamma_2)$ tal que 
	\begin{itemize}
		\item Para todo $v,w \in V(\Gamma_1)$ existe constante $A > 0$ tal que 
		\[
		\frac{1}{A} d_X(v,w) - A \le d_Y(f(v),f(w)) \le A d_X(v,w) + A
		\]
		\item Existe una constante real $C \ge 0$ tal que para todo vértice $y \in V(\Gamma_2)$ existe $v \in V(\Gamma_1)$ tal que 
		\[
		d(y,f(v)) \le C
		\]
	\end{itemize}
	
	Para esto si tenemos una cuasisometría $g$ como espacios métrices definamos una sobre los vértices $f$ de la siguiente manera.
	Por cada vértice $v \in V(\Gamma_1)$ consideremos $g(v) \in \Gamma_2$. 
	Si $g(v) \in V(\Gamma_2)$ tomamos $f(v)=g(v)$.
	El otro caso es que $g(v)$ cae en el medio de alguna arista $xy \in E(\Gamma_2)$. 
	Supongamos que $d(g(v),x) \le \frac{1}{2}$ porque la distancia máxima con alguno de los vértices debe ser no más de $\frac{1}{2}$ porque justamente las aristas son isométricas con el intervalo $[0,1]$. 
	En este caso definamos $f(v) = x$.
	Notemos ahora que la distancia entre dos vértices por medio de $f$ a lo sumo aumenta. 
	Tenemos que 
	\[
	d(f(v),f(w) \le d (g(v),g(w)) + 1
	\]
	para $v,w \in V(\Gamma_1)$ porque a lo sumo las imágenes por $g$ están a distancia $\frac{1}{2}$ de alguno de los vértices.
	De esta manera notemos que si tomamos la constante $C+1$ nos sirve.
	Para ver que la imagen es cuasidensa idénticamente tomando $C+1$ nos sirve.
	En definitiva $f$ definida de esta manera es una cuasisometría con constante $C+1$.	
\end{obs}


Todo espacio métrico es cuasisométrico consigo mismo por medio de la identidad.
La composición de cuasisometrías también sigue siendo una cuasisometría.
Con esta proposición vemos que ser cuasisométricos es una relación de equivalencia entre los espacios métricos. 


Intuitivamente una cuasisometría entre espacios métricos nos dice que estos resultan ser bastante similares al menos desde cierta distancia. 
\medskip

\begin{ej}
	Ejemplo de $\ZZ \times \ZZ$ para ver que no es cuasisométrico con un árbol o bien podría rehacer el ejemplo de la parte anterior.
\end{ej}

Todo grafo de Cayley lo podemos pensar como un espacio métrico tal como lo definimos en la sección \ref{seccion_treewidth}.  
Veamos que esto no depende de los generadores que hayamos elegido.

\begin{prop}
	Sea $G$ grupo \fg por $\Sigma$ y $\Delta$ conjuntos finitos entonces $\text{Cay}(G,\Sigma)$ y $\text{Cay}(G, \Delta)$ son cuasisométricos entre sí.
\end{prop}

\begin{proof}
	Resultado estándar. Ver \cite{bridson2013metric}.
\end{proof}



Así otra manera de pensar que un grafo de un grupo virtualmente libre es casi un árbol es pedirle que sea cuasisométrico con un árbol. 
Veamos que estas categorización es equivalente a pedirle que el treewidth sea finito que era la otra caracterización que habíamos obtenido anteriormente.

\begin{prop} \label{treewidth-inv}
	El treewidth finito es un invariante por cuasisometría para grafos con grado acotado uniformemente.
\end{prop}
%\begin{proof}[Intento de demo distinta. No salió]
%Si tenemos una cuasisometría $f:\Gamma_1 \to \Gamma_2$ tal que $\Gamma_1$ tiene treewidth finito $k \in \NN$, nos gustaría ver que $\Gamma_2$ también tiene treewidth finito.
%Dada la cuasisometría $f$ consideremos $C \ge A + B$.
%
%Partamos de una descomposición en un árbol $T$ para $\Gamma_1$.
%La idea es usar la cuasisometría para empujar esta descomposición a una (con bolsones posiblemente más grandes) en $\Gamma_2$.
%Por cada bolsón $X_t \subset V(\Gamma_1)$ vamos a considerar un bolsón en el otro grafo,
%\[
%Y_t = B(f(X_t),Ck) \cap V(\Gamma_2).
%\] 
%Esto es todos los vértices de $\Gamma_2$ que estén a distancia igual o menor que $Ck$ de la imagen del bolsón $X_t$ por medio de $f$. 
%Donde $k$ es la constante del treewidth del grafo $\Gamma_1$ y $C$ la constante de la cuasisometría anteriormente definida.
%
%Como ya sabemos que $T$ el grafo subyacente de la descomposición es un árbol basta ver que cumple las tres propiedades que debe cumplir toda descomposición \ref{desc-arbol}.
%
%\begin{enumerate}
%	\item Queremos ver que dado $v \in V(\Gamma_2)$ existe $t \in V(T)$ tal que $v \in Y_t$.
%	Por ser un embedding cuasisométrico debe existir $y \in \Gamma_1$ (no necesariamente algún vértice) tal que $d(f(y),v) \le C$. 
%	Tomemos $x \in V(\Gamma_1)$ de manera que $d(x,y) < 1$. 
%	Usando desigualdad triangular notemos que 
%	\[
%	d(f(x),v) \le d(f(x),f(y)) + d(v,f(y)) \le 2C
%	\]
%	donde $d(f(x),f(y)) \le C$ porque $f$ es una cuasisometría.
%	Mientras que la otra cota vale porque justamente así tomamos a $f(y)$. 
%	Como $k \le 2$ tenemos que $2C \le Ck$ y así vemos que $v \in Y_t$ tal como queríamos ver.
%	\item Dado una arista $vw \in E(\Gamma_2)$ veamos que $v, w \in Y_t$ para algún mismo bolsón $Y_t$.
%	Por la cuenta anterior sabemos que todo $v \in V(\Gamma_2)$ es tal que existe algún $x \in V(\Gamma_1)$ de manera que $d(f(x),v) \le C + 1$.
%	Si tomamos este mismo $x$ notemos que como $d(v,w) = 1$ entonces $d(w,x) \le C+2$. 
%	Esto nos dice que $v,w \in Y_t$ donde $Y_t$ es la bolsa correspondiente a $f(x)$. 
%	\item Sea $v \in Y_t \cap Y_s$ queremos ver que $v \in Y_r$ para todo $r$ en la geodésica que une $t$ con $s$ en el árbol $T$.
%	Alcanza con tomarnos $Y_t$ tal que exista $x \in V(\Gamma_1)$ con $x \in X_t$ de manera que $d(f(x),v) \le 2C$.	
%	Esto lo podemos hacer porque $f$ es una cuasisometría. 
%	Fijemos uno de los bolsones y veamos que para toda geodésicas que parten del vértice del árbol $t$ y terminan en $s$ son tales que todos los vértices $r \in V(T)$ que aparecen cumplen que $v \in Y_r$. 
%	Si no pudiéramos hacer esto tendríamos que nuestro grafo $T$ tiene un ciclo pero esto es absurdo puesto que es un árbol.
%	Supongamos que $sr$ es una arista de $T$ y $rt$ es otra arista.
%%	Los casos de una geodésica en general se reducen a este usando inducción en la longitud del camino.
%%	Sabemos que existe $z \in X_r \cap X_t$ por la proposición \ref{}. 
%%	De esta manera como $|X_t| \le k$ tenemos que $d(x,z) \le k$. 
%%	Usando que $f$ es una cuasisometría tenemos que $d(f(x),f(z) \le Ck$.
%%	
%%	Entonces veamos de acotar $d(f(z),v)$ y así concluir que $x \in Y_r$ tal como queríamos ver.
%%	Usando la desigualdad triangular,
%%	\[
%%	d(f(z),v) \le d(f(z),f()) + d(v,f(x)) \le Ck + C + 2
%%	\]
%%	y esto termina la demostración porque inductivamente lo podemos ver para cualquier $r$ en la geodésica.
%
%\red{Ahora creo que esto no es cierto. La tercera condición se rompa incluso en un ejemplo finito.}
%\end{enumerate}
%
%\end{proof}

\begin{proof}
	Si tenemos una cuasisometría $f:\Gamma_1 \to \Gamma_2$ tal que $\Gamma_2$ tiene treewidth finito $k \in \NN$, nos gustaría ver que $\Gamma_1$ también tiene esta propiedad.
	Consideremos $l$ tal que $d(f(v),f(w)) \le l$ para vértices $v,w \in V(\Gamma_1)$ que estén conectados por una arista.
	Esto lo podemos tomar porque al ser una cuasisometría 
	\[
	d(f(v),f(w)) \le C d(v,w) + C  \le 2C
	\]
	entonces basta con tomar $l \ge C+1$.
	
	Veamos de armarnos la descomposición en un árbol $T$ para $\Gamma_1$.	
	Tomaremos como árbol para descomposición al mismo $T$ que usamos para $\Gamma_2$.
	Sean $X_t$ los bolsones de esta descomposición. 
	Recordemos que por \ref{prop-vecinos-desc} si tomamos $N^l(X_t)$ los vecinos del bolsón $X_t$ que están a distancia no mayor a $l$ seguimos teniendo una descomposición.  
	Consideraremos los bolsones $Y_t = f^{-1}(N^l(X_t))$ de vértices en $\Gamma_1$. 
	
	Debemos ver que cumplen las tres propiedades.
	
	\begin{enumerate}
		\item[\textbf{T1.}] La primera se cumple puesto que los bolsones $X_t$ cubren $V(\Gamma_2)$. 
		De esta manera $\cup_{t \in T} N^l(X_t) = V(\Gamma_2)$ y por lo tanto tomando preimagen tenemos que
		\[
		\bigcup_{t \in V(T)} f^{-1} (N^l (X_t)) = \bigcup_{t \in V(T)} Y_t = f^{-1} (V(\Gamma_2)) = V(\Gamma_1)
		\] 
		donde usamos que la preimagen de la unión es la unión de las preimágenes.
		\item[\textbf{T2.}] La segunda condición usamos que si hay una arista $xy \in E(\Gamma_2)$ luego debe ser que $d(f(x),f(y)) \le l$ por como tomamos a $l$.
		De esta manera como $f(x) \in X_t$ para algún $t \in V(T)$, notemos que $f(y) \in N^l(X_t)$ también. 
		Tomando preimagen tenemos que $x,y \in f^{-1}(N^l(X_t))$ y esto es que justamente $x,y \in Y_t$ para un mismo $t \in V(T)$ tal como queríamos ver.		
		\item[\textbf{T3.}] Para la tercera condición si $x \in Y_t \cap Y_s$ queremos ver que $x \in Y_r$ para todo $r \in V(T)$ que aparezca en la geodésica de $s$ a $t$.
		Como la preimagen de una intersección es lo mismo que la intersección de las preimágenes entonces 
		\[
		x \in f^{-1}(N^l(X_t)) \cap f^{-1}(N^l(X_s)) = f^{-1}(N^l(X_t) \cap N^l (X_s)
		\]
		de esta manera debe existir $v \in V(\Gamma_2)$ tal que $v \in N^l(X_s) \cap N^l(X_t)$.
		Ahora usamos que esta es una descomposición sobre $\Gamma_2$ para notar que $v \in N^l(X_r)$.
		Tomando preimagen tenemos que $x \in Y_r$ tal como queríamos ver.
	\end{enumerate}
	
	Finalmente debemos ver que el tamaño de los bolsones está acotado uniformemente.
	Esto es que exista $k \in \NN$ tal que $|Y_t| \le k$ para todo $t \in V(T)$.
	Como $\Gamma_2$ tiene treewidth finito tenemos que $|X_t| \le M$ uniformemente para todo $y \in V(T)$ para cierta $M$. 
	Como el grado de los grafos está acotado uniformemente por alguna constante $d$ tenemos que 
	\[
	|N^l(X_t)| \le d^l |X_t| \le d^l M.
	\]
	Finalmente notemos que al ser $f$ una cuasisometría tenemos que $|f^{-1}(v)| \le k$ para todo $v \in V(\Gamma_2)$.
	Esto lo podemos ver porque si $f(x) = v = f(y)$ entonces
	\[
	\frac{1}{C}d(x,y) - C \le d( f(x), f(y) ) = 0 \implies d(x,y) \le C^2 < \infty
	\]
	y esta cota es uniforme para todo $v \in \Gamma_2$. 
	Así vemos que,
	\[
	|Y_t| = |f^{-1}(N^l(X_t))| \le C^2 d^l M < \infty
	\]
	y tomamos $k$ suficientemente grande para que haga valer esto.
	Concluímos así que la descomposición que nos armamos para $\Gamma_1$ tiene treewidth finito.
\end{proof}

A partir de este resultado podemos ver que la otra manera que teníamos de pensar a los grafos que se parecen a árboles resulta ser más débil. 
El siguiente resultado lo demostramos en el caso general de un grafo tal que los grados de sus vértices están acotados uniformemente. 
Como caso particular tenemos los grafos de Cayley de grupos finitamente generados.


\begin{prop} 
	Un grafo $X$ de grado acotado uniformemente cuasisométrico con un árbol tiene treewidth finito.
\end{prop}
\begin{proof}	
	%Para la ida veamos de armarnos la cuasisometría a partir de la descomposición en árbol del grafo $X$. 
	%Definamos entonces la cuasisometría $q: T \to X$ a partir de mandar $t \to x_t$ donde $x_t \in X_t$ es algún elemento del bolsón correspondiente a $t \in V(T)$ que sabemos que no es vacío. 
	%Esta función es una cuasisometría con constante $C=1$ y $A = bs(T) + 1$.
	%
	%Hay que ver la distancia dentro de un bolsón esté controlada.
	%Si los bolsones los tomamos conexos \ref{} entonces notemos que la mayor distancia posible es...
	Dado que el grafo $X$ es cuasisométrico a un árbol $T$, por la observación \ref{desc-arbol-arbol} este tiene treewidth exactamente 1.
	Por la proposición anterior \ref{treewidth-inv} como es un invariante por cuasisometría vemos que $X$ debe tener treewidth finito tal como queríamos ver.
\end{proof}

\begin{coro}
	Todo grupo independiente de contexto es tal que su grafo de Cayley es cuasisométrico a un árbol.
\end{coro}
\begin{proof}
	Por \ref{teo_schupp_muller_ic_desc} todo grupo independiente de contexto tiene treewidth finito y usando la proposición recién demostrada concluímos que su grafo de Cayley es cuasisométrico a un árbol.
\end{proof}


\begin{prop}\label{cuasisometria-subgrupo-ind-finito}
	Sea $G$ grupo \fg entonces si $H$ es un subgrupo de índice finito resulta que son cuasisométricos.
\end{prop}
\begin{proof}
	Resultado estándar pero no tan elemental de demostrar. 
	Es un corolario de Milnor Schwarz.
	Ver \cite{loh2017geometric}.
\end{proof}

Esto nos dice que todo grupo es cuasisométrico con los subgrupos de índice finito. 
De esta manera obtenemos el siguiente resultado,

\begin{prop}
	Todo grafo de Cayley de un grupo virtualmente libre es cuasisométrico a un árbol.
\end{prop}

\begin{proof}
	Si $G$ es virtualmente libre entonces existe $F \le G$ grupo libre de índice finito.
	Por la prop anterior \ref{cuasisometria-subgrupo-ind-finito} tenemos que el grafo de Cayley de $G$ es cuasisométrico con el de $H$.
	Por el resultado \ref{} sabemos que los grafos de grupos libres son árboles y así queda demostrada la proposición.
\end{proof}

	
	
\end{document}