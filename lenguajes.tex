% !TeX TS-program = 
\documentclass[tesis.tex]{subfiles}

\newcommand{\APND}{automáta de pila no determinístico }
\newcommand{\APD}{automáta de pila determinístico }

\begin{document}
\chapter{Teoría de lenguajes.}	
En esta sección vamos a introducir los elementos básicos de la teoría de lenguajes formales que utilizaremos en este trabajo. 

Consideremos un conjunto finito no vacío $\Sigma$ que llamaremos el alfabeto y $\Sigma^k$ el conjunto de sucesiones finitas de elementos $a_1 \dots a_k$ con  $a_i \in \Sigma$. Los elementos de $\Sigma$ se llaman letras y los elementos de $\Sigma^k$ serán palabras de longitud $k$ sobre $\Sigma$. 
\begin{deff}
	El \blue{monoide libre} sobre un alfabeto $\Sigma$ es el siguiente conjunto
	\begin{equation*}
	\Sigma^{*} = \bigcup_{k=0}^{\infty} \Sigma^k
	\end{equation*}
	con la operación $\cdot$ que es la concatenación de palabras. Es decir dadas $w_1 \in \Sigma^{k}, w_2 \in \Sigma^{l}$ luego $w_1 \cdot w_2 \in \Sigma^{k+l} \subset \Sigma^*$. El elemento neutro es la palabra vacía que corresponde a la copia de $\Sigma^0$ que es la única palabra sin letras. 
\end{deff}
\begin{obs}
	El monoide es libre con la siguiente propiedad: si tenemos una función del alfabeto $f: \Sigma \to M$ donde $M$ es algún monoide entonces existe un único morfismo de monoides $\overline f: \Sigma^{*} \to M$ que hace conmutar al siguiente diagrama.	
	
	\begin{center}
	\begin{tikzcd}
		\Sigma \arrow[r, "f"] \arrow[d, hook] & M \\
		\Sigma^* \arrow[ru, "\overline f"]    &  
	\end{tikzcd}
	\end{center}
	
\end{obs}

Si $w$ es una palabra sobre el alfabeto $\Sigma$ luego una subpalabra $u$ de $w$ es una palabra $u \in \Sigma^*$ tal que $w = vuz$ para algunas $v, z \in \Sigma^*$. Si $w = vu$ entonces $v$ es un prefijo de $w$ y $u$ es un subfijo de $w$.

\begin{deff}
	Un \blue{lenguaje} sobre un alfabeto $\Sigma$ es un subconjunto de $\Sigma^*$.
\end{deff}


\begin{comment}
	\begin{obs}
	De esta manera podemos pensar a una {presentación} de un monoide $M$  como un epimorfismo de monoides $\pi: \Sigma^{*} \to M$. Esto nos dice que el monoide $M$ tiene como generadores a $\Sigma$ y sus relaciones están dadas por el núcleo del morfismo $\pi$.
	\end{obs}
\end{comment}
\section{Gramáticas.}
Vamos a considerar lenguajes definidos a partir de lo que se conoce como una gramática. Esto es esencialmente unh conjunto de reglas que al irse aplicando nos permiten generar todas las palabras del lenguaje.

\begin{deff}
	Una gramática es una tupla ${\cal G} = (V, \Sigma, P, S)$ que la interpretamos de la siguiente manera,
	\begin{itemize}
		\item $V$ es un conjunto finito de variables;
		\item $S \in V$ es el símbolo inicial;
		\item $\Sigma$ es un conjunto finito de símbolos terminales ;
		\item $P \subseteq (V \cup \Sigma)^* \times (V \cup \Sigma)^*$ es un conjunto finito de producciones.
	\end{itemize}
\end{deff}

Las variables en algunos casos son también llamados símbolos no terminales. Las producciones son reglas recursivas que nos sirven para definir a un lenguaje. A la regla $(\omega, \lambda)$ la vamos a denotar $\omega \to \lambda$.

La gramática permite generar al lenguaje de la siguiente manera. Arrancamos con alguna producción que tenga al símbolo inicial $S$ a su izquierda y la reemplazamos por lo que aparece a la derecha. Es decir si miramos $S \to \omega$ luego reemplazamos a $S$ por $\omega$ y seguimos. Por cada subpalabra de $\omega\in (V \cup \Sigma)^*$ que aparezca a la izquierda de alguna de las producciones la reemplazamos por la que aparece a la derecha. 

Más en general definimos la relación $\Rightarrow$ que formaliza esta idea. Para $\omega, \xi \in \Sigma^*$  $\omega \Rightarrow_{\cal G} \xi$ significa que a partir de las reglas de nuestra gramática podemos derivar a $\xi$ usando las producciones sucesivamente. Tomamos la clausura transitiva, reflexiva y simétrica de esta relación que denotaremos $\overset{*}{\Rightarrow_{\cal G}}$.


De esta manera podemos definir un lenguaje a partir de la gramática como las palabras en $\Sigma^*$ tales que por medio de las producciones las podemos generar. Formalmente
\[
L({\cal G}) = \{ w \in \Sigma^* \ | \ S \overset{*}{\Rightarrow_{\cal G}} w   \}.
\]

De esta manera podemos empezar a clasificar los lenguajes por medio de qué tipo de gramáticas los generan. 

\begin{deff}
	Los lenguajes regulares son lenguajes generados por gramáticas en las cuales las producciones son del estilo
	\begin{enumerate}
		\item $A \to \lambda$
		\item $A \to a$
		\item $A \to a B$
	\end{enumerate}
	donde $A, B \in V$ y $a \in \Sigma$.
\end{deff}


\begin{comment}
	\begin{deff}
	Un automáta finito es una tupla ${\cal M} = (Q,\{q_0\},\Sigma,\delta,F)$ que la interpretamos de la siguiente manera:
	\begin{itemize}
	\item $Q$ son los estados.
	\item $q_0$ es el estado inicial.
	\item $\Sigma$ es el alfabeto.
	\item $\Delta$ es la función de transición entre los estados.
	\item $F$ es un conjunto de estados que llamaremos finales.
	\end{itemize}
	\end{deff}
	
	A los automátas finitos los interpretamos como grafos con algunas reglas para movernos sobre ellos. Para empezar $Q$ representan los vértices de nuestro grafo.
\end{comment}






%definición automáta finito
%definición de lenguaje regular por medio del automáta, las otras defs no son necesarias
% comentar que los regulares son cerrados para grupos? quizá pueda servir para después

\section{Conos de lenguajes y el problema de la palabra.}

En esta sección consideraremos un grupo $G$ finitamente generado y una presentación $\pi: \Sigma^* \twoheadrightarrow  G$ de este grupo con finitos generadores $\Sigma$. 

El problema de la palabra es uno de los problemas de teoría de grupos más centrales al área. Explicitamente el problema consiste en dada una palabra $\omega \in \Sigma^*$ en los generadores del grupo encontrar un algoritmo para decidir si esta palabra es la identidad del grupo o no. Notemos que para poder pensar este problema estamos fijando de antemano una presentación posible del grupo.

Dado un grupo finitamente presentado por alguna presentación $\pi$ consideramos el siguiente lenguaje 

$$\text{WP}_\pi (G) = \{ \omega \in \Sigma^* \ | \ \pi(\omega)=1 \}$$

que llamaremos \emph{el problema de la palabra de la presentación $\pi$ de $G$}. 



Así como definimos los lenguajes regulares vamos a preocuparnos por otros tipos de lenguajes siempre y cuando cumplan las siguientes condiciones.

\begin{deff}
	Una familia de lenguajes $\CC$ es un \emph{cono} si para todo $L \in \CC$ resulta que:
	\begin{itemize}
		\item Es cerrado por imágenes de morfismos de monoides. Sea $L \subset \Sigma^*$ luego si existe $\phi:\Sigma^* \to \Delta^*$ morfismo de monoides debe ser que $\phi(L) \in \CC$.
		\item Es cerrado por preimagenes de morfismos de monoides. Si $L \subset \Sigma^*$ luego si existe $\phi:\Sigma^* \to \Delta^*$ morfismo de monoides luego debe ser que $\phi^{-1}(L) \in \CC$. 
		\item Es cerrado por intersecciones con lenguajes regulares. Si $R$ es un lenguaje regular sobre $\Sigma^*$ luego $L \cap R \in \CC$ también.
	\end{itemize}
\end{deff} 

En particular los conos de lenguajes cumplen la siguiente propiedad.

\begin{prop}
	Sea $WP_\pi(G)$ el problema de la palabra de cierto grupo en alguna presentación $\pi$ y $\CC$ algún cono de lenguajes. Si $WP(G) \in \CC$ luego valen las siguientes afirmaciones:
	\begin{itemize}
		\item $WP(G) \in \CC$ para cualquier presentación elegida $\pi$.
		\item $WP(H) \in \CC$ para todo subgrupo $H$ finitamente generado de $G$.
		\item $WP(H) \in \CC$ para todo subgrupo $H$ de índice finito en $G$.
	\end{itemize} 
\end{prop}

Esto nos dice que en particular es interesante estudiar el problema de la palabra sobre conos. 

\section{Lenguajes independientes de contexto.}


Un \textit{lenguaje independiente de contexto} es un lenguaje tal que es generado por una gramática donde todas sus producciones son del estilo
\[
 \omega \to \lambda
\]
y cumplen que $|\omega| = 1$. 

\begin{ej}
	Sea el alfabeto $\Sigma = \{ a,b \}$. Si $w=a_1 \dots a_k$ es una palabra sobre $\Sigma^*$ entonces podemos considerar a $w^r$ que es la palabra inversa dada por leerla de derecha a izquierda y tiene la siguiente pinta $w^r= a_k \dots a_1$. Consideremos sobre $\Sigma$ el lenguaje $L = \{ w \in \Sigma^* \ | \ w = w^r  \}$ tal que este es el lenguaje de los palíndromos. Construyamos una gramática independiente de contexto para este lenguaje. Sea $w$ una palabra en $L$ luego sabemos que si su longitud es mayor que uno entonces debe ser que $w = a u a$ o $w = b u b$ para cierta palabra $u \in L$ dado que $u$ necesariamente tiene que ser un palíndromo porque $w$ lo es. Esto sucede para todas las palabras del lenguaje exceptuando las palabras de longitud uno que son justamente las letras del alfabeto. De esta manera podemos considerar la siguiente gramática ${\cal G}  =  (\{S\}, \Sigma, P, S )$ donde las producciones $P$ están dadas por :
	\begin{align*}
	S  & \to \epsilon \\ S &\to a \\ S &\to b \\ S &\to  aSa \\ S &\to bSb. \\
	\end{align*}
	Para ver que esta gramática genera al lenguaje $L$ notemos que si tomamos una palabra en el lenguaje $w \in L$ luego si no es una letra al ser palíndromo sobre el alfabeto $\Sigma$ necesariamente debe comenzar con $a$ o con $b$ y de esta manera tenemos que la primer derivación es $S \to aSa$ o $S \to bSb$. Dado que la subpalabra $w = aua$ que se obtiene de $w$ sin considerar la primera y última letra es un palíndromo (e incluso podría ser la palabra vacía) luego podemos repetir este proceso para llegar a $S \implies^* w$ después de finitos pasos. Por otro lado toda palabra generada por esta gramática es un palíndromo porque todas las reglas son tales que agregan una letra al principio y la misma al final o simplemente agregan letras que también son palíndromos.
	
\end{ej}

\subsection{Autómatas de pila.}
Así como las gramáticas nos permiten generar un lenguaje tenemos las máquinas que nos permiten aceptar un lenguaje. En nuestro caso en particular vamos a usar automátas de pila no determinísticos para aceptar los lenguajes independientes de contexto.

\begin{deff}
	Un \blue{automáta de pila finito no determinístico} es una tupla ${\cal M } = (Q, \Sigma, Z, \delta, q_0, F, Z_0)$ que la interpretamos de la siguiente manera,
	\begin{itemize}
		\item $Q$ denota los finitos estados;
		\item $\Sigma$ es el alfabeto finito del lenguaje que queremos reconocer;
		\item $\Gamma$ es el alfabeto finito de la pila;
		\item $\delta$ es la función de transición donde $\delta: Q  \times \Sigma \times \Gamma \to {\cal P}( Q  \times \Gamma^*)$;
		\item $F \subseteq Q$ es el conjunto de estados finales;
		\item $q_0$ es el estado inicial;
		\item $Z_0$ es el símbolo inicial en la pila.
	\end{itemize}
\end{deff}


\paragraph{Funcionamiento del automáta.}
%Un prefijo de $w$ será una subpalabra $\gamma$ tal que $w=\gamma w'$ visto con la concatenación de palabras en $\Sigma^*$.

El automáta de pila finito funciona de la siguiente manera. Dada una palabra $w \in \Sigma$ queremos saber si es aceptada por el autómata de pila o no. Para eso vamos a ir leyendo esta palabra de izquierda a derecha. Al comenzar a leer esta palabra estamos en el estado $q_0$ que distinguimos como el estado inicial de nuestro automáta. Nos fijamos en la función de transición que es una función parcial cuánto nos da evaluada  $\delta(\lambda,q_0,\gamma)$ para algún $\gamma$ prefijo de $w$ y donde estamos mirando a $\lambda \in Z^*$ el elemento neutro de este monoide. Esto se corresponde a la idea de que al comenzar nuestra pila está vacía. En tal caso nuestra función de transición nos da un resultado que es un par $(z,q)$ donde $z \in Z^{*}$ es lo que nos va a quedar en la pila y $q$ es el nuevo estado al cual nos movimos. Notemos que nuestra función de transición no tiene porqué tener un $(z,q)$ tal que podamos movernos o podría ser bien que tenga más de uno. En este caso diremos que nuestro automáta es \textit{no determinístico} dado que en algunos casos existe más de una opción.

En general estamos en la siguiente situación en el proceso de aceptar la palabra $w$. Tenemos algún $z \in Z^{*}$ en la pila que al ser una pila lo leeremos de derecha a izquierda, estamos en algún estado $q \in Q$ y nos quedaré una subpalabra $\gamma$ de $w$ para leer. Una \textit{configuración} de nuestro automáta entonces es una manera de describir en que situación de aceptar o no aceptar una palabra y la denotamos $(zq\gamma)$. 

Cuando hayamos visto toda la palabra o no tengamos manera de movernos de estado nos fijamos si el estado $p$ en el que estamos es final, es decir si $p \in F$. En tal caso la palabra $w$ es aceptada por el automáta. Formalmente estaremos en alguna configuración $zq$ para $z \in Z^*$ y $q \in Q$ y no nos queda nada de la palabra $w$ porque ya la consumimos toda.

\paragraph{Descripción instántanea del automata.} Veamos ahora como describir formalmente el funcionamiento de un automáta a partir de lo que estamos haciendo en cierto instante. 
Consideremos que estamos en el instante que nuestra subpalabra que nos queda por leer es $w$, estamos en un estado $q$ y en nuestra pila tenemos la palabra $\gamma$, entonces vamos a representar al instante por medio de esta tupla $(q,w,\gamma)$.
Si ahora tenemos la posibilidad de movernos a otro estado $p$ tal que $(p,w',\alpha\beta) \in \delta (q,aw',x\beta)$ donde $aw' = w$ con $a$ alguna letra posiblemente vacía y similarmente $x \beta = \gamma$ con $x \in \Gamma^*$ una letra de $\Gamma$ posiblemente vacía. Este movimiento lo denotamos como $(q,aw',x\beta) \vdash (p,w',\alpha \beta)$. Podemos considerar la clausura transitiva de esta relación sobre los triples $Z^* \times Q \times \Sigma^*$ que denotaremos $\vdash^*$.


\paragraph{El lenguaje aceptado por un automáta de pila.} Notemos que en particular el automáta de pila nos da un lenguaje que está formado por las palabras $w$ en el alfabeto de la entrada del automáta $\Sigma$ que son aceptadas. En general diremos que un automáta acepta un lenguaje $L$ si su lenguaje aceptado es exactamente $L$. Este lenguaje aceptado por el automáta $\cal M$ en algunas casos para hacer énfasis en el automáta lo denotaremos $L {(\cal M)}$ y siguiendo la notación formal anterior lo podemos describir de la siguiente manera,
\begin{equation*}
	L( {\cal M}) = \{ w \in \Sigma^* \ | \ (q_0,w,Z_0) \vdash^* (q, \epsilon, \gamma), \ \ q \in F, \ \gamma \in \Gamma^*      \}
\end{equation*}
donde $(q_0, w, Z_0)$ es el instante inicial en el cual tenemos la palabra $w$ en el estado inicial $q_0$ con el símbolo $Z_0$ de la pila. Al finalizar deberíamos estar en un estado $q$ final y lo que nos queda en la pila $\gamma$ es totalmente irrelevante en este caso.

\begin{obs}
	Al ser un automáta de pila no determinístico existen posiblemente más de una manera de consumir alguna palabra en el automáta. Es así que por la definición que dimos la palabra es aceptada por el automáta si al menos alguna de estas derivaciones la lleva a ser aceptada. No importa si existen derivaciones que no lo hagan mientras una sí lo haga. 
\end{obs}
\medskip
\begin{teo}
Un lenguaje $L$ es independiente de contexto sii es aceptado por un automáta de pila no determinístico.
\end{teo}

\begin{proof}
	\red {HACER}
\end{proof}

De esta manera sabemos que en particular el lenguaje de los palíndromos definido anteriormente es aceptado por un automáta de pila no determinístico. Veamos como construirnos este automáta en este caso en especial.
\begin{ej}
	Sea nuestro alfabeto $\Sigma = \{ a, b\}$ y $L$ el lenguaje de los palíndromos sobre este alfabeto. Consideremos el siguiente autómata de pila $M=(Q,\{q_0\} ,\{a,b\}, \{a,b,\$\}, \$, \{q_1\})$ donde nuestra pila tiene el mismo alfabeto que el de entrada con un símbolo extra que es \$ que va a ser nuestro símbolo inicial de la pila.
	
	\begin{tikzpicture}[->,>=stealth',shorten >=1pt,auto,node distance=3.5cm,
	scale = 1,transform shape]
	
	\node[state,initial] (q_0) {$q_0$};
	\node[state,accepting] (q_1) [right of=q_0] {$q_1$};
	
	\path (q_0) edge [bend right]             node {} (q_1);
	\path (q_1) edge [loop above] node {}    (   );
	\path (q_0) edge [loop above] node {}    (   );
	\end{tikzpicture}

	El automáta tiene dos estados, el inicial y el final. La idea es que en el primer estado vamos apilando la palabra y en la segunda vamos desapilando la palabra anteriormente apilada. Nuestra función de transición va a ser la siguiente,
	\begin{itemize}
		\item $\delta(q_0,a,Z) = (q_0,aZ)$ 
		\item $\delta(q_0,b,Z) = (q_0,ba)$ 
		\item $\delta(q_1,a,a) = (q_0,\lambda)$.
		\item $\delta(q_1,b,b) = (q_0,\lambda)$.
		\item $\delta (q_0, \lambda, Z) = (q_1,Z)$
		\item $\delta (q_0, X, Z) = (q_1, Z)$ 
	\end{itemize}
 		donde $Z=a,b, \$$ es decir cualquier cosa del alfabeto de la pila y $X = a, b$ cualquier elemento de nuestro alfabeto. El automáta en el primer estado apila lo que sea que estemos leyendo sin importar lo que esté en el tope de pila. En el segundo estado desapila cada vez que lo que estemos leyendo coincida con el tope de pila. Finalmente para ir del estado inicial al final tenemos en cuenta dos casos. Nos podemos mover por $\lambda$ es decir sin consumir ninguna letra de la palabra de la entrada o leyendo alguna de las letras de la palabra. Estos casos se corresponden a que el palíndromo tenga longitud par o tenga longitud impar.	
\end{ej}


Hasta ahora definimos los automátas de pila no determinísticos que aceptan por estado final. Otra definición posible de lenguaje aceptado podría ser que acepten por pila vacía. Es decir que una vez que consumimos la palabra $w$ de entrada llegamos a una configuración $(q, \lambda, \lambda)$ donde $\lambda$ es la palabra vacía de ambos alfabetos respectivamente. Formalmente notaremos al lenguaje aceptado por pila vacía por un automáta $\cal M$
de la siguiente manera,
\begin{equation*}
	L({\cal M}) = \{ w \in \Sigma^* \ | \ (q_0,w,Z_0) \vdash^* (q, \lambda, \lambda), \ q \in Q, u \in \Sigma^*    \}
\end{equation*}
donde arrancamos en el estado inicial y llegamos a algún estado $q$ cualesquiera y nuestra pila está vacía así como lo que nos queda por leer de la palabra. En este caso obtenemos que nuestro lenguaje es aceptado por un automáta de pila no determinístico por pila vacía.


El siguiente resultado nos dice que en el caso que nuestro automáta sea no determinístico es equivalente usar una u otra manera de definir a nuestro lenguaje.

\medskip
\begin{teo}
Un lenguaje $L$ es aceptado por un automáta de pila no determinístico por estado final sii es aceptado por un automáta de pila no determinístico por pila vacía.
\end{teo}

\begin{proof}
	\red{HACER}
\end{proof}

\section{Automátas de pila determinísticos.} Si en la definición anterior del automáta pedimos que la función de transición de una configuración dada tenga a lo sumo un valor entonces nuestro automáta lo vamos a llamar determinístico. Formalmente esto es que 
\[
|\delta(q,a, z)| \le 1 \ \ \ \forall z \in Z, \ p \in Q, \ a \in \Sigma \cup \{ \lambda \}.
\]
En cierta manera estamos diciendo que de un instante dado solo tenemos a lo sumo una única posibilidad de movernos a otro estado. A los lenguajes aceptados por un \APD los pensamos que son aceptados por un estado final.


 Por lo tanto para cada palabra tenemos un único camino en el automáta para saber si es aceptada o no a diferencia de un \APND  que podría tener varios caminos posibles para cada palabra.
\begin{obs}
	Todo \APD en particular es no determinístico y por lo tanto la clase de lenguajes aceptados por los primeros están contenidos en la clase de los segundos.
\end{obs}

Veamos que esta contención es estricta. Para eso volvamos a considerar el ejemplo del lenguaje de los palíndromos.
\begin{ej}
	El lenguaje $L = \{ w \in \{ a,b \}^*  \ : \ w = w^r \}$ no es aceptado por un \APD pero sí por uno no determinístico. Supongamos que $M$ es un \APD que lo acepta. Notemos que para cualquier palabra $w \in \{ a,b \}^*$ debe ser que al consumirla la pila no puede quedar vacía dado que $ww^r \in L$ y en tal caso no aceptaría a esta palabra. Esto es que la pila nunca está vacía sea cual sea la configuración que lleguemos. Para cada palabra arbitraria $w$ existe otra $x_w$ tal que al procesar $wx_w$ lo que nos queda en la pila es de tamaño mínimo con respecto a todas las palabras $wx$. Sea entonces lo que tiene en la pila la palabra $\alpha_w$ que sabemos es de longitud mínima. Si consideramos palabras del estilo $wx_wz$ sabemos que la longitud de lo que quede en la pila no puede disminuir. Ahora consideremos dos palabras del estilo $r=wx_w, s=uy_u$ tales que sus pilas son de longitud mínima al terminar de recorrer las palabras y que resultan tener el mismo tope de pila y terminar en el mismo estado. Podemos asegurar la existencia de estas palabras debido a que tenemos finitos estados y combinaciones de tope de pilas dado que el automáta de pila es finito pero tenemos infinitas palabras que cumplen esta propiedad. Ahora basta con elegir $z$ de modo que $tz$ sea palíndromo pero que $sz$ no lo sea. 
	
	Veamos que podemos elegir a $z$ para que una de las concatenaciones $tz,sz$ sea palíndromo y la otra no. Partamos en distintos casos. Si $|t|=|s|$ basta con tomar $z=s^r$. Si $|t|\neq |s|$ y supongamos que $s$ tiene longitud menor y no es prefijo de $t$ entonces de nuevo podemos tomar el palíndromo $ss^r$ tal que $ts^r$ no es un palíndromo. Finalmente queda el caso que una es un prefijo de la otra, supongamos $t=su$. Si elegimos $x=a,b$ tal que $ux$ no sea un palíndromo luego la palabra $ss^r$ es un palíndromo pero $suxs^r$ no lo es.
	
	Esto muestra que si bien $sz \notin L$ y $tz \in L$ el \APD no va a poder diferenciarlas por lo tanto no es posible que este lenguaje sea aceptado por un \APD tal como queríamos ver.
	
\end{ej}

\subsection{Automátas de pila determinísticos especiales.} Consideremos ahora un \APD tal que acepta tanto por estado final como por pila vacía. A estos los llamaremos \textit{automátas de pila determinístico especiales}.  
Estos automátas son los que nos surgen de la construcción del automáta del problema de la palabra  para grupos virtualmente libres. 

\begin{ej}
	Sea el lenguaje $L = \{ a^m b^n  : m \ge n \ge 1 \}$ este no es un lenguaje independiente de contexto determinístico especial pero sí es determinístico. 
	
	Construyamos un \APD que acepte a $L$. Sea $$M = (\{q_0,q_1,q_2\}, \{q_0\}, \{a,b\}, \{a,b,Z_0\}, Z_0, q_2) $$ el siguiente \APD que representamos así:
	
	\begin{tikzpicture}[->,>=stealth',shorten >=1pt,auto,node distance=3.5cm,
	scale = 1,transform shape]
	
	\node[state,initial] (q_0) {$q_0$};
	\node[state] (q_1) [right of=q_0] {$q_1$};
	\node[state,accepting] (q_2) [right of=q_1] {$q_2$};
	
	\path (q_0) edge    [bend right]          node {$b, a | \lambda$} (q_1)
	(q_0) edge    [loop above]          node {$a, Z | aZ$} (q_0)
	(q_1) edge      [bend right]      node {$\lambda, a | Z$}   (q_2)
	(q_1) edge    [loop above]           node {$b, a | \lambda$} (q_1);
	
	\end{tikzpicture}
	
	donde $Z$ es cualquier letra del alfabeto de la pila. El automáta en el estado inicial $q_0$ apila a todas las $a$ y cambia al estado $q_1$ cuando lee por primera vez una $b$ y en ese caso desapila la a que está en el tope de la pila. En el estado $q_1$ sigue desapilando cada vez que ve una $b$. Finalmente va al estado $q_2$ cuando en la pila sigue quedando $a$ y ya leímos toda la palabra y en tal caso la acepta porque significa que vimos como máximo tantas $b$ como $a$ y este es el lenguaje que buscábamos generar.
	
	El lenguaje no es aceptado por un \APD por pila vacía dado que tiene la propiedad de los prefijos. Es decir que existen palabras que están en el lenguaje tales que alguno de sus prefijos también están. Por ejemplo consideremos $a^m b^i$ y  $a^m b^j$ para $m \ge 2$ e $i < j \le m$. Esto es porque si $M = (Q, \Sigma, Z, Z_0, \delta, q_0 , F)$ fuera un \APD que acepta por pila vacía a este lenguaje tendríamos que $(q_0,a^mb^i,Z_0) \vdash^* (q,\lambda, \lambda)$ con $q$ un estado final pero como tiene la pila vacía no podemos continuar aceptando a la palabra $a^mb^j$ ya que por la definición que empleamos el automáta necesita leer algún elemento de la pila. De esta manera vemos que este lenguaje no puede ser aceptado por pila vacía y estado final por un \APD concluyendo que los lenguajes determinísticos especiales forman un subfamilia propia de los independientes de contexto determinísticos.
	
	\begin{comment}[Esto es con la otra definición de automáta de pila.]
			Por el principio del palomar podemos ver que este lenguaje no es aceptado por \APD por pila vacía y estado final a la vez. Si así lo fuera supongamos que existe un automáta que lo acepta $M = (Q, \Sigma, Z, Z_0, \delta, q_0 , F)$. Debido a que tiene finitos estados podemos elegirnos $m$ suficientemente grande tal que existan palabras $a^mb^j, a^mb^i$ con $1 \le i < j < m$ y elegidas de manera que terminen en el mismo estado final $p$. Notemos que $(q_0,a^mb^ib^{m-j+1},Z_0) \vdash^* (q,\lambda, Z)$ donde $q$ es un estado final dado que $m-j+1+i \le m$. Por otro lado debe ser que $(q_0, a^mb^jb^{m-j+1}) \vdash^* (q,\lambda, Z) $ pero notemos que $a^mb^{m-1} \notin L$. Por lo tanto llegamos a una contradicción y de esta manera vemos que $L$ no es aceptado por un \APD especial tal como queríamos ver.
	\end{comment}

	
\end{ej}





\section{Lenguajes poly independientes de contexto.}

En esta sección vamos a introducir otra familia de lenguajes que generalizan levemente a los lenguajes independientes de contexto.





\section{Sistemas de reescritura.}

Un \emph{sistema de reescritura} podemos pensarlo como un grafo en el sentido de Serre. En este caso los vértices son los \emph{objetos} mientras que las aristas las llamamos \emph{movimientos}.  Si nuestro sistema de reescritura $\Gamma$ tiene un movimiento de un objeto $a$ en otro objeto $b$ diremos que $a$ puede ser \emph{reescrito} a $b$ y lo denotaremos $a \to_{\Gamma} b$, omitiendo aclarar que estamos considerando el sistema $\Gamma$ en todo caso que no sea ambiguo. Un camino en el grafo lo llamaremos una \emph{derivación}.  

Nosotros queremos que estos sistemas de reescritura sean tales que si tomamos una sucesión de movimientos en algún momento se estabilice y en tal caso llamaremos \emph{terminante.} A su vez los objetos tales que no puedan ser modificados por ningún movimiento llamaremos objetos \emph{terminales.} 

Dado un sistema de reescritura $\Gamma$ por medio de la clausura transitiva que denotaremos $\overset{*}{\rightarrow}_{\Gamma}$ obtenemos una relación transitiva sobre los objetos. Nuestro interés justamente va a estar en estudiar las relaciones de equivalencia que surgen de sistemas de reescritura particulares. 

\begin{ej}
	Considerar como objetos los enteros y como movimiento dividir al número por dos si es posible. Este sistema de reescritura tiene como objetos terminales los números impares.
\end{ej}

Otra característica que le vamos a pedir a los sistema de reescritura es que sean \emph{confluentes}. Informalmente esto es que a partir de un objeto, si usamos dos derivaciones distintas entonces eventualmente estas derivaciones se encuentran en algún objeto independientemente de qué derivaciones tomemos. Esto es que dadas derivaciones $a \overset{*}{\rightarrow} b, a \overset{*}{\rightarrow} c$ existe un objeto $d$ tal que $b \overset{*}{\rightarrow} d$ y $c \overset{*}{\rightarrow} d$.

En particular los sistemas que nos van a interesar en este trabajo son sistemas tales que son confluentes y terminantes que son conocidos como \emph{Church-Roser}.

\subsection{Sistemas de reescritura de cadenas.}
En nuestro caso en particular vamos a trabajar con sistemas de reescritura donde los objetos son palabras sobre algún alfabeto $\Sigma$ y los movimientos son reglas del estilo $w \to v$ que interpretamos de la siguiente manera. Si tenemos alguna palabra que tenga como subpalabra a $w$ podemos modificarla por $v$. Esto lo representamos por el triple $(p, w \to v, q)$ donde $p,q$ son las subpalabras que vienen antes de $w$ tales que pueden ser vacías. 

Al sistema de reescritura $\Gamma$ con alfabeto $\Sigma$ y las reglas $\cal R$ lo vamos a denotar por sr$\left< \Sigma, {\cal R} \right>$. Estos sistemas también son llamados \emph{sistemas de Thue}.






\end{document}

