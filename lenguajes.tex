% !TeX TS-program = 
\documentclass[tesis.tex]{subfiles}

%\newcommand{\ic}{independiente de contexto }
%\newcommand{\APND}{automáta de pila no determinístico }
%\newcommand{\APD}{automáta de pila determinístico }
%\newcommand{\gramatica}{{\cal G} = (V, \Sigma, P, S)}
%\newcommand{\deriva}{\overset{*}{\Rightarrow_{\cal G}}}
%\newcommand{\lengderivado}{L({\cal G})}
%\newcommand{\fg}{grupo finitamente generado }

\begin{document}
\chapter{Grupos independientes de contexto.}
Este capítulo sigue las ideas de los trabajos \cite{muller1983groups} y \cite{muller1985theory} completando muchos detalles que son omitidos por estos.

\section{El problema de la palabra.}


Sea $G$ un grupo finitamente generado por $A$ entonces si tomamos $B = A \cup A^{-1}$ el conjunto simétrico de generadores resulta que $G$ es finitamente generado como monoide por $B$, en otras palabras, existe un epimorfismo de monoides $\pi:B^* \to G$.
El problema de la palabra consiste en dadas $w,w' \in B^*$ determinar si $\pi(w) {=} \pi(w')$.
Equivalentemente esto es ver si $\pi(w'w^{-1}) = 1$.
Definimos entonces el siguiente lenguaje.

\begin{deff}
	Dado $G$ un grupo finitamente generado 
	y $\Sigma$ conjunto finito con un epimorfismo de monoides $\pi:\Sigma^* \to G$ entonces
	el \emph{problema de la palabra de $G$ para los generadores $\Sigma$} es el siguiente lenguaje	
	\[ 
	\text{WP} (G, \Sigma) = \{ w \in \Sigma^* \ | \ \pi(w)= 1 \}
	\]
\end{deff}

\begin{obs}
	En esta definición que damos del problema de la palabra no estamos considerando conjuntos simétricos de generadores sino que estamos considerando conjuntos de generadores como monoides arbitrarios.
	Esto es que $\Sigma$ no necesariamente cumple que $\Sigma = A \cup A^{-1}$ para $A$ conjunto finito de generadores de $G$ como grupo.
\end{obs}


Dado que este lenguaje depende del conjunto de generadores elegido nos gustaría ver qué condiciones tienen que cumplir los lenguajes o bien qué condiciones deberán cumplir los grupos para que no exista esta dependencia en los generadores.
En este contexto nos van a interesar las familias de lenguajes que cumplen las siguientes propiedades.
\medskip
\begin{deff}
	Una familia de lenguajes $\CC$ es un \emph{cono} si para todo $L \in \CC$ tal que $L \subseteq \Sigma^*$ resulta que:
	\begin{itemize}
		\item[\textbf{C1.}] Es cerrado por imágenes de morfismos de monoides. 
		Si existe $\phi:\Sigma^* \to \Delta^*$ morfismo de monoides debe ser que $\phi(L) \in \CC$.
		\item[\textbf{C2.}] Es cerrado por preimagenes de morfismos de monoides. 
		Si existe $\phi:\Delta^* \to \Sigma^*$ morfismo de monoides  debe ser que $\phi^{-1}(L) \in \CC$. 
		\item[\textbf{C3.}] Es cerrado por intersecciones con lenguajes regulares. 
		Si $R \subseteq \Sigma^*$ es un lenguaje regular entonces $L \cap R \in \CC$.
	\end{itemize}
\end{deff} 


Los conos de lenguajes cumplen la siguiente propiedad de gran importancia para el estudio del problema de la palabra.
\medskip
\begin{prop}\label{prop-cono-wp}
	Sea $WP(G, \Sigma)$ el lenguaje del problema de la palabra de cierto grupo $G$ para algunos generadores $\Sigma$ y $\CC$ cono de lenguajes. 
	Si $WP(G, \Sigma) \in \CC$ luego valen las siguientes afirmaciones:
	\begin{itemize}
		\item[\textbf{W1.}] $WP(G, \Delta) \in \CC$ para cualquier conjunto de generadores $\Delta$.
		\item[\textbf{W2.}] $WP(H) \in \CC$ para todo subgrupo $H$ \fg de $G$.
	\end{itemize} 
\end{prop}
\begin{proof}
	Vamos a probar \textbf{W1}.
	Como $\Delta$ y $\Sigma$ son dos conjuntos finitos de generadores de $G$ entonces tenemos dos epimorfismos de monoides:
	$\delta: \Delta^* \to G$ y $\pi:\Sigma^* \to G$.
	Si $\Delta = \{ b_{1}, \dots, b_{n} \}$ entonces definimos $\Delta' = \{ b'_{1}, \dots, b'_{n}   \} \subseteq \Sigma^*$ de manera que $b'_{i} \in \pi^{-1}(\delta(b_{i}))$ para todo $1 \le i \le n$.
	Tenemos una función biyectiva $f:\Delta \to \Delta'$ definida como $f(b_{i}) = b'_{i}$ para todo $1 \le i \le n$.
	Por la propiedad universal de los monoides libres tenemos que existe $\ol f: \Delta^* \to \Sigma^*$ morfismo tal que hace conmutar al siguiente diagrama
	\begin{center}
		\begin{tikzcd}
			\Delta^* \arrow[r, "\ol{f}",dashed ]  & \Sigma^* \\
			\Delta \arrow[ru, "f",  swap]  \arrow[u,"\iota",swap]  &  
		\end{tikzcd}
	\end{center}
	
	Por la construcción que hicimos del morfismo $\ol f$ tenemos que el siguiente diagrama también conmuta:
	\begin{center}
		\begin{tikzcd}
			\Delta^* \arrow[r, "\delta"] \arrow[d,"\ol{f}",swap] & G \\
			\Sigma^* \arrow[ru, "\pi",swap]    &  
		\end{tikzcd}
	\end{center}
	
	Notemos que como el diagrama conmuta tenemos que 
	\[
	\ol{f}^{-1}(\pi^{-1}(1)) = \delta^{-1}(1)
	\]
	lo que implica que por la definición del lenguaje del problema de la palabra que
	\[
	\ol{f}^{-1}(WP(G,\Sigma)) = WP(G, \Delta).
	\]
	Como $WP(G, \Sigma) \in \CC$ resulta que $\ol{f}^{-1}(WP(G,\Sigma)) \in \CC$ por lo tanto $WP(G, \Delta) \in \CC$ tal como queríamos ver.
	
	Veamos ahora que vale \textbf{W2}. 
	Sea $\Sigma$ conjunto de generadores de $H$.
	Podemos extender este conjunto a otro conjunto finito $\Sigma'$ tal que $\Sigma'$ genera a $G$ y $\Sigma \subseteq \Sigma'$. 
	De esta manera la siguiente igualdad es inmediata
	\[
	WP(H, \Sigma) = WP(G, \Sigma') \cap \Sigma^*
	\]
	y dado que $WP(G, \Sigma') \in \CC$ es un cono entonces $WP(H, \Sigma) \in \CC$.
	
\end{proof}

La moraleja de este resultado es que si $\CC$ es un cono entonces están bien definidos los grupos $G$ tales $WP(G) \in \CC$, es decir que no depende qué generadores tomemos del grupo. 
 

\section{Propiedades de los lenguajes independientes de contexto.}

En esta sección vamos a probar algunas propiedades que tienen los lenguajes independientes de contexto vistos como una clase de lenguajes.
La referencia de estos resultados es \cite{hopcraft-ullman}.


La familia de lenguajes independientes de contexto va a ser la siguiente:
\[
	\text{IC} = \{  L \mid \exists \Sigma, \ L \subseteq \Sigma^*, \ \ L \  \text{es independiente de contexto} \}.
\]
de manera similar podemos definir la familai de lenguajes regulares como
\[
	\text{REG} = \{  L \mid \exists \Sigma, \ L \subseteq \Sigma^*, \ \ L \  \text{es regular}  \}.
\]

La primera propiedad que vamos a ver de los lenguajes independiente de contexto es que son cerrados con respecto a la intersección con lenguajes regulares.

\begin{prop}\label{intersecciones-reg-ic}
	Sea $L \in \text{IC}$ y sea $R \in \text{REG}$ tales que existe $\Sigma$ de manera que $L,R \subseteq \Sigma^{*}$ entonces $L \cap R \in \text{IC}$.	
\end{prop}

\begin{proof}
	Sea $L$ un lenguaje \ic tal que $L \subset \Sigma^*$ y sea $R$ un lenguaje regular tal que $R \subset \Sigma^*$. 
	Queremos ver que el lenguaje $L \cap R$ es aceptado por un \APND.
	
	Si $L$ es aceptado por un \APND ${\cal M } = (Q, \Sigma, Z, \delta, q_0, F, Z_0)$ y R es aceptado por un autómata no determinístico ${\cal M' } = (Q', \Sigma, \delta', q_0', F',)$.
	
	Consideremos el siguiente \APND
	\[
		{\cal N } = (Q \times Q', \Sigma, Z, \delta \times \delta', (q_0,q_0'), F \times F', Z_0).
	\]	
	
	Queremos ver que $L({\cal N}) = L \cap R $.	
	Probemos primero la siguiente afirmación.
	\[
	((q_0,q_0'), w, Z_0) \vdash^*_{\cal N}  ((q_i,q_j), \epsilon, z) \ \iff (q_0, w, Z_0) \vdash^*_{\cal M}  (q_i, \epsilon, z) \ \text{y} \ (q_0', w) \vdash^*_{\cal M'} (q_j, \epsilon)  	
	\]
	Para ver esto veremos las dos implicaciones a la vez haciendo inducción en la longitud de la palabra $w$.
	
	En el caso base $|w| = 1$ de manera que $w = a \in \Sigma$.
	Este caso tenemos la igualdad porque justamente la función de transición del \APND $\cal N$ es $\delta \times \delta'$.
	
	Para el paso inductivo consideremos que $|w|=n$ de manera que $w=ua$ con $|u|=n-1$ y $a \in \Sigma$.
	Luego tenemos que $((q_0,q_0'), u, Z_0) \vdash^*_{\cal N}  ((q_k,q_l), \epsilon, z') \ \iff (q_0, u, Z_0) \vdash^*_{\cal M}  (q_k, \epsilon, z') \ \text{y} \ (q_0', u) \vdash^*_{\cal M'} (q_l, \epsilon)$.
	Nuevamente usamos nuestra definición de la función de transición $\delta \times \delta'$ para concluir que
	\[ 
	((q_i,q_j),\epsilon, z) \in \delta \times \delta'((q_k,q_l), a, z') \iff  (q_i, a, z') \in \delta(q_k, a, z') \ \text{y} \ (q_j, a) \in \delta(q_l, a).
	\]	
	Con esto terminamos de probar la afirmación.
	
	Finalmente para ver que $L({\cal N})  =  L \cap R$ usamos que por nuestra afirmación si $w \in L \cap R$ entonces es aceptado por lo dos autómatas $\cal M$ y por $\cal M'$, esto es que $(q_0, w, Z_0) \vdash^* (q_F, \epsilon, z)$ y que $(q_0', w) \vdash^* (q_F', \epsilon)$ y equivalentemente $((q_0,q_0'),w,Z_0) \vdash^* ((q_F, q_F'),\epsilon, z)$. 
	Dado que los estados finales del autómata $\cal N$ resultan ser $F \times F'$ obtenemos que $w \in L({\cal N}) \iff w \in L \cap R$.
	Concluimos así que el lenguaje $L \cap R$ resulta ser \ic tal como queríamos ver.		
\end{proof}

Otras propiedades que nos van a interesar tienen que ver con la relación que tienen con los morfismos de monoides.
Precisamente probaremos que los lenguajes \ic como clase son cerrados por imágenes de morfismos de monoides y por preimágenes de morfismos de monoides.
 

\begin{prop}\label{morfismos-monoides-ic}

		Sea $L \in \text{IC}$, $\Sigma$ tal que $L \subseteq \Sigma^{*}$ y $h:\Sigma^{*} \to \Delta^*$ un morfismo de monoides 
		 entonces $h(L) \in \text{IC}$.
		 
\end{prop}

\begin{proof}
		Consideramos $L$ lenguaje \ic sobre $\Sigma$.
		Esto nos dice que existe una gramática $\gramatica $ tal que $L(\cG) = L$.
		Si tenemos un morfismo de monoides $h: \Sigma^* \to \Delta^*$ queremos ver que existe una gramática $\cG'$ tal que $L(\cG') = h(L)$.
		Para eso consideramos $\cG  = (V, \Sigma, P', S)$ donde la única diferencia es que reemplazamos en cada producción $P \in \cG$ a cada letra $a \in \Sigma$ por $h(a) \in \Delta^*$.
		La gramática sigue siendo \ic, nos alcanza con ver que genera al lenguaje que queremos.
		
		Miramos las dos contenciones a la vez.		
		Si tomamos una palabra $w \in L$ luego tenemos que $h(w) \in h(L)$ por lo tanto tiene que haber una sucesión de producciones $S  \implies^* w  $ en la gramática $\cG$.
		Si tomamos esta derivación y cambiamos cada producción $A \to \alpha \in P$ por la correspondiente $A \to \beta \in P'$, donde $\beta = h(\alpha)$, en la gramática $\cG'$ ahora obtenemos la palabra $h(w)$.
		Volviendo para atrás obtenemos que $L(\cG') = h(L)$ tal como queríamos ver.

\end{proof}


\begin{prop}
	Sea $L \in \text{IC}$, $\Sigma$ tal que $L \subseteq \Sigma^{*}$ y $h:\Delta^{*} \to \Sigma^*$ un morfismo de monoides 
	entonces $h^{-1}(L) \in \text{IC}$.
\end{prop}

\begin{proof}
	Sea $L$ un lenguaje \ic sobre el alfabeto $\Sigma$ tal que es aceptado por un \APND ${\cal M } = (Q, \Sigma, Z, \delta, q_0, F, Z_0)$.
	Consideremos un morfismo de monoides $h: \Delta^* \to \Sigma^*$. 
	Queremos ver que el lenguaje $h^{-1}(L) \subset \Delta^*$ resulta ser \ic. 
	Primero introduzcamos la constante $ n = \max_{a \in \Delta} |h(a)|$.
	
	Ahora consideremos el siguiente \APND 
	\[
	{\cal M' } = (Q', \Delta, Z, \delta, q_0', F', Z_0)
	\]
	donde $Q' = Q \times \Sigma^{\le n}$, donde $\Sigma^{\le n} = \{ w \in \Sigma^* : |w| \le n  \}$.
	Los estados finales $F' = F \times \{ \epsilon \}$.
	El estado inicial es $q_0' = (q_0, \epsilon)$.
	Finalmente nuestra función de transición resulta ser 
	\[
	\delta'((q,v), a ,z) = 
	\begin{cases}
		\{(q,h(a), z )\}  \ &\text{si} \ v=\epsilon, z \in Z \\
		\{(p,u),z' \} \ &\text{si} \ (p,z') \in \delta(q,y,z), v=yu \\  
		\emptyset \ &\text{caso contrario}.
	\end{cases}
	\]
	
	Probemos primero la siguiente afirmación.
	Para cada $w \in \Delta^*$ tenemos la siguiente equivalencia,
	\[
	(q_0',w,z) \vdash^*_{\cal M'} ((q,\epsilon), \sigma) \iff (q_0,h(w), z) \vdash^*_{\cal M} (q, \sigma).
	\]
	
	Para probar esta equivalencia lo hacemos por inducción en la longitud de $|w|$.
	Para el caso base tenemos que $w = a \in \Delta$.
	En este caso tenemos que $(q_0', a, z) \vdash ((q,h(a)), z)$.
	Una vez en este estado por como definimos la función de transición tenemos que $ (q_0,h(a), z) \vdash (q, \sigma)  \iff ((q_0,h(a)), \epsilon ,z) \vdash ((q,\epsilon), \sigma)$.
	
	El caso general tenemos una palabra $w$ tal que $|w|=n$.
	Sabemos que la afirmación vale para cualquier palabra de longitud menor a $n$.
	En particular si $w=ua$ con $|u|=n-1$ y $a \in \Delta$, tenemos que por la hipótesis inductiva que 
	\[
	(q_0',u,z) \vdash^*_{\cal M'} ((q,\epsilon), \sigma) \iff (q_0,h(u), z) \vdash^*_{\cal M} (q, \sigma)
	\]
	entonces de nuevo por el mismo razonamiento que hicimos para el caso base tenemos probada la afirmación.
	
	Para concluir la demostración notemos que por nuestra afirmación una palabra $w \in L(\cal M')$ si y solo si $h(w) \in L$. 
	Esto nos dice que el lenguaje aceptado por estado final de $\cal M'$ resulta ser $L' = \{ w \in \Delta^* : h(w) \in L \} = h^{-1}(L)$.
\end{proof}

Finalmente la herramienta principal que tenemos para ver que cierto lenguaje $L$ no es \ic es usar el siguiente lema.

\begin{lema}[Pumping] \label{pumping}
	Sea $L$ un lenguaje independiente de contexto entonces existe una constante $n \ge 0$ tal que para todas las palabras $w \in L$ de longitud al menos $n$ existe una factorización $w = uvxwy$ con $|vwx| \le n$ y $|vx| > 0$ tal que para todo $i \in \NN$ vale que $uv^iwx^iy \in L$.
\end{lema}

\begin{proof}
	Resultado estándar. Ver \cite{hopcraft-ullman}.
\end{proof}

\section{Grupos independiente de contexto.}

La flia de lenguajes $\text{IC}$ es un cono.
Esto se puede ver a partir de las proposiciones  \ref{intersecciones-reg-ic} y \ref{morfismos-monoides-ic}.
Esto nos dice que la siguiente definición no depende de los generadores que tomemos.

\begin{deff}
	Si $G$ es un \fg tal que es generado por un conjunto finito $\Sigma$ y el lenguaje $WP(G, \Sigma)$ es independiente de contexto entonces diremos que $G$ es un \emph{grupo \ic }.
\end{deff}


\begin{ej} Consideremos los siguientes ejemplos.	
	\begin{enumerate}[E1.]
		\item 
		Dado $F$ grupo libre generado por un conjunto finito $A$ tal que $1 \notin A$. 
		Sea  $B = A \cup A^{-1}$ el conjunto de generadores simétrico de $A$. 
		Probemos que $\text{WP}(F,B)$ es un lenguaje independiente de contexto.
		
		Consideremos el siguiente autómata de pila,
		\[
		M = (\{ q_{0},q_{1} \}, B, B \cup \{ 1, \$ \}, \delta, \$, \{q_{1}\}).
		\]
		
		Por como lo definimos tiene dos estados y uno solo es final.
		El alfabeto de entrada es el conjunto de generadores $B$ mientras que el de pila es $B \cup \{ 1 , \$ \}$ donde $\$$ es el símbolo inicial de la pila.
		Representamos al autómata de la siguiente manera donde $b_{i} \in B$ genérico y $b \in B \cup \{1\}$ entonces si $b_{i} \cdot b$ es el producto en $F$ de ambos generadores
		
		\begin{center}
			\begin{tikzpicture}[->,>=stealth',shorten >=1pt,auto,node distance=6.5cm,
				scale = 1,transform shape]
				
				\node[state,initial] (q_0) {$q_0$};
				\node[state,accepting] (q_1) [right of=q_0] {$q_1$};
				
				\path 
				(q_1) edge    [bend right,swap]          node {$b_{i}, b \mid b_{i} \cdot b $ \ } (q_0)
				(q_0) edge    [bend right, swap]          node {$\epsilon, 1 \mid \epsilon $ \ } (q_1)
				(q_0) edge    [loop above]          node {$b_{i}, b \mid  b_{i} \cdot b$ \ }  (q_0)
				(q_1) edge    [loop above]          node {$\epsilon, \$ \mid \epsilon   $ \ }  (q_1);;
				
			\end{tikzpicture}
		\end{center}
		
				
		Consideremos el lenguaje aceptado por estado final, esto es:
		\[
		L({\cal M}) = \{  w \in B^* \mid (q_{0},w,1)   \vdash^*  (q_{1}, \epsilon, y)  \}.
		\]
		donde en particular notemos que $w \in L(\cal M)$ si y solo sí $w \in N(\cal M)$. 
		Esto nos dice que si $(q_{0},w,1)   \vdash^*  (q_{1}, \epsilon, y)$ entonces $y = \epsilon$.
		En particular $\cal M$ es un autómata especial determinístico.
		
		Notemos que $L({\cal M}) = WP(F,B)$ porque si $w \in L(\cal M)$ entonces
		\[
			(q_{0},w,1) \overset{*}{\vdash} (q_{0},\epsilon,1) \vdash (q_{1},\epsilon, \epsilon)
		\]
		donde en en el estado $q_{0}$ apilamos a $w$ y hacemos el producto en el grupo libre en la pila.
		De esta manera $w \in L({\cal M}) \iff w \in WP(F,B)$.
		
		\item 	$\ZZ \times \ZZ$ no es un grupo independiente de contexto.
		Si tomamos los siguientes generadores como monoide $\Sigma = \{ a,b,c \}$ tal que tenemos un morfismo de monoides $\pi: \Sigma^* \to \ZZ \times \ZZ$ dado por $\pi(a)=(1,0), \pi(b)=(0,1), \pi(c)=(-1,-1)$.
		Bajo esta presentación 
		\[
		WP(\ZZ \times \ZZ, \Sigma) = \{ w \in \Sigma^*  : \ \exists n \in \NN, \ |w|_a = |w|_b = |w|_c = n \}.
		\]
		Este lenguaje no es independiente de contexto.
		Para eso usemos el lema del pumping \ref{pumping} para probarlo por contradicción.
		Si fuera \ic debería existir una constante $n \ge 0$ tal que hace valer las hipotesis del lema.
		Consideremos la palabra $w = a^n b^n c^n \in WP(G, \Sigma)$.
		Si tenemos una factorización 
		\[
		uvwxy = a^nb^nc^n
		\]
		si $|vwx| \le n$ implica que no todas las letras aparecen en $vwx$.
		Supongamos que la letra que no aparece es $c$.
		Por otro lado como $|vx| \le 0$ esto nos dice que al menos una letra aparece en la subpalabra $vx$.
		Si tomamos $i=0$ notemos que la palabra $uwy \in WP(G,\Sigma)$ pero esto es una contradicción porque la cantidad de $c$ en esta palabra es mayor que de $a$ o $b$.
	\end{enumerate}
\end{ej}





\section{Teorema de Muller--Schupp.}

Nuestro objetivo es probar que todo grupo \vl resulta ser \ic pero para esto debemos primero probar algunos resultados sobre grupos virtualmente libres.



\begin{lema}\label{lema_int_normal}
	Sea $G$ grupo, sea $H$ subgrupo de $G$ luego el subgrupo $K = \bigcap_{g \in G} gHg^{-1}$ es un subgrupo normal de $G$.
\end{lema}
\begin{proof}
	Ver \cite{}.
\end{proof}

Los siguientes resultados nos dicen que la clase de grupos virtualmente libres resulta ser cerrada por subgrupos de índice finito y que a su vez los subgrupos libres pueden ser tomados para que sean normales.

\begin{prop}\label{prop_vls}
	Para todo grupo $G$ \vl \ valen las siguientes propiedades.
	\begin{enumerate}
		\item Si $F$ es un subgrupo libre de índice finito de $G$ entonces podemos tomarnos otro subgrupo $F'$ de manera que sea normal, libre y de índice finito.
		\item Si $H$ es un subgrupo de $G$ de índice finito entonces $H$ también resulta ser \vl.
	\end{enumerate}
\end{prop}

\begin{proof}
	Vamos a probar \textbf{1}.
	Si $G$ es virtualmente libre y $F$ es un subgrupo libre tenemos que la cantidad de conjugados de $F$ es finita por el lema \ref{lema_normalizador_conjugados}.
	Por el lema \ref{lema_int_normal} tenemos que el siguiente subgrupo de $G$ es normal
	\[
	F' = \bigcap_{g \in G} gFg^{-1}
	\]
	donde la cantidad de grupos que estamos intersecando es finita.
	Veamos que este subgrupo $F'$ nos sirve. 
	
	
	Como $F$ tiene índice finito y todos sus conjugados al ser isomorfos a $F$ también tienen índice finito entonces $F'$ tiene índice finito por el lema \ref{lema_indice_interseccion}.
	Como $G$ es finitamente generado por ser virtualmente libre entonces usando \ref{lema_subg_fg} obtenemos que $F'$ es finitamente generado.
	Finalmente notemos que es libre por el resultado \ref{coro_niels_sch} que nos dice que todo subgrupo de un grupo libre es libre, en particular al ser $F'$ subgrupo de $F$ que es libre obtenemos que $F'$ es libre tal como queríamos ver.
	
	Probemos \textbf{2}. 
	Por el lema \ref{lema_subg_fg} obtenemos directamente que $H$ es un grupo finitamente generado.
	Si $F$ es un libre de índice finito en $G$ podemos tomar $H \cap F$ que es libre por ser subgrupo de un libre de acuerdo al resultado \ref{coro_niels_sch}.
	El índice resulta ser finito puesto que 
	\[
	[H:F\cap H] \le [G:F] < \infty.
	\]
\end{proof}


\begin{obs}\label{obs_presentacion_vl}
	Dado $G$ \vl \ vamos a construirnos una presentación en particular.
	\todo{Debería ser consistente con la notación para las acciones por izq o der.}
	Como es un grupo \vl \ tenemos $F$ subgrupo libre que podemos tomarlo normal por \ref{prop_vls} y $G/F$ grupo finito de manera que podemos escribir a $G$ como una extensión de estos dos grupos por medio de la siguiente forma
	\begin{center}
		\begin{tikzcd}
			1 \arrow[r] & F \arrow[r, "\iota"] & G \arrow[r, "\pi"] & G/F \arrow[r] & 1
		\end{tikzcd}
	\end{center}
	donde $\iota: F \to G$ es la inclusión como subgrupo y $\pi: G \to G/F$ es la proyección.

	
	Consideremos que $F$ es libre generado por $A = \{ a_i : 1 \le i \le n \}$ bajo la suposición que $1 \notin A$.
	Por otro lado el cociente $G/F$ resulta ser finito entonces $G/F = \{ q_i : 1 \le i \le m \}$.
	Elegimos un transversal a derecha $T = \{ t_1, \dots, t_m \}$ de manera que $\pi(t_i)= q_i$ y fijamos para que $t_1 = 1$.
	
	Dado que es un transversal tenemos que se deben cumplir las siguientes dos relaciones para todo $a_l \in A$ y $t_i,t_j,t_k \in T$. 
	\begin{enumerate}
		\item $t_ia_{l}t_i^{-1} = u_{il}$ donde $u_{il} \in F$ usando que $F$ es normal.
		\item Si tenemos que $q_iq_j = q_k$ entonces tenemos la siguiente igualdad 
		$(Ft_{i}) (Ft_{j}) = Ft_{k}$
		de donde se desprende que debe existir $z_{ij} \in F$ tal que 
		 $t_it_j = z_{ij}t_k$.
	\end{enumerate}
	Sean
	\[
		W = A \cup T, \quad R = \{t_ia_{l}t_i^{-1} = u_{il},  t_it_j = z_{ij}t_k \quad \forall (i,j,k,l) :   1 \le i,j,k \le m, 1 \le l  \le n \}
	\]
	Vamos a probar que $G \equiv \langle W \mid R \rangle$ 
	Denotemos a este grupo presentado como $H = \langle W \mid R \rangle$ y sea $F_{W}$ el grupo libre sobre sus generadores.
	
	Notemos que $W$ genera a $G$ porque
	dado $g \in G$ luego existe $a \in F$ y $t \in T$ de manera que $g = at$.
	Como $F$ es libre generado por $A \subseteq W$ y $T \subseteq W$ entonces $g$ está generado por $W$. 
	Esto nos dice que tenemos un epimorfismo de grupos $\varphi: F_{W} \to G$.
	
	Por como elegimos las relaciones $r \in R$ de $H$ tenemos que el grupo $G$ las cumple.
	De esta manera por la propiedad universal del cociente tenemos un epimorfismo de grupos $\ol \varphi$ tal que hace conmutar al siguiente diagrama	
	\begin{center}
		\begin{tikzcd}
			F_{W} \arrow[dd] \arrow[rr, "\varphi"]          &  & G \\
			&  &   \\
			H \arrow[rruu, "\overline \varphi"', dashed] &  &  
		\end{tikzcd}
	\end{center}
	
	Vimos que $\ol \varphi$ es un epimorfismo, veamos ahora que es un monomorfismo.
	Sea $w \in H$ luego queremos ver que $\ol \varphi(w) = 1 \iff w = 1$.
	Por el mismo procedimiento que hicimos en la demostración del lema \ref{lema_subg_fg} tenemos que podemos llevar a $w$ (vista como una palabra en $F_{W}$) a que sea de la siguiente pinta
	\[
		w = yt
	\]
	donde $y \in F$ reducida y $t \in T$.
	Por lo tanto $\ol \varphi (w) = \ol \varphi(yt) = yt$.
	Entonces $yt=1$ si y solamente si $y=1$ y $t=1$. 
	Esto nos dice que $\ol \varphi(w) = 1$ si y solamente si $w=1$. 
	Concluimos así que $\ol \varphi$ es un isomorfismo de grupos tal como queríamos ver y por lo tanto  $\langle W \mid R \rangle$ resultó ser una presentación de $G$.
\end{obs}
\medskip

Ya estamos en condiciones de probar uno de los resultados centrales de este trabajo.


\begin{teo}\cite{muller1983groups}
	Todo grupo virtualmente libre es independiente de contexto.
\end{teo}


\begin{proof}
	Sea $G$ grupo \vl  \ y consideremos una presentación $\langle W  \mid  R \rangle$ como la que construimos en \ref{obs_presentacion_vl} respetando la misma notación que introducimos en esta observación.
	Para probar que es \ic vamos a construir un autómata de pila $\cal M$ de manera que  $L ( {\cal M})= \text{WP}(G,W)$.
	Antes de definirlo vamos a definir algunos conjuntos finitos que nos van a servir para esta construcción.
	
	Por nuestra definición del conjunto de relaciones $R$ teníamos que para todo $t_{i}, t_{j}, t_{k} \in T, a_{j} \in A$ existían $u_{ij}, z_{ijk} \in F$ de manera que 
	$t_j a_i = u_{ij} t_j $ y que $ t_it_j = z_{ijk}t_k $. 
	Sea entonces el conjunto finito 
	\[
		U =  \{u_{ij} : 1 \le i,j \le n\}  \cup  \{z_{ijk} : 1 \le i,j,k \le n\} 
	\]
	donde tomamos la escritura de todas las palabras en $U$ para que sean reducidas.
	Sea $B = A \cup A^{-1}$ el conjunto simétrico de generadores de $A$.
	Consideremos 
	\[
		U/B^* = \{ u' \in B^* : \exists v \in B^*, \  u'v \in U  \}
	\]
	el conjunto de los prefijos de las palabras en $U$  tal que es un conjunto finito también.
	Sea entonces $Q = T \times (T \cup A) \times U/B^* $.
	
	Con estas definiciones ya podemos definir al autómata
	\[
	{\cal M }= (Q, W , B \cup \{ 1 \}, \delta, (1,1,1), 1 ,\{(1,1,1)\})
	\]
	El alfabeto de entrada es $W$ que es el conjunto de generadores del grupo.
	El alfabeto de la pila es $B \cup 1$ que es el conjunto simétrico de generadores del subgrupo libre $F$.
	Nuestro estado inicial que también es el final corresponde a $(1,1,1)$.
	
	Ahora podemos definir la función de transición. 
	Sea $w_i \in W$ algún generador del grupo luego tenemos que
	\[
	\delta(w_i,(t_j,1,1), s) =
		\begin{cases}
		 ((t_j,a_i,1), s) \quad \text{si} \ w_{i}=a_{i} \\
		 ((t_k,t_i,1), s) \quad \text{si} \ w_{i}=t_{i} \ \text{y} \ t_{j}t_{i} = t_{k}
	\end{cases}\]
	de manera que si estamos en $T \times \{ 1 \} \times \{ 1 \}$ podemos pasar a la segunda coordenada correspondiente a la letra de $W$ que hayamos leído.	
	Ahora en esta instancia lo que vamos a hacer es la reducción del producto de $u_{ij} \cdot v$ en el caso del estado $(t_j,a_i,1)$ o bien el de $z_{ijk} \cdot v$ en el caso del estado $(t_k,t_i,1)$.
	Esta reducción la hacemos una letra a la vez.
	Sea entonces $v \in U$ la palabra correspondiente al estado $(t_{j},w_{i},1)$.
	En este caso obtenemos:
	\begin{equation*}
		\delta(\epsilon,(t_j,w_i,u), s) = ((t_j,w_i,u a_i), v \cdot a_i ) \quad \ \text{si} \ ua_{i} \in \text{Pre}(v).
	\end{equation*}
	siempre y cuando $ua_i$ sea un prefijo de la palabra de $U$ correspondiente.
	La notación $v \cdot a_{i}$ representa el producto de $v \in F$ con $a_{i}$ un generador en $F$.
	Notemos que para cada configuración del tipo $(w,(t_{j}, w_{i}, u),s)$ existe una única transición posible dado que fijamos de antemano alguna escritura única en los generadores $A$ para cada palabra $u \in U$.
	Finalmente la función de transición la definimos para que podamos volver una vez que ya reducimos toda la palabra $v \in U$.
	Para eso tenemos 
	\begin{equation*}
		\delta(\epsilon,(t_j,w_i,v), s) = ((t_j,1,1), s) \quad \ \text{si} \ v \in U.		
	\end{equation*}
	en el caso que $v \in U$, es decir que ya hicimos toda la reducción.
	Finalmente definimos 
	\[
		\delta (\epsilon, (1,1,1), 1) = ((1,1,1), \epsilon)
	\]
	para poder terminar de vaciar la pila y así poder aceptar por pila vacía también.
	
	Con esto tenemos definida la función de transición y así al autómata $\cal M$. 
	Notemos que por como lo definimos resulta ser un autómata de pila determinístico.
		
	Una vez definido este autómata consideremos ahora el lenguaje aceptado por pila vacía y estado final a la vez:
	\[
	{\cal L }(M) = \{  w \in W^* \mid (w,(1,1,1),1)   \vdash^*  (\epsilon, (1,1,1), \epsilon)  \}
	\]
	Debemos ver que el autómata acepta justamente al lenguaje que queremos. 
	Esto es que $ {\cal L }(M) = \text{WP}(G,W) $ 
	
	Sea $w \in W^*$ entonces veamos que el autómata $\cal M$ hace las operaciones que describimos en la observación \ref{obs_presentacion_vl} para reescribir a $w = st$ donde $s \in F$ reducida y $t \in T$.
	Para esto veamos primero dos observaciones del funcionamiento de este autómata.
	
	\begin{enumerate}[i)]
		\item Primero notemos que si $w = a_{i}w'$ donde $a_{i} \in B$ entonces tenemos la siguiente derivación:
		\[
		(a_{i}w', ( 1,1,1), 1 ) \vdash (w', (1,a_{i},1),1) \vdash (w',(1,a_{i},a_{i}),a_{i}) \vdash (w', (1,1,1), a_{i})
		\]
		donde usamos que $u_{1i} = a_{i}$
		y similarmente en el caso que $w = t_{i}w'$ donde $t_{i} \in T$ tenemos la siguiente derivación:
		\[
		(t_{i}w', (1,1,1), 1) \vdash (w', (t_{i},t_{i},1),1) \vdash (w',(t_{1},t_{i},1),1) \vdash (w', (t_{i},1,1), 1)
		\]
		donde usamos que $z_{1i1} = 1$.
		
		\item Por otro lado observemos que dada una configuración del tipo $(w, ( t_{k}, w_{i}, u), s)$ con $w_{i} \neq 1$ entonces separamos en dos casos.
		Si $w_{i} = t_{j}$ sea $v = z_{ijk}$ y si $w_{i} = a_{l}$ sea $v = u_{il}$ entonces tenemos que si $u \in \text{Pre}(v)$ de manera que $uv' = v$ entonces
		\[
		(w, ( t_{k}, w_{i}, u), s) \overset{*}{\vdash} (w, ( t_{k}, 1, 1), s \cdot v').
		\]
	\end{enumerate}
	
	
	
	
	
	
	En particular de estas dos observaciones obtenemos que toda palabra $w \in W^*$ satisface que 
	\[
		(w, (1,1,1), 1) \overset{*}{\vdash} (\epsilon, (t, 1,1), s)
	\]
	donde $s \in F$ es el elemento del grupo libre que se consigue luego de hacer todas las operaciones que describimos en la observación \ref{obs_presentacion_vl}. 
	En particular del autómata obtenemos que si miramos la primera coordenada de nuestro estado tenemos que $w = st$.
	
	De esta manera $w \in L({\cal M}) \iff w = 1 $ y así obtenemos que $WP(G,W) = L(\cal M)$ tal como queríamos ver.
	
	
%	Dada una palabra $w \in W^*$ por como es esta presentación sabemos que se puede escribir como $w = vt$ donde $v \in F$ reducida y $t \in  T$. 
%	De esta manera $w \in \text{WP}(G,W)$ si y solo si $v=1, t=1$.
%	Notemos que el autómata en toda transición no hace más que reescribir la cadena de izquierda a derecha tal como hicimos en la observación \ref{obs_presentacion_vl}.
%	Esto es que cuando termina de consumir la cadena de entrada llegamos a la configuración $(1, t_i, v)$ es decir que $w = vt_i$.
%	Por lo tanto como aceptamos por estado final y pila vacía esto nos dice que $w \in \text{WP}(G,W)$ si y solo sí $w \in {\cal L}(M)$.
	
	Con esto probamos que los grupos virtualmente libres son \ic usando la equivalencia \ref{teo_ic_apnd}.
	
\end{proof}

\begin{center}
	\missingfigure[figwidth=6cm]{Ejemplo de un grupo que sepa hacer?}
\end{center}
\todo[]{Agregar citas y uno o dos párrafos precisos sobre la complejidad computacional del problema.}




\end{document}

