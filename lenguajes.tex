% !TeX TS-program = 
\documentclass[tesis.tex]{subfiles}

%\newcommand{\ic}{independiente de contexto }
%\newcommand{\APND}{automáta de pila no determinístico }
%\newcommand{\APD}{automáta de pila determinístico }
%\newcommand{\gramatica}{{\cal G} = (V, \Sigma, P, S)}
%\newcommand{\deriva}{\overset{*}{\Rightarrow_{\cal G}}}
%\newcommand{\lengderivado}{L({\cal G})}
%\newcommand{\fg}{grupo finitamente generado }

\begin{document}
\chapter{Grupos independientes de contexto.}
Este capítulo sigue las ideas de los trabajos \cite{muller1983groups} y \cite{muller1985theory} completando muchos detalles que son omitidos por estos.



\section{Propiedades de los lenguajes independientes de contexto.}

En esta sección vamos a considerar a los lenguajes \ic como una clase. 
Probaremos algunas propiedades que tienen los lenguajes independientes de contexto vistos como clase.
La referencia de estos resultados estándares es \cite{hopcraft-ullman}.

\begin{prop}\label{intersecciones-reg-ic}
	Los lenguajes \ic son cerrados por intersecciones con lenguajes regulares.
\end{prop}

\begin{proof}
	Sea $L$ un lenguaje \ic tal que $L \subset \Sigma^*$ y sea $R$ un lenguaje regular tal que $R \subset \Sigma^*$. 
	Queremos ver que el lenguaje $L \cap R$ es aceptado por un \APND.
	
	Si $L$ es aceptado por un \APND ${\cal M } = (Q, \Sigma, Z, \delta, q_0, F, Z_0)$ y R es aceptado por un autómata no determinístico ${\cal M' } = (Q', \Sigma, \delta', q_0', F',)$.
	
	Consideremos el siguiente \APND
	\[
		{\cal N } = (Q \times Q', \Sigma, Z, \delta \times \delta', (q_0,q_0'), F \times F', Z_0).
	\]	
	
	Queremos ver que $L({\cal N}) = L \cap R $.	
	Probemos primero la siguiente afirmación.
	\[
	((q_0,q_0'), w, Z_0) \vdash^*_{\cal N}  ((q_i,q_j), \lambda, z) \ \iff (q_0, w, Z_0) \vdash^*_{\cal M}  (q_i, \lambda, z) \ \text{y} \ (q_0', w) \vdash^*_{\cal M'} (q_j, \lambda)  	
	\]
	Para ver esto veremos las dos implicaciones a la vez haciendo inducción en la longitud de la palabra $w$.
	
	En el caso base $|w| = 1$ de manera que $w = a \in \Sigma$.
	Este caso tenemos la igualdad porque justamente la función de transición del \APND $\cal N$ es $\delta \times \delta'$.
	
	Para el paso inductivo consideremos que $|w|=n$ de manera que $w=ua$ con $|u|=n-1$ y $a \in \Sigma$.
	Luego tenemos que $((q_0,q_0'), u, Z_0) \vdash^*_{\cal N}  ((q_k,q_l), \lambda, z') \ \iff (q_0, u, Z_0) \vdash^*_{\cal M}  (q_k, \lambda, z') \ \text{y} \ (q_0', u) \vdash^*_{\cal M'} (q_l, \lambda)$.
	Nuevamente usamos nuestra definición de la función de transición $\delta \times \delta'$ para concluir que
	\[ 
	((q_i,q_j),\lambda, z) \in \delta \times \delta'((q_k,q_l), a, z') \iff  (q_i, a, z') \in \delta(q_k, a, z') \ \text{y} \ (q_j, a) \in \delta(q_l, a).
	\]	
	Con esto terminamos de probar la afirmación.
	
	Finalmente para ver que $L({\cal N})  =  L \cap R$ usamos que por nuestra afirmación si $w \in L \cap R$ entonces es aceptado por lo dos autómatas $\cal M$ y por $\cal M'$, esto es que $(q_0, w, Z_0) \vdash^* (q_F, \lambda, z)$ y que $(q_0', w) \vdash^* (q_F', \lambda)$ y equivalentemente $((q_0,q_0'),w,Z_0) \vdash^* ((q_F, q_F'),\lambda, z)$. 
	Dado que los estados finales del autómata $\cal N$ resultan ser $F \times F'$ obtenemos que $w \in L({\cal N}) \iff w \in L \cap R$.
	Concluimos así que el lenguaje $L \cap R$ resulta ser \ic tal como queríamos ver.		
\end{proof}

Otras propiedades interesantes tienen que ver con su relación con morfismos de monoides. 

\begin{prop}\label{morfismos-monoides-ic}
	Los lenguajes \ic son cerrados por:
	\begin{enumerate}
		\item Imágenes de morfismos de monoides.
		\item Preimágenes de morfismos de monoides.
	\end{enumerate}
\end{prop}

\begin{proof}
	\begin{enumerate}
		\item Consideramos $L$ lenguaje \ic sobre $\Sigma$.
		Esto nos dice que existe una gramática $\gramatica $ tal que $L(\cG) = L$.
		Si tenemos un morfismo de monoides $h: \Sigma^* \to \Delta^*$ queremos ver que existe una gramática $\cG'$ tal que $L(\cG') = h(L)$.
		Para eso consideramos $\cG  = (V, \Sigma, P', S)$ donde la única diferencia es que reemplazamos en cada producción $P \in \cG$ a cada letra $a \in \Sigma$ por $h(a) \in \Delta^*$.
		La gramática sigue siendo \ic, nos alcanza con ver que genera al lenguaje que queremos.
		
		Miramos las dos contenciones a la vez.		
		Si tomamos una palabra $w \in L$ luego tenemos que $h(w) \in h(L)$ por lo tanto tiene que haber una sucesión de producciones $S  \implies^* w  $ en la gramática $\cG$.
		Si tomamos esta derivación y cambiamos cada producción $A \to \alpha \in P$ por la correspondiente $A \to \beta \in P'$, donde $\beta = h(\alpha)$, en la gramática $\cG'$ ahora obtenemos la palabra $h(w)$.
		Volviendo para atrás obtenemos que $L(\cG') = h(L)$ tal como queríamos ver.
		
		\item Sea $L$ un lenguaje \ic sobre el alfabeto $\Sigma$ tal que es aceptado por un \APND ${\cal M } = (Q, \Sigma, Z, \delta, q_0, F, Z_0)$.
		Consideremos un morfismo de monoides $h: \Delta^* \to \Sigma^*$. 
		Queremos ver que el lenguaje $h^{-1}(L) \subset \Delta^*$ resulta ser \ic. 
		Primero introduzcamos la constante $ n = \max_{a \in \Delta} |h(a)|$.
		
		Ahora consideremos el siguiente \APND 
		\[
			{\cal M' } = (Q', \Delta, Z, \delta, q_0', F', Z_0)
		\]
		donde $Q' = Q \times \Sigma^{\le n}$, donde $\Sigma^{\le n} = \{ w \in \Sigma^* : |w| \le n  \}$.
		Los estados finales $F' = F \times \{ \lambda \}$.
		El estado inicial es $q_0' = (q_0, \lambda)$.
		Finalmente nuestra función de transición resulta ser 
		\[
		\delta'((q,v), a ,z) = 
		\begin{cases}
		\{(q,h(a), z )\}  \ &\text{si} \ v=\lambda, z \in Z \\
		\{(p,u),z' \} \ &\text{si} \ (p,z') \in \delta(q,y,z), v=yu \\  
		\emptyset \ &\text{caso contrario}.
		\end{cases}
		\]
		
		Probemos primero la siguiente afirmación.
		Para cada $w \in \Delta^*$ tenemos la siguiente equivalencia,
		\[
			(q_0',w,z) \vdash^*_{\cal M'} ((q,\lambda), \sigma) \iff (q_0,h(w), z) \vdash^*_{\cal M} (q, \sigma).
		\]
		
		Para probar esta equivalencia lo hacemos por inducción en la longitud de $|w|$.
		Para el caso base tenemos que $w = a \in \Delta$.
		En este caso tenemos que $(q_0', a, z) \vdash ((q,h(a)), z)$.
		Una vez en este estado por como definimos la función de transición tenemos que $ (q_0,h(a), z) \vdash (q, \sigma)  \iff ((q_0,h(a)), \lambda ,z) \vdash ((q,\lambda), \sigma)$.
		
		El caso general tenemos una palabra $w$ tal que $|w|=n$.
		Sabemos que la afirmación vale para cualquier palabra de longitud menor a $n$.
		En particular si $w=ua$ con $|u|=n-1$ y $a \in \Delta$, tenemos que por la hipótesis inductiva que 
		\[
		(q_0',u,z) \vdash^*_{\cal M'} ((q,\lambda), \sigma) \iff (q_0,h(u), z) \vdash^*_{\cal M} (q, \sigma)
		\]
		entonces de nuevo por el mismo razonamiento que hicimos para el caso base tenemos probada la afirmación.
		
		Para concluir la demostración notemos que por nuestra afirmación una palabra $w \in L(\cal M')$ si y solo si $h(w) \in L$. 
		Esto nos dice que el lenguaje aceptado por estado final de $\cal M'$ resulta ser $L' = \{ w \in \Delta^* : h(w) \in L \} = h^{-1}(L)$.
	\end{enumerate}
\end{proof}

Finalmente la herramienta principal que tenemos para ver que cierto lenguaje $L$ no es \ic es usar el siguiente lema.

\begin{lema}[Pumping] \label{pumping}
	Sea $L$ un lenguaje independiente de contexto entonces existe una constante $n \ge 0$ tal que para todas las palabras $w \in L$ de longitud al menos $n$ existe una factorización $w = uvxwy$ con $|vwx| \le n$ y $|vx| > 0$ tal que para todo $i \in \NN$ vale que $uv^iwx^iy \in L$.
\end{lema}

\begin{proof}
	Resultado estándar. Ver \cite{hopcraft-ullman}.
\end{proof}

\section{Autómatas de pila determinísticos.} 
Si en la definición anterior del autómata pedimos que la función de transición de una configuración dada tenga a lo sumo un valor entonces nuestro autómata lo vamos a llamar determinístico. 
Formalmente esto es que 
\[
|\delta(q,a, z)| \le 1 \ \ \ \forall z \in Z, \ p \in Q, \ a \in \Sigma \cup \{ \lambda \}.
\]
En cierta manera estamos diciendo que de un instante dado solo tenemos a lo sumo una única posibilidad de movernos a otro estado. 
A los lenguajes aceptados por un \APD los pensamos que son aceptados por estado final.


Por lo tanto para cada palabra tenemos un único camino en el autómata para saber si es aceptada o no a diferencia de un \APND  que podría tener varios caminos posibles para cada palabra.
\begin{obs}
	Todo \APD en particular es no determinístico y por lo tanto la clase de lenguajes aceptados por los primeros están contenidos en la clase de los segundos.
\end{obs}

Veamos que esta contención es estricta. Para eso volvamos a considerar el ejemplo del lenguaje de los palíndromos.
\begin{ej}
	El lenguaje 
	\[
	L = \{ w \in \{ a,b \}^*  \ : \ w = w^r \}
	\]
	 no es aceptado por un \APD pero sí por uno no determinístico. 
	Supongamos que $M$ es un \APD que lo acepta. 
	
	Notemos que para cualquier palabra $w \in \{ a,b \}^*$ debe ser que al consumirla la pila no puede quedar vacía dado que $ww^r \in L$ y en tal caso no aceptaría a esta palabra. 
	Esto es que la pila nunca está vacía sea cual sea la configuración que lleguemos. 
	Para cada palabra arbitraria $w$ existe otra $x_w$ tal que al procesar $wx_w$ lo que nos queda en la pila es de tamaño mínimo con respecto a todas las palabras $wx$. 
	Sea entonces lo que tiene en la pila la palabra $\alpha_w$ que sabemos es de longitud mínima. Si consideramos palabras del estilo $wx_wz$ sabemos que la longitud de lo que quede en la pila no puede disminuir. 
	Ahora consideremos dos palabras del estilo 
	
	\[
	r=wx_w, \ s=uy_u
	\]
	 tales que sus pilas son de longitud mínima al terminar de recorrer las palabras y que resultan tener el mismo tope de pila y terminar en el mismo estado. 
	Podemos asegurar la existencia de estas palabras debido a que tenemos finitos estados y combinaciones de tope de pilas dado que el autómata de pila es finito pero tenemos infinitas palabras que cumplen esta propiedad. 
	Ahora basta con elegir $z$ de modo que $tz$ sea palíndromo pero que $sz$ no lo sea. 
	
	Veamos que podemos elegir a $z$ para que una de las concatenaciones $tz,sz$ sea palíndromo y la otra no. Partamos en distintos casos. 
	
	Si $|t|=|s|$ basta con tomar $z=s^r$. 
	Si $|t|\neq |s|$ y supongamos que $s$ tiene longitud menor y no es prefijo de $t$ entonces de nuevo podemos tomar el palíndromo $ss^r$ tal que $ts^r$ no es un palíndromo.
	 
	Finalmente queda el caso que una es un prefijo de la otra, supongamos $t=su$. 
	Si elegimos $x=a,b$ tal que $ux$ no sea un palíndromo luego la palabra $ss^r$ es un palíndromo pero $suxs^r$ no lo es.
	
	Esto muestra que si bien $sz \notin L$ y $tz \in L$ el \APD no va a poder diferenciarlas por lo tanto no es posible que este lenguaje sea aceptado por un \APD tal como queríamos ver.
	
\end{ej}

\section{Autómatas de pila determinísticos especiales.} 
Consideremos ahora un \APD tal que acepta tanto por estado final como por pila vacía. A estos los llamaremos \textit{autómatas de pila determinístico especiales}.  
Estos autómatas son los que nos surgen de la construcción del autómata del problema de la palabra  para grupos virtualmente libres. 

\begin{ej}
	Sea el lenguaje $L = \{ a^m b^n  : m \ge n \ge 1 \}$ este no es un lenguaje independiente de contexto determinístico especial pero sí es determinístico. 
	
	Construyamos un \APD que acepte a $L$. Sea $$M = (\{q_0,q_1,q_2\}, \{q_0\}, \{a,b\}, \{a,b,Z_0\}, Z_0, q_2) $$ el siguiente \APD que representamos así:
	
	\begin{center}
		\begin{tikzpicture}[->,>=stealth',shorten >=1pt,auto,node distance=3.5cm,
		scale = 1,transform shape]
		
		\node[state,initial] (q_0) {$q_0$};
		\node[state] (q_1) [right of=q_0] {$q_1$};
		\node[state,accepting] (q_2) [right of=q_1] {$q_2$};
		
		\path (q_0) edge    [bend right]          node {$b, a | \lambda$} (q_1)
		(q_0) edge    [loop above]          node {$a, Z | aZ$} (q_0)
		(q_1) edge      [bend right]      node {$\lambda, a | Z$}   (q_2)
		(q_1) edge    [loop above]           node {$b, a | \lambda$} (q_1);
		
		\end{tikzpicture}
	\end{center}
	
	
	
	donde $Z$ es cualquier letra del alfabeto de la pila. El autómata en el estado inicial $q_0$ apila a todas las $a$ y cambia al estado $q_1$ cuando lee por primera vez una $b$ y en ese caso desapila la a que está en el tope de la pila. En el estado $q_1$ sigue desapilando cada vez que ve una $b$. Finalmente va al estado $q_2$ cuando en la pila sigue quedando $a$ y ya leímos toda la palabra y en tal caso la acepta porque significa que vimos como máximo tantas $b$ como $a$ y este es el lenguaje que buscábamos generar.
	
	El lenguaje no es aceptado por un \APD por pila vacía dado que tiene la propiedad de los prefijos. 
	Es decir que existen palabras que están en el lenguaje tales que alguno de sus prefijos también están. 
	Por ejemplo consideremos
	 \[
	 	a^m b^i, \ a^m b^j \ \
	 	\text{para}  \
	 	m \ge 2,  \ i < j \le m
	 \] 
	Esto es porque si $M = (Q, \Sigma, Z, Z_0, \delta, q_0 , F)$ fuera un \APD que acepta por pila vacía a este lenguaje tendríamos que 
	\[
		(q_0,a^{m}b^{i},Z_0) \vdash^* (q,\lambda, \lambda)
	\]
	con $q$ un estado final pero como tiene la pila vacía no podemos continuar aceptando a la palabra $a^mb^j$ ya que por la definición que empleamos el autómata necesita leer algún elemento de la pila. 
	
	De esta manera vemos que este lenguaje no puede ser aceptado por pila vacía y estado final por un \APD concluyendo que los lenguajes determinísticos especiales forman un subfamilia propia de los independientes de contexto determinísticos.
	
%[Esto es con la otra definición de automáta de pila.]
%			Por el principio del palomar podemos ver que este lenguaje no es aceptado por \APD por pila vacía y estado final a la vez. Si así lo fuera supongamos que existe un automáta que lo acepta $M = (Q, \Sigma, Z, Z_0, \delta, q_0 , F)$. Debido a que tiene finitos estados podemos elegirnos $m$ suficientemente grande tal que existan palabras $a^mb^j, a^mb^i$ con $1 \le i < j < m$ y elegidas de manera que terminen en el mismo estado final $p$. Notemos que $(q_0,a^mb^ib^{m-j+1},Z_0) \vdash^* (q,\lambda, Z)$ donde $q$ es un estado final dado que $m-j+1+i \le m$. Por otro lado debe ser que $(q_0, a^mb^jb^{m-j+1}) \vdash^* (q,\lambda, Z) $ pero notemos que $a^mb^{m-1} \notin L$. Por lo tanto llegamos a una contradicción y de esta manera vemos que $L$ no es aceptado por un \APD especial tal como queríamos ver.
\end{ej}


\section{El problema de la palabra.}

Todo grupo $G$ lo podemos ver como un monoide. 
En particular si tenemos un grupo $G$ tal que es finitamente generado como grupo por algún conjunto finito $X$ entonces es finitamente generado como monoide por el conjunto simétrico de generadores $Y = X \cup X^{-1}$.
Sea entonces $\Sigma$ conjunto finito de generadores como monoide de $G$
o lo que resulta equivalente, tenemos un epimorfismo de monoides desde el monoide libre $\pi: \Sigma^* \twoheadrightarrow  G$. 

El problema de la palabra es uno de los problemas de teoría de grupos más centrales al área. Explícitamente el problema consiste en encontrar un algoritmo que dada una palabra $\omega \in \Sigma^*$ decida si esta palabra es la identidad del grupo o no. 

\begin{deff}
El \emph{problema de la palabra de $G$ para los generadores $\Sigma$} es el siguiente lenguaje,	
\[ \text{WP} (G, \Sigma) = \{ \omega \in \Sigma^* \ | \ \omega \underset{G}= 1 \}\]
\end{deff}

Dado que este lenguaje depende del conjunto de generadores elegido nos gustaría ver qué condiciones tienen que cumplir los lenguajes o los grupos para que no exista esta dependencia en los generadores.


En el contexto del problema de la palabra nos van a interesar las familias de lenguajes que cumplen las siguientes propiedades.
\medskip
\begin{deff}
	Una familia de lenguajes $\CC$ es un \emph{cono} si para todo $L \in \CC$ resulta que:
	\begin{itemize}
		\item[\textbf{C1.}] Es cerrado por imágenes de morfismos de monoides. Sea $L \subset \Sigma^*$ luego si existe $\phi:\Sigma^* \to \Delta^*$ morfismo de monoides debe ser que $\phi(L) \in \CC$.
		\item[\textbf{C2.}] Es cerrado por preimagenes de morfismos de monoides. Sea $L \subset \Sigma^*$ luego si existe $\phi:\Sigma^* \to \Delta^*$ morfismo de monoides  debe ser que $\phi^{-1}(L) \in \CC$. 
		\item[\textbf{C3.}] Es cerrado por intersecciones con lenguajes regulares. Si $R$ es un lenguaje regular sobre $\Sigma^*$ entonces $L \cap R \in \CC$ también resulta serlo.
	\end{itemize}
\end{deff} 


Los conos de lenguajes cumplen la siguiente propiedad de gran importancia para el estudio del problema de la palabra.
\medskip
\begin{prop}\label{prop-cono-wp}
	Sea $WP(G, \Sigma)$ el lenguaje del problema de la palabra de cierto grupo $G$ para algunos generadores $\Sigma$ y $\CC$ cono de lenguajes. 
	Si $WP(G, \Sigma) \in \CC$ luego valen las siguientes afirmaciones:
	\begin{itemize}
		\item[\textbf{W1.}] $WP(G, \Delta) \in \CC$ para cualquier conjunto de generadores $\Delta$.
		\item[\textbf{W2.}] $WP(H) \in \CC$ para todo subgrupo $H$ \fg de $G$.
	\end{itemize} 
\end{prop}
\begin{proof}
		Para ver \textbf{W1} formamos el siguiente diagrama conmutativo,
		\begin{center}
			\begin{tikzcd}
				\Delta^* \arrow[r, "\delta"] \arrow[d,"f",swap] & G \\
				\Sigma^* \arrow[ru, "\pi",swap]    &  
			\end{tikzcd}
		\end{center}
		donde $f$ es algún morfismo de monoides y donde usamos la propiedad universal de los monoides libres.
		Notemos que $WP(G, \Delta) = \delta^{-1}(1)$ y como el diagrama conmuta tenemos que 
		\[
		f^{-1}(\delta^{-1}(1)) = f^{-1}(WP(G,\Sigma)) = WP(G, \Delta).
		\]
		Dado que esto es un cono obtenemos lo que queríamos ver puesto que es cerrado por preimágenes de morfismos de monoides.
		
		
		
		Veamos ahora que vale \textbf{W2}. 
		Sea $\Sigma'$ conjunto de generadores de $H$.
		Siempre podemos extenderlo a $\Sigma$ tal que $\Sigma$ genere a $G$. 
		De esta manera 
		\[
		WP(H, \Sigma') = WP(H, \Sigma) \cap \Sigma'^*
		\]
		y de vuelta como es un cono la intersección con lenguajes regulares nos da un lenguaje en el cono. 

\end{proof}


Esto nos dice que es interesante estudiar el problema de la palabra justamente sobre conos.

\section{Grupos independientes de contexto.}


\begin{ej}\label{ic-cono}
	Los lenguajes independientes de contexto forman un cono.
	Esto se puede ver a partir de las proposiciones  \ref{intersecciones-reg-ic} y \ref{morfismos-monoides-ic}.
\end{ej}

Esto nos dice que la siguiente definición no depende de los generadores que tomemos.

\begin{deff}
	Si $G$ es un grupo \fg tal que para ciertos generadores $\Sigma$ resulta que $WP(G, \Sigma)$ es independiente de contexto entonces diremos que $G$ es un \emph{grupo \ic }.
\end{deff}


\begin{ej} Consideremos los siguientes ejemplos.	
	\begin{enumerate}[E1.]
		\item 
		Dado $F$ grupo libre de rango finito supongamos generado por el conjunto finito $X$. 
		Sea  $Y = X \cup X^{-1}$ el conjunto de generadores simétrico de $X$. 
		Probemos que $\text{WP}(F,Y)$ es un lenguaje independiente de contexto.
		
		Consideremos el siguiente autómata de pila,
		\[
		M = (\{ 1 \}, Y, Y, \delta, 1, \{1\}).
		\]
		
		Por como lo definimos tiene un solo estado y el alfabeto tanto de entrada como el de la pila que usa es el conjunto de generadores $Y$.
		La función de transición $\delta$ está definida de la siguiente manera,
		\[
		\delta(1, y_i, u)=\left\{
		\begin{array}{ll}
		(1 , u )  &\ \text{si} \ a = 1  \\
		(1, y_i \cdot u') &\ \text{si} \  u = y_jw'  \\
		\end{array}
		\right.
		\]
		donde al apilar hacemos la multiplicación $y_i \cdot u$ en el grupo libre $F$.
		
		Consideremos el lenguaje aceptado por pila vacía, esto es
		
		\[
		{\cal L }(M) = \{  y \in Y^* \mid (y,1,1)   \vdash^*  (1, 1, 1)  \}.
		\]
		
		Notemos que ${\cal L }(M) = \text{WP}(F,Y)$ porque justamente en la pila apilamos lo que vamos leyendo de izquierda a derecha y desapilamos cuando leemos el inverso del generador que está en el tope de la pila.
		Esto es que desapilamos cuando un generador aparece después de su inverso en la palabra.

		
		\item 	$\ZZ \times \ZZ$ no es un grupo independiente de contexto.
		Si tomamos los siguientes generadores como monoide $\Sigma = \{ a,b,c \}$ tal que tenemos un morfismo de monoides $\pi: \Sigma^* \to \ZZ \times \ZZ$ dado por $\pi(a)=(1,0), \pi(b)=(0,1), \pi(c)=(-1,-1)$.
		Bajo esta presentación 
		\[
		WP(\ZZ \times \ZZ, \Sigma) = \{ w \in \Sigma^*  : \ \exists n \in \NN, \ |w|_a = |w|_b = |w|_c = n \}.
		\]
		Este lenguaje no es independiente de contexto.
		Para eso usemos el lema del pumping \ref{pumping} para probarlo por contradicción.
		Si fuera \ic debería existir una constante $n \ge 0$ tal que hace valer las hipotesis del lema.
		Consideremos la palabra $w = a^n b^n c^n \in WP(G, \Sigma)$.
		Si tenemos una factorización 
		\[
		uvwxy = a^nb^nc^n
		\]
		si $|vwx| \le n$ implica que no todas las letras aparecen en $vwx$.
		Supongamos que la letra que no aparece es $c$.
		Por otro lado como $|vx| \le 0$ esto nos dice que al menos una letra aparece en la subpalabra $vx$.
		Si tomamos $i=0$ notemos que la palabra $uwy \in WP(G,\Sigma)$ pero esto es una contradicción porque la cantidad de $c$ en esta palabra es mayor que de $a$ o $b$.
	\end{enumerate}
\end{ej}

\section{Teorema de Muller--Schupp.}


Una pregunta natural es intentar entender la relación entre la clasificación del lenguaje del problema de la palabra de un grupo dado y las distintas familias de grupos que le corresponden. 
La siguiente demostración generaliza la construcción del ejemplo del problema de la palabra de un grupo libre.


\begin{teo}\cite{muller1983groups}
	Todo grupo virtualmente libre es independiente de contexto.
\end{teo}


\begin{proof}
	Sea $G$ grupo \vl  \ y consideremos una presentación $\langle W  \mid  R \rangle$ como la que construimos en la observación \ref{obs_presentacion_vl}.
	Veamos de construir un autómata de pila de manera que acepte al lenguaje $\text{WP}(G,W)$.
	Antes de definirlo consideremos algunos conjuntos finitos que nos van a servir para definir al autómata.
	
	Si tenemos que $ t_j y_i = u_{ij} t_j $ y que $ t_it_j = z_{ijk}t_k $ luego consideremos la siguiente definición.
	El conjunto finito $U = \{ u_{ij}, \ z_{ijk} : 1 \le i,j,k \le n \}$.
	Consideremos $\text{Pre}(U) = \{ u' \in Y^* : u'v \in U  \}$ el conjunto de los prefijos de las palabras en $U$ que sabemos es finito también.	
	Sea entonces el conjunto finito $Q = T \times T \cup Y \times \text{Pre}(U) $.
	
	Con estas definiciones ahora nuestro autómata lo definimos así: 
	\[
	{\cal M }= (Q, W , Y, \delta, (1,1,1), \{(1,1,1)\})
	\]
	El conjunto $Q$ van a ser nuestros estados.
	El alfabeto de entrada es $W$ que es el conjunto de generadores del grupo.
	El alfabeto de la pila es $Y$ que es el conjunto de generadores del subgrupo libre $F$.
	Nuestro estado inicial que también es el final corresponde a $(1,1,1)$.
	
	Ahora podemos definir la función de transición. 
	Sea $w_i \in W$ algún generador del grupo luego tenemos que
	\begin{align*}
		\delta(y_iw',(t_j,1,1), v) &= (w', (t_j,y_i,1), v) \\
		\delta(t_iw',(t_j,1,1), v) &= (w', (t_k,t_i,1), v) \\
	\end{align*}
	de manera que si estamos en $T \times \{ 1 \} \times \{ 1 \}$ podemos pasar a la segunda coordenada correspondiente a la letra de $W$ que hayamos leído.	
	Ahora en esta instancia lo que vamos a hacer es la reducción del producto de $u_{ij} \cdot v$ o bien el de $z_{ijk} \cdot v$ respectivamente una letra a la vez.
	Para esto tenemos
	\begin{equation*}
		\delta(w',(t_j,w_i,u), v) = (w', (t_j,w_i,y_iu), y_i \cdot v) .
	\end{equation*}
	siempre y cuando $y_iu$ sea un prefijo de la palabra de $U$ correspondiente.
	Notemos que sigue siendo determinístico porque fijamos de antemano alguna escritura única en los generadores $Y$ para cada palabra $u \in U$.
	En la pila hacemos la reducción en el grupo libre de multiplicar por una letra.
	Finalmente la función de transición la definimos para que podamos volver una vez que ya reducimos toda la palabra $u \in U$.
	Para eso tenemos 
	\begin{equation*}
		\delta(w',(t_j,w_i,u), v) = (w', (t_j,1,1), v).		
	\end{equation*}
	en el caso que $u \in U$, es decir que ya hicimos toda la reducción.
	
	Una vez definido este autómata consideremos ahora el lenguaje aceptado por pila vacía y estado final a la vez,
	
	\[
	{\cal L }(M) = \{  w \in W^* \mid (w,1,1)   \vdash^*  (1, 1, 1)  \}
	\]
	Debemos ver que el autómata acepta justamente al lenguaje que queremos. 
	Esto es que $ {\cal L }(M) = \text{WP}(G,W) $ 
	
	
	Dada una palabra $w \in W^*$ por como es esta presentación sabemos que se puede escribir como $w = vt$ donde $v \in F$ reducida y $t \in  T$. 
	De esta manera $w \in \text{WP}(G,W)$ si y solo si $v=1, t=1$.
	Notemos que el autómata en toda transición no hace más que reescribir la cadena de izquierda a derecha tal como hicimos en la observación \ref{obs_presentacion_vl}.
	Esto es que cuando termina de consumir la cadena de entrada llegamos a la configuración $(1, t_i, v)$ es decir que $w = vt_i$.
	Por lo tanto como aceptamos por estado final y pila vacía esto nos dice que $w \in \text{WP}(G,W)$ si y solo sí $w \in {\cal L}(M)$.
	
	Con esto probamos que los grupos virtualmente libres son \ic usando la equivalencia \ref{teo_ic_apnd}.
	
\end{proof}
\todo[inline]{Agregar citas y precisión a este párrafo.}

Un comentario interesante es que el automáta que construímos para aceptar el problema de la palabra del grupo \vl resulta ser un automáta de pila determinístico especial.
En particular esto nos dice que existen algoritmos más óptimos para el problema de la palabra.




\end{document}

