\documentclass[tesis.tex]{subfiles}
\begin{document}
\section{Teoría de lenguajes.}	
En esta sección vamos a introducir los elementos básicos de la teoría de lenguajes formales que utilizaremos en este trabajo. 
\begin{deff}
Un conjunto finito $\Sigma$ lo vamos a llamar un \blue{alfabeto}. Una \blue{palabra} $w$ en este alfabeto es un elemento de $\Sigma^k$ para algún $k \le 0$ y su \blue{longitud} es este número k.
\end{deff}
La analogía es que un lenguaje formal va a ser un conjunto de palabras sobre este alfabeto. Más formalmente para poder hablar de palabras sobre un alfabeto necesitamos definir el monoide libre sobre el alfabeto.
\begin{deff}
	El \blue{monoide libre} sobre un alfabeto $\Sigma$ es el siguiente conjunto
	\[
	\Sigma^{*} = \bigcup_{k=0}^{\infty} \Sigma^k
	\]
	tal que es un monoide con la operación $\cdot$ que es la concatenación de palabras. Es decir dadas $w_1 \in \Sigma^{k}, w_2 \in \Sigma^{l}$ luego $w_1 \cdot w_2 \in \Sigma^{k+l} \subset \Sigma^*$. El elemento neutro es la palabra vacía que es la única palabra de longitud $0$. 
\end{deff}

\begin{obs}
	Este monoide es libre con la siguiente propiedad. Si tenemos una función del alfabeto $f: \Sigma \to M$ donde $M$ es algún monoide entonces existe un único morfismo de monoides $\overline f: \Sigma^{*} \to M$ que hace conmutar al siguiente diagrama.
\end{obs}

\begin{obs}
	De esta manera podemos pensar a una {presentación} de un monoide $M$  como un epimorfismo de monoides $\pi: \Sigma^{*} \to M$. De esta manera nuestro $M$ tiene como generadores a $\Sigma$ y sus relaciones están dadas por el núcleo del morfismo $\pi$.
\end{obs}

\begin{deff}
	Un \blue{lenguaje} $L$ es un subconjunto de $\Sigma^{*}$. 
\end{deff}

\subsection{Lenguajes independientes de contexto.}
\subsubsection{Gramáticas.}
En esta sección vamos a ver una manera de clasificar a los lenguajes. Para eso nos vamos a enfocar primero en cómo generarlos. Una herramienta para poder generarlos son las gramáticas.

\begin{deff}
	Una gramática es una tupla ${\cal G} = (V, \Sigma, P, S)$ que la interpretamos de la siguiente manera,
\begin{itemize}
		\item $V$ es un conjunto finito de variables;
		\item $S \in V$ es el símbolo inicial;
		\item $\Sigma$ es un conjunto finito de símbolos terminales ;
		\item $P \subseteq (V \cup \Sigma)^* \times (V \cup \Sigma)^*$ es un conjunto finito de producciones.
\end{itemize}
\end{deff}

La gramática permite generar al lenguaje de la siguiente manera...


Un lenguaje independiente de contexto es un lenguaje tal que es generado por una gramática donde sus producciones son del estilo..

\subsubsection{Autómatas.}
Así como las gramáticas nos permiten generar un lenguaje tenemos las máquinas que nos permiten aceptar un lenguaje. En nuestro caso en particular vamos a usar automátas de pila para aceptar los lenguajes independientes de contexto.

\begin{deff}
	Un \blue{automáta de pila finito} es una tupla ${\cal M } = (Q, \Sigma, Z, \delta, q_0, F)$ que la interpretamos de la siguiente manera,
	\begin{itemize}
		\item $Q$ denota los finitos estados;
		\item $\Sigma$ es el alfabeto finito del lenguaje que queremos reconocer;
		\item $Z$ es el alfabeto finito de la pila;
		\item $\delta$ es la función de transición donde $\delta: Z^* \times Q \times \Sigma^* \to Z^* \times Q^*$;
		\item $F \subseteq Q$ es el conjunto de estados finales;
		\item $q_0$ es el estado inicial;
	\end{itemize}
\end{deff}

El automáta de pila finito funciona de la siguiente manera. Dada una palabra $w \in \Sigma$ queremos saber si está palabra es aceptada por el autómata de pila o no. Un prefijo de $w$ será una subpalabra $\gamma$ tal que $w=\gamma w'$ visto con la concatenación de palabras en $\Sigma^*$. 
Para eso vamos a ir leyendo esta palabra de izquierda a derecha. Al comenzar a leer esta palabra estamos en el estado $q_0$ que distinguimos como el estado inicial de nuestro automáta. Nos fijamos en la función de transición que es una función parcial cuánto nos da evaluada  $\delta(\lambda,q_0,\gamma)$ para algún $\gamma$ prefijo de $w$ y donde estamos mirando a $\lambda \in Z^*$ el elemento neutro de este monoide. Esto se corresponde a la idea de que al comenzar nuestra pila está vacía. En tal caso nuestra función de transición nos da un resultado que es un par $(z,q)$ donde $z$ es lo que nos va a quedar en la pila y $q$ es el nuevo estado al cual nos movimos.

si tenemos a algún par $(Z_0 q_0 w, z q)$ donde la primer coordenada nos dice qué tenemos en el tope de la pila, en qué estado estamo y qué palabra estamos viendo.

Dentro de las transiciones $\delta$ buscamos alguna
 


\end{document}

