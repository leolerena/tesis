% !TeX TS-program = 
\documentclass[tesis.tex]{subfiles}

\newcommand{\APND}{automáta de pila no determinístico}
\newcommand{\APD}{automáta de pila determinístico }

\begin{document}
\chapter{Teoría de lenguajes.}	
En esta sección vamos a introducir los elementos básicos de la teoría de lenguajes formales que utilizaremos en este trabajo. 

Consideremos un conjunto finito no vacío $\Sigma$ que llamaremos el alfabeto y $\Sigma^k$ el conjunto de sucesiones finitas de elementos $a_1 \dots a_k$ con  $a_i \in \Sigma$. Los elementos de $\Sigma$ se llaman letras y los elementos de $\Sigma^k$ serán palabras de longitud $k$ sobre $\Sigma$. 
\begin{deff}
	El \blue{monoide libre} sobre un alfabeto $\Sigma$ es el siguiente conjunto
	\begin{equation*}
	\Sigma^{*} = \bigcup_{k=0}^{\infty} \Sigma^k
	\end{equation*}
	con la operación $\cdot$ que es la concatenación de palabras. Es decir dadas $w_1 \in \Sigma^{k}, w_2 \in \Sigma^{l}$ luego $w_1 \cdot w_2 \in \Sigma^{k+l} \subset \Sigma^*$. El elemento neutro es la palabra vacía que corresponde a la copia de $\Sigma^0$ que es la única palabra sin letras. 
\end{deff}
\begin{obs}
	El monoide es libre con la siguiente propiedad: si tenemos una función del alfabeto $f: \Sigma \to M$ donde $M$ es algún monoide entonces existe un único morfismo de monoides $\overline f: \Sigma^{*} \to M$ que hace conmutar al siguiente diagrama.	
	
	\begin{center}
	\begin{tikzcd}
		\Sigma \arrow[r, "f"] \arrow[d, hook] & M \\
		\Sigma^* \arrow[ru, "\overline f"]    &  
	\end{tikzcd}
	\end{center}
	
\end{obs}

Si $w$ es una palabra sobre el alfabeto $\Sigma$ luego una subpalabra $u$ de $w$ es una palabra $u \in \Sigma^*$ tal que $w = vuz$ para algunas $v, z \in \Sigma^*$. Si $w = vu$ entonces $v$ es un prefijo de $w$ y $u$ es un subfijo de $w$.

\begin{deff}
	Un \blue{lenguaje} sobre un alfabeto $\Sigma$ es un subconjunto de $\Sigma^*$.
\end{deff}


\begin{comment}
	\begin{obs}
	De esta manera podemos pensar a una {presentación} de un monoide $M$  como un epimorfismo de monoides $\pi: \Sigma^{*} \to M$. Esto nos dice que el monoide $M$ tiene como generadores a $\Sigma$ y sus relaciones están dadas por el núcleo del morfismo $\pi$.
	\end{obs}
\end{comment}

\section{Lenguajes independientes de contexto.}
\subsection{Gramáticas.}
Vamos a considerar lenguajes definidos a partir de lo que se conoce como una gramática. Esto es esencialmente unh conjunto de reglas que al irse aplicando nos permiten generar todas las palabras del lenguaje.

\begin{deff}
	Una gramática es una tupla ${\cal G} = (V, \Sigma, P, S)$ que la interpretamos de la siguiente manera,
\begin{itemize}
		\item $V$ es un conjunto finito de variables;
		\item $S \in V$ es el símbolo inicial;
		\item $\Sigma$ es un conjunto finito de símbolos terminales ;
		\item $P \subseteq (V \cup \Sigma)^* \times (V \cup \Sigma)^*$ es un conjunto finito de producciones.
\end{itemize}
\end{deff}

Las variables en algunos casos son también llamados símbolos no terminales. Las solemos notar en mayúscula y representan lenguajes. Las producciones son reglas recursivas que nos sirven para definir a un lenguaje. A la regla $(\omega, \lambda)$ la vamos a denotar $\omega \to \lambda$.

La gramática permite generar al lenguaje de la siguiente manera. Arrancamos con la producción que tiene al símbolo inicial $S$ a su izquierda y la reemplazamos por lo que aparece a la derecha. Es decir si miramos $S \to \omega$ luego reemplazamos a $S$ por $\omega$ y seguimos. Por cada subpalabra de $\omega\in (V \cup \Sigma)^*$ que aparezca a la izquierda de alguna de las producciones la reemplazamos por la que aparece a la derecha. 

Más en general definimos la relación $\Rightarrow$ que formaliza esta idea. Para $\omega, \xi \in \Sigma^*$  $\omega \Rightarrow_{\cal G} \xi$ significa que a partir de las reglas de nuestra gramática podemos derivar a $\xi$ usando las producciones sucesivamente. Tomamos la clausura transitiva, reflexiva y simétrica de esta relación que denotaremos $\overset{*}{\Rightarrow_{\cal G}}$.


De esta manera podemos definir un lenguaje a partir de la gramática como las palabras en $\Sigma^*$ tales que por medio de las producciones las podemos generar. Formalmente
\[
L({\cal G}) = \{ w \in \Sigma^* \ | \ S \overset{*}{\Rightarrow_{\cal G}} w   \}.
\]



Un \textit{lenguaje independiente de contexto} es un lenguaje tal que es generado por una gramática donde todas sus producciones son del estilo
\[
 \omega \to \lambda
\]
y cumplen que $|\omega| = 1$. 

\begin{ej}[Derivación de un lenguaje ind de contexto para ver el funcionamiento]
	Sea el alfabeto $\Sigma = \{ a,b \}$. Si $w=a_1 \dots a_k$ es una palabra sobre $\Sigma^*$ entonces podemos considerar a $w^r$ que es la palabra inversa dada por leerla de derecha a izquierda y tiene la siguiente pinta $w^r= a_k \dots a_1$. Consideremos sobre $\Sigma$ el lenguaje $L = \{ w \in \Sigma^* \ | \ w = w^r  \}$ tal que este es el lenguaje de los palíndromos. Construyamos una gramática independiente de contexto para este lenguaje. Sea $w$ una palabra en $L$ luego sabemos que si su longitud es mayor que uno entonces debe ser que $w = a u a$ o $w = b u b$ para cierta palabra $u \in L$ dado que $u$ necesariamente tiene que ser un palíndromo porque $w$ lo es. Esto sucede para todas las palabras del lenguaje exceptuando las palabras de longitud uno que son justamente las letras del alfabeto. De esta manera podemos considerar la siguiente gramática ${\cal G}  =  (\{S\}, \Sigma, P, S )$ donde las producciones $P$ están dadas por :
	\begin{align*}
	S  & \to \epsilon \\ S &\to a \\ S &\to b \\ S &\to  aSa \\ S &\to bSb. \\
	\end{align*}
	Para ver que esta gramática genera al lenguaje $L$ notemos que si tomamos una palabra en el lenguaje $w \in L$ luego si no es una letra al ser palíndromo sobre el alfabeto $\Sigma$ necesariamente debe comenzar con $a$ o con $b$ y de esta manera tenemos que la primer derivación es $S \to aSa$ o $S \to bSb$. Dado que la subpalabra $w = aua$ que se obtiene de $w$ sin considerar la primera y última letra es un palíndromo (e incluso podría ser la palabra vacía) luego podemos repetir este proceso para llegar a $S \implies^* w$ después de finitos pasos. Por otro lado toda palabra generada por esta gramática es un palíndromo porque todas las reglas son tales que agregan una letra al principio y la misma al final o simplemente agregan letras que también son palíndromos.
	
\end{ej}

\subsection{Autómatas.}
Así como las gramáticas nos permiten generar un lenguaje tenemos las máquinas que nos permiten aceptar un lenguaje. En nuestro caso en particular vamos a usar automátas de pila no determinísticos para aceptar los lenguajes independientes de contexto.

\begin{deff}
	Un \blue{automáta de pila finito no determinístico} es una tupla ${\cal M } = (Q, \Sigma, Z, \delta, q_0, F, Z_0)$ que la interpretamos de la siguiente manera,
	\begin{itemize}
		\item $Q$ denota los finitos estados;
		\item $\Sigma$ es el alfabeto finito del lenguaje que queremos reconocer;
		\item $\Gamma$ es el alfabeto finito de la pila;
		\item $\delta$ es la función de transición donde $\delta: Q  \times \Sigma^* \times \Gamma^* \to Q  \times \Sigma^* \times \Gamma^*$;
		\item $F \subseteq Q$ es el conjunto de estados finales;
		\item $q_0$ es el estado inicial;
		\item $Z_0$ es el símbolo inicial en la pila.
	\end{itemize}
\end{deff}


\paragraph{Funcionamiento del automáta.}
%Un prefijo de $w$ será una subpalabra $\gamma$ tal que $w=\gamma w'$ visto con la concatenación de palabras en $\Sigma^*$.

El automáta de pila finito funciona de la siguiente manera. Dada una palabra $w \in \Sigma$ queremos saber si es aceptada por el autómata de pila o no. Para eso vamos a ir leyendo esta palabra de izquierda a derecha. Al comenzar a leer esta palabra estamos en el estado $q_0$ que distinguimos como el estado inicial de nuestro automáta. Nos fijamos en la función de transición que es una función parcial cuánto nos da evaluada  $\delta(\lambda,q_0,\gamma)$ para algún $\gamma$ prefijo de $w$ y donde estamos mirando a $\lambda \in Z^*$ el elemento neutro de este monoide. Esto se corresponde a la idea de que al comenzar nuestra pila está vacía. En tal caso nuestra función de transición nos da un resultado que es un par $(z,q)$ donde $z \in Z^{*}$ es lo que nos va a quedar en la pila y $q$ es el nuevo estado al cual nos movimos. Notemos que nuestra función de transición no tiene porqué tener un $(z,q)$ tal que podamos movernos o podría ser bien que tenga más de uno. En este caso diremos que nuestro automáta es \textit{no determinístico} dado que en algunos casos existe más de una opción.

En general estamos en la siguiente situación en el proceso de aceptar la palabra $w$. Tenemos algún $z \in Z^{*}$ en la pila que al ser una pila lo leeremos de derecha a izquierda, estamos en algún estado $q \in Q$ y nos quedaré una subpalabra $\gamma$ de $w$ para leer. Una \textit{configuración} de nuestro automáta entonces es una manera de describir en que situación de aceptar o no aceptar una palabra y la denotamos $(zq\gamma)$. 

Cuando hayamos visto toda la palabra o no tengamos manera de movernos de estado nos fijamos si el estado $p$ en el que estamos es final, es decir si $p \in F$. En tal caso la palabra $w$ es aceptada por el automáta. Formalmente estaremos en alguna configuración $zq$ para $z \in Z^*$ y $q \in Q$ y no nos queda nada de la palabra $w$ porque ya la consumimos toda.

\paragraph{Descripción instántanea del automata.} Veamos ahora como describir formalmente el funcionamiento de un automáta a partir de lo que estamos haciendo en cierto instante. 
Consideremos que estamos en el instante que nuestra subpalabra que nos queda por leer es $w$, estamos en un estado $q$ y en nuestra pila tenemos la palabra $\gamma$, entonces vamos a representar al instante por medio de esta tupla $(q,w,\gamma)$.
Si ahora tenemos la posibilidad de movernos a otro estado $p$ tal que $(p,w',\alpha\beta) \in \delta (q,aw',x\beta)$ donde $aw' = w$ con $a$ alguna letra posiblemente vacía y similarmente $x \beta = \gamma$ con $x \in \Gamma^*$ una letra de $\Gamma$ posiblemente vacía. Este movimiento lo denotamos como $(q,aw',x\beta) \vdash (p,w',\alpha \beta)$. Podemos considerar la clausura transitiva de esta relación sobre los triples $Z^* \times Q \times \Sigma^*$ que denotaremos $\vdash^*$.


\paragraph{El lenguaje aceptado por un automáta de pila.} Notemos que en particular el automáta de pila nos da un lenguaje que está formado por las palabras $w$ en el alfabeto de la entrada del automáta $\Sigma$ que son aceptadas. En general diremos que un automáta acepta un lenguaje $L$ si su lenguaje aceptado es exactamente $L$. Este lenguaje aceptado por el automáta $\cal M$ en algunas casos para hacer énfasis en el automáta lo denotaremos $L {(\cal M)}$ y siguiendo la notación formal anterior lo podemos describir de la siguiente manera,
\begin{equation*}
	L( {\cal M}) = \{ w \in \Sigma^* \ | \ (q_0,w,Z_0) \vdash^* (q, \epsilon, \gamma), \ \ q \in F, \ \gamma \in \Gamma^*      \}
\end{equation*}
donde $(q_0, w, Z_0)$ es el instante inicial en el cual tenemos la palabra $w$ en el estado inicial $q_0$ con el símbolo $Z_0$ de la pila. Al finalizar deberíamos estar en un estado $q$ final y lo que nos queda en la pila $\gamma$ es totalmente irrelevante en este caso.

\begin{obs}
	Al ser un automáta de pila no determinístico existen posiblemente más de una manera de consumir alguna palabra en el automáta. Es así que por la definición que dimos la palabra es aceptada por el automáta si al menos alguna de estas derivaciones la lleva a ser aceptada. No importa si existen derivaciones que no lo hagan mientras una sí lo haga. 
\end{obs}
\medskip
\begin{teo}
Un lenguaje $L$ es independiente de contexto sii es aceptado por un automáta de pila no determinístico.
\end{teo}

\begin{proof}
	Ver \cite{}.
\end{proof}

De esta manera sabemos que en particular el lenguaje de los palíndromos definido anteriormente es aceptado por un automáta de pila no determinístico. Veamos como construirnos este automáta en este caso en especial,
\begin{ej}
	Sea nuestro alfabeto $\Sigma = \{ a, b\}$ y $L$ el lenguaje de los palíndromos sobre este alfabeto. 
	\red{Agregar tikz del automáta y una breve explicación de como funciona.}
	El automáta tiene dos estados: uno inicial y el otro final. La idea es que en el primer estado leemos la palabra y vamos apilando...
\end{ej}


Hasta ahora definimos los automátas de pila no determinísticos que aceptan por estado final. Otra definición posible de lenguaje aceptado podría ser que acepten por pila vacía. Es decir que una vez que consumimos la palabra $w$ de entrada llegamos a una configuración $(\lambda q)$ donde $\lambda$ es la palabra vacía de $Z^*$ el alfabeto de la pila. Formalmente notaremos al lenguaje aceptado por pila vacía por un automáta $\cal M$
de la siguiente manera,
\begin{equation*}
	L({\cal M}) = \{ w \in \Sigma^* \ | \ (q_0,w,Z_0) \vdash^* (q, u, \epsilon), \ q \in Q, u \in \Sigma^*    \}
\end{equation*}
donde arrancamos en el estado inicial y llegamos a algún estado $q$ cualesquiera, alguna subpalabra $u$ de $w$ y nuestra pila está vacía. En este caso obtenemos que nuestro lenguaje es aceptado por un automáta de pila no determinístico por pila vacía.


El siguiente resultado nos permite utilizar las dos maneras de aceptar un lenguaje tal como nos convenga.

\medskip
\begin{teo}
Un lenguaje $L$ es aceptado por un automáta de pila no determinístico por estado final sii es aceptado por un automáta de pila no determinístico por pila vacía.
\end{teo}

\begin{proof}
	Ver \cite{}
\end{proof}

\paragraph{Automatas determinísticos.} Si en la definición anterior del automáta pedimos que la función de transición de una configuración dada tenga a lo sumo un valor entonces nuestro automáta lo vamos a llamar determinístico. Formalmente esto es que 
\[
|\delta(z,p,\gamma)| \le 1 \ \ \ \forall z \in Z^*, \ p \in Q, \ \gamma \in \Sigma^{*}.
\]
En cierta manera estamos diciendo que de un instante dado solo tenemos a lo sumo una única posibilidad de movernos a otro. Por lo tanto para cada palabra tenemos un único
\begin{obs}
	Todo automáta de pila determínistico en particular es no determinístico y por lo tanto la clase de lenguajes aceptados por los primeros están contenidos en la clase de los segundos.
\end{obs}

Lo que vale es que esta contención es estricta. Para eso consideremos nuevamente el lenguaje de palíndromos. Para ver que no es aceptado por un \APD.

\paragraph{Automátas de pila determinísticos especiales.} Consideremos ahora un \APD tal que acepta tanto por estado final como por pila vacía. A estos los llamaremos \textit{automátas de pila determinístico especiales}.  
Estos automátas son los que nos surgen de la construcción del automáta del problema de la palabra de la siguiente sección para grupos virtualmente libres. Notemos que los lenguajes que aceptan son una subclase de los independientes de contexto. 

\begin{ej}
	Sea el lenguaje $L = \{ a^n b^m   \}$
\end{ej}

\section{Sistemas de reescritura.}

Un \emph{sistema de reescritura} podemos pensarlo como un grafo en el sentido de Serre. En este caso los vértices son los \emph{objetos} mientras que las aristas las llamamos \emph{movimientos}.  Si nuestro sistema de reescritura $\Gamma$ tiene un movimiento de un objeto $a$ en otro objeto $b$ diremos que $a$ puede ser \emph{reescrito} a $b$ y lo denotaremos $a \to_{\Gamma} b$, omitiendo aclarar que estamos considerando el sistema $\Gamma$ en todo caso que no sea ambiguo. Un camino en el grafo lo llamaremos una \emph{derivación}.  

Nosotros queremos que estos sistemas de reescritura sean tales que si tomamos una sucesión de movimientos en algún momento se estabilice y en tal caso llamaremos \emph{terminante.} A su vez los objetos tales que no puedan ser modificados por ningún movimiento llamaremos objetos \emph{terminales.} 

Dado un sistema de reescritura $\Gamma$ por medio de la clausura transitiva que denotaremos $\overset{*}{\rightarrow}_{\Gamma}$ obtenemos una relación transitiva sobre los objetos. Nuestro interés justamente va a estar en estudiar las relaciones de equivalencia que surgen de sistemas de reescritura particulares. 

\begin{ej}
	Considerar como objetos los enteros y como movimiento dividir al número por dos si es posible. Este sistema de reescritura tiene como objetos terminales los números impares.
\end{ej}

Otra característica que le vamos a pedir a los sistema de reescritura es que sean \emph{confluentes}. Informalmente esto es que a partir de un objeto, si usamos dos derivaciones distintas entonces eventualmente estas derivaciones se encuentran en algún objeto independientemente de qué derivaciones tomemos. Esto es que dadas derivaciones $a \overset{*}{\rightarrow} b, a \overset{*}{\rightarrow} c$ existe un objeto $d$ tal que $b \overset{*}{\rightarrow} d$ y $c \overset{*}{\rightarrow} d$.

En particular los sistemas que nos van a interesar en este trabajo son sistemas tales que son confluentes y terminantes que son conocidos como \emph{Church-Roser}.

\subsection{Sistemas de reescritura de cadenas.}
En nuestro caso en particular vamos a trabajar con sistemas de reescritura donde los objetos son palabras sobre algún alfabeto $\Sigma$ y los movimientos son reglas del estilo $w \to v$ que interpretamos de la siguiente manera. Si tenemos alguna palabra que tenga como subpalabra a $w$ podemos modificarla por $v$. Esto lo representamos por el triple $(p, w \to v, q)$ donde $p,q$ son las subpalabras que vienen antes de $w$ tales que pueden ser vacías. 

Al sistema de reescritura $\Gamma$ con alfabeto $\Sigma$ y las reglas $\cal R$ lo vamos a denotar por sr$\left< \Sigma, {\cal R} \right>$. Estos sistemas también son llamados \emph{sistemas de Thue}.






\end{document}

