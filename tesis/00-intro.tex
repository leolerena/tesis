\documentclass[tesis.tex]{subfiles}
\begin{document}
	\chapter*{Introducción.}
	
	Si $G$ es un \fg el problema de la palabra es el siguiente problema: dada una palabra $w$ en los generadores y sus inversos el objetivo es decidir si $w$ representa la identidad del grupo.
	Este problema no resulta ser un problema fácil, de hecho existen grupos finitamente presentados tales que su problema de la palabra no es algorítimicamente decidible.
	A pesar de la existencia de estos resultados negativos, para muchos grupos es algorítimicamente decidible.
	Esto hace que resulte interesante estudiar este problema con un enfoque en la complejidad computacional.
	En este trabajo estudiamos el problema de la palabra desde el punto de vista de los lenguajes formales.
	Esto quiere decir que dado un grupo finitamente generado le asociamos el conjunto de palabras sobre los generadores y sus inversos tales que representan a la identidad del grupo.
	Este lenguaje formal es el lenguaje asociado al problema de la palabra.

	% Una manera de definir los lenguajes regulares es un lenguaje que es aceptado por un autómata finito.

	% El primer trabajo que abordó el problema de la palabra con esta perspectiva fue el trabajo de Animisov \cite{anisimov1971languages} en el cual probó que los grupos finitos son exactamente los grupos que su problema de la palabra resulta ser un lenguaje regular.
	
	El primer trabajo que estudió el problema de la palabra con esta perspectiva fue el de Animisov \cite{anisimov1971languages}. 
	En ese trabajo se consideró la familia de lenguajes regulares, que pueden ser definidos como los lenguajes que son aceptados por un aútomata finito.
	% El logro del trabajo fue conectar la teoría de lenguajes con la teoría de los grupos. 
	El resultado principal al que llegó fue que un grupo es tal que el lenguaje asociado a su problema de la palabra es un lenguaje regular si y solo si este grupo es finito.
		
	\begin{figure}[H]
		\centering
		\begin{tikzpicture}[->,>=stealth',shorten >=1pt,auto,node distance=3.5cm,
				scale = 1,transform shape]
				\node[state,initial,accepting] (1) [] {$1$};
				\node[state] (a) [above right of=1] {$a$};
				\node[state] (b) [below right of=1] {$b$};
				\node[state] (ab) [below right of=a] {$ab$};

					\path (1) edge    [bend left]          node {$a$ \ } (a);
					\path (a) edge    [bend left,swap]          node {$a^{-1}$ \ } (1);
					\path (1) edge    [bend left]          node {$b$ \ } (b);
					\path (b) edge    [bend left, swap]          node {$b^{-1}$ \ } (1);
					\path (a) edge    [bend left]          node {$b$ \ } (ab);
					\path (ab) edge   [bend left, swap]          node {$b^{-1}$} (a);
					\path (b) edge    [bend left]          node {$a$ \ } (ab);
					\path (ab) edge   [bend left, swap]          node {$a^{-1}$ \ } (b);
			\end{tikzpicture}
		\caption{Este autómata acepta el problema de la palabra del grupo $\ZZ/2\ZZ \times \ZZ/2\ZZ$ presentado como $\langle a, b \mid a^2, b^2, abab \rangle$. Observar la semejanza con el grafo de Cayley para la misma presentación.}
	\end{figure}
	Este resultado abrió un abanico de posibilidades. 
	Si uno considera una familia de lenguajes, ¿qué grupos tienen el lenguaje asociado al problema de la palabra en esa familia?
	En particular en ese trabajo Animisov preguntó: "Si el problema de la palabra de un grupo $G$ es un lenguaje independiente de contexto, ¿qué dice esto sobre la estructura algebraica del grupo?".
	Esta pregunta fue resuelta en un trabajo influyente de Muller y Schupp \cite{muller1983groups}.
	Su respuesta es que los grupos virtualmente libres son exactamente los grupos que tienen un lenguaje asociado al problema de la palabra independiente de contexto.
	A diferencia del caso de los lenguajes regulares, el caso de los lenguajes independientes de contexto resultó ser mucho más complejo de probarlo.
	% Esta dificultad radica en que no resultó posible probar el teorema de forma directa.
	Para probar este resultado Muller-Schupp se valieron de varias áreas de la matemática y varias caracterizaciones equivalentes de los grupos virtualmente libres.

	Desde aquel entonces aparecieron aún más demostraciones del teorema de Muller-Schupp, todas ellas con su riqueza y sus nuevas caracterizaciones de los grupos virtualmente libres.
	Todas estas demostraciones emplean las más diversas áreas de matemática probando así la riqueza detrás de estos grupos.
    Para ejemplificar listamos algunas de estas posibles caracterizaciones de los grupos virtualmente libres: (1) grupos fundamentales de grafos de grupos finitos, (2) grupos finitamente generados tales que sus grafos de Cayley tienen treewidth finito, (3) grupos universales de pregrupos finitos, (4) grupos con presentaciones finitas dadas por sistemas geodésicos de reescritura, (5) grupos finitamente generados con teoría monádica de segundo orden decidible, etc. 
	Para el lector interesado en entender algunas de estas construcciones consultar los siguientes trabajos \cite{diekert2017context}, \cite{diekert2013context}, \cite{khukhro_characterisation_2020}, \cite{kuske2005logical}, \cite{muller1985theory}, \cite{muller1983groups}, \cite{antolin2011cayley} y \cite{araujo2017geometric}.

	El objetivo de esta tesis es estudiar una parte de todas estas posibles caracterizaciones de los grupos virtualmente libres y probar un camino de equivalencias entre ellas.
	El camino de equivalencias elegido está basado en el camino propuesto por los trabajos \cite{diekert2013context} y \cite{diekert2017context}.
	Para este objetivo utilizamos herramientas de topología, de teoría de grupos, de teoría de lenguajes formales y de teoría de grafos, 
	mostrando así cómo todas estas áreas de matemática se ponen en común para describir a los grupos virtualmente libres.
	Es particularmente interesante ver cómo las propiedades sintácticas de un grupo afectan la geometría de su grafo de Cayley y  también entender cómo las acciones del grupo repercuten en su problema de la palabra.
	

	El trabajo está estructurado de manera que en cada capítulo (exceptuando el de 
	preliminares) probamos una caracterización equivalente de los grupos \vls.
	A continuación resumimos brevemente los contenidos de cada capítulo:
	\begin{itemize}
		\item 
			El primer capítulo es un capítulo introductorio.
			En este se abordan las ideas elementales de teoría de lenguajes formales, teoría de grupos y de teoría de grafos que se utilizan a lo largo del trabajo.
			% En el primer capítulo introducimos las ideas más elementales que usaremos de la teoría de grupos, de la teoría de grafos y de la teoría de lenguajes.
			Estos temas tratados son en su gran mayoría temas clásicos y pueden encontrarse por ejemplo en los siguientes libros \cite{lyndon1977combinatorial}, \cite{diestel2005graph}, \cite{gallier2022mathematical} y \cite{hopcraft-ullman}.
		
		\item 
			En el segundo capítulo se da una introducción a la teoría de Bass--Serre. 
			Esta área conecta la estructura de un grupo con sus acciones en árboles.
			A partir de una acción de un grupo sobre un árbol definimos un grafo de grupos y viceversa, a partir de un grafo de grupos construimos un grupo (denominado el grupo fundamental del grafo de grupos) y un árbol (denominado el árbol de Bass--Serre) en el cuál actúa.
			Uno de los resultados fundacionales del área es que estas dos construcciones son inversas \ref{teo_Serre}.
			El resultado central de este capítulo para el propósito de este trabajo es \ref{teo_karrass_solitar} que prueba que el grupo fundamental de los grafos de grupos con grupos finitos resulta ser un grupo virtualmente libre.

		\item 
			En el tercer capítulo se introducen los grupos independientes de contexto.
			Estos son grupos para los cuales el lenguaje asociado a su problema de la palabra resulta ser un lenguaje independiente de contexto.
			Si bien este lenguaje depende del conjunto de generadores se prueba que para cualquier otro conjunto de generadores también resulta ser un lenguaje independiente de contexto.
			Esta propiedad es válida porque los lenguajes independientes de contexto resultan ser un cono de lenguajes.
			Finalmente se prueba el teorema central del capítulo que es el de Muller--Schupp \ref{teo_Muller_Schupp} que todo grupo virtualmente libre es \ic porque existe un autómata de pila que acepta el lenguaje correspondiente a su problema de la palabra. 
			
			
		
		\item 
			En el cuarto capítulo se introducen las descomposiciones en un árbol y a partir de estas descomposiciones definimos un número natural o infinito que es el treewidth de un grafo.
			Se prueban varios propiedades elementales que tienen estas descomposiciones y se da la construcción principal para nuestro trabajo que es una descomposición en un árbol para un grafo de Cayley de un grupo finitamente generado arbitrario \ref*{desc-grafo-cayley}.
			A partir de esto se definen los grupos de treewidth finito que son los grupos para los cuales esta descomposición tiene treewidth finito.
			% Probamos que si para un grupo y un conjunto finito de generadores del grupo esta descomposición en un árbol tiene treewidth finito entonces para cualquier otro conjunto de generadores la descomposición en un árbol también va a tener treewidth finito.
			Finalmente se prueba el resultado central del capítulo \ref{teo_ic_implica_tw} que dice que los grupos \ic resultan tener treewidth finito.
			
		\item 
			En el quinto capítulo se termina de cerrar el camino de equivalencias de caracterizaciones de los grupos virtualmente libres.
			Para esto se da una rápida introducción a la teoría de los cortes de los grafos con el foco en los grafos accesibles.  
			Estos resultan ser una familia de grafos que contienen a la familia de grafos con treewidth finito.
			Se prueba que si se considera un grafo accesible luego se puede construir un árbol a partir de los cortes del grafo.
			Este árbol lo denominamos el árbol de estructura.
			El resultado central del capítulo es \ref{coro_tw_finito_implica_pi1} que dice que el grupo de automorfismos de un grafo con treewidth finito actúa con finitas órbitas sobre su árbol de estructura. 
			De esta manera por medio del resultado central de la teoría de Bass--Serre  \ref{teo_Serre} se obtiene que los grupos con treewidth finito son isomorfos a grupos fundamentales de grafos de grupos finitos y así se termiona de probar todas estas equivalencias.
	\end{itemize}
		
	
	
	
	El siguiente esquema muestra las relaciones que existen entre los resultados probados en cada capítulo.
	
	\[	
	\begin{tikzpicture}
		\path 
		(0,0) node(a) [rectangle,draw] {Grupo fundamental de un grafo de grupos finito}
		(5,-3) node(b) [rectangle,draw] {Treewidth finito}
		(0,-6) node(c) [rectangle,draw] {Independiente de contexto}
		(-5,-3) node (d) [rectangle,draw] {Virtualmente libre};
		\draw   
		(d) edge[<-,line width=1.0pt,"Teorema \ref{teo_karrass_solitar}"] (a) 
		(c) edge[<-,line width=1.0pt,"Teorema \ref{teo_Muller_Schupp}"] (d)
		(b) edge[<-,line width=1.0pt,"Teorema \ref{teo_ic_implica_tw}"] (c)
		(a)  edge[<-,line width=1.0pt,"Teorema \ref{coro_tw_finito_implica_pi1}"] (b);
	\end{tikzpicture}
	\]
	
	
	
	
	
	
	

\end{document}