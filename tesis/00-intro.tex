\documentclass[tesis.tex]{subfiles}
\begin{document}
	\chapter*{Introducción.}
	
	Si $G$ es un \fg el problema de la palabra es el siguiente problema: dada una palabra $w$ en los generadores y sus inversos el objetivo es decidir si $w$ representa la identidad del grupo.
	Este problema no resulta ser un problema fácil, de hecho existen grupos finitamente presentados tales que su problema de la palabra no es decidible.
	A pesar de la existencia de estos resultados negativos, para muchos grupos el problema de la palabra es decidible.
	Esto hace que resulte interesante estudiar el problema de la palabra desde un punto de vista de la complejidad computacional.
	En nuestro trabajo, siguiendo la teoría de lenguajes formales, el problema de la palabra de un grupo finitamente generado es el conjunto de palabras sobre los generadores y sus inversos tales que representan a la identidad del grupo.
	Bajo esta definición el problema de la palabra resulta ser un lenguaje formal.
	El primer trabajo que abordó el problema de la palabra con esta perspectiva fue el trabajo de Animisov \cite{anisimov1971languages} en el cual probó que los grupos finitos son exactamente los grupos que su problema de la palabra resulta ser un lenguaje regular.

	
	\begin{figure}[H]
		\centering
		\begin{tikzpicture}[->,>=stealth',shorten >=1pt,auto,node distance=3.5cm,
				scale = 1,transform shape]
				\node[state,initial,accepting] (1) [] {$1$};
				\node[state] (a) [above right of=1] {$a$};
				\node[state] (b) [below right of=1] {$b$};
				\node[state] (ab) [below right of=a] {$ab$};

					\path (1) edge    [bend left]          node {$a$ \ } (a);
					\path (a) edge    [bend left,swap]          node {$a^{-1}$ \ } (1);
					\path (1) edge    [bend left]          node {$b$ \ } (b);
					\path (b) edge    [bend left, swap]          node {$b^{-1}$ \ } (1);
					\path (a) edge    [bend left]          node {$b$ \ } (ab);
					\path (ab) edge   [bend left, swap]          node {$b^{-1}$} (a);
					\path (b) edge    [bend left]          node {$a$ \ } (ab);
					\path (ab) edge   [bend left, swap]          node {$a^{-1}$ \ } (b);
			\end{tikzpicture}
		\caption{Este autómata acepta el problema de la palabra del grupo $\ZZ/2\ZZ \times \ZZ/2\ZZ$ presentado como $\langle a, b \mid a^2, b^2, abab \rangle$. Observar la semejanza con el grafo de Cayley para la misma presentación.}
	\end{figure}
	En ese trabajo Animisov preguntó: "Si el problema de la palabra de un grupo $G$ es un lenguaje independiente de contexto qué dice esto sobre la estructura algebraica del grupo?".
	Esta pregunta fue resuelta en un trabajo influyente de Muller y Schupp \cite{muller1983groups}.
	Su respuesta es que los grupos virtualmente libres son exactamente los grupos que tienen un problema de la palabra independiente de contexto.
	Desde aquel entonces se encontraron una enorme cantidad de caracterizaciones equivalentes para los grupos virtualmente libres. 
	Estas caracterizaciones emplean las más diversas áreas de matemática demostrando la riqueza detrás de los grupos virtualmente libres.
    Para ejemplificar listamos algunas de estas posibles caracterizaciones de los grupos \vl: (1) grupos fundamentales de grafos de grupos finitos, (2) grupos finitamente generados tales que sus grafos de Cayley tienen treewidth finito, (3) grupos universales de pregrupos finitos, (4) grupos con presentaciones finitas dadas por sistemas geodésicos de reescritura, y (6) grupos finitamente generados con teoría monádica de segundo orden decidible. 
	Para el lector interesado en ver aún más caracterizaciones consultar los siguientes trabajos \cite{diekert2017context}, \cite{antolin2011cayley} y \cite{araujo2017geometric}.

	El objetivo de esta tesis es estudiar algunas de estas posibles caracterizaciones de los grupos virtualmente libres y probar un camino de equivalencias entre todas estas caracterizaciones.
	Para este objetivo utilizamos caracterizaciones provenientes de la topología, de la teoría de grupos, de la teoría de lenguajes formales y de la teoría de grafos, 
	mostrando así cómo todas estas áreas de matemática se ponen en común para describir un objeto.
	Es particularmente interesante ver cómo las propiedades sintácticas de un grupo afectan la geometría de su grafo de Cayley y  también entender cómo las acciones del grupo repercuten en su problema de la palabra.
	Muchos de los resultados y construcciones que aparecen en este trabajo son resultados conocidos que no aparecen detallados en la literatura.

	El trabajo está estructurado de manera que en cada capítulo (exceptuando el de 
	preliminares) probamos una caracterización equivalente de los grupos \vls.
	A continuación resumimos brevemente los contenidos de cada capítulo:
	\begin{itemize}
		\item 
			En el primer capítulo introducimos las ideas más elementales que usaremos de la teoría de grupos, de la teoría de grafos y de la teoría de lenguajes.
			Estos temas tratados son en su gran mayoría temas clásicos y pueden encontrarse por ejemplo en los siguientes libros \cite{lyndon1977combinatorial}, \cite{diestel2005graph}, \cite{gallier2022mathematical} y \cite{hopcraft-ullman}.
		
		\item 
			En el segundo capítulo damos una introducción a la teoría de Bass--Serre que conecta la estructura de un grupo con sus acciones en árboles.
			A partir de una acción de un grupo sobre un árbol nos define un grafo de grupos y viceversa, a partir de un grafo de grupos podemos armarnos un grupo (denominado el grupo fundamental del grafo de grupos) y un árbol en el cuál actúa.
			Uno de los resultados centrales de la teoría de Bass--Serre es que estas construcciones son inversas y esto lo probamos en el teorema \ref{teo_Serre}.
			El resultado central de este capítulo para nuestro trabajo es \ref{teo_karrass_solitar} que prueba que el grupo fundamental de los grafos de grupos con grupos finitos resulta ser un grupo virtualmente libre.

		\item 
			En el tercer capítulo introducimos los grupos independientes de contexto, que son los grupos que el lenguaje del problema de la palabra, esto es las palabras en los generadores que representan la identidad, resulta ser un lenguaje independiente de contexto.
			 
			En principio, como este lenguaje depende del conjunto de generadores elegido esta definición no está justificada.
			Entonces nos vemos obligados a probar que si un grupo tiene un lenguaje del problema de la palabra independiente de contexto entonces para cualquier conjunto de generadores también resulta ser un lenguaje independiente de contexto.
			Esta propiedad es válida porque los lenguajes independientes de contexto resultan ser un cono de lenguajes.
			Finalmente probamos el teorema central del capítulo que es el de Muller--Schupp \ref{teo_Muller_Schupp} que todo grupo virtualmente libre es \ic porque existe un autómata de pila que acepta el lenguaje correspondiente a su problema de la palabra. 
			
			
		
		\item 
			En el cuarto capítulo introducimos para grafos no dirigidos las descomposiciones en un árbol y a partir de estas descomposiciones definimos un número natural o infinito que es el treewidth de un grafo.
			Probamos varios propiedades elementales que tienen estas descomposiciones y construimos una descomposición en un árbol para un grafo de Cayley de un grupo finitamente generado arbitrario.
			Para que el treewidth finito sea un invariante para un grupo y no dependa del grafo de Cayley utilizamos cuasisometrías para probar que el treewidth finito es un invariante por cuasisometría y así como todos los grafos de Cayley de un grupo finitamente generado son cuasisométricos entre sí obtenemos que está bien definida la noción de un grupo con treewidth finito.
			Finalmente probamos el resultado central del capítulo \ref{teo_ic_implica_tw} que nos dice que los grupos \ic resultan tener treewidth finito.
			
		\item 
			En el quinto capítulo terminamos de cerrar las equivalencias de los grupos virtualmente libres.
			Para esto esbozamos rápidamente la teoría de los cortes de los grafos y nos enfocamos en los grafos accesibles que resultan ser una familia que contiene a los grafos que tienen treewidth finito.
			Probamos que bajo estas hipótesis podemos construirnos a partir de los cortes de los grafos un árbol que denominamos el árbol de estructura del grafo.
			El resultado central del capítulo es \ref{coro_tw_finito_implica_pi1} que dice que el grupo de automorfismos de un grafo con treewidth finito actúa con finitas órbitas sobre su árbol de estructura. 
			De esta manera manera por medio del resultado central de la teoría de Bass--Serre  \ref{teo_Serre} obtenemos que los grupos con treewidth finito son grupos fundamentales de grafos de grupos finitos y así terminamos de probar todas estas equivalencias.
	\end{itemize}
		
	
	
	
	El siguiente esquema muestra las relaciones que existen entre los resultados probados en cada capítulo.
	
	\[	
	\begin{tikzpicture}
		\path 
		(0,0) node(a) [rectangle,draw] {Grupo fundamental de un grafo de grupos finito}
		(5,-3) node(b) [rectangle,draw] {Treewidth finito}
		(0,-6) node(c) [rectangle,draw] {Independiente de contexto}
		(-5,-3) node (d) [rectangle,draw] {Virtualmente libre};
		\draw   
		(d) edge[<-,line width=1.0pt,"Teorema \ref{teo_karrass_solitar}"] (a) 
		(c) edge[<-,line width=1.0pt,"Teorema \ref{teo_Muller_Schupp}"] (d)
		(b) edge[<-,line width=1.0pt,"Teorema \ref{teo_ic_implica_tw}"] (c)
		(a)  edge[<-,line width=1.0pt,"Teorema \ref{coro_tw_finito_implica_pi1}"] (b);
	\end{tikzpicture}
	\]
	
	
	
	
	
	
	

\end{document}