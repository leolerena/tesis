\documentclass[tesis.tex]{subfiles}


 
\begin{document}
	
\chapter{Cortes de grafos y árboles de estructura.}

Las principales referencias de este capítulo son Diekert \cite{diekert2013context}, Thomassen \cite{thomassen1993vertex} y Krön \cite{kron2010cutting}.
En la sección \ref{secc_cortes_grafos} damos varias definiciones básicas que usaremos en el capítulo. 

En la sección \ref{secc_grafos_accesibles} definimos una familia de grafos que denominamos grafos accesibles.
Esta definición de un grafo accesible generaliza la idea de los grafos de Cayley de grupos accesibles.
En el teorema \ref{teo_treewidth_fin_accesible} probamos que los grupos que tienen treewidth finito tienen grafos de Cayley accesibles. 

En la sección \ref{secc_copt} definimos los cortes óptimos que son una familia de cortes que tienen la propiedad de estar siempre anidados entre sí.
El resultado principal de esta sección es el teorema \ref{teo_copt_anidados}.

En la sección \ref{secc_arbol_estr} dado un grafo accesible y localmente finito definimos un árbol que denominamos el árbol de estructura.
Este árbol lo definimos a partir de los cortes óptimos del grafo.

En la sección \ref{secc_accion_arbol_estr} construimos una acción del grupo de automorfismos de un grafo accesible y localmente finito sobre su árbol de estructura.
El resultado central es \ref{coro_tw_finito_implica_pi1} que dice que los grupos con treewidth finito son el grupo fundamental de un grafo de grupos con grupos finitos.



%En este capítulo vamos a probar que todo grupo que tiene treewidth finito resulta ser el grupo fundamental de un grafo de grupos finitos, de esta manera terminando de probar la equivalencia de todas las definiciones de los grupos virtualmente libres.
%El objetivo es partir de un grupo con treewidth finito y ver qué actúa sobre un cierto árbol por lo tanto por la teoría de Bass--Serre culminamos la demostración.
%El árbol que vamos a construir es el árbol de estructura del grafo.
%Vamos a ver en qué caso se puede construir y cómo se construye a partir de cortes del grafo. 
%La notación y el orden presentado sigue el del trabajo \cite{diekert2013context}, completando algunos detalles que ahí se omiten.
%\todo{Agregaría una introducción como la de los otros caps}
%
%De ahora en más consideraremos que $\Gamma$ es un grafo conexo, infinito y localmente finito salvo que se mencione lo contrario.
%En particular estos grafos incluyen a los grafos de Cayley de grupos finitamente generados.

\section{Cortes de grafos.}\label{secc_cortes_grafos}

\begin{deff}
	Sea $\Gamma$ un grafo no dirigido.
	Un conjunto $C \subset V(\Gamma)$ es un \emph{corte} si cumple las siguientes condiciones:
	\begin{itemize}
		\item $C$ y $\ol C$ son conexos y no vacíos.
		\item $|\delta C| < \infty$.
	\end{itemize}
	Si $|\delta C| \le k$ decimos que $C$ es un \emph{$k$-corte}.
\end{deff}	

Equivalentemente un subconjunto de vértices $C$ es un corte si $\Gamma \setminus \delta C$ tiene dos componentes conexas no vacías.

\begin{figure}[]
	\centering
	\begin{tikzpicture}
		[scale=0.60,V/.style = {circle, draw,align= center, minimum size=0.25cm,
			minimum size=1em,inner sep=2,
			fill=applegreen!15,font=\scriptsize	},fill fraction/.style={path picture={
				\fill[#1] 
				(path picture bounding box.south) rectangle
				(path picture bounding box.north west);
		}},
		fill fraction/.default=astral!90
		]
		
		\begin{scope}[nodes=V,xshift=4.5cm, yshift=-4cm]
			\node (1) at (0,0) {};
			\node (a) at (2,0)  {};
			\node (b) at (-2,2)     {};
			\node (bb) at (-2,-2)    {};
			\node (ab) at (4,2)      {};
			\node (ba) at (-4,2)     {};
			\node (bba) at (-4,-2)     {};
			\node (aba) at (6,2)    {};
			\node (abba) at (6,-2)    {};
			\node (bab) at (-6,3)    {};
			\node (bbab) at (-6,-3)    {};
			\node (bbabb) at (-6,-1)    {};
			\node (ababb) at (8,1)    {};
			
			\node (abbab) at (8,-1)     {};
			\node (abbabb) at (8,-3)  {};
			
		\end{scope}
		
		
		
		
		\draw   (1)  edge[-] (a);
		\draw   (1)  edge[-] (b);
		\draw   (1)  edge[-] (bb);
		\draw   (b)  edge[-] (bb);
		\draw   (b)  edge[-] (ba);
		\draw   (b)  edge[-] (abbab);
		\draw   (aba)  edge[-] (ababb);
		\draw   (abba)  edge[-] (ababb); 
		\draw   (abba)  edge[-] (ab); 
		\draw   (a)  edge[-] (ab);
		\draw   (1)  edge[-] (bba);
		\draw   (bbab)  edge[-] (bba);
		\draw   (abbab)  edge[-] (abbabb);
		\draw   (bbabb)  edge[-] (bbab);
		\draw   (bbabb)  edge[-] (bab);
		\draw   (bbabb)  edge[-] (bab);
		\draw   (abbab)  edge[-] (ababb);
		
		
		\draw [ultra thick,cadmiumred, label=above:$A$] (5.25,-5) to[out=45,in=-15] (5.25,-2);
		\node (9) at (5.25,-1.5) {\textcolor{cadmiumred}{$\beta C$}};
	\end{tikzpicture}
\caption{\small{ $C$ es un $2$-corte en este caso. }}
\end{figure}


La siguiente observación elemental va a ser fundamental a los argumentos de este capítulo.

\begin{obs}\label{obs_camino_corte}
	Si $\Gamma$ es un grafo conexo, $C$ un corte de $\Gamma$, $x \in C$ e $y \in \ol C$ y 
	$\alpha$ un camino que une $x$ con $y$ 
	entonces tenemos que debe existir $v \in \delta C$ tal que $v$ pertenece al camino $\alpha$.
\end{obs} 


En particular nos van a interesar los cortes para separar caminos en grafos y en cierta manera usar los cortes para distinguir caminos infinitos.
Idealmente queremos que nuestros cortes nos separen en partes infinitas al grafo pero esto no siempre es posible.

\begin{ej}\label{ej:grilla_corte}
	Consideremos $\ZZ \times \ZZ = \langle a,b \mid aba^{-1}b^{-1} \rangle$ y 
	$\Gamma = \text{Cay}(\ZZ^2, \{ a, b,a^{-1},b^{-1} \})$ su grafo de Cayley.
	
	Este grafo cumple que todo conjunto $C \subseteq V(\Gamma)$ conexo tal que $|\delta C| < \infty$ es tal que $C$ es finito o bien $\ol C$ es finito.
	Para probar esto sea $v \in \delta C$ tal que
	maximice $\underset{w \in \delta C}{\max} \ d(1,w)$.
	Si consideramos una bola $B$ centrada en $1$ de radio mayor a esta distancia entonces tenemos que $C \subseteq B$ o bien $\ol C \subseteq B$ dado que, en caso contrario como $\ol B$ es conexo entonces usando la observación \ref{obs_camino_corte} tendríamos que habría algún $v' \in \ol B$ tal que $v' \in \beta C$ contradiciendo la elección del radio de $B$.
	Dado que $\Gamma$ tiene el grado acotado uniformemente obtenemos que  $|B| < \infty$ lo que implica que $|C| < \infty$ o bien que $|\ol C| < \infty$.
\end{ej}

Probaremos ahora una propiedad fundamental de los cortes que es que fijada una arista solo hay finitos cortes que tienen a esta arista dentro de su borde.	
Dado un grafo $\Gamma$ y $\{x,y\} \in E(\Gamma)$  usaremos la notación $\Gamma \setminus \{ x,y \}$ para referirnos al grafo no dirigido definido por $V(\Gamma \setminus \{ x,y \}) = V(\Gamma)$ y $E(\Gamma \setminus \{ x,y \}) = E(\Gamma) \setminus \{\{x,y\}\}$.

\begin{lema}\label{obs_kCorte_restriccion}
	Sea $\Gamma$ un grafo conexo infinito y $k \ge 2$.
	Si $C$ es un $k$-corte en $\Gamma$ y $\{ x,y  \} \in \delta C$ entonces $C$ es un $(k-1)$-corte en $ \Gamma \setminus \{ x,y \} $.
\end{lema}

\begin{proof}
%	Para ver esto notemos que $\Gamma \setminus \{ x,y \}$ es tal que si restringimos $C$ a este grafo nos queda que $|\delta C| = k-1$.
	Los conjuntos de vértices $C$ y $\ol C$ siguen siendo conexos y no vacíos si los miramos en $\Gamma \setminus \{ x,y \}$.
	Por otro lado tenemos que su borde de aristas en el grafo $\Gamma \setminus \{x,y\}$ es $\delta C \setminus \{x,y\}$, por lo tanto $C$ resulta ser un $(k-1)$-corte de $\Gamma \setminus \{x,y\}$.
\end{proof}

\begin{lema}\label{lema_corte_disconexo_vertices}
	Sea $\Gamma$ un grafo conexo.
	Sea $Y \subseteq E(\Gamma)$ un subconjunto finito de manera que $\Gamma \setminus Y$ resulta ser disconexo.
	Entonces si existe un corte $C$ tal que $Y \subseteq \delta C$ luego tiene que valer que $Y = \delta C$.
\end{lema}

\begin{proof}
	Si $\Gamma \setminus Y$ resulta ser disconexo entonces para que exista algún corte $C$ con $Y \subseteq \delta C$ es necesario que $\Gamma \setminus Y$ tenga exactamente dos componentes conexas.
	Si tuviera más de una componente conexa y existiera un corte $C$ que cumpla que $Y \subseteq \delta C$ entonces $\Gamma \setminus \delta C$ tendría más de dos componentes conexas, negando que $C$ es un corte.	
	
	Llamemos a las dos componentes conexas de $\Gamma \setminus Y$ como $D$ y  $\ol D$.
	Supongamos que $C$ es un corte tal que $Y \subseteq \delta C$. 
	Notemos que $C$ y $\ol C$ son conexos si los vemos en $\Gamma \setminus Y$.
	En este caso tenemos que necesariamente $C \cap  D = \emptyset$ o $C \cap \ol D = \emptyset$ en $\Gamma \setminus Y$.
	Caso contrario como $C \cup \ol C = V(\Gamma)$ entonces negaríamos que $D$ y $\ol D$ desconectan a $\Gamma \setminus Y$.
	Concluimos que necesariamente $C = D$ o bien $C = \ol D$ y así que $\delta C = Y$ tal como queríamos ver.
	
\end{proof}

\begin{prop}\label{lema_aristas_finitos_kcortes}
	Sea $\Gamma$ un grafo conexo y sea $\{x,y\} \in E(\Gamma)$ entonces para todo $k \in \NN$ existen finitos $k$-cortes $C$ tales que $\{x,y\} \in \delta C$.
\end{prop}

\begin{proof}
	Vamos a probar esta afirmación por inducción en $k$.
	
	En el caso base tenemos que $k=1$. 
	Si $\Gamma \setminus \{x,y\}$ es disconexo y separa al grafo en dos componentes conexas no vacías entonces tenemos un $1$-corte. 
	Caso contrario no existe un $1$-corte $C$ tal que $\delta C = \{x,y\}$.
	
	Para el paso inductivo supongamos que vale para $(k-1)$-cortes y veamos que vale para $k$-cortes.
	Sea $C$ un $k$-corte sobre $\Gamma$, por la observación \ref{obs_kCorte_restriccion} tenemos que $C$ es un $(k-1)$-corte para el grafo $\Gamma \setminus \{x,y\}$.
	Como suponemos que $k \ge 2$ entonces por el lema \ref{lema_corte_disconexo_vertices} el grafo $\Gamma \setminus \{x,y\}$ tiene que ser conexo caso contrario no existiría tal corte $C$.
	Esto nos dice que debe existir un camino $\alpha$ tal que $\alpha = (x_{0}, \dots, x_{n})$ con $x_{0} = x$ y $x_{n} = y$.
	Por la observación \ref{obs_camino_corte} tenemos que alguna de las aristas $\{x_{i},x_{i+1}\}$ es tal que pertenece a $\delta C$.
	Por cada una de estas aristas si aplicamos la hipótesis inductiva concluimos que existen finitos $(k-1)$-cortes que la contienen.
	Como el camino $\alpha$ tiene finitas aristas entonces tenemos finalmente que existen finitos $(k-1)$-cortes en $\Gamma \setminus \{x,y\}$ y así que existen finitos $k$-cortes en $\Gamma$ tales que contienen a $\{x,y\}$ tal como queríamos ver.
	
\end{proof}




\begin{coro}\label{lema_finitos_kcortes}
	Sea $\Gamma$ un grafo conexo y localmente finito.
	Sea $S \subset V(\Gamma)$ un conjunto finito y $k\ge 1$.
	Entonces existen finitos $k$-cortes $C$ con $\beta C \cap S \neq \emptyset$.
\end{coro}	

\begin{proof}
	Como el grafo $\Gamma$ es localmente finito entonces existen finitas aristas adyacentes a $S$.
	Aplicando el lema \ref{lema_aristas_finitos_kcortes} a cada una de estas finitas aristas obtenemos el resultado.	
\end{proof}
	





\begin{deff}
	Dos cortes $C,D \in V(\Gamma)$ están \emph{anidados} si vale alguna de las cuatro inclusiones $C \subseteq D, C \subseteq \ol D, \ol C \subseteq \ol D, \ol C \subseteq D$.
	Los conjuntos $C \cap D, C \cap \ol D, \ol C \cap D, \ol C \cap \ol D$ los llamamos las \emph{esquinas}.
	Dos esquinas $C \cap D$ y $\ol C \cap \ol D$ son \emph{esquinas opuestas}.
	Caso contrario diremos que son \emph{esquinas adyacentes}.
\end{deff}

\begin{lema}
	Dos cortes $C,D$ están anidados si y solo si alguna de las cuatro esquinas es vacía.
\end{lema}
\begin{proof}
	Para la ida supongamos que $C \subseteq D$ luego tiene que valer que $C \cap \ol D = \emptyset$.
	Para la vuelta si por ejemplo $C \cap D = \emptyset$ como vale que $C = (C \cap D) \cup (C \cap \ol D)$ esto nos dice que $C \subseteq \ol D$.
	
\end{proof}

\begin{lema}\label{lema_corte_borde_anidado}
	Sean $C,D$ cortes tales que $\beta C \subseteq D$ entonces $C$ y $D$ están anidados.
\end{lema}
\begin{proof}
	Si $C,D$ no estuvieran anidados tendríamos que $C \cap \ol D \neq \emptyset \neq \ol C \cap \ol D$.
	Esto nos diría que podríamos tomarnos un camino entre $c_0 \in C \cap \ol D$ y $c_1 \in \ol C \cap \ol D$ tal que esté contenido en $\ol D$.
	Esto es una contradicción porque este camino tendría que pasar por $\beta C$ y sabemos que $\beta C \subseteq D$.
	
\end{proof}


\begin{lema}
	Dado $C$ corte y $k \in \NN$ tenemos que 
	\[
	| \{  D : C, D \ \text{no están anidados y} \ D \ \text{es un k-corte}   \} | < \infty
	\]
\end{lema}
\begin{proof}
	Por el lema \ref{lema_corte_borde_anidado} no pueden haber cortes que no estén anidados $D$ tales que $\beta C \subseteq D$.
	Por el lema \ref{lema_finitos_kcortes} concluimos que hay finitos $k$-cortes $D$ tales que $\beta C \cap \beta D \neq \emptyset$ dado que $\beta C$ es un conjunto finito.
				
\end{proof}


\section{Grafos accesibles.}\label{secc_grafos_accesibles}

\begin{deff}
	Dado $\Gamma$ un grafo no dirigido conexo un camino simple $r$ es un \emph{rayo} si es un camino indexado por los naturales,
	\[
	r = (v_0, \dots, v_{n}, \dots).
	\]	
	Un camino simple $\alpha$ es un \emph{camino infinito} si está indexado por los enteros,
	\[
	\alpha = ( \dots, v_{-n}, \dots, v_0, \dots, v_{n}, \dots ).
	\]
\end{deff}		


En particular utilizaremos en varias ocasiones los siguientes resultados.


\begin{lema}[König]\label{lema_Konig}
	Si $\Gamma$ es un grafo conexo e infinito entonces existe un rayo $r$ sobre $\Gamma$.
\end{lema}
\begin{proof}
	Ver \cite[p.215]{diestel2005graph}.
\end{proof}


\begin{lema}\label{lema_camino_infinito_cortado}
	Sea $\Gamma$ un grafo no dirigido infinito y conexo y $C \subseteq V(\Gamma)$ un corte de $\Gamma$ de manera que $|C| = |\ol C| = \infty$ entonces existe $\alpha$ un camino infinito tal que 
	$|\alpha \cap C| = \infty = |\alpha \cap \ol C|$.
\end{lema}

\begin{proof}
	Fijemos dos vértices $u \in C$ y $v \in \ol C$ de manera que $\{u,v\} \in \delta C$.
	Vamos a construir dos rayos, uno en $C$ que empieza en $u$ y el otro en $\ol C$ empezando en $v$.
	Por \ref{lema_Konig} existe $r_{1}$ un rayo dentro de $C$.
	Si este rayo no pasa por el vértice $u$ nos tomamos un camino simple finito que una $u$ con el origen de este rayo.
	Consideremos entonces el primer vértice del rayo que interseca a este camino, sea este $u_{1}$.
	Si consideramos la concatenación de este camino finito hasta $u_{1}$ y luego la continuación del rayo a partir de $u_{1}$ obtenemos 
	un rayo $r_{2}$ de manera que $r_{2} = (u, \dots, u_{1}, \dots)$.
	Para obtener el otro rayo y así llegar a tener un camino infinito $\alpha$ hacemos lo mismo a partir del vértice $v \in \ol C$. 
	
	Para finalizar concatenamos estos dos rayos con la arista $\{u,v\}$ para así llegar a un camino infinito $\alpha$ que cumple lo pedido. 
	
\end{proof}


No necesariamente vale la vuelta.
Esto es que si tenemos un camino infinito $\alpha$ entonces existe un corte $C$ tal que cumple que $|\alpha \cap C| = \infty = |\alpha \cap \ol C|$.
Por ejemplo basta tomar la grilla \ref{ej:grilla_corte}.

Esto nos dice que dado un corte podemos tener caminos que son separados por el corte pero no vale que dado un camino tenemos que un corte lo separa.
Esto nos lleva a dar la siguiente definición.

\begin{deff}
	Sea $\Gamma$ un grafo no dirigido conexo y sea $\alpha$ un camino infinito en este grafo.
	Definimos el conjunto de \emph{cortes del camino} como 
	\[
	{\cal C}(\alpha) = \{ C \subset V(\Gamma) \mid  C \ \text{es un corte y} \ |\alpha \cap C| = \infty = |\alpha \cap \ol C| \}
	\] 
\end{deff}

\begin{obs}
	Dado $\Gamma$ un grafo no dirigido conexo y $\alpha$ un camino infinito en $\Gamma$ entonces: ${\cal C}(\alpha) \neq \emptyset$ si y solo sí existe un corte $C$ de manera que $\alpha \setminus \delta C$ tiene dos componentes conexas infinitas.
\end{obs}


\begin{deff}
	Dado $\Gamma$ un grafo infinito y conexo,
	definimos una relación sobre los rayos $ r,r' $ de $\Gamma$ de la siguiente manera:
	\[
	r \sim r' \iff \forall C \ \text{corte de $\Gamma$}. \ \ 
	(|r \cap C| = \infty \iff |r' \cap C| = \infty). 
	\]
	A las clases de equivalencia $\omega$ de esta relación se las llama \emph{ends} del grafo.
\end{deff}



\begin{figure}[H]
	\centering
	\begin{tikzpicture}[scale=0.85]
%		\draw[step=2.0,black,thin] (0,0) grid (4,4);
		\draw [ultra thick, label=above:$A$] (0,0)[bend right=8] to (3,3) to[bend right=8] (1,4);
		\draw [ultra thick, label=above:$A$] (2,0)[bend left=10] to (5,2) to[bend right=8] (10,0);		
		\draw [ultra thick, label=above:$A$] (10,2) to[bend left=20] (3.5,4.5);
		
		\draw [very thick, cadmiumred, <-, label=l:r] (10,1) to (3.5,3.25);
		\draw [very thick, cadmiumred, ->] (1.5,1)[bend right=10] to (5,2.5) to[bend right=8] (10,0.65);
		\draw [very thick, astral, ->] (1.75,1.25) to (4.5,3.25) to[bend left=8] (1.5,4.75);
		\draw [very thick, applegreen, ->] (5,2.95) to[bend right=10] (1,0);
		
		\draw [ultra thick, astral, densely dotted,bend right=10] (1,3.5) to (5.5,4);
		\draw [ultra thick, applegreen, densely dotted,bend left=10] (0,0.75) to (3.5,0.75);
		\draw [ultra thick, cadmiumred, densely dotted,bend left=10] (8-0.5,0.75-0.5) to (9.75,2.75);
		
		\node[label=right:{\Large{$\Gamma$}}] at (8,4.5) {};
		
	\end{tikzpicture}
	\caption{El grafo $\Gamma$ tiene tres ends. Estos son $\textcolor{cadmiumred}{\omega_{1}}, \textcolor{astral}{\omega_{2}}$ y $\textcolor{applegreen}{\omega_{3}}$.}
\end{figure}


Por resultados anteriores \ref{lema_camino_infinito_cortado} tenemos que un grafo $\Gamma$ conexo y localmente finito tiene más de un end si y solo sí existe un camino infinito $\alpha$ tal que ${\cal C}_{\alpha} \neq \emptyset$.

Equivalentemente diremos que un grafo $\Gamma$ localmente finito y conexo \emph{tiene más de un end} si existe un conjunto finito $S$ de vértices de manera que $\Gamma \setminus S$ tiene más de dos componentes conexas infinitas.

\TODO{Que quede claro que esto es una curiosidad?}
Un dato importante es que en el contexto de los grupos la cantidad de ends de sus grafos de Cayley resulta ser un invariante del grupo.
En \cite{stallings1971group} Stallings probó que dado un \fg con más de un end entonces se parte como el producto amalgamado o producto HNN sobre un subgrupo finito no trivial.
Si alguno de estos factores resulta tener más de un end este procedimiento continúa.
En el caso que el procedimiento termine Stallings denominó a estos grupos como grupos accesibles. 

En esta sección vamos a introducir la definición de los grafos accesibles que generaliza esta definición de Stallings.
Esta noción fue originalmente introducida en el trabajo \cite{thomassen1993vertex}.
En ese trabajo prueban que un grupo es accesible siguiendo la definición clásica si y solo sí todo grafo de Cayley de este grupo resulta ser accesible.
\begin{deff}
	Un grafo $\Gamma$ infinito y conexo es \emph{accesible} si existe $k \in \NN$ de manera que todo camino infinito $\alpha$ cumple que ${\cal C}(\alpha) = \emptyset$ o bien ${\cal C}(\alpha)$ contiene un $k$-corte.
\end{deff}
En particular de la definición obtenemos que los grafos accesibles tienen más de un end.
Equivalentemente podemos definir a un grafo como accesible si existe $k \in \NN$ tal que para todo par de ends distintos $\omega_1, \omega_2$ existe un k-corte $C$ de manera que separa a ambos ends.


Veamos que bajo esta definición los grafos con treewidth finito son accesibles.
Esto no nos va a asegurar que ser accesible en este sentido es una propiedad intrínseca de un grupo.
Para probar esto vamos a probar un lema un poco más técnico que generaliza esta afirmación.

\begin{lema}\label{lema_corte_treewidth}
	Sea $\Gamma$ un grafo infinito y conexo con treewidth finito y grado uniformemente acotado.
	Entonces existe $k \in \NN$ tal que:
	para todo $r$ rayo, todo $v_0 \in V(\Gamma)$ y todo $n \in \NN$ debe existir un $k-$corte $D$ que cumple las siguientes propiedades: $d(v_0,\ol D) \ge n, v_0 \in D, |\ol D \cap r| = \infty$. 
\end{lema}

\begin{proof}
	Sea $d$ una cota para el grado de los vértices del grafo $\Gamma$ y sea $m$ el treewidth del grafo.
	Propondremos $k = dm$ como la constante que buscamos.
	
	Sea $(T,f)$ descomposición en un árbol para $\Gamma$ tal que $bs(\Gamma,T,f) = m$.
	Sea $r$ un rayo sobre $\Gamma$ y $v_{0} \in V(\Gamma)$ un vértice.
	Consideremos $t_0 \in V(T)$ de manera que $v_0 \in f(t_{0})$.
	Si tenemos dos vértices $u_1,u_2 \in \ol{ f(t_{0})}$ tales que $u_1$ y $u_2$ están en bolsones $f(t_{1}), f(t_{2})$ de manera que $t_1$ y $t_2$ están en componentes conexas distintas de $T \setminus t_0$, luego como los árboles son únicamente geodésicos si usamos	la proposición \ref{prop-camino-desc} obtenemos que todo camino $\alpha$ que pase por $u_1$ y por $u_2$ debe pasar en algún momento por $f(t_{0})$.
	En particular estamos considerando un rayo $r$ que resulta ser un camino simple entonces dado que $|f({t_0})| < \infty$ obtenemos que debe haber una única componente conexa de $\ol{f({t_0})}$ tal que interseca infinitas veces al rayo $r$.
	Nombraremos a esta componente conexa $C_{t_0,r}$.
	Tenemos así que, nuevamente por \ref{prop-camino-desc}, exactamente una de las dos componentes conexas de $T \setminus \{t_{0}\}$ la unión de los bolsones de sus vértices contienen a $C_{t_{0},r}$.
	
	Sea ahora $t_{1} \in V(T)$ tal que es adyacente a $t_{0}$ y cumple que está en la componente conexa de $T \setminus \{t_{0}\}$ que cubre a $C_{t_{0},r}$.  
	Elegimos la componente conexa de $\ol{f(t_{1})}$ que interseca infinitas veces al rayo $r$.
	Repetimos este procedimiento para llegar a un rayo sobre el árbol
	$(t_0,t_1, \dots t_n \dots)$ y una sucesión de componentes conexas correspondientes $(C_{t_0,r}, C_{t_1,r}, \dots, C_{t_n, r}, \dots)$.
	
	
	
	Sea $n \in \NN$ arbitrario luego si elegimos $l \in \NN$ suficientemente grande podemos garantizar que
	\[
	f({t_l}) \cap B_n(v_0) = \emptyset.
	\]
	Esto porque por la proposición \ref{prop_tw_finitos_bolsones} tenemos que la descomposición se puede tomar de manera que cada vértice aparezca en finitos bolsones.
	Por otro lado como $\Gamma$ es localmente finito esto nos dice que la bola $|B_n(v_0)|$ es finita por lo tanto hay finitos bolsones en los cuales pueden aparecer los vértices de $B_{n}(v_{0})$.
	
	\begin{figure}[H]
		\centering
		\begin{tikzpicture}[scale=0.85]
			%		\draw[step=2.0,black,thin] (0,0) grid (4,4);
			\draw [ultra thick, label=above:$A$] (0,0)[bend right=8] to (3,3) to[bend right=8] (1,4);
			\draw [ultra thick, label=above:$A$] (2,-1)[] to (3.1,0.35) to  (5,2)  to[bend right=8] (10,0);		
			\draw [ultra thick, label=above:$A$] (10,2) to[bend left=20] (3.5,4.5);
			
			\draw [very thick, cadmiumred, <-, label=l:r] (10,1) to (3.5,3.25);
			%			\draw [very thick, cadmiumred, ->] (1.5,1)[bend right=10] to (5,2.5) to[bend right=8] (10,0.65);
			%			\draw [very thick, carrotorange, ->] (1.75,1.25) to (4.5,3.25) to[bend left=8] (1.5,4.75);
			%			\draw [very thick, applegreen, ->] (5,2.95) to[bend right=10] (1,0);
			
			%			\draw [ultra thick, carrotorange, densely dotted,bend right=10] (1,3.5) to (5.5,4);
			
			%			\draw [ultra thick, cadmiumred, densely dotted,bend left=10] (8-0.5,0.75-0.5) to (9.75,2.75);
			
			\node[circle,fill = black,,minimum size=.1cm, inner sep=0pt,label= left:$v_{0}$] at (2.65,1.5) {};
			\node[circle,fill = black,,minimum size=.1cm, inner sep=0pt,] at (4.05,2.5) {};
			\node[circle,fill = black,,minimum size=.1cm, inner sep=0pt,] at (4.05-0.2,2.5-0.2) {};
			\node[circle,fill = black,,minimum size=.1cm, inner sep=0pt,] at (4.05-0.4,2.5-0.4) {};
			\node[label=right:{{$f(t_{0})$}}] at (3.35,0.5) {};
			
			
			\draw [very thick, bend right=10] (1.70,1.60) to (3.1,0.35);
			\draw [very thick, bend right=10] (1.70+1,1.58+1.2) to (3.2+0.75,0.35+0.75);
			
			\draw [very thick, bend right=10] (5.5,1.65) to (5.75,3);
			\draw [very thick, bend right=10] (5.95,1.35) to (6.35,2.75);
			\node[label=right:{{$f(t_{l})$}}] at (5.65,3.25) {};
			
			\node[label=right:{\Large{$ C_{t_l, r} $}}] at (7.05,1.25) {};
			\node[cadmiumred, label=right:{\Large{\textcolor{cadmiumred}{$r$}}}] at (9.25,1.55) {};
		\end{tikzpicture}
		\caption{Tenemos que $f(t_{l})$ está lo suficientemente lejos de $v_{0}$.}
	\end{figure}
	
	Ahora vamos a buscar el corte $D$ que nos cumpla lo pedido.
	Para eso vamos a necesitar un conjunto conexo con complemento conexo y no vacío tal que su borde sea finito.
	Consideremos $D$ la componente conexa de $v_0$ en $\ol{ C_{t_l, r}}$.
	
	\begin{itemize}
		\item 	
		Primero veamos que $\ol D$ es conexo.
		Para ver esto notemos que $C_{t_{l},r} \subseteq \ol D$. 
		Si tenemos un elemento $u_1 \in \ol D$ tiene que estar conectado en $\Gamma$ con algún vértice de $C_{t_l, r}$.
		Esto se debe a que el grafo $\Gamma$ es conexo por lo que $u_{1}$ tiene al menos un camino que lo conecta con el conjunto de vértices $C_{t_{l},r}$.
		Si este camino interseca a $D$ tendríamos que $u_{1} \in D$ porque sería la misma componente conexa de $\ol{ C_{t_l, r}}$.
		De esta manera vimos que todo elemento de $\ol D$ está conectado con  $C_{t_l, r}$, que es conexo, evitando pasar por $D$ y así concluimos que $\ol D$ es conexo.
		\item 
		Tanto $D$ como $\ol D$ son no vacíos porque $C_{t_l, r}$ es infinito y está contenido en $\ol D$ y por otro lado porque $v_0 \in D$.
		\item
		Finalmente veamos que $D$ es un $k-$corte.
		Notemos que si $v \in D \cap \beta D$ luego tenemos que existe $u \in \ol D \cap \beta D$ tal que $\{u,v\} \in E(\Gamma)$.
%		Vamos a ver de acotar $|\beta D \cap D|$ y $|\beta D \cap \ol D|$.
		
		Por lo visto anteriormente todo camino que va de $\ol D$ a $D$ tiene que pasar necesariamente por $C_{t_l, r}$. 
		Entonces tenemos que $u \in C_{t_l, r}$.
		En este caso notemos que $v \in f{(t_l)}$ caso contrario tendríamos que $v \in C_{t_l, r}$ dado que esta es una de las componentes conexas de $\Gamma \setminus f{(t_l)}$ y asumimos que $v \in D \subseteq \ol{C_{t_{l},r}}$.
		De esta manera vimos que $\beta D \cap D \subseteq f{(t_l)}$.
		Como toda arista $\delta D$ tiene uno de sus vértices en $f(t_{l})$ y el grafo tiene grado acotado uniformemente esto nos dice que $|\delta D| \le dm = k$ tal como queríamos ver. 
		
	\end{itemize}
	
Finalmente notemos que por la construcción $v_{0} \in D$.
Como $d(v_{0}, f(t_{l})) \ge n$ dado que así elegimos a $t_{l}$ 
y vimos que $D \cap \delta D \subseteq f(t_{l})$
entonces se garantiza que $d(v_{0}, \ol D) \ge n$. 
Como $|r \cap C_{t_{l},r}| = \infty$ y $C_{t_{l},r} \subseteq \ol D$ obtenemos lo que queríamos probar.
	
\end{proof}

\begin{obs}\label{obs_corte_tw}
	En particular notemos que el resultado del lema anterior \ref{lema_corte_treewidth} sigue siendo válido si reemplazamos a $r$ el rayo por $\alpha$ un camino infinito.
	Esto porque a partir de un camino infinito podemos armarnos un rayo.
\end{obs}

Ahora sí con este resultado podemos probar que los grafos con treewidth finito y grado uniformemente acotado son accesibles.

\begin{teo}\label{teo_treewidth_fin_accesible}
	Sea $\Gamma$ un grafo no dirigido infinito y conexo con treewidth finito y con grado uniformemente acotado.
	Entonces $\Gamma$ es accesible.
\end{teo}
\begin{proof}
	Tomemos $\alpha$ un camino infinito tal que ${\cal C}(\alpha) \neq \emptyset$ y $C \in {\cal C}(\alpha)$.
	Veamos de construirnos un $k$-corte $D$ de manera que $D \in {\cal C}(\alpha)$.
	
	Fijemos $v_0  \in \beta C$ y consideremos el siguiente número natural
	\[
	n = \max \{ d(v_0,w) : w \in \beta C  \}.
	\]
	Por el lema anterior \ref{lema_corte_treewidth} y la observación \ref{obs_corte_tw} podemos ver que existe un $k$-corte $D$ de manera que $v_{0} \in D$, $d(v_0, \ol D) \ge n+1$ y que cumple que $|\ol D \cap \alpha| = \infty$.
	Por como tomamos a este corte $D$ cumple que $\beta C \subset D$ y así por el lema \ref{lema_corte_borde_anidado} tenemos que el corte $C$ cumple que 
	$C \subseteq D$ o bien $\ol C \subseteq D$.
	En cualquiera de estos dos casos como $|\alpha \cap C| = |\alpha \cap \ol  C| = \infty$ obtenemos así que $|D \cap \alpha|=\infty$ y así que $D \in {\cal C}(\alpha)$ tal como queríamos ver. 
	
\end{proof}


\section{Cortes óptimos.}\label{secc_copt}

Por el lema \ref{lema_corte_borde_anidado} tenemos que el siguiente número natural está bien definido.

\begin{deff}
	Dado $\Gamma$ grafo conexo y localmente finito, $C$ un corte y $k \in \NN$ constante definimos
	\[
	m_k(C) = | \{  D : C, D \ \text{no están anidados y} \ D \ \text{es un k-corte}   \} |. 
	\]
\end{deff}

\begin{deff}
	Dado $\alpha$ camino infinito definimos el conjunto de sus \emph{cortes mínimos} como
	\[
	\cam = \{  C \in \ca : |\delta C| \ \text{es mínimo}  \}
	\]
	
	Dado un grafo $\Gamma$ definimos el conjunto de sus \emph{cortes mínimos} como 
	\[
	{\cal C}_{\text{min}} = \bigcup \ \{ \ \cam : \alpha \ \text{es un camino infinito}  \}
	\]
\end{deff}




\begin{lema}\label{lema_esquinas_caminos}
	
	Sea $\Gamma$ un grafo infinito y conexo y sean $C,D$ cortes tales que $C \in {\cal C}(\alpha)$ y $D \in {\cal C}(\beta)$ para $\alpha$ y $\beta$ caminos infinitos.
	Entonces sucede alguno de estos dos casos:
	\begin{enumerate}
		\item $C \in {\cal C}(\beta)$ y $D \in {\cal C}(\alpha)$ y existen esquinas opuestas $E, E'$ tales que $|E \cap \alpha| = \infty = |E' \cap \alpha|.$
		
		\item Existen $E, E'$ esquinas opuestas tales que
		\[
		|E \cap \alpha| = |E' \cap \beta | = \infty. 
		\]
		
	\end{enumerate}
\end{lema}
\begin{proof}
	Dado $C \in {\cal C}(\alpha)$ y $D \in {\cal C}(\beta)$ tenemos que necesariamente no pueden haber dos esquinas adyacentes tales que ambas cumplan que intersecan finitamente a $\alpha$ y a $\beta$.
	Por otro lado deben existir al menos dos esquinas tales que intersecan infinitamente a $\alpha$ y otras dos tales que intersecan infinitamente a $\beta$. 
	
	Estas restricciones nos dejan en dos casos posibles.
	El primer caso que consideramos es el que tenemos dos esquinas $E, E'$ tales que $|E \cap \alpha| = |\ol{E} \cap \alpha| = \infty$ y $|E' \cap \beta| = |\ol{E'} \cap \beta|=\infty$.
	Esto nos dice que $C \in {\cal C}(\beta)$ y que $D \in {\cal C}(\alpha)$.
	En cualquier otro caso obtenemos que existen $E,E'$ esquinas opuestas de manera que $|E \cap \alpha| = |E' \cap \beta | = \infty$.
	
\end{proof}

\begin{figure}[H]
	
	\centering
	\begin{tikzpicture}[scale=0.85]
		\draw[step=2.0,black,thin] (0,0) grid (4,4);
		\draw [ultra thick,cadmiumred, label=above:$A$] (0,0) to[out=45,in=115] (1,1) to[out=-180+115,in=10] (1,4);
		\draw [ultra thick,astral] (2.2,0) to[out=105,in=15] (2,2) to[out=+105,in=-100] (4,4);
		
		\draw[step=2.0,black,thin] (8,0) grid (12,4);
		\draw [ultra thick,cadmiumred, label=above:$A$] (8,0) to[out=45,in=115] (9,1) to[out=-180+115,in=120] (12,3);
		\draw [ultra thick,astral, label=above:$A$] (8,0.75) to[out=45,in=115] (10,1) to[out=-180+115,in=-120] (11,4);
		
		\node (1) at (3.5,4.25)  {$C \cap D$};
		\node (2) at (3.5+8,4.25)  {$C \cap D$};
		\node (3) at (0.5,4.25)  {$\ol C \cap D$};
		\node (4) at (0.5+8,4.25)  {$\ol C \cap D$};
		\node (5) at (0.5,-0.25)  {$\ol C \cap \ol D$};
		\node (6) at (0.5+8,-0.25)  {$\ol C \cap \ol D$};
		\node (7) at (3.5,-0.25)  {$C \cap \ol D$};
		\node (8) at (3.5+8,-0.25)  {$ C \cap \ol D$};
		
		\node (9) at (1.25,3.5) {\textcolor{cadmiumred}{$\alpha$}};
		\node (9) at (11.75,2.75) {\textcolor{cadmiumred}{$\alpha$}};
		
		\node (9) at (3.25,3.5) {\textcolor{astral}{$\beta$}};
		\node (9) at (3.25+7.25,3.5) {\textcolor{astral}{$\beta$}};
		
	\end{tikzpicture}
	\caption{\small{En la izquierda tenemos que si tomamos $E = \ol C \cap D$ y $E' = C \cap \ol D$ estas esquinas cumplen que $|E \cap \alpha| = \infty$ y que $|E' \cap \beta | = \infty$.
			En la derecha tenemos que nos alcanza con tomar $E = \ol C \cap \ol D$ y $E' = C \cap D$.
			En este caso $C \in {\cal{C}}(\beta)$ y $D \in {\cal{C}}(\alpha)$.}}
\end{figure}


\begin{prop}\label{prop_esquinas_minimales}
	Sea $\Gamma$ un grafo infinito, conexo, localmente finito y accesible.
	Entonces para todo par de cortes minimales $C,D \in \cmin$ existen dos esquinas opuestas $E,E'$ tales que $E,E' \in \cmin$.
\end{prop}

\begin{proof}
	Sea $C \in \cmin(\alpha)$ y $D \in \cmin(\beta)$ luego por el lema \ref{lema_esquinas_caminos} podemos separar en dos casos.
	\begin{enumerate}
		\item Si $C \in {\cal C}(\beta)$ y $D \in {\cal C}(\alpha)$ entonces esto nos dice que $|\delta C| = |\delta D|$ dado que ambos cortes son minimales.
		Esto implica que $D \in \cam$. 
		En este caso simplemente nos olvidamos del camino $\beta$ y consideramos $C,D \in \cam$.
		Luego tenemos que existen dos esquinas opuestas $E,E'$ tales que $|E \cap \alpha| = \infty$ y $|E' \cap \alpha| = \infty$.
		\item Existen esquinas $E,E'$ tales que $|E \cap \alpha| = \infty = |E' \cap \beta|$.
	\end{enumerate}	
	
	Si es necesario renombramos a los cortes para que $E = C \cap D$ y $E' = \ol C \cap \ol D$.
	
	
	Debemos ver que estas esquinas nos sirven pero en principio no sabemos si son cortes siquiera.
	Sea $F$ componente conexa de $E$ tal que $|F \cap \alpha| = \infty$.
	Notemos que $\ol C \cup \ol D \subseteq \ol F$ por lo tanto tenemos que $|\ol F \cap \alpha|=\infty$.
	Más aún podemos ver que $\ol F$ es conexo.
	Para eso consideremos $v,w \in \ol F$.
	Si tanto $v,w \in \ol C \cup \ol D$ entonces como este conjunto es un conexo y está contenido en $\ol F$ no hay nada para hacer.
	En el caso que $v \in (C \cap D) \setminus F$ luego consideramos $\gamma$ un camino en $\Gamma$ tal que los una.
	Sea $u$ el primer vértice tal que $u \in E \cap \delta E$ entonces necesariamente el camino $(v, \dots, u)$ está contenido en la componente conexa de $v$ en $E$ y esta componente conexa no es $F$. 
	Como $u \in \delta E$ luego existe $u' \in \ol C \cup \ol D$ tal que $u'$ es adyacente a $u$. 
	Luego como $\ol C \cup \ol D$ es conexo tenemos un camino entre $u'$ y $w$ contenido en $\ol F$ y así cerramos nuestra demostración.
	Similarmente encontramos un conjunto de vértices $F'$ que cumple esto para $\beta$.
	Notemos que $|\delta F| < \infty$, esto se debe a que si $\{ v,w \} \in \delta F$ luego necesariamente $\{v,w\} \in \delta E$.
	Por otro lado $\delta E = \delta (C \cap D)  \subseteq \delta C \cup \delta D$ y $|\delta C| + |\delta D| < \infty$. 
	Con esto probamos que $F \in {\cal C}(\alpha)$ y que $F' \in {\cal C}(\beta)$.
	
	%	Si ambos vértices están en un mismo corte ya está porque es conexo por lo tanto tenemos un camino que los une y no toca a $F$.
	%	Si no están en un mismo corte, como podría ser que $v \in C \cap \ol D$ y $w \in \ol C \cap D$ luego usando la conexión de $\ol D$ y de $\ol C$ podemos unir ambos con algún vértice en $\ol C \cap \ol D$ y así vemos que es conexo.
	%	Con esto concluimos que $F$ es un corte que separa en dos partes a $\alpha$.
	%	Hacemos algo similar para construirnos un corte $F' \subseteq E'$ que separe en dos partes a $\beta$.
	
	Veamos ahora que $F$ y que $F'$ son cortes minimales.
	Como $\delta E \subseteq \delta C \cup \delta D$ y $\delta E' \subseteq \delta C \cup \delta D$ entonces $\delta E \cup \delta E' \subseteq \delta C \cup \delta D$.
	Consideramos las siguientes desigualdades:
	\begin{align*}
		|\delta E \cup \delta E'| & \le |\delta C \cup \delta D| \\
		|\delta E| + |\delta E'| - |\delta (E) \cap \delta (E')| & \le |\delta C| + |\delta D| - |\delta C \cap \delta D|
	\end{align*}
	si $\{x,y\} \in \delta E \cap \delta E'$ esto nos dice que $x \in E$ e $y \in \delta E'$ por lo que $x \in C \cap D$ e $y \in \ol C \cap \ol D$ lo que implica que $\{x,y\} \in \delta C \cap \delta D$.
	Entonces tenemos que $|\delta (E) \cap \delta (E')| \le |\delta C \cap \delta D|$ por lo tanto vale la siguiente desigualdad
	\[
	|\delta E| + |\delta E'| \le |\delta C| + |\delta D|.
	\]	
	Por otro lado como $F$ y $F'$ son cortes tales que $|\delta F| \le |\delta E|$ y $|\delta F'| \le |\delta E'|$ luego obtenemos que
	\[
	|\delta F| + |\delta F'| \le |\delta C| + |\delta D|
	\] 
	y dado que $C,D$ son cortes minimales obtenemos así que $|\delta C| \le |\delta F|$ y que $|\delta D| \le |\delta F'|$ concluyendo que $|\delta F | = |\delta C|, |\delta F'| = |\delta D|$ y por lo tanto que $F,F'$ son cortes minimales.
	Por el mismo razonamiento obtenemos que $|\delta E| = |\delta C|$ y que $|\delta E'| = |\delta D|$.
	
	Probemos ahora que $E=F$, equivalentemente que $E$ es conexo y que por lo tanto esto implicaría que $E \in \cam$.
	Para esto notemos que si $v \in E \setminus F$ luego necesariamente por el mismo argumento anterior que usamos para probar que $\ol F$ es conexo tenemos que debe existir $u$ tal que $u \in \ol C \cup \ol D$ y conectado con $v' \in E \setminus F$.
	Si existiera esta arista $\{v',u \}$ luego esto nos diría que $\{v',u \} \in \delta E \setminus \delta F$ lo que contradice que $|\delta E| = |\delta F| = |\delta C|$.
	Similarmente podemos probar que $E' = F'$.
	Concluimos así que $E, E' \in \cmin$.
\end{proof}



Así como definimos cortes mínimos ahora vamos a definir cortes óptimos que serán un subconjunto de ellos.
Consideremos ahora que el grafo $\Gamma$ es accesible y su constante es $k$.
En este caso notaremos para cada corte $C$ el siguiente valor
\[
m(C) = m_k(C).
\]


\begin{deff}
	Dado $\Gamma$ grafo infinito, conexo, localmente finito y accesible y $\alpha$ camino infinito en $\Gamma$ sea
	\[
		m_\alpha = \min \{ m(C) : C \in \cam \}.
	\]
	De esta manera los \emph{cortes óptimos} del camino $\alpha$ serán
	\[
		\copta = \{ C \in \cam : m(C) = m_\alpha  \}
	\]
	y el \emph{conjunto de los cortes óptimos} análogamente será el conjunto que contenga a todos ellos
	\[
		\copt = \bigcup \ \{ \copta : \alpha \ \text{es un camino infinito}  \}.
	\]
\end{deff}








El siguiente es el resultado central y más importante de los cortes óptimos.

\begin{teo}\label{teo_copt_anidados}
	Sea $\Gamma$ un grafo infinito, conexo, localmente finito y accesible.
	Entonces todo par de cortes óptimos $C,D \in \copt$ está anidado.
\end{teo}
\begin{proof}
	Dado que $C, D \in \copt$ entonces tenemos dos caminos $\alpha$, $\beta$ tales que $C \in \copt(\alpha)$ y $D \in \copt(\beta)$.
	Supongamos sin pérdida de generalidad que $m_{\alpha} \ge m_{\beta}$.
	
	La idea de la demostración es suponer que estos dos cortes no están anidados y llegar a una contradicción.
	Para esto queremos encontrar cortes $E, E' \in \copt$ de manera que estén anidados con $C$ y con $D$ y que hagan valer la siguiente desigualdad:
	\[
	m(E) + m(E') < m(C) + m(D).
	\]   
	La validez de esta desigualdad nos diría que alguno de los dos cortes $C,D$ no es óptimo tal como supusimos.
	
	Por la proposición \ref{prop_esquinas_minimales} tenemos que existen dos esquinas opuestas $ E, E' $ tales que son minimales.	
	Si es necesario renombramos a los cortes para que $E = C \cap D$ y $E' = \ol C \cap \ol D$.
	Queremos ver ahora que son $E,E'$ son cortes óptimos.	
	Para esto probaremos primero que si $F$ es un corte anidado con $C$ o con $D$ entonces está anidado con $E$ o con $E'$.
	Separamos en dos casos que son simétricos.
	Si $F \cap C = \emptyset$ entonces $F \cap E= \emptyset$ y similarmente si $F \cap \ol C = \emptyset$ entonces $F \cap E' = \emptyset$.
	Similarmente para $D$.
	Esto nos va a decir que todos los cortes que estén anidados con alguno de los dos cortes originales $C,D$ va a estar anidado con alguno de los dos propuestos $E,E'$.
	
	Por ahora tenemos la siguiente desigualdad de conjuntos
	\[
		|\{ F : F \ \text{no está anidado con} \ E \ \text{o con} \  E' \}| \le |\{ F : F \  \text{no está anidado con} \ C \ \text{o con} \ D \}|
	\]
	notemos que esta desigualdad la podemos escribir como:
	\begin{align*}
		m(E) + m(E') - &|\{ F : F \ \text{no está anidado con} \ E \ \text{ni con} \ E' \}| \le \\
		&m(C) + m(D) - |\{ F : F \ \text{no está anidado con} \ C \ \text{ni con} \ D \}|.
	\end{align*}
	Ahora vamos a probar que 
	\[
		|\{ F : F \ \text{no está anidado con} \ E \ \text{ni con} \ E' \}| \le |\{ F : F \ \text{no está anidado con} \ C \ \text{ni con} \ D \}|.
	\]
	Tenemos cuatro casos a analizar.
	\begin{enumerate}
		\item $C \subseteq F$ y $D \subseteq \ol F$.
		Este caso está descartado porque tendríamos que $E=\emptyset$ y sabemos que $E$ es infinito. 
		\item $\ol C \subseteq F$ y $\ol D \subseteq \ol F$.
		Similar al caso anterior tendríamos que $E' = \emptyset$ y es una contradicción.
		\item $\ol C \subseteq F$ y $D \subseteq \ol F$.
		Esto nos dice que $E \subseteq \ol F$ y que $E' \subseteq F$.
		\item $ C \subseteq F$ y $D \subseteq \ol F$.
		Esto nos dice que $E \subseteq F$ y que $E' \subseteq \ol F$.
	\end{enumerate}
	Por lo tanto $|\{ F : F \ \text{no está anidado con} \ E \ \text{ni con} \ E' \}| \le |\{ F : F \ \text{no está anidado con} \ C \ \text{ni con} \ D \}|$ y esto nos dice que:
	\[
		m(E) + m(E') \le m(C) + m(D)
	\]
	pero nosotros queremos ver que es estricta para terminar de probar este teorema.
	Para eso notemos que $C$ está anidado con $E$ puesto que $\ol C \cap E = \ol C \cap (C \cap D) = \emptyset$.
	Idénticamente vemos que está anidado con $E'$.
	Pero por suposición tenemos que $C$ y $D$ no están anidados así que tenemos una desigualdad estricta y por lo tanto llegamos a una contradicción (que alguno de los dos cortes $C$ o $D$ no son óptimos) de suponer que los cortes $C,D$ no están anidados.
\end{proof}

\section{Árbol de estructura.}\label{secc_arbol_estr}


Los árboles de estructura fueron introducidos y demostradas todas las propiedades que describimos en esta sección en los trabajos de Dunwoody \cite{dunwoody1979accessibility} y \cite{dunwoody1982cutting}.
La teoría se profundizó enormemente en \cite{dicks1989groups} codificando la construcción en un problema de un anillo booleano asociado al grafo. 

Así como en la sección anterior, todos los grafos que vamos a considerar a partir de ahora son accesibles.
Vamos a construir un árbol a partir de los cortes óptimos.
Para esto lo primero que vamos a ver es que los cortes óptimos forman un poset localmente finito respecto a la inclusión.

\begin{lema}\label{lema_intermedios}
	Sea $\Gamma$ un grafo infinito, conexo, localmente finito y accesible.
	Si $C,D \in \copt$ luego 
	\[
	|\{ E \in \copt : C \subseteq E \subseteq D \}| < \infty
	\]
\end{lema}
\begin{proof}
	Tomemos un camino $\gamma$ tal que salga de $c \in C$ y termine en $d \in \ol D$.
	
	Si $E$ es un corte luego tiene que separar a $\gamma$ caso contrario tendríamos que $d \in E \cap \ol D$.
	Como el grafo que estamos considerando es accesible sabemos que $E \in \copt$ implica que $E$ es un $k$-corte.
	Si ahora usamos el lema \ref{lema_finitos_kcortes} con el conjunto finito $\gamma$ esto nos dice que solo existen finitos cortes $E$ que cumplen lo pedido.
\end{proof}


Ahora vamos a definir la siguiente relación sobre los cortes óptimos.

\begin{deff}
	Dos cortes $C \sim D \in \copt$ si y solo sí
	\[
		C = D \ \text{o bien} \ \ol C \subsetneq D \ \text{y} \ \forall E \in \copt : \ol C \subsetneq E \subseteq D \implies D = E
	\]
\end{deff}

\begin{obs}\label{obs_copt_igualdad}
	Sean $C,D \in \copt$ cortes óptimos tales que $C \subseteq D$ y $C \sim D$ entonces necesariamente $C = D$.
\end{obs}

\begin{obs}\label{obs_cortes_maximal}
	Dados cortes $C,D \in \copt$ tales que $C \subsetneq D$ luego por el lema \ref{lema_intermedios} debe existir corte $E$ tal que $C \subseteq E \subsetneq D$ y maximal con respecto a esta propiedad. 
	Esto implica que en particular vale que $\ol E \sim D$.
\end{obs}

\begin{prop}
	La relación $(\copt, \sim)$ es de equivalencia.
\end{prop}
\begin{proof}
	Por como la definimos es clara que es reflexiva.
	Para ver que es simétrica notemos que si $C \sim D$ y son cortes distintos entonces  
	\[
		\ol C \subsetneq D \implies \ol D \subsetneq C
	\]
	por lo tanto volviendo a tomar complemento obtenemos que
	\[
(	\forall E \in \copt : \ol C \subsetneq E \subseteq D \implies D = E) \implies 
	(\forall E \in \copt : \ol D \subsetneq E \subseteq C \implies C = E).
	\]
	
	Finalmente nos queda ver que la relación resulta ser transitiva.
	Esto es que para ciertos cortes óptimos tenemos que $C \sim D$, $D \sim E$ y queremos ver que $C \sim E$.
	Una primera observación que podemos hacer es que si partimos que $C \neq D \neq E$ entonces dado que $C \sim D$ luego $\ol D \subseteq C$ y similarmente $\ol D \subseteq E$ por lo tanto tenemos que $\ol D \subseteq C \cap D$ y así que $C \cap D \neq \emptyset$.

	Por el teorema \ref{teo_copt_anidados} tenemos que $C,E$ están anidados.
	Veamos que sucede en los cuatro posibles casos.
	\begin{itemize}
		\item $C \subseteq E$. 
		Como $D \sim C$ tenemos que $ \ol D \subseteq C $ y como $D \sim E$ entonces para todo $\forall F \in \copt : \ol D \subsetneq F \subseteq E \implies F = E)$. 
		Si tomamos $F = C$ luego obtenemos que $C = E$ lo que nos dice que $C \sim E$ tal como queríamos ver.
		
		
		\item $E \subseteq C$ análogo al caso anterior.
		\item $E \subseteq \ol C$.
		Tenemos que por la observación anterior que $C \cap E \neq \emptyset$ contradiciendo que $E \subseteq \ol C$.
		\item $\ol C \subseteq E$.
		Como $C \cap E \neq \emptyset$ esto nos dice que $\ol C \subsetneq E$ por lo tanto tenemos que si $F \in \copt$ luego para todo $F \in \copt$ tal que $\ol C \subsetneq F \subseteq E$ queremos ver que $F=E$.
		Lo separamos en cuatro casos nuevamente gracias al teorema \ref{teo_copt_anidados} dado que $F$ debe estar anidado con $D$.
		\begin{itemize}
			\item $D \subseteq F$.
			Esto nos dice que $D \subseteq E$ entonces por la observación \ref{obs_copt_igualdad} obtenemos que $D = E$ y así $C \sim E$.
			
			\item $F \subseteq D$.
			Tenemos que $C \sim D$ entonces $\ol C \subsetneq F \subset D$ implica que $F = D$ y así tenemos que $D \subseteq E$ y por lo tanto nuevamente por la observación \ref{obs_copt_igualdad} que $C \sim E$.
			
			\item $\ol D \subseteq F$. 
			Tenemos que dado que $D \sim E$ entonces $\ol D \subsetneq F \subseteq E$ implica que $F = E$.
			
			\item $F \subseteq \ol D$.
			Entonces tenemos que $\ol C \subseteq F \subseteq \ol D$ entonces por la observación \ref{obs_copt_igualdad} concluimos que $\ol C = \ol D$ lo que implica que $C = D$.
		\end{itemize}
	\end{itemize}
\end{proof}

\begin{ej}
	Sea $G = \ZZ / 2\ZZ \ast \ZZ / 3\ZZ $ con presentación $G \simeq \langle a,b \mid a^2, b^3 \rangle$.
	Sea el grafo de Cayley  $\text{Cay}(G, \{a,b,b^2\})$.
	
	Notemos que toda arista de la pinta $\{g,ga\} \in E(\Gamma)$ define un corte con $\delta C = \{g,ga\}$.
	Es decir todas las aristas con etiqueta $a$ nos definen un $1$-corte.
	Este corte naturalmente es minimal porque no podría ser que $\delta C$ sea menor. Vale la siguiente igualdad:
	\[
	\cmin = \{ C \ \text{corte}  \mid \exists g \in G, \ \delta C = \{g,ga\}  \}. 
	\]
	Notemos que dado $C \in \cmin$ luego necesariamente tenemos que $m(C) = 0$. 
	Esto se debe a que si $D \in \cmin$ luego $\delta D \in C$ o bien $\delta D \in \ol C$ esto nos dice que $D \subseteq C$ o bien que $\ol D \subseteq C$ y así que $C$ y $D$ están anidados.
	Concluimos que $\cmin = \copt$.
	
	Veamos ahora si fijamos $C \in \copt$ como es la clase de equivalencia $[C]$.
	Para esto notemos que $|[C]| = 3.$
	Esto se debe a que si $\delta C = \{ g,ga \}$ entonces los siguientes cortes $D,E \in \copt$ son tales que $C \sim D \sim E$.
	
	\begin{figure}[H]
		\centering
		\begin{tikzpicture}
			[scale=0.8, V/.style = {circle, draw,align= center, minimum size=0.25cm,
				minimum size=0.75em,inner sep=0.75,
				fill=carrotorange!15,font=\scriptsize	},fill fraction/.style={path picture={
					\fill[#1] 
					(path picture bounding box.south) rectangle
					(path picture bounding box.north west);
			}},
			fill fraction/.default=astral!90
			]
			
			\begin{scope}[nodes=V,xshift=4.5cm, yshift=-4cm]
				\node (1) at (0,0) {};
				\node (2) at (2,0)  {};
				\node (3) at (-2,2)     {};
				\node (4) at (-2,-2)    {};
				\node (5) at (-3,-3)      {};
				\node (6) at (-3,3)     {};
				
				
			\end{scope}
			
			\node[right=.1cm of 2] {$\dots$};
			\node[left=.1cm of 6] {$\dots$};
			\node[left=.1cm of 5] {$\dots$};
			%			\node[left=.1cm of babb] {$\dots$};
			%			\node[left=.1cm of bbabb] {$\dots$};
			%			\node[left=.1cm of bbab] {$\dots$};
			%			\node[right=.1cm of abab] {$\dots$};
			%			\node[right=.1cm of ababb] {$\dots$};
			
			
			\draw   (1)  edge[-] (2);
			\draw   (1)  edge[-] (3);
			\draw   (1)  edge[-] (4);
			\draw   (3)  edge[-] (6);
			\draw   (3)  edge[-] (4);
			\draw   (4)  edge[-] (5);
			
			\draw [ultra thick, cadmiumred, bend left=20] (5.25,-3) to (5.25,-5);
			\node at (4.85,-3) {$C$};
			
			\draw [ultra thick, cadmiumred, bend left=10] (1.5,-2.5) to (1.5+1.5,-2.5+1.5);
			\node at (1.95,-2.75) {$D$};
			
			\draw [ultra thick, cadmiumred, bend left=10] (3.5-0.4,-7-0.4) to (1.75-0.4,-5.65-0.4);
			\node at (3.5-0.7,-7+0.4) {$E$};
			
		\end{tikzpicture}
		\caption{\small{Los tres cortes $C,D,E$ están en la misma clase de equivalencia.}}
	\end{figure}

	Si $F$ es otro corte tal que $\ol F \subsetneq C$ tenemos que necesariamente vale que $\ol F \subsetneq \ol D$ o bien $\ol F \subsetneq \ol E$.
\end{ej}

\begin{deff}
	Sea $\Gamma$ un grafo conexo, accesible y localmente finito y sea $\copt$ el conjunto de sus cortes óptimos.
	El \emph{árbol de estructura} $T(\copt)$ de un grafo $\Gamma$ es el siguiente grafo dirigido:
	\begin{align*}
		V(T(\copt)) &= \{ [C] : C \in \copt \} \\
		E(T(\copt)) &= \{ \{ [C], [\ol C] \} : C \in \copt   \}
	\end{align*}
\end{deff}

En particular notemos que está bien definido como grafo no dirigido dado que para todo $C \in \copt$ tenemos que $C \nsim \ol C$ por lo que no tiene bucles.
Por otro lado si $C \sim D$ entonces no puede pasar que $\ol C \sim \ol D$ caso contrario obtendríamos que $\ol C \subsetneq D$ y que $D \subsetneq \ol C$ por lo tanto $\ol C \subsetneq \ol C$ y esto es una contradicción.

\begin{obs}
	Por nuestra construcción el árbol de estructura no tiene porqué ser localmente finito.
	El grado de cada vértice $[C] \in V(T(\copt))$ está acotado por el órden de la clase de equivalencia de $C$ y éste podría ser infinito.
\end{obs}


\begin{prop}
	El grafo $T(\copt)$ es un árbol.
\end{prop}

\begin{proof}
	Primero veamos que no tiene ciclos.
	Supongamos que $\gamma$	es un ciclo
	\[
		\gamma = ([C_0], [C_1], [C_2], \dots [C_{n-1}], [C_0])
	\]
	donde $\ol C_{i} \sim C_{i+1}$ para todo $i=1 \dots n-1$.
	Notemos que $\ol C_{i-1} \neq C_{i}$ para todo $i=1 \dots n-1$ sino tendríamos que $C_{i-1} = \ol{ C_{i}}$ y esto implicaría que $C_{i-1} = \ol{C_{i}} \sim C_{i-1}$
	y esto diría que el camino $\gamma$ no es simple.
	
	De esta manera tenemos una cadena de inclusiones
	\begin{equation}\label{eq:inclusiones}
			C_0 \subsetneq C_1 \subsetneq \dots  \subsetneq C_{n-1}
	\end{equation}
	tal que $\ol{ C_{n-1}} \sim C_0$.
	Veamos que no puede haber una arista $\{[{ C_{n-1}}], [C_0]\}$.
	Si la hubiera tendría que valer alguna de estas dos condiciones:
	\begin{enumerate}
		\item $\ol{C_{n-1}} = C_{0}$. 
		Esto contradice lo visto en \ref{eq:inclusiones} dado que en tal caso
		 llegamos a que $\ol{C_{n-1}} \subsetneq C_{n-1}$.
		
		\item ${C_{n-1}} \subsetneq C_{0}$ pero esto nos da un absurdo porque en tal caso por \ref{eq:inclusiones} obtenemos que $C_{0} \subsetneq C_{0}$.
	\end{enumerate}
	Por lo tanto el grafo resulta ser acíclico.
	
	Veamos ahora que es conexo.
	Sean $[C], [D] \in V(T(\copt))$, vamos a construir un camino entre ellos.
	Por el teorema \ref{teo_copt_anidados} tenemos que necesariamente $C,D \in \copt$ están anidados.
	Sin pérdida de generalidad podemos suponer que $C \subseteq D$.
	Por el lema \ref{lema_intermedios} sabemos que hay finitos cortes $E$ intermedios.
	Tomemos entonces una sucesión creciente no refinable de cortes de manera que  
	\[
		C=C_0 \subsetneq C_1 \subsetneq \dots \subsetneq C_n = D
	\]
	luego por como tomamos estos cortes obtenemos que $\ol C_i \sim C_{i+1}$ por lo tanto obtenemos el siguiente camino $\gamma = ([C],[C_1], \dots, [C_{n-1}],[D])$ en el grafo $T(\copt)$ probando así que es conexo.	
\end{proof}





\section{Acciones sobre el árbol de estructura.}\label{secc_accion_arbol_estr}

En esta sección vamos a ver que los cortes óptimos no solamente están anidados entre sí sino que si un grupo $G$ actúa sobre el grafo $\Gamma$ luego actúa sobre estos cortes óptimos.
Esta acción se traslada naturalmente a una acción sobre el árbol de estructura.



\begin{prop}
	Sea $\Gamma$ un grafo accesible, conexo y localmente finito tal que $\aut(\Gamma)$ actúa con finitas órbitas sobre este grafo.
	Entonces $\aut(\Gamma)$ actúa sobre $\copt$.
\end{prop}

\begin{proof}
	Para eso notemos que si $C$ es un corte luego si $\varphi \in \aut(\Gamma)$ tenemos que $\varphi(C)$ es conexo y no vacío. 
	Por otro lado al ser un morfismo del grafo $\Gamma$ obtenemos que $\ol{\varphi(C)} = \varphi(\ol C)$ por lo tanto su complemento sigue siendo conexo.
	Al ser un automorfismo tenemos que $\varphi(\beta C) = \beta (\varphi C)$ y por este motivo tenemos que $|\varphi (\beta C)| = |\beta (\varphi C)|$ y esto nos dice que manda $k$-cortes en $k$-cortes.
	
	
	Veamos que manda un corte minimal en uno minimal. 
	Para eso si $C \in {{\cal C}}_{\text{min}}(\alpha)$ para cierto camino infinito $\alpha$ luego notemos que $\varphi(\alpha)$ es otro camino infinito y por ser un automorfismo obtenemos que $\varphi(C) \in {\cal C}({\varphi(\alpha)})$ y que $|\delta(\varphi(C))| = |\delta C|$.
	De esta manera $C \in \cam \iff \varphi(C) \in \cmin(\varphi(\alpha))$.
	
	
	Veamos ahora que manda cortes óptimos en cortes óptimos.
	Sean $C,D \in {\cal C}_{\text{min}}$ cortes que no están anidados luego necesariamente $\varphi(C)$ y $\varphi(D)$ no están anidados tampoco.
	Caso contrario si lo estuvieran supongamos que $\varphi(C) \cap \varphi(D) = \emptyset$ luego tendríamos que dado que $\varphi \in \aut(\Gamma)$ esto dice que preserva las intersecciones y tiene una inversa por lo que $C \cap D = \emptyset$ y esto contradice que $C, D$ no están anidados.
	Esto nos dice que $\aut(\Gamma)$ actúa sobre los cortes óptimos porque dado $C \in \copta$ luego $\varphi(C) \in {\cal C}_{\text{opt}}(\varphi(\alpha))$.
	
\end{proof}

Más aún podemos extender esta acción al árbol de estructura.

\begin{coro}
	Sea $\Gamma$ un grafo accesible, conexo y localmente finito tal que $\aut(\Gamma)$ actúa con finitas órbitas sobre este grafo.
	Entonces $\aut(\Gamma)$ actúa sobre $T(\copt)$.
\end{coro}

\begin{proof}
	Sean $C \sim D \in \copt$ cortes óptimos relacionados probemos que para todo $\varphi \in \aut(\Gamma)$ vale que $\varphi(C) \sim \varphi(D)$.
	Para eso notemos que si para todo $E \in \copt$ tenemos que 
	$\ol C \subsetneq E \subset D$ implica que $E = D$ luego
	\[
	\ol {\varphi(C)} \subsetneq E \subset \varphi(D) \iff \ol C \subsetneq \varphi^{-1}(E) \subset D
	\]  
	donde usamos que $\varphi$ es un automorfismo y que actúa sobre los cortes óptimos.
	Entonces como $C \sim D$ luego $\varphi^{-1}(E) = D$ lo que implica que $E = \varphi(D)$ tal como queríamos ver.
	Esto nos dice que la acción dada por $\varphi([C]) = [\varphi (C)]$ está bien definida y por lo tanto actúa sobre los vértices de $T(\copt)$.
	Para ver que actúa sobre el árbol notemos que 
	\[
		\varphi(\{[C],[\ol C]\}) = (\{[\varphi (C)], [\varphi (\ol C)]\}) = (\{[\varphi (C)], [\ol{\varphi (C)}]\})
	\]
	donde usamos que $\varphi(\ol{C}) = \ol{\varphi(C)}$ dado que $\varphi \in \aut(\Gamma)$.
	De esta manera vimos que $\aut(\Gamma)$ actúa sobre $T(\copt)$.
	
\end{proof}


\begin{lema}\label{lema_finitas_orbitas}
	Sea $\Gamma$ un grafo accesible, conexo y localmente finito.
	Si $\aut(\Gamma)$ actúa con finitas órbitas sobre $\Gamma$ luego actúa con finitas órbitas sobre $\copt$ y sobre el árbol de estructura $T(\copt)$.
\end{lema} 
\begin{proof}
	Veamos primero que actúa con finitas órbitas sobre $\copt$.
	Para eso tenemos que existe $A \in V(\Gamma)$ dominio fundamental finito de la acción de $\aut(\Gamma)$ sobre $\Gamma$.
	Por lo tanto para todo corte $C \in \copt$ podemos tomar otro corte $D \in \copt$ de manera que para cierta $\varphi \in \aut(\Gamma)$ vale que $C  = \varphi(D)$ y tal que $\delta D \cap A \neq \emptyset$.
	Entonces la acción de $\aut(\Gamma)$ sobre $\copt$ tiene el siguiente dominio fundamental
	\[
		B = \{  D \in \copt \mid \delta D \cap A \neq \emptyset \}.
	\]
	
	Como el grafo $\Gamma$ es accesible existe $k$ de manera que todo corte óptimo $D$ es un $k$-corte.
	Por el resultado \ref{lema_finitos_kcortes} tenemos que $|B| < \infty$ y así vimos que $\aut(\Gamma)$ actúa con finitas órbitas sobre $\copt$. 
	
	Para ver que son finitas las órbitas sobre el árbol de estructura notemos que al ser $V(T(\copt))$ isomorfo a un cociente de $\copt$ luego tenemos que en particular las órbitas sobre el árbol de estructura también tienen que ser finitas.
	
\end{proof}

\begin{obs}
	Un caso particular que nos interesa es cuando el grafo $\Gamma$ es el grafo de Cayley para cierto \fg $G$.
	En este caso tenemos que $G$ es un subgrupo de $\aut(\Gamma)$ y como $G$ tiene una única órbita se sigue que $|\aut(\Gamma)/\Gamma| = 1$ de manera que en particular actúa con finitas órbitas.		
\end{obs}

Nuestro objetivo ahora es entender cómo es la acción sobre el árbol de estructura. 
Lo primero que vamos a buscar entender es cómo son los estabilizadores de esta acción.

\begin{lema}\label{lema_nlC_cap_olC_conexo}
	Sea $\Gamma$ un grafo accesible, conexo y localmente finito.
	Entonces existe $l \in \NN$ de manera que $N^l(C) \cap \ol C$ es conexo para todo $C \in \copt$.
\end{lema}
\begin{proof}
	Primero consideremos algún $C \in \copt$ y después veamos de extender este resultado para todos los cortes óptimos.
	
	Dado que $\ol C$ es conexo tenemos que para todo par de vértices $v,w \in \delta C \cap \ol C$ existe un camino $\alpha_{vw}$ tal que $\alpha_{vw} \subset \ol C$ y es de longitud minimal.
	Sea $l \in \NN$ tal que $l \ge \sup_{v,w \in \delta C \cap \ol C} d(C,\alpha_{vw})$ que está bien definido dado que existen finitos pares de vértices $v,w \in \delta C \cap \ol C$.
	
	Afirmamos que $N^{l}(C) \cap \ol C$ es conexo.
	Para eso sean $w,w' \in N^{l}(C) \cap \ol C$ luego tenemos que existe un camino tal que conecta a $w$ con $v  \in \delta C \cap \ol C$ y un camino que conecta a $w'$ con $v' \in \delta C \cap \ol C$ y ambos caminos están contenidos en $N^{l}(C) \cap \ol C$.
	Por la elección de $l$ tenemos que $v$ y $v'$ están conectados en $N^l(C) \cap \ol C$ probando así que es conexo.
	

	Para finalizar la demostración observemos que por el lema anterior \ref{lema_finitas_orbitas} tenemos finitas órbitas de la acción de $\aut(\Gamma)$ sobre los cortes óptimos.
	Para cada representante de órbita $C_i \in \copt$ donde $i=1,\dots,n$ podemos tomar $l_i \in \NN$ de manera que $ N^{l_{i}}(C_{\textit{}}) \cap \ol C_{i} $ sea conexo. 
	Si tomamos $l = \underset{{i=1,\dots,n}}{\max} l_i$ vemos que nos sirve para todos los representantes de órbitas y así para todos los cortes óptimos.
	
\end{proof}
\begin{obs}
	En particular notemos que si $\Gamma$ es un grafo localmente finito accesible entonces  $|N^l(C) \cap \ol C|<\infty$. 
	Esto se puede ver porque si $v \in N^l(C) \cap \ol C$ luego existe $w \in \beta C \cap C$ tal que esté en el camino que une $v$ con $C$.
	Esto nos dice que $N^l(C) \cap \ol C = N^l(\beta C  \cap C) \cap \ol C$. 
	El conjunto $N^l(\beta C  \cap C) \cap \ol C$ es finito porque los vecinos de un conjunto finito en un grafo localmente finito forman un conjunto finito.
\end{obs}

Ahora definimos un objeto que nos va a ayudar a entender como son estos estabilizadores.

\begin{deff}
	Sea $\Gamma$ un grafo accesible, conexo y localmente finito
	tal que $\aut(\Gamma)$ actúa con finitas órbitas sobre $\copt$.
	Sea $l \in \NN$ tal que $N^l(C) \cap \ol C$ es conexo para todo $C \in \copt$.
	El \emph{bloque} asignado a $[C] \in V(T(\copt))$ está definido como
	\[
		B([C]) = \bigcap_{D \sim C} N^l (D).
	\]	
\end{deff}



Veamos que estos bloques no son vacíos.

\begin{lema}\label{lema_bloques_def_equiv}
	Tenemos la siguiente igualdad de conjuntos,
	\[
		B([C]) = \bigcap_{D \sim C} D \cup \bigcup_{D \sim C} N^l(D) \cap \ol D
	\]
\end{lema}

\begin{proof}
	Veamos las dos inclusiones.
	
	\begin{itemize}
		\item \textbf{$(\subseteq)$}. 
		Sea $v \in B([C])$. 
		Si $v \in \bigcap_{D \sim C} D$ ya está. 
		Caso contrario tenemos que debe existir $D \in [C]$ tal que $v \in N^l(D) \cap \ol D$ y por lo tanto queda probada esta inclusión.
		\item \textbf{$(\supseteq)$}. 
		Nos alcanza con ver que $ N^l(D) \cap \ol D \subseteq B([C])$ para todo $D \in [C]$.
		Para eso notemos que al estar en la misma clase de equivalencia tenemos que 
		\[
			N^l(D) \cap \ol D \subseteq \ol D \subsetneq C \subseteq N^l(C)
		\]
		terminando de probar esta inclusión y así obteniendo la igualdad de conjuntos.
	\end{itemize}
\end{proof}




Estos bloques aparte cumplen que son conexos.

\begin{lema}\label{lema_conexion_B(C)}
	Sea $\Gamma$ grafo infinito, conexo, localmente finito y accesible tal que $\aut(\Gamma)$ actúa con finitas órbitas sobre $\Gamma$.
	Existe una constante $\kappa \in \NN$ tal que para todo $C \in \copt$ y todo $S \subseteq B([C])$ tenemos que: 
	Cuando dos vértices $u,v \in B([C]) \setminus N^\kappa(S)$ se pueden conectar por algún camino en $\Gamma \setminus N^{\kappa}(S)$ entonces se pueden conectar por algún camino en $B([C]) \setminus S$. 
\end{lema}

\begin{proof}
	% Agregar dibujito.
	La acción de $\aut(\Gamma)$ sobre $\copt$ tiene finitas órbitas entonces por lo probado en el lema \ref{lema_nlC_cap_olC_conexo} podemos considerar 
	$\kappa = \max \{ d(u,v) \ | \ D \in \copt, \  u,v \in N^l(D) \cap \ol D  \}$. 	
	Veamos que esta constante nos sirve.
	
	Vamos a probarlo por inducción en la longitud del camino.
	Sea $\gamma$ camino en $\Gamma \setminus N^{\kappa}(S)$ tal que une $u$ con $v$.
	Queremos ver de acortar este camino usando el  que $N^{l}(D) \cap \ol D$ es conexo.
	Si $\gamma \subseteq B([C])$ no hay nada que probar.
	Entonces consideremos el primer vértice $v_m$ de $\gamma$ tal que $v_m \notin B([C])$ y esto quiere decir que para cierto $D \in [C]$ tenemos que $v_m \notin N^l(D)$.
	Por como tomamos a $v_{m}$ y por el lema \ref{lema_bloques_def_equiv} tenemos que $v_{m-1} \in N^l(D) \cap \ol D$.
	
		
	Consideremos ahora el primer vértice $v_n$ de $\gamma$ con $n > m$ tal que $v_n \in N^l(D)$.
	Por ser el primer vértice que aparece en el camino después de $v_{m}$ y por el lema \ref{lema_bloques_def_equiv} tiene que valer que $v_n \in N^l(D) \cap \ol D$.
	
	Dado que $N^l(D) \cap \ol D$ es conexo tenemos un camino de longitud minimal que une $v_{m-1}$ con $v_n$ tal que está contenido en $N^l(D) \cap \ol D$ y en particular por el lema \ref{lema_bloques_def_equiv} está contenido en $B([C])$.
	
	Este camino evita $S$ porque para todo $w$ en el camino vale que 
	\[
		d(v_{n},w) \le d(v_{n},v_{m-1}) \le \kappa
	\]
	dado que $v_{n}, v_{m-1} \in \delta D \cap \ol D$ y dado que el camino tiene longitud mínima.
	Luego si algún $w$ del camino estuviera en $S$ tendríamos que 
	$d(v_n,S) \le \kappa$ contradiciendo que partimos de un camino $\gamma$ que está contenido en $\Gamma \setminus N^{\kappa}(S)$.	
	
\end{proof}


\begin{obs}
	Como corolario del resultado anterior obtenemos que para todo $C \in \copt$ vale que $B([C])$ es conexo.
	Esto porque podemos tomar $S = \emptyset$.
\end{obs}

\begin{lema}\label{lema_accion_vecinos}
	Sea $\Gamma$ grafo accesible, infinito, conexo, localmente finito tal que $\aut(\Gamma)$ actúa con finitas órbitas sobre $\Gamma$, $D$ corte y $g \in \aut(\Gamma)$. 
	Entonces vale la siguiente igualdad de conjuntos:
	\[
		g \cdot N^l(D) = N^l(g \cdot D).
	\]
\end{lema}

\begin{proof}
	Veamos las dos contenciones.
	
	\begin{itemize}
		\item $(\subseteq)$. 
		Sea $v \in N^l(D)$ luego existe $w \in D$ tal que $d(v,w) \le l$.
		Como multiplicar por $g$ es un automorfismo manda vértices adyacentes en vértices adyacentes, en particular no aumenta la distancia.
		Esto es que $d(g\cdot v,g\cdot w) \le l$ concluyendo así que $g \cdot v \in N^l(g \cdot D)$.  
		\item $(\supseteq)$.
		Si $v \in N^l(g\cdot D)$ luego tenemos que debe existir $w \in D$ tal que $d(g \cdot w, v) \le l$.
		Ahora como multiplicar por $g$ es un automorfismo con inversa multiplicar por $g^{-1}$ obtenemos que 
		\[
			d(g^{-1} \cdot v, w) \le l 
			\implies g^{-1}v \in N^l(D) \implies 
			 v \in g\cdot N^l(D).
		\]
	\end{itemize}
\end{proof}

Dado $C \in \copt$ definimos el estabilizador $G_{[C]} = \{ g \in G \mid g \cdot [C] \subseteq [C]  \}$.

\begin{lema}\label{lema_g_actua_b(C)}
	Sea $\Gamma$ grafo accesible, infinito, conexo, localmente finito tal que $G$ actúa con finitas órbitas sobre $\Gamma$, $D$ corte y $g \in G$ tal que $g \in G_{[C]}$ luego tenemos que 
	$g \cdot B([C]) = B([C])$.
\end{lema}
\begin{proof}
	Si $v \in B([C])$ luego $v \in \bigcap_{C \sim D} N^l(D)$.
	Por lo tanto usando que multiplicar por $g$ es un automorfismo del grafo obtenemos la siguiente igualdad
	\[
		g\cdot v \in \bigcap_{C \sim D} g \cdot N^l(D)
	\]
	por lo que por el lema \ref{lema_accion_vecinos} concluímos que 
	\[
		g\cdot v \in \bigcap_{C \sim D}   N^l(g\cdot D).
	\]	
	
	Por nuestra hipótesis tenemos que $g \cdot D \sim C$ para todo $D \in [C]$ por lo tanto renombrando obtenemos la siguiente igualdad
	\[
	g\cdot v \in \bigcap_{C \sim D}   N^l(D) = B([C]),
	\]
	tal como queríamos ver.
	
\end{proof}

\begin{lema}\label{lema_gC_actua_C}
	Sea $\Gamma$ un grafo conexo, localmente finito y accesible tal que $G$ actúa sobre $\Gamma$ con finitas órbitas y sea $C \in \copt$.
	Entonces $G_{[C]}$ actúa con finitas órbitas sobre $[C]$.
\end{lema}
\begin{proof}
	Podemos ver que $G_{[C]}$ actúa sobre $[C]$ si restringimos la acción de $G$ al subgrupo $G_{[C]}$.
	Como $G$ actúa con finitas órbitas tenemos que existen finitos cortes $D_{1}, \dots, D_{k} \in [C]$ de manera que las órbitas ${\cal O}(D_{i})$ son distintas para todo $1 \le i \le k$.
	Sea $D \in [C]$ entonces existe $g \in G$ y $D_{i}$ tal que $D = gD_{i}$ y como $D \sim gD_{i} \sim C$ esto dice que $gC \sim C$ lo que implica que 
	$g \in G_{[C]}$.
	De esta manera tenemos que $G_{[C]}$ también actúa con finitas órbitas sobre $[C]$. 
\end{proof}


\begin{lema}\label{lema_existe_m_borde_corte}
	Sea $\Gamma$ un grafo conexo, localmente finito y accesible tal que $G$ actúa sobre $\Gamma$ con finitas órbitas.
	Entonces existe $m \in \NN$ tal que para todo $v \in B([C])$ existe un corte $D \in [C]$ que cumple que
	$d(v,\beta D) \le m$.
\end{lema}
\begin{proof}
	Probemos entonces que existe tal corte $D$.
	Sea $v \in B([C])$ luego tenemos que si existe un corte $D \in [C]$ tal que  $v \in N^l(D) \cap \ol D$ ya está porque podemos tomar $m=l$.
	Entonces nos interesa el caso que $v \in D$ para todo $D \sim C$.
	Sabemos que $G$ actúa con finitas órbitas sobre el grafo $\Gamma$ por lo tanto debe existir $U \subseteq V(\Gamma)$ finito tal que $B([C]) \subseteq G \cdot U$.
	Esto nos dice que $v = g \cdot u$ con $u \in U$.
	Dado que $U$ es finito tenemos que debe existir $m \ge 1$ tal que $d(u,\beta C) \le m$ para todo $u \in U$.
	Esto nos dice que usando el lema \ref{lema_accion_vecinos} luego
	\[
	d(g \cdot u, \beta (g \cdot C)) \le m,
	\]
	por lo tanto $v \in B([C])$ está en una bola de radio $m$ de $\beta (g \cdot C)$.
	Sea $w \in \beta (g \cdot C)$ tal que $d(v,w) \le m$.
%	Si vemos que hay finitas órbitas de la acción de $G_{[C]}$ sobre $\bigcup_{D \sim C} \beta (g \cdot C)$ entonces dado que el grafo es localmente finito, por el mismo argumento de antes tendríamos que hay finitas posibles órbitas.
	
	Podemos suponer que $w \in \beta (g \cdot C) \cap B([C])$, caso contrario tendríamos que $w \in \beta (g \cdot C)$ y para todo $D \sim C$ tendríamos que $w \in \ol D$ lo que implicaría  que $d(v,\beta D) \le m$ y en este caso no quedaría nada para probar dado que habríamos encontrado al corte $D \in [C]$ que cumple que $d(v, \beta D) \le m$. 
	
	Probemos que $\beta (g \cdot C) \subseteq \bigcup_{D \sim C} \beta D$.
	Entonces supongamos que $w \in \beta (g \cdot C) \cap B([C])$ y más aún nos queda considerar el caso que $w \in D$ para todo $D \sim C$ y $w \in \beta (g \cdot C)$. 
	
	Por la proposición \ref{teo_copt_anidados} tenemos que $g \cdot C$ y $C$ están anidados.
	Renombrando a $g \cdot C$ de ser necesario obtenemos dos casos en particular, $\ol{ g \cdot C} \subsetneq C$ y $C \subseteq \ol {g \cdot C}$.
	
	El caso que $C \subseteq \ol{ g \cdot C}$ tenemos que $ g \cdot C \subseteq \ol C$ lo que nos dice que $\beta (g \cdot C) \subseteq \beta \ol C \cup \ol C$.
	Como suponemos que $w \in C$ luego esto nos dice que $w \in \beta \ol C = \beta C$ tal como queríamos ver.
	
	Consideremos ahora el caso que ${\ol{ g \cdot C}} \subsetneq C$.
	Por la obs \ref{obs_cortes_maximal} tenemos que existen finitos cortes intermedios y debe existir $D \in \copt$ tal que 
	$\ol{g \cdot C}  \subseteq \ol D \subsetneq C$ y resulta maximal respecto a la inclusión y así por definición $D \sim C$.
	Como $w \in \bigcap_{C \sim D} D  \subseteq B([C])$ luego tenemos que 
	$w \in D \cap \beta (g \cdot C)$.
	Como tenemos la siguiente contención $\beta (g \cdot C) \subseteq \beta D \cup \ol D$ obtenemos que
	\[
	w \in D \cap (\beta D \cup \ol D) \implies w \in \beta D
	\]
	tal como queríamos ver.
\end{proof}


\begin{lema}\label{lema_accion_b(C)}
	Sea $\Gamma$ un grafo conexo, localmente finito y accesible tal que $G$ actúa sobre $\Gamma$ con finitas órbitas.
	Entonces $G_{[C]}$ actúa con finitas órbitas sobre $B([C])$.
\end{lema}
\begin{proof}
	Como $G$ actúa con finitas órbitas sobre $\copt$ en particular lo hace sobre el subconjunto $\bigcup_{D \sim C} \beta D$.
	Como corolario de \ref{lema_gC_actua_C} vemos que $G_{[C]}$ actúa con finitas órbitas en el conjunto $ \bigcup_{D \sim C} \beta D $.
	
	Por el lema previo \ref{lema_g_actua_b(C)} vimos que $G_{[C]}$ actúa sobre $B([C])$.
	Queremos ver que lo hace con finitas órbitas.
	Para eso sabemos que existe $m \in \NN$ tal que para todo $v \in B([C])$ existe un corte $D \in [C]$ que cumple que
	$d(v,\beta D) \le m$ gracias al lema \ref{lema_existe_m_borde_corte}.
	
	Con esto nos alcanza para ver que $G_{[C]}$ actúa con finitas órbitas sobre $B([C])$.
	Nos armarnos un dominio fundamental finito de la acción de $G_{[C]}$ en $B([C])$ de la siguiente manera.
	Dado que existen finitos $v_{1}, \dots, v_{n}$ tales que $v_{i} \in \beta D_{i}$ para algún $D_{i} \in [C]$ tales que son dominio fundamental de la acción de $G_{[C]}$ en $\bigcup_{D \sim C} \beta D$ podemos considerar el conjunto 
	$A  = \bigcup_{i=1 \dots n} N^m(v_{i})$. 
	Luego este conjunto es un dominio fundamental de la acción de $G_{[C]}$ en $B([C])$ y es finito porque $\Gamma$ es localmente finito.
\end{proof}

\begin{lema}\label{lema_bloques_1_end}
	Sea $\Gamma$ un grafo infinito, conexo, localmente finito y accesible.
	Los bloques $B([C])$ tienen a lo sumo un solo end.
\end{lema}
\begin{proof}
	Si el bloque tuviera más de un end entonces debería existir $\alpha$ camino infinito y $S \subseteq B([C])$ finito de manera que $\alpha$ tiene infinitos vértices en dos componentes conexas distintas de $B([C]) \setminus S$.
	Para todo $D \in [C]$ tenemos que $\alpha \subseteq B([C]) \subseteq N^l(D)$ y por el lema \ref{lema_nlC_cap_olC_conexo} sabemos que $|N^l(D) \cap \ol D | < \infty$.
	Esto nos dice que en particular tenemos que $|\alpha \cap \ol D| < \infty$.
	
	% por el lema anterior tenemos al menos un corte que lo parte
	El conjunto $N^l(S)$ sigue siendo finito por lo tanto tenemos que $\alpha$ tiene infinitos vértices en $\Gamma \setminus N^l(S)$.
	Por el lema \ref{lema_conexion_B(C)} tenemos que dados dos vértices de $B[C]$ luego todo camino en $\Gamma \setminus N^l(S)$ que los conecte se puede tomar para que esté dentro de $B([C]) \setminus S$. 
	De esta manera tienen que existir dos componentes conexas de $\Gamma \setminus N^l(S)$ tales que tengan infinitos vértices de $\alpha$. 
	Caso contrario negaríamos que existen dos componentes conexas de $B([C]) \setminus S$ que tienen infinitos vértices de $\alpha$.
	Esto nos dice que ${\cal C}_{\alpha} \neq \emptyset$.
	Como el conjunto de cortes es no vacío debe existir $E \in \copt \cap {\cal C}_{\alpha}$.
	Como $E \in {\cal C}_{\alpha}$ luego tenemos que $|\alpha \cap E| = \infty = |\alpha \cap \ol E|$.
	Nuestro objetivo es llegar a una contradicción de suponer la existencia de este corte. 	
	% contradicción.
	
	Dado que $E,C \in \copt$ luego están anidados gracias a la proposición \ref{teo_copt_anidados}.
	Dos posibilidades están descartadas. 
	Es imposible que $E \subseteq \ol C$ o bien $\ol E \subseteq \ol C$ porque en ambos casos contradecimos que $E \in {\cal C}_{\alpha}$.
	Nos queda considerar los otros dos casos que son $E \subsetneq C$ o bien $\ol E \subsetneq C$, ambos casos los tratamos análogamente.
	Por la observación \ref{obs_cortes_maximal} tenemos existe $D \in [C]$ tal que 
	\[
		E \subseteq \ol D \subsetneq C
	\]
	pero como justamente $D \sim C$ tenemos que $E \cap \alpha \subseteq \ol D \cap \alpha$ pero $|\alpha \cap \ol D| < \infty$ y esto es una contradicción porque suponíamos que $E \in {\cal C}_{\alpha}$.
\end{proof}


\begin{lema}\label{lema_geodesica_biinfinita}
	Sea $\Gamma$ un grafo conexo, localmente finito e infinito de manera que $\aut(\Gamma)$ actúa con finitas órbitas sobre $\Gamma$.
	Entonces $\Gamma$ tiene una geodésica infinita.
\end{lema}
\begin{proof}
	Dada $\gamma$ una geodésica de longitud impar denotaremos $m(\gamma)$ su vértice medio.	
	El grafo es infinito entonces existen geodésicas de cualquier tamaño arbitrario.
	En particular como la acción de $\aut(\Gamma)$ tiene un dominio fundamental finito obtenemos que tiene que haber algún vértice $v_0$ de manera que $m(\gamma) = v_0$ para infinitas geodésicas $\gamma$ de longitud arbitraria.
	
	En base a este vértice vamos a construir un árbol $T$ de la siguiente manera.
	La raíz va a ser $v_0$ y los vértices van a ser $(v_{-n},\dots, v_0,\dots,v_n)$ geodésicas que tienen a $v_0$ como punto medio.
	Las aristas entonces van a ser 
	\[
		((v_{-n},\dots, v_0,\dots,v_n), (v_{-n-1},\dots, v_0,\dots,v_{n+1}))
	\]
	siempre y cuando $(v_{-n-1},\dots, v_0,\dots,v_{n+1})$ siga siendo una geodésica con $v_0$ su punto medio.
	Dado que el grafo $\Gamma$ es localmente finito por suposición tenemos que $T$ también lo va a ser.
	
	Finalmente para encontrar la geodésica infinita lo que hacemos es usar el lema de König \ref{lema_Konig} en $T$ para obtener un rayo en $T$, que se corresponde justamente con una geodésica infinita en $\Gamma$ tal como buscábamos.
	
\end{proof}

\begin{lema}\label{lema_tw_mas_1_end}
	Sea $\Gamma$ un grafo infinito, conexo y con treewidth finito de manera que $\aut(\Gamma)$ actúa con finitas órbitas sobre $\Gamma$.
	Entonces $\Gamma$ tiene más de un end.
\end{lema}
\begin{proof}
	Este grafo resulta ser accesible por el teorema \ref{teo_treewidth_fin_accesible}.
	Consideremos $m = \max \{  d(u,v) : u,v \in \beta C \cap \ol C \}$ para todo $C \in \copt$.
	Esto lo podemos considerar porque por el lema \ref{lema_finitas_orbitas} tenemos que $\aut(\Gamma)$ actúa con finitas órbitas sobre $\copt$.
	
	Supongamos que $\Gamma$ no tiene más de un end.
	Sea $\alpha = (\dots v_{-n}, \dots, v_{0}, \dots, v_{n} \dots)$ una geodésica infinita como en el lema \ref{lema_geodesica_biinfinita}.
	Usando \ref{lema_corte_treewidth} tenemos que debe existir un $k$-corte $C$ de manera que $d(v_{0}, \ol C) > m$, $v_{0} \in C$ y tal que $|\alpha \cap \ol C|=\infty$.
	Como suponemos que el grafo no tiene más de un end entonces necesariamente vale que ${\cal{C}}_{\alpha} = \emptyset$. 
	Veamos que esto nos da una contradicción.
	
	Como $|\alpha \cap C| < \infty$ debe existir $i,j \in \NN$ de manera que $v_{-i} \in \beta C \cap \ol C$ y $v_{j} \in \beta C \cap \ol C$.
	Dado que $\alpha$ es geodésico tenemos que
	\[
		d(v_{-i},v_{j}) = d(v_{-i},v_{0}) + d(v_{0}, v_{j}) = 2m
	\]
	y esto es una contradicción por como elegimos a $m$. 
	Concluyendo así que $\Gamma$ tiene más de un end.
	
\end{proof}

Consideremos $\cG$ una clase de grupos tal que es cerrada por subgrupos de índice finito.
En nuestro caso en particular nos va a interesar la clase de grupos finitos pero otras clases interesantes que cumplen esta propiedad son la de los grupos virtualmente libres por lo visto en la proposición \ref{prop_vls}.
Recordemos que un grupo $G$ es virtualmente $\cG$ si existe $H$ subgrupo de $G$ de índice finito tal que $H \in \cG$.

\begin{lema}\label{lema_accion_virtualmente_g}
	Sea $\Gamma$ un grafo conexo, localmente finito, accesible tal que $\aut(\Gamma)$ actúa con finitas órbitas y sea $G$ grupo que actúa en $\Gamma$ por automorfismos.
	Consideremos que para todo vértice $v \in V(\Gamma)$ tenemos que $G_{v} \in \cG$.
	Entonces dado $U \subseteq V(\Gamma)$ finito tenemos que $G_{U} = \{ g \in G \mid g \cdot { U} \subset { U} \}$ es virtualmente $\cG$.
\end{lema}

\begin{proof}
	Tenemos el morfismo $G_{U} \to \Sy(U)$ que tiene como núcleo al subgrupo $\bigcap_{u \in U} G_u$ de índice finito en $G_U$.
	Veamos que esta intersección es un grupo que está en $\cG$.
	
	Para todo $k \in \NN$ tenemos que $G_{v}$ actúa sobre $B_{k}(v) = \{ w \in V(\Gamma) : d(w,v) \le k  \}$.
	Elegimos $k$ de manera que ${ U} \subset B_{k}(v)$.
	Dado que $\Gamma$ es localmente finito tenemos que $B_{k}(v)$ es finito y así que si $N$ es el núcleo de la acción $G_{v} \to B_{k}(v)$ entonces $N$ tiene índice finito y por lo tanto $N$ está en $\cG$.
	Como $[G_{v} : N] < \infty$ esto implica que $ [\bigcap_{u \in U} G_{u} : N] < \infty$
	y esto implica que $ \bigcap_{u \in U} G_{u}$ está en $\cG$ y por lo tanto $G_{ U}$ es virtualmente $\cG$.
\end{proof}


\begin{prop}\label{prop_clase_grupos_accion}
	Sea $\Gamma$ un grafo conexo, localmente finito, accesible tal que $\aut(\Gamma)$ actúa con finitas órbitas y sea $G$ grupo que actúa en $\Gamma$ por automorfismos.
	Consideremos que para todo vértice $v \in V(\Gamma)$ tenemos que $G_{v} \in \cG$.
	Entonces:
	\begin{enumerate}
		\item $G$ actúa con estabilizadores virtualmente $\cG$ en las aristas de $T(\copt)$.
		\item Si $B([C])$ es finito para todo $C \in \copt$ entonces $G$ actúa con estabilizadores virtualmente $\cG$ en los vértices de $T(\copt)$.
	\end{enumerate}	
\end{prop}

\begin{proof}
	Probemos \textbf{1}.
	Tenemos que si $g \in G_{\{C,\ol C\}}$ luego esto nos dice que $g \cdot C = C $ y que $g \cdot \ol C = \ol C$. 
	En particular obtenemos que $g \cdot \beta C \subset \beta C$.
	Esto nos dice que $g \in G_{\beta C}$ y como $|\beta C| < \infty$ entonces por el lema \ref{lema_accion_virtualmente_g} obtenemos que $G_{\{C,\ol C\}}$ es virtualmente $\cG$ tal como queríamos ver.
	
	como el grupo $G$ actúa con finitas órbitas y $\beta C$ es finito debe ser que existe $U \in V(\Gamma)$ de manera que $g \in G_{U}$.
	Como $N \le G_{\{[C],[\ol C]\}}$ luego tenemos que $G_{\{[C],[\ol C]\}}$ es virtualmente $\cG$.
	%Acá esto sale con la cuenta de los índices si quiero escribir más detalladamente.
	
	Probemos que vale \textbf{2}.
	Por el lema \ref{lema_accion_b(C)} tenemos que si $B([C])$ es finito entonces $G_{[C]}$ actúa con finitas órbitas sobre $B([C])$.
	En particular tenemos que si $g \in G_{[C]}$ luego $g \in G_{B([C])}$ y como $|B([C])| < \infty$ esto nos dice que por el lema \ref{lema_accion_virtualmente_g} que $G_{[C]}$ es virtualmente $\cG$ tal como queríamos ver.
	
\end{proof}



\begin{teo}\label{teo_tw_accion_central}
	Sea $\Gamma$ grafo conexo, localmente finito con treewidth finito.
	Sea $G$ grupo que actúa en $\Gamma$ por medio de automorfismos con finitas órbitas y tal que para cada vértice $v \in V(\Gamma)$ tenemos que el estabilizador $G_{v}$ es finito.
	Entonces $G$ actúa en $T(\copt)$ con finitas órbitas y con estabilizadores finitos.
\end{teo}
\begin{proof}
	Por la proposición \ref{lema_finitas_orbitas} tenemos que $G$ actúa con finitas órbitas sobre $T(\copt)$.
	
	Veamos ahora que los estabilizadores son finitos.
	Para eso queremos ver primero que $B([C])$ es finito.
	Para esto notemos que $B([C])$ tiene treewidth finito por ser subgrafo de un grafo con treewidth finito y aparte resulta ser accesible por el teorema \ref{teo_treewidth_fin_accesible}.
	Esto nos dice que $G_{[C]}$ actúa con finitas órbitas sobre $B([C])$.
	Ahora supongamos que $B([C])$ es infinito, en este caso por el lema \ref{lema_tw_mas_1_end} tendríamos que debería tener más de un end pero anteriormente vimos en el lema \ref{lema_bloques_1_end} que estos grafos no pueden tener más de end.
	De esta manera descartamos que $B([C])$ es infinito y por lo tanto es finito.
	Ahora para finalizar usamos el resultado \ref{prop_clase_grupos_accion} para la clase $\cG$ de los grupos finitos.
\end{proof}



\begin{coro}\label{coro_tw_finito_implica_pi1}
	Si $G$ es un grupo con treewidth finito entonces $G$ es el grupo fundamental de un grafo de grupos con grupos finitos en sus vértices y aristas.
\end{coro}
\begin{proof}
	Usamos el resultado anterior \ref{teo_tw_accion_central} para obtener que actúa sobre un árbol con finitas órbitas y con estabilizadores finitos.
	Por el teorema central de Bass Serre \ref{teo_Serre} concluímos lo que queríamos probar.
\end{proof}



\listoftodos

\end{document}