\usepackage{amsmath,amsfonts,amsthm,amssymb,mathtools,sectsty}
\pagenumbering{gobble}
\usepackage{subcaption}
%\usepackage{graphicx}
%\usepackage[pdftex,dvipsnames]{xcolor}
\usepackage{cancel}
\usepackage{graphicx}
\usepackage{marginnote}
\usepackage{mathabx}
\usepackage{float}
\setlength{\marginparwidth}{2cm}

% Tikz y las librerías para automátas
\usepackage{tikz-cd}
\usepackage{tikz}
\usetikzlibrary{arrows,automata}
\usetikzlibrary{babel} %para evitar que se jodan los automatas de tikz
\usetikzlibrary{graphs} 
\usetikzlibrary{calc}
\usetikzlibrary{positioning}
%\usetikzlibrary{shapes.geometric}  % for [ellipse], [diamond], etc

%\usepackage[backend=biber,...]{biblatex} 

%Referencias; me gustaría que backref funcione pero no es importante tampoco.
\usepackage[pagebackref]{hyperref}
% Para modificar el estilo de las referencias
\hypersetup{
	colorlinks,
	linkcolor={astral},
	citecolor={red!70!black},
	urlcolor={red!80!black}
}
\definecolor{astral}{RGB}{46,116,181}
\colorlet{chulo}{blue!70!purple}
\colorlet{rojo}{purple!45!black}
\definecolor{carrotorange}{rgb}{0.93, 0.57, 0.13}
\definecolor{brightcerulean}{rgb}{0.11, 0.67, 0.84}
\definecolor{brightube}{rgb}{0.82, 0.62, 0.91}
\definecolor{cadmiumred}{rgb}{0.89, 0.0, 0.13}
\definecolor{applegreen}{rgb}{0.55, 0.71, 0.0}
\definecolor{aurometalsaurus}{rgb}{0.43, 0.5, 0.5}

%%%%%%%%%%%%%%%%%%%%% ENUMERAR CON COSAS QUE NO SEAN SOLO NÚMEROS %%%%%%%%%%
\usepackage[shortlabels]{enumitem}
\setlist[enumerate]{font=\bfseries}



%%%%%%%%%%%%%%%%%%%%%%% TÍTULOS DE SECCIONES MÁS FANCIES %%%%%%%%%%%%%%
\usepackage{titlesec}
\setcounter{secnumdepth}{3} % Hasta que profundidad quiero numerar, 4 sería los párrafos.
\titleformat{\section}[block]{\color{astral!50!black}\Large\bfseries\filcenter}{\S\thesection.}{1em}{}
\titleformat{\subsection}[hang]{\color{astral!50!black}\large\bfseries\filcenter}{\S\thesubsection.}{1em}{}

%%%%%%%%%%%%%%%%%%%%%% COLORES PÁRRAFOS Y CAPÍTULOS %%%%%%%%%%%%%%%%%%%%%%%%
%\paragraphfont{\color{astral!70!black}}
\chapterfont{\color{astral!40!black}}
%\subsectionfont{\color{astral!60!black} }
%\sectionfont{\color{astral!50!black} }

\usepackage{mathpazo}
\usepackage{amssymb}
%\usepackage{thmtools}

%Esto sirve para armar grafos de Cayley de una manera más copada.
\usetikzlibrary{lindenmayersystems,arrows.meta}
\pgfdeclarelindenmayersystem{cayley}{
	\rule{G->G-G+++G--G}
	\symbol{R}{
		\pgflsystemstep=0.5\pgflsystemstep
	}

}

\usepackage[framemethod=tikz]{mdframed}

%%%%%%%%%%%%%  TEOREMAS  %%%%%%%%%%%%%%%%%
\theoremstyle{plain} %% el estilo clásico
\newtheorem{teo}{\color{rojo}{ { Teorema}}}[section]
\newtheorem{prop}[teo]{\color{rojo} {Proposición}}
\newtheorem{lema}[teo]{\color{rojo} {Lema}}
\newtheorem{coro}[teo]{\color{rojo} {Corolario}}
\newtheorem*{aff}{ {Afirmación}}
% Si pongo [theorem] siguen la numeración de los teoremas. 
% e.j. Teo 1, Lema 2, Teo 3, Teo 4 ...
\theoremstyle{definition}
\newtheorem{deff}[teo]{{ Definición}}{\smallskip}
\newtheorem{ej}[teo]{{Ejemplo}}{\smallskip}

% Remarks
\theoremstyle{remark}
\newtheorem{obs}[teo]{ {Observación}}{\smallskip}

%%%%%%%%%% FRAMES PARA TEOREMAS A LO HATCHER %%%%%%%%%%%%%%%%%%%%%%%%%

\surroundwithmdframed[outerlinewidth=0.4pt,
innerlinewidth=0.4pt,
align=center,
middlelinewidth=1pt,
middlelinecolor=white,
innertopmargin=-4pt,
innerbottommargin=0pt,
innerrightmargin=4pt,
innerleftmargin=4pt,
bottomline=false,topline=false,rightline=false]{teo}
\surroundwithmdframed[outerlinewidth=0.4pt,
innerlinewidth=0.4pt,
align=center,
middlelinewidth=1pt,
middlelinecolor=white,
innertopmargin=-4pt,
innerbottommargin=0pt,
innerrightmargin=4pt,
innerleftmargin=4pt,
bottomline=false,topline=false,rightline=false]{lema}

\surroundwithmdframed[outerlinewidth=0.4pt,
innerlinewidth=0.4pt,
align=center,
middlelinewidth=1pt,
middlelinecolor=white,
innertopmargin=-4pt,
innerbottommargin=0pt,
innerrightmargin=4pt,
innerleftmargin=4pt,
bottomline=false,topline=false,rightline=false]{prop}


\surroundwithmdframed[outerlinewidth=0.4pt,
innerlinewidth=0.4pt,
align=center,
middlelinewidth=1pt,
middlelinecolor=white,
innertopmargin=-4pt,
innerbottommargin=0pt,
innerrightmargin=4pt,
innerleftmargin=4pt,
bottomline=false,topline=false,rightline=false]{coro}

%==================================================================%

% DEMOS EN NEGRITA.
\renewenvironment{proof}{{\textbf{Demostración.}}}{ \hfill $\blacksquare$ \medskip} 

%% ========== Para escribir pseudo ==========
%\usepackage{algorithm}
%\usepackage[noend]{algpseudocode}  % "noend" es para no mostrar los endfor, endif
%%\algrenewcommand\alglinenumber[1]{\tiny #1:}  % Para que los numeros de linea del pseudo sean pequeños
%\renewcommand{\thealgorithm}{}  % Que no aparezca el numero luego de "Algorithm"
%\floatname{algorithm}{ }    % Entre {  } que quiero que aparezca en vez de "Algorithm"
%
%% traducciones
%\algrenewcommand\algorithmicwhile{\textbf{mientras}}
%\algrenewcommand\algorithmicdo{\textbf{hacer}}
%\algrenewcommand\algorithmicreturn{\textbf{devolver}}
%\algrenewcommand\algorithmicif{\textbf{si}}
%\algrenewcommand\algorithmicthen{\textbf{entonces}}
%\algrenewcommand\algorithmicfor{\textbf{para}}
%
%%% indentar dentro de los algoritmos
%\algdef{SE}[SUBALG]{Indent}{EndIndent}{}{\algorithmicend\ }%
%\algtext*{Indent}
%\algtext*{EndIndent}

% =========================================================
\usepackage[colorinlistoftodos,prependcaption,textsize=tiny]{todonotes}



%Comandos útiles.
\newcommand\RP{\mathbb{RP}}
\newcommand{\norm}[1]{\left\lVert#1\right\rVert}
\newcommand{\RR}{\mathbb{R}}
\newcommand{\CC}{\mathbb{C}}
\newcommand{\NN}{\mathbb{N}}
\newcommand{\ZZ}{\mathbb{Z}}
\newcommand{\Om}{\Omega}
\newcommand{\A}{\mathcal A}
\newcommand\ol{\overline}
\newcommand{\blue}{\textcolor{chulo}}
\newcommand{\red}{\textcolor{rojo}}
\newcommand{\Gg}{\mathfrak g}
\newcommand{\SL}{SL_2(\mathbb Z)}
\newcommand{\stab}{\text{Stab}}
\newcommand{\ic}{independiente de contexto }
\newcommand{\APND}{automáta de pila no determinístico }
\newcommand{\APD}{automáta de pila determinístico }
\newcommand{\gramatica}{{\cal G} = (V, \Sigma, P, S)}
\newcommand{\deriva}{\overset{*}{\to_{\cal G}}}
\newcommand{\tto}{\overset{*}{\to}}
\newcommand{\lengderivado}{L({\cal G})}
\newcommand{\fg}{grupo finitamente generado }
%\newcommand{\ol}{\overline{}}
\newcommand{\aut}{\text{Aut}}
\newcommand{\Sy}{\text{Sym}} 

\newcommand{\partes}[1]{{\cal{P}}(#1)} 



\newcommand{\fp}{grupo finitamente presentado }
\newcommand{\vl}{virtualmente libre }
\newcommand{\vls}{virtualmente libres}
\newcommand{\WP}{\text{WP}(G, \Sigma)}

\newcommand{\cG}{ {\cal G} }
\newcommand{\cGg}{{\cal G} = (V, \Sigma, P, S)}
\newcommand{\cH}{ {\cal H} }
\newcommand{\Xm}{\widetilde X}
%\newcommand{\ol}{\overline{}}

\newcommand{\ca}{{\cal C}(\alpha)}
\newcommand{\cmin}{{\cal C}_{\text{min}}}
\newcommand{\cam}{{\cal C}_{\text{min}}(\alpha)}
\newcommand{\copta}{{\cal C}_{\text{opt}}(\alpha)}
\newcommand{\copt}{{\cal C}_{\text{opt}}}
\newcommand*{\rows}{6}

\newcommand{\TODO}[1]{\textcolor{red}{TODO: #1}}



%%%%%%%%%%%%%%  SETUP DE LA PÁGINA %%%%%%%%%%%%%%%%%
%\usepackage{fancyhdr} 
\pagestyle{headings} 
\pagenumbering{arabic} 
%\foot[C]{\textbf{\thepage}} % except the center
%\setlength{\headheight}{42pt}% ...at least 51.60004pt
%\renewcommand{\headrulewidth}{0.8pt}
%\head[L]{\thepage} 
%\head[R]{\textsl{\leftmark}} 
%\fancyfoot[C]{\thepage}

\usepackage{float}


\usepackage{subfiles} % mejor ponerlos al final