\documentclass[tesis.tex]{subfiles}

\begin{document}
	
\chapter{Grafos de Cayley.} \label{seccion_treewidth}

Las principales referencias de este capítulo son \cite{diekert2017context}, \cite{kuske2005logical} y \cite{diestel2005graph}.

En la primer sección \ref{secc_tw} introducimos las descomposiciones en árboles para grafos no dirigidos y la definición del treewidth finito.
El ejemplo \ref{desc-grafo-cayley} de la descomposición en un árbol para el grafo de Cayley de un grupo finitamente generado va a ser crucial en la demostración del teorema principal de este capítulo.


En la segundo sección \ref{secc_qi} introducimos brevemente ideas elementales detrás de las cuasisometrías (en nuestro caso restringidas al contexto de grafos).
La idea es poder usar estas herramientas para probar que la definición de treewidth no depende de los generadores.

Finalmente en la sección \ref{secc_MuSch} probamos uno de los resultados centrales del trabajo que establece que todo grupo independiente de contexto tiene treewidth finito.



\section{Treewidth.}\label{secc_tw}

Dado que el grafo de Cayley de todo grupo libre se puede tomar para que sea un árbol (ver \ref{lema_cayley_libre_arbol}) es razonable pensar que todo grupo virtualmente libre es tal que su grafo de Cayley se parece a un árbol. 
Vamos a dar una primera definición que nos permite formalizar esta idea de que un grafo se parezca a un árbol.


\begin{deff}\label{desc-arbol}
	Una \emph{descomposición en un árbol} de un grafo $\Gamma$ es un par $(T,f)$ donde
	$T$ es un árbol y $f$ un mapa 
	\[
	f: V(T) \to 2^{V(\Gamma)}
	\]
	Que cumple las siguientes condiciones:
	\begin{enumerate}
		\item[\textbf{T1.}] Para todo vértice $v \in V(\Gamma)$ debe existir $t \in V(T)$ tal que $v \in f(t)$. 
		\item[\textbf{T2.}] Para toda arista $\{v,w\} \in E(\Gamma)$ 
		debe existir $t \in V(T)$ tal que $v,w \in f(t)$.
		\item[\textbf{T3.}] Si $v \in V(\Gamma)$ es tal que $v \in f(t) \cap f(s)$ luego $v \in f(r)$ para todo $r \in V(T)$ en la geodésica que va desde $s$ a $t$ dentro de $T$.  
	\end{enumerate}
	Dado $t \in V(T)$ llamaremos a $f(t) \in 2^{V(\Gamma)}$ un \emph{bolsón}.
	 
\end{deff}
Notemos que la tercer condición \textbf{T3} se puede reescribir de la siguiente manera:
dado $v \in V(\Gamma)$ entonces el conjunto $T_{v} = \{ t \in V(T) :  v \in f(t) \}$ forma un subárbol de $T$.
\smallskip

Si queremos que una descomposición en un árbol de un grafo modele la idea que el grafo se parece a un árbol entonces vamos a requerir que los 
bolsones no tengan muchos vértices. 
Esto nos conduce a la siguiente definición.



\begin{deff}
	Dado $\Gamma$ un grafo no dirigido y $(T,f)$ una descomposición en un árbol de $\Gamma$ el \emph{bagsize} de esta descomposición es el siguiente valor:
	\begin{equation*}
		bs(\Gamma,T,f) = \sup_{t \in V(T)} |f(t)| - 1
	\end{equation*}
	Un grafo $\Gamma$ tiene \emph{treewidth finito} si existe una descomposición en un árbol de bagsize finito.	
\end{deff}

\begin{ej}
	Sea $\Gamma$ el grafo definido por los vértices 
	\[
		V = \{ v_{1}, v_{2},v_{3}, v_{4},v_{5},v_{6} \}
	\]
	 y por las aristas
	\[
		 E = \{ \{ v_{1}, v_{2}\}, \{ v_{1},v_{3} \},\{v_{2} ,v_{3} \} , \{ v_{2} , v_{5} \}, \{ v_{3} , v_{4} \}, \{ v_{5}, v_{4} \}, \{ v_{4}, v_{6} \} \}. 
	\]
	Entonces una descomposición en un árbol para $\Gamma$ está dada por $T$ árbol con vértices  $V(T)=\{ t_{1}, t_{2}, t_{3}  \}$ y con aristas $E(T) = \{ \{t_{1}, t_{2} \}, \{ t_{2}, t_{3} \} \}$ y un mapa 
	definido por $f(t_{1}) = \{ v_{1}, v_{2}, v_{3} \}$, $f(t_{2}) = \{ v_{2}, v_{3}, v_{4}, v_{5} \}$ y $f(t_{3}) = \{ v_{4}, v_{6} \}$.
	Representamos gráficamente esta descomposición pintando los vértices del grafo $\Gamma$ con el color correspondiente al bolsón del vértice del árbol $T$ que pertenece.
	\begin{center}
		\begin{tikzpicture}[scale=0.8,V/.style = {circle, draw, fill=gray!15,font=\footnotesize},fill fraction/.style={path picture={
					\fill[#1] 
					(path picture bounding box.south) rectangle
					(path picture bounding box.north west);
			}},
			fill fraction/.default=gray!50
,			]
			\node[align=center] at (-2.5, 1) {Grafo \textbf{$T$}};
			\begin{scope}[nodes=V, xshift=-3.5cm, yshift=0cm]
				\node (1) [fill = brightcerulean!80] {$t_1$};
				\node (2) [below=of 1, fill=cadmiumred!75] {$t_{2}$};
				\node (3) [below=of 2, fill=brightube!90] {$t_3$};
			\end{scope}	
			\draw   (1)  edge[-,] (2)
			(2)  edge[-] (3);
			
			\node[align=center] at (5.5, 1) {Grafo \textbf{$\Gamma$}};
			\begin{scope}[nodes=V,xshift=4.5cm, yshift=0cm]
				\node (1)  [fill = brightcerulean!80] {$v_{1}$};
				\node (2) [below=of 1, fill = brightcerulean!80, fill fraction=cadmiumred!75]{$v_{2}$};
				\node (3) [right=of 2, fill = brightcerulean!80, fill fraction=cadmiumred!75	]    {$v_{3}$};
				\node (4) [below=of 2,fill = cadmiumred!75, fill fraction=brightube!90]    {$v_{4}$};
				\node (5) [below=of 3, fill = cadmiumred!75]    {$v_{5}$};
				\node (6) [below=of 4,fill = brightube!90]    {$v_{6}$};
			\end{scope}
			\draw   (1)  edge[-] (2);
			\draw   (1)  edge[-] (3);
			\draw   (2)  edge[-] (3);
			\draw   (3)  edge[-] (4);
			\draw   (4)  edge[-] (5);
			\draw   (2)  edge[-] (5);
			\draw   (4)  edge[-] (6);
		\end{tikzpicture}
	\end{center}
	
	Por como lo construimos tenemos que $bs(\Gamma, T, f) = 3$.
\end{ej}

\begin{obs}
	Si $\Gamma$ es un grafo finito entonces $\Gamma$ tiene treewidth finito. 
\end{obs}


\begin{deff}
	Dado $\Gamma$ un grafo no dirigido llamaremos a $\Gamma'$ \emph{la subdivisión baricéntrica de $\Gamma$ }al grafo no dirigido dado por
	$V(\Gamma') = V(\Gamma) \cup E(\Gamma)$ y $E(\Gamma') = \{ \{ v, \{v,w \} \} \mid \ \text{si} \ \{ v,w \} \in E(\Gamma) \}$.
\end{deff}




\begin{obs}\label{obs_sub_bari_arbol}
	Si $T$ es un árbol entonces $T'$ la subdivisión baricéntrica de $T$ también es un árbol.
\end{obs}

\begin{prop}\label{desc-arbol-arbol}
	Todo árbol $T$ tiene treewidth igual a  $1$.
\end{prop}

\begin{proof}
	Consideremos el siguiente par $(T',f)$ donde $T'$ es la subdivisión baricéntrica de $T$ y por lo tanto es un árbol por \ref{obs_sub_bari_arbol} y donde $f$ está definida como
	\[
	f(t) = 
	\begin{cases}
		\{ v \} \ & \text{si} \ t = v \in V(T) 				\\
		\{ v,w  \} \ &\text{si} \ t = \{ v,w\} \in E(T).
	\end{cases}
	\]
	
	
	Por como definimos a $f$ vale que $|f(t)| \le 2$ para todo $t \in V(T)$.
	De esta manera si vemos que $(T',f)$ se trata de una descomposición en un árbol para $T$ tendremos probado que $bs(T,T',f) = 1$ tal como queríamos ver.
	
	\begin{enumerate}
		\item[\textbf{T1.}] 
		Sea $t \in V(T)$ luego consideremos $f(t) = \{ t \}$ dado que $t \in V(T')$ por construcción de la subdivisión baricéntrica.
		
		\item[\textbf{T2.}] 
		Dada una arista $\{t,s\} \in E(T)$ consideramos $f(\{ t,s \}) = \{ t,s \} $ de manera que tanto $t$ como $s$ están en un mismo bolsón.
		
		\item[\textbf{T3.}] 
		Sea $t \in V(T)$, queremos ver que $\{ t' \in V(T') :  t \in f(t') \}$ resulta ser un subárbol de $T'$.		
		Para empezar tenemos que $t \in f(t')$ si y solo sí sucede alguno de dos casos:
		$t' = t$ o bien ocurre que $t' = \{ t,s \}$ para cierta arista $\{t,s\} \in E(T)$.
		Estos vértices de $T'$ están conectados y en particular son un subgrafo dado que $ \{t, \{t,s\}\} \in E(T')$ para $t \in V(T)$ y $\{t,s\} \in E(T)$. 
	\end{enumerate}
	
\end{proof}
	
	


Probaremos algunas proposiciones útiles de las descomposiciones en árboles de grafos.
Esta primer proposición generaliza la propiedad \textbf{T3} de una descomposición.
%Sean $t,s \in V(T)$ vértices cualesquiera entonces existe una única geodésica que las une que denotamos $[t,s]$.
\begin{prop}\label{prop-camino-desc}
	Sea $\Gamma$ grafo no dirigido.
	Sea $(T,f)$ una descomposición en un árbol de $\Gamma$.
	Sean $t_{1},t_{2},t_{3} \in V(T)$ tales que $[t_1,t_2]$ pasa por $t_{3}$.
	Sean $v,w \in V(\Gamma)$ de manera que $v \in f(t_{1})$ y tal que $w \in f(t_{2})$.
	Si $\gamma = v_0 \dots v_n$ es algún camino en $\Gamma$ conectándolos de manera que $v_{0}=v$ y $v_{n} = w$	
	entonces debe existir algún $ 0 \le i \le n$ de manera que $v_i \in f(t_{3})$. 
\end{prop}

\begin{proof}	
	Vamos a demostrarlo haciendo inducción en la longitud $|\gamma|$.
	 
	El caso base es que $|\gamma| = 0$ por lo tanto $v=w$. 
	En este caso por ser $(T,f)$ una descomposición en un árbol tenemos que usando \textbf{T3} vale que $\{  t \in V(T) \mid v \in f(t) \}$ es un subárbol, por lo que $v \in f(t_{3})$.
	
	Para el paso inductivo supongamos que vale para caminos de longitud $n$.
	Sea $t' \in V(T)$ de manera que $v_{0}, v_{1} \in f(t')$ que existe por la propiedad \textbf{T2}.
	Dado que 
	$[t_{1}, t_{2}] = [t_{1}, t'][t',t_{2}]$
	luego $t_{3}$ está en alguna de las dos geodésicas: $[t_{1},t']$ o bien en  $[t',t_{2}]$.
	
	Si $t_{3}$ está en la primera geodésica entonces $t' = t_{3}$ y ya está porque nos alcanza con tomar $i=1$ de manera que $v_1 \in f(t_{3})$.
	En el otro caso consideramos el camino de longitud $n$ dado por $(v_1, v_2, \dots v_n)$ y usando la hipótesis inductiva llegamos al resultado.	
\end{proof}

\begin{prop}\label{prop_tw_finitos_bolsones}
	Sea $\Gamma$ un grafo no dirigido localmente finito con treewidth finito entonces podemos tomar una descomposición en un árbol $(T,f')$ de $\Gamma$ de manera que para todo $v \in V(\Gamma)$ valga que:
	\[
	|T_{v}| = | \{  t \in V(T) : \ v \in f'(t)  \} | < \infty.
	\]
\end{prop}
\begin{proof}
	Sea $(T,f)$ una descomposición en un árbol de $\Gamma$ tal que tiene bagsize finito.
	Por la propiedad \textbf{T2} tenemos que para cada arista 
	$\{u,v\} \in E(\Gamma)$ existe al menos un vértice $t_{uv} \in V(T)$ de manera que $u,v \in f(t_{uv})$.
	
	Fijamos $v \in V(\Gamma)$ un vértice  de $\Gamma$.
	Consideremos el siguiente conjunto de vértices de $T$
	\[
		\{  t_{uv} \in V(T) \mid \exists u \in V(\Gamma). \ \{ u,v \} \in f(t_{uv})  \}.
	\]
	Notemos que es un conjunto finito porque por hipótesis $\Gamma$ es un grafo localmente finito.
	Sea $T'$ el subárbol finito de $T$ generado a partir de estos finitos vértices.
	
	Vamos a definir una descomposición en un árbol de $\Gamma$ usando este subárbol de manera que $|T_{v}| < \infty$.
	Nuestra descomposición resulta ser $(T,f_{v})$ con $f_{v}$ definida de la siguiente manera
	
	\[
	f_{v}(t) = 
	\begin{cases}
		f(t)  \ \  &\text{si} \ \  t \in V(T') \\
		f(t) \setminus \{  v \} \ \  &\text{si} \ \  t \notin V(T')
	\end{cases}
	\]
	
	
	Chequeamos que sea trata de una descomposición en un árbol.
	Las primeras dos condiciones \textbf{T1} y \textbf{T2} se siguen cumpliendo.
	La condición \textbf{T3} es válida porque si tomamos $w \in V(\Gamma)$, tal que $w \neq v$, por nuestra construcción $T_{w}$ era un subárbol de $T$ y similarmente para el vértice $v$ tenemos que 
	\[
		\{ t' \in V(T') \mid v \in f_{v}(t') \}
	\]	
	es un subárbol generado a partir de una cantidad finita de vértices por lo tanto $T_{v}$ también es un árbol.
	Con esto probamos que $(T,f_{v})$ es una descomposición en un árbol y como $|f_{v}(t)| \le |f(t)|$ para todo $t \in V(T)$ luego obtenemos que el bagsize de $(T,f_{v})$ también es finito.
	
	Hacemos este procedimiento para cada vértice $v \in V(\Gamma)$ de manera iterativa para obtener la descomposición en un árbol que buscábamos.
	
\end{proof}

\begin{deff}
	Sea $\Gamma$ un grafo no dirigido y $C \subseteq V(\Gamma)$ un conjunto de vértices entonces definimos los \emph{vecinos de $C$} por medio de 
	\[
	N(C) = \{ v \in V(\Gamma) \mid \exists w \in C, \ \{v,w \} \in E(\Gamma) \}.
	\]
	De esta manera dado $l \ge 2$ podemos definir recursivamente los \emph{l-ésimos vecinos} por medio de $N^l(C) = N(N^{l-1}(C))$.
\end{deff}


	Si al grafo no dirigido $\Gamma$ lo interpretamos como un espacio métrico entonces podemos escribir esta definición de la siguiente manera un poco más concisa
	\[
		N^l (C) = \{ v \in V(\Gamma) : \ \exists w \in C, \  d(v,w) \le l  \}.
	\]
	Veamos ahora que dada una descomposición en un árbol si tomamos los vecinos de los bolsones podemos armarnos otra descomposición en un árbol.


\begin{prop}\label{prop-vecinos-desc}
	Sea $\Gamma$ un grafo no dirigido y sea $(T,f)$ una descomposición en un árbol para $\Gamma$.
	Dado $l \in \NN$ consideramos $(T,g)$ tal que $g(t) = N^l(f(t))$ entonces $(T,g)$ resulta ser una descomposición en un árbol para $\Gamma$.
\end{prop}
\begin{proof}
	Probemos este resultado haciendo inducción en $l$.
	
	Consideremos el caso base $l=1$.
	En este caso, las dos primeras condiciones de la descomposición en un árbol \textbf{T1} y \textbf{T2} se siguen cumpliendo porque no hicimos más que agrandar los bolsones. 
	Esto es que $f(t) \subseteq g(t)$.
	
	Debemos ver que $(T,g)$ cumple \textbf{T3}.
	Queremos ver que si fijamos $v \in V(\Gamma)$ el conjunto 
	\[
		T_{v} = \{ t \in V(T) \mid v \in g(t)  \}
	\]
	forma un subárbol de $T$. 
	Para ver esto nos basta con ver que $T_{v}$ es conexo.
	Como los árboles son únicamente geodésicos podemos ver equivalentemente lo siguiente:
	dados $t_{1}, t_{2} \in T_{v}$ entonces para todo $t_{3}$ que esté en la geodésica $[t_{1}, t_{2}]$ vale que $t_{3} \in T_{v}$. 
	
	Como $v \in g(t_{1})$ entonces existe $w \in f(t_{1})$ adyacente (o bien podría ser exactamente $v$) a $v$.
	Similamente existe $u \in f(t_{2})$ adyacente (o bien podría ser exactamente $v$) a $v$.
	
	Consideramos el camino de tres vértices $\gamma = (w,v,u)$ y usamos la proposición \ref{prop-camino-desc} para concluir que, sin pérdida de generalidad, $v \in f(t_{3})$.
	De esta manera vemos que necesariamente $v \in g(t_{3})$ tal como queríamos ver.
		
	El paso inductivo se sigue directamente del caso base.
	Esto se debe a que si sabemos que $(T,g)$ con $g(t) = N^{l-1}(f(t))$ forma una descomposición de un árbol entonces podemos usar el caso base para ver que los vecinos de esta descomposición siguen siendo otra descomposición en un árbol y con esto terminamos de probarlo porque $N^l(f(t)) = N (N^{l-1}(f(t)))$.
\end{proof}
\medskip

\begin{coro}\label{coro_vecinos_bagsize_grado}
	Sea $\Gamma$ un grafo no dirigido con grado acotado uniformemente y con treewidth finito si $(T,f)$ es una descomposición en un árbol para $\Gamma$ tal que tiene bagsize finito luego dado $l \in \NN$ consideramos $(T,g)$ tal que $g(t) = N^l(f(t))$ entonces $(T,g)$ resulta ser una descomposición en un árbol para $\Gamma$ con bagsize finito.
\end{coro}
\begin{proof}
	Para ver esto basta notar que si existe $k \in \NN$ de manera que  $|f(t)| \le k$ para todo $t \in V(T)$ entonces como $\Gamma$ tiene grado acotado uniformemente por $n \in \NN$ luego para todo $t \in V(T)$ vale que
	\[
		|N^{l} (f(t))| \le kn < \infty.
	\]
	
\end{proof}

\begin{deff}
	Dado un grafo no dirigido $\Gamma$ y un conjunto de vértices $C \subseteq V(\Gamma)$ definimos el \emph{borde de vértices} de $C$ como
	\[
	\beta C =  N(C) \cap N(\ol{ C}).
	\] 
	El \emph{borde de aristas} de $C \subseteq V(\Gamma)$ se define como
	\[
	\delta C = \{  \{v,w \} \in E(\Gamma) \mid v \in C, w \in \ol C    \}.
	\]
\end{deff}


\begin{obs}\label{desc-grafo-cayley}%[Descomposición válida para todo grupo finitamente generado].
	
	Construyamos una descomposición en un árbol que podemos hacer en general para todos los grafos de Cayley de grupos finitamente generados. 
	Sea $G$ un grupo finitamente generado por $A$ y $\Gamma = \text{Cay}(G,A)$ el respectivo grafo de Cayley de $G$. 
	
	
	Definimos $V_l = \Gamma \setminus N^l(\{1\}) $ tal que $V_0 = \Gamma \setminus \{1\}$. 
	Los vértices del árbol $T$ van a estar dados por los siguientes conjuntos:
	\[
	V(T) = \{  \beta C : \exists l \in \NN / C \subseteq V_l \ \text{componente conexa} \} \cup \{ 1 \}.
	\]
	estamos tomando todos los posibles bordes de vértices 
	de todas las componentes conexas de todos los $V_l$ con $l \in \NN$.
	 
	Antes de definir las aristas notemos lo siguiente.
	Si $C$ es una componente conexa de $V_{l+1}$ entonces tiene que existir $D$ componente conexa de $V_{l}$ de manera tal que $C \subseteq D$.
	Esto se debe a que $V_{l+1} \subseteq V_{l}$ y a su vez como $C$ es conexo si interseca a una componente conexa necesariamente tiene que estar incluido en esta componente.
	Las aristas van a estar dadas por lo siguiente:
	\[
	E(T) = \{ \{ \beta C, \beta D \} : \exists l \in \NN / \ C \subseteq D \subseteq V_l \land \ C \subseteq V_{l+1}  \} \cup \{  \{1,\beta C\} : C \subseteq V_0  \}
	\]
	
	Probemos que el grafo $T$ resulta ser un árbol. 
	
	\begin{enumerate}[$\bullet$]
		\item \textbf{$T$ es conexo.}
		Probaremos la siguiente afirmación que va a implicar la conectividad de $T$:
		Para todo $\beta C \in V(T)$ existe un camino que lo conecte con $1$.
		Lo probaremos por inducción en $l$ siendo $l \in \NN$ tal que $C$ es una componente conexa de $V_{l}$.
		
		El caso base es $l = 0$ y en este caso tenemos que $\{ 1, \beta C  \}$ para toda componente conexa $C \in V_{0}$ por definición de las aristas de $T$.
		
		Para el paso inductivo suponemos que todo $\beta D$ para $D$ componente conexa de $V_{l}$ está conectado con $1$.
		Sea entonces $C$ componente conexa de $V_{l+1}$ entonces necesariamente tiene que existir $D$ componente conexa de $V_{l}$ tal que $C \subseteq D$ dado que $C$ es conexo y $V_{l+1} \subseteq V_{l}$.
		Esto nos dice que $\{ \beta C, \beta D \} \in V(\Gamma)$ y así como $\beta D$ está conectado con $1$ vimos que $\beta C$ también lo está.
		
		\item \textbf{$T$ es acíclico.}
		Probemos que todo camino cerrado en $T$ no es simple.
		Para eso sea $\gamma$ un camino cerrado (no constante sino no hay nada que probar) en $T$ tal que empieza y termina en $\beta E$ donde $E$ es alguna componente conexa para algún $V_{k}$.
		Sea $C$ tal que $\beta C$ es uno de los vértices del camino $\gamma$ y es componente conexa de $V_{l}$ con $l$ maximal en el camino.
		Supongamos sin pérdida de generalidad que $E$ es distinto de $C$ porque sino podemos modificar el comienzo y final del camino porque es un camino cerrado.
		Necesariamente tenemos que si $\{ \beta D, \beta C \}$ es una arista del camino entonces $D$ es componente conexa de $V_{l-1}$.
		Esto nos dice que el camino $\gamma$ tiene la siguiente pinta
		\[
			\gamma = (\beta E, \dots, \beta D, \beta C, \beta D, \dots, \beta E)
		\]
		por lo que el camino no es simple dado que tiene dos veces a la misma arista $\{ \beta C, \beta D \}$.
		
	\end{enumerate}
	  
	Sea entonces $f: V(T) \to 2^{V(\Gamma)}$ la función definida por $f(\beta C) = \beta C$ donde en el lado izquierdo lo miramos como un vértice del árbol $T$ y en el lado derecho como un conjunto de vértices de $\Gamma$.
	Afirmamos que $(T,f)$ es una descomposición en un árbol para el grafo de Cayley $\Gamma$.
	Para verlo debemos ver que cumple las tres condiciones \ref{desc-arbol} de la definición. 
	\begin{enumerate}
		\item[\textbf{T1.}] Sea $g \in V(\Gamma)$ luego si $d(1,g) = l$
		tenemos que $g \in C \subseteq V_{l-1}$ para cierta $C$ componente conexa de $V_{l-1}$ y en particular $g \in \beta C$.		 
		
		
		\item[\textbf{T2.}] Sea una arista $\{g,h\} \in E(\Gamma)$. 
		Separamos en dos casos dependiendo si $d(g,1) = d(h,1)$ o si no sucede esto.
		
		Supongamos que ambas están a la misma distancia del vértice $1$. 
		En tal caso sea $l$ tal que 
		\[
			d(g,1)= l = d(h,1)
		\] 
		luego tiene que existir $C$ componente conexa de $V_{l-1}$ 
		tal que $g,h \in C$ dado que ambos vértices están conectados.
		Tenemos en particular que $g,h \in \beta C$.
		
		
		El otro caso es que las distancias al vértice $1$ son distintas aunque necesariamente están restringidas a que sean del tipo
		\[
			d(g,1)= l < l+1 = d(h,1)
		\] 
		y en este caso resulta que $g,h \in \beta C$ si $C$ es la componente conexa que contiene a $h$ en $V_l$.
		
		\item[\textbf{T3.}] Queremos ver que si fijamos $g \in V(\Gamma)$ entonces el subgrafo $T_{g} = \{ t \in V(T) \mid g \in f(t) \} $ es un subárbol de $T$.
		Para eso notemos que si $d(g,1) = l$ luego $g \in \beta V_{l-1}$ y $g \in \beta V_{l}$ y para cualquier otro valor $j$ no vale que $g \in \beta V_{j}$.
		
		Existe una única componente conexa $D \in V_{l-1}$ de manera que $g \in \beta D$. 
		Por otro lado como el grafo de Cayley de un \fg es localmente finito entonces existen finitas componentes conexas $C_{1}, \dots, C_{k}$ de $V_{l}$ de manera que $g \in \beta C_{1}, \dots, g \in \beta C_{k}$.

		Notemos que $T_{g}$ es conexo porque para todo $\beta C_{i}$ vale que $\{ \beta D, \beta C_{i} \} \in E(T)$ por lo que $T_{g}$ resulta ser un subárbol de $T$ tal como queríamos ver.
		 
	\end{enumerate}
\end{obs}

\begin{ej}\label{ej_tw_c2c3}
	Sea $G = \ZZ / 2\ZZ \ast \ZZ / 3\ZZ $ tal que tiene la siguiente presentación $G \simeq \langle a,b \mid a^2, b^3 \rangle$.
	Tomamos el conjunto simétrico de generadores $\{ a, b, b^{-1} \}$ tal que al grafo $\text{Cay}(G, \{a,b,b^2\})$ lo representamos de la siguiente manera:
	
	
	\begin{center}
		\begin{tikzpicture}
			[scale=0.60,V/.style = {circle, draw,align= center, minimum size=0.5cm,
					minimum size=2em,inner sep=2,
					 fill=astral!15,font=\scriptsize	},fill fraction/.style={path picture={
							\fill[#1] 
							(path picture bounding box.south) rectangle
							(path picture bounding box.north west);
					}},
			fill fraction/.default=astral!90
			]
	
			\begin{scope}[nodes=V,xshift=4.5cm, yshift=-4cm]
					 \node (1) at (0,0) {$1$};
					\node (a) at (2,0)  {$a$};
					 \node (b) at (-2,2)     {$b$};
					\node (bb) at (-2,-2)    {$b^2$};
					\node (ab) at (4,2)      {$ab$};
					 \node (abb) at (4,-2)     {$ab^2$};
					\node (ba) at (-4,2)     {$ba$};
					\node (bba) at (-4,-2)     {$b^2a$};
					\node (aba) at (6,2)    {$aba$};
					\node (abba) at (6,-2)    {$ab^2a$};
					\node (bab) at (-6,3)    {$bab$};
					\node (babb) at (-6,1)     {$bab^2$};
					\node (bbab) at (-6,-3)     {$b^2ab$};
					\node (bbabb) at (-6,-1)    {$b^2ab^2$};
					\node (abab) at (8,3)    {$abab$};
					\node (ababb) at (8,1)    {$abab^2$};
					
					\node (abbab) at (8,-1)     {$ab^2ab$};
					\node (abbabb) at (8,-3)  {$ab^2ab^2$};
					
				\end{scope}
			
			\node[right=.1cm of abbab] {$\dots$};
			\node[right=.1cm of abbabb] {$\dots$};
			\node[left=.1cm of bab] {$\dots$};
			\node[left=.1cm of babb] {$\dots$};
			\node[left=.1cm of bbabb] {$\dots$};
			\node[left=.1cm of bbab] {$\dots$};
			\node[right=.1cm of abab] {$\dots$};
			\node[right=.1cm of ababb] {$\dots$};
			
			
			\draw   (1)  edge[-] (a);
			\draw   (1)  edge[-] (b);
			\draw   (1)  edge[-] (bb);
			\draw   (b)  edge[-] (bb);
			\draw   (b)  edge[-] (ba);
			\draw   (bb)  edge[-] (bba);
			\draw   (a)  edge[-] (ab);
			\draw   (abb)  edge[-] (a);
			\draw   (ab)  edge[-] (abb);
			\draw   (ab)  edge[-] (aba);
			\draw   (aba)  edge[-] (abab);
			\draw   (ababb)  edge[-] (aba);
			\draw   (ababb)  edge[-] (abab);
			\draw   (abb)  edge[-] (abba);
			\draw   (abba)  edge[-] (abbab);
			\draw   (abba)  edge[-] (abbabb);
			\draw   (bba)  edge[-] (bbabb);
			\draw   (bba)  edge[-] (bbab);
			\draw   (ba)  edge[-] (bab);
			\draw   (babb)  edge[-] (bab);
			\draw   (abbab)  edge[-] (abbabb);
			\draw   (ba)  edge[-] (babb);
			\draw   (bbabb)  edge[-] (bbab);
			\draw   (abbab)  edge[-] (abbabb);
	\end{tikzpicture}
	\end{center}	
	
	
	
	
	
	
	
	
	
		
%	\begin{tikzpicture}
%		[V/.style = {circle, draw,align= center, minimum size=0.5cm,
%			minimum size=2em,inner sep=2,
%			 fill=gray!15,font=\scriptsize	},fill fraction/.style={path picture={
%				\fill[#1] 
%				(path picture bounding box.south) rectangle
%				(path picture bounding box.north west);
%		}},
%		fill fraction/.default=gray!50
%		]
%		\node[align=center] at (5, 0) {Grafo \textbf{$\text{Cay}(\ZZ/2\ZZ \ast \ZZ / 3\ZZ, \{ a,b,b^2 \})$}};
%		\begin{scope}[nodes=V,xshift=4.5cm, yshift=-4cm]
%			 \node (1) at (0,0) [fill = brightcerulean!80,  fill fraction=brightube!90] {$1$};
%			\node (a) at (2,0)  [fill = cadmiumred!75,fill fraction=brightcerulean!80]{$a$};
%			 \node (b) at (-2,2) [fill fraction = carrotorange!90,fill =brightube!90	]    {$b$};
%			\node (bb) at (-2,-2) [fill fraction = aurometalsaurus!90,  fill =brightube!90]    {$b^2$};
%			\node (ab) at (4,2)  [fill fraction= cadmiumred!75,fill = astral!90]    {$ab$};
%			 \node (abb) at (4,-2) [fill fraction = cadmiumred!75,fill = applegreen!90]    {$ab^2$};
%			\node (ba) at (-4,2) [fill = carrotorange!90]    {$ba$};
%			\node (bba) at (-4,-2) [fill = aurometalsaurus!90]    {$b^2a$};
%			\node (aba) at (6,2) [fill = astral!90]    {$aba$};
%			\node (abba) at (6,-2) [fill = applegreen!90]    {$ab^2a$};
%			\node (bab) at (-6,3) [fill = carrotorange!90]    {$bab$};
%			\node (babb) at (-6,1) [fill = carrotorange!90]    {$bab^2$};
%			\node (bbab) at (-6,-3) [fill = aurometalsaurus!90]    {$b^2ab$};
%			\node (bbabb) at (-6,-1) [fill = aurometalsaurus!90]    {$b^2ab^2$};
%			\node (abab) at (8,3) [fill = astral!90]    {$abab$};
%			\node (ababb) at (8,1) [fill = astral!90]    {$abab^2$};
%			
%			\node (abbab) at (8,-1) [fill = applegreen!90]    {$ab^2ab$};
%			\node (abbabb) at (8,-3) [fill = applegreen!90]    {$ab^2ab^2$};
%			
%		\end{scope}
%		
%		\node[right=.1cm of abbab] {$\dots$};
%		\node[right=.1cm of abbabb] {$\dots$};
%		\node[left=.1cm of bab] {$\dots$};
%		\node[left=.1cm of babb] {$\dots$};
%		\node[left=.1cm of bbabb] {$\dots$};
%		\node[left=.1cm of bbab] {$\dots$};
%		\node[right=.1cm of abab] {$\dots$};
%		\node[right=.1cm of ababb] {$\dots$};
%		
%		
%		\draw   (1)  edge[-] (a);
%		\draw   (1)  edge[-] (b);
%		\draw   (1)  edge[-] (bb);
%		\draw   (b)  edge[-] (bb);
%		\draw   (b)  edge[-] (ba);
%		\draw   (bb)  edge[-] (bba);
%		\draw   (a)  edge[-] (ab);
%		\draw   (abb)  edge[-] (a);
%		\draw   (ab)  edge[-] (abb);
%		\draw   (ab)  edge[-] (aba);
%		\draw   (aba)  edge[-] (abab);
%		\draw   (ababb)  edge[-] (aba);
%		\draw   (ababb)  edge[-] (abab);
%		\draw   (abb)  edge[-] (abba);
%		\draw   (abba)  edge[-] (abbab);
%		\draw   (abba)  edge[-] (abbabb);
%		\draw   (bba)  edge[-] (bbabb);
%		\draw   (bba)  edge[-] (bbab);
%		\draw   (ba)  edge[-] (bab);
%		\draw   (babb)  edge[-] (bab);
%		\draw   (abbab)  edge[-] (abbabb);
%		\draw   (ba)  edge[-] (babb);
%		\draw   (bbabb)  edge[-] (bbab);
%		\draw   (abbab)  edge[-] (abbabb);
%	\end{tikzpicture}

Consideremos entonces cómo es la descomposición en un árbol construida en \ref{desc-grafo-cayley} para este grafo de Cayley.
El árbol $T$ lo representamos de la siguiente manera: 
   
%   	\begin{figure}[H]
%   		\centering
%   	\begin{tikzpicture}[scale=0.9,V/.style = {circle, draw,align= center, minimum size=0.5cm,
%   		minimum size=3em,inner sep=2,
%   		fill=applegreen!15,font=\small	},fill fraction/.style={path picture={
%   			\fill[#1] 
%   			(path picture bounding box.south) rectangle
%   			(path picture bounding box.north west);
%   	}},
%   	fill fraction/.default=gray!50
%   	]
%   	\node[align=center] at (9, 2) {};
%   	\begin{scope}[nodes=V,xshift=4.5cm, yshift=-4cm]
%   		\node (1) at (0,2)  {$\{ b,b^2,1 \}$};
%   		\node (2) at (0,0)  {$1$};
%   		\node (3) at (0,-2)  {$\{ 1,a \}$};
%   		\node (4) at (3,4)     {$\{ b,ba \}$};
%   		\node (5) at (-3,4)   {$\{b^2,b^2a\}$};
%   		\node (6) at (0,-4)  {$\{ a,ab,ab^2 \}$};
%  
%   		\node (7) at (5,7)  {$\{ba,bab,bab^2\}$};
%   		\node (8) at (-5,7)    {$\{b^2a,b^2ab,b^2ab^2\}$};
%   		\node (9) at (5,-4)  {$\{ab,aba\}$};
%   		\node (10) at (-5,-4)  {$\{ ab^2, ab^2a \}$};
%   		\node (11) at (8,-8)  {$\{aba,abab,abab^2\}$};
%   		\node (12)  at (-8,-8)   {$\{ab^2a,ab^2ab,ab^2ab^2\}$};
% 
%   	\end{scope}
%   
%   \node[below=.1cm of 12] {$\vdots$};
%   	\node[below=.1cm of 11] {$\vdots$};
%   	\node[above=.1cm of 7] {$\vdots$};
%   	\node[above=.1cm of 8] {$\vdots$};
%   
%   
%   \draw (1) edge[-] (2);
%   \draw (2) edge[-] (3);
%   \draw (1) edge[-] (5);
%   \draw (1) edge[-] (4);
%   \draw (7) edge[-] (4);
%   \draw (8) edge[-] (5);
%   \draw (3) edge[-] (6);
%   \draw (6) edge[-] (9);
%   \draw (6) edge[-] (10);
%   \draw (9) edge[-] (11);
%   \draw (10) edge[-] (12);
%   \end{tikzpicture}
%   	\end{figure}
%
\begin{figure}[H]
	\centering
	\begin{tikzpicture}[scale=0.70, V/.style = {circle, draw,align= center, minimum size=0.5cm,
			minimum size=3em,inner sep=2,
			fill=applegreen!15,font=\tiny	},fill fraction/.style={path picture={
				\fill[#1] 
				(path picture bounding box.south) rectangle
				(path picture bounding box.north west);
		}},
		fill fraction/.default=gray!50
		]
		\node[align=center] at (9, 2) {};
		\begin{scope}[nodes=V,xshift=4.5cm, yshift=-4cm]
			\node (1) at (0,0)  {$\{ b,b^2,1 \}$};
			\node (2) at (2,0)  {$1$};
			\node (3) at (4,0)  {$\{ 1,a \}$};
			\node (4) at (-2,3)     {$\{ b,ba \}$};
			\node (5) at (-2,-3)   {$\{b^2,b^2a\}$};
			\node (6) at (6,0)  {$\{ a,ab,ab^2 \}$};
			
			\node (7) at (-4,5)  {$\{ba,bab,bab^2\}$};
			\node (8) at (-4,-5)    {$\{b^2a,b^2ab,b^2ab^2\}$};
			\node (9) at (8,3)  {$\{ab,aba\}$};
			\node (10) at (8,-3)  {$\{ ab^2, ab^2a \}$};
			\node (11) at (10,5)  {$\{aba,abab,abab^2\}$};
			\node (12) at (10,-5)[] {$\{ab^2a,ab^2ab,ab^2ab^2\}$};
			
		\end{scope}
		
		\node[right=.1cm of 12] {$\dots$};
		\node[right=.1cm of 11] {$\dots$};
		\node[ left=.1cm of 7] {$\dots$};
		\node[ left=.1cm of 8] {$\dots$};
		
		
		\draw (1) edge[-] (2);
		\draw (2) edge[-] (3);
		\draw (1) edge[-] (5);
		\draw (1) edge[-] (4);
		\draw (7) edge[-] (4);
		\draw (8) edge[-] (5);
		\draw (3) edge[-] (6);
		\draw (6) edge[-] (9);
		\draw (6) edge[-] (10);
		\draw (9) edge[-] (11);
		\draw (10) edge[-] (12);
	\end{tikzpicture}
\end{figure}


Cada vértice $t \in V(T)$ es el bolsón correspondiente para la descomposición en un árbol de $\text{Cay}(G,\{ a,b,b^2 \})$.	
\end{ej}


\section{Cuasisometrías.}\label{secc_qi}

Nuestro objetivo es introducir las cuasisometrías para garantizar que la propiedad de tener treewidth finito no dependa de los generadores elegidos.
Esta subsección sigue los textos de \cite{bridson2013metric} y de \cite{loh2017geometric}.

\begin{deff}
	Sean $(X,d_X),(Y,d_Y)$ espacios métricos. 
	Una \emph{cuasisometría} es una función $\phi:X \to Y$ tal que:
	\begin{itemize}
		\item[\textbf{Q1.}] Existe constante $A > 0$ tal que para todo par de puntos $x_1,x_2 \in X$ hace valer la siguientes desigualdades
		\[
		\frac{1}{A} d_X(x_1,x_2) - A \le d_Y(\phi(x_1),\phi(x_2)) \le A d_X(x_1,x_2) + A
		\]
		\item[\textbf{Q2.}] Existe una constante $C \ge 0$ tal que para todo punto $y \in Y$ debe existir $x \in X$ de manera que 
		\[
		d(y,\phi(x)) \le C
		\]
	\end{itemize}
\end{deff}

Intuitivamente una cuasisometría entre espacios métricos nos dice que estos resultan ser globalmente similares y las diferencias que tienen localmente están uniformemente acotadas. 

\medskip
\begin{prop}\label{prop_qi_simetrica}
	Sean $(X,d_X),(Y,d_Y)$ espacios métricos. 
	Una función $\phi:X \to Y$ es una cuasisometría si y solo sí existe 
	$\psi:Y \to X$ cuasisometría y constantes positivas $C,D \in \RR$ tales que:
	\begin{itemize}
		\item $d(\psi \circ \phi (x), x) < C$ para todo $x \in X$.
		\item $d(\phi \circ \psi (y), y) < D$ para todo $y \in Y$. 
	\end{itemize}
\end{prop}
\begin{proof}
	Ver \cite[pp.84-85]{loh2017geometric}.
\end{proof}

\begin{prop}\label{prop_qi_transitiva}
	Sean $(X,d_X),(Y,d_Y), (Z, d_{Z})$ espacios métricos y sean $\phi:X \to Y$ y $\psi:Y \to Z$ cuasisometrías,
	 entonces $\psi \circ \phi: X \to Z$ es una cuasisometría.
\end{prop}
\begin{proof}
	Ver \cite[p.86]{loh2017geometric}.
\end{proof}




La relación de cuasisometría es reflexiva.
La proposición \ref{prop_qi_simetrica} nos dice que la relación de cuasisometría entre espacios métricos es simétrica.
Por la proposición \ref{prop_qi_transitiva} tenemos que esta relación es transitiva también. 
Esto nos dice que la relación de cuasisometría es una relación de equivalencia entre los espacios métricos.
\begin{deff}
	Dados $(X,d_{X})$ y $(Y,d_{Y})$ espacios métricos diremos que son espacios métricos \emph{cuasisométricos} si existe $\phi:X \to Y$ cuasisometría. 
	%	y denotaremos $X \overset{q.i}{\sim} Y$ la relación de cuasisometría entre espacios métricos.
\end{deff}



En nuestro caso en particular los espacios métricos que nos van a interesar son los grafos no dirigidos.
Más aún nos vamos a interesar en los grafos de Cayley de grupos finitamente generados.
El siguiente resultado que enunciamos sin demostración nos garantiza que todos los grafos de Cayley para un grupo finitamente generado son cuasisométricos entre sí.
\begin{prop}
	Sea $G$ grupo finitamente generado por $\Sigma$ y por $\Delta$ entonces $(\text{Cay}(G,\Sigma), d_{\Sigma})$ y $(\text{Cay}(G, \Delta), d_{\Delta})$ son cuasisométricos entre sí.
\end{prop}

\begin{proof}
	Ver \cite[p.89]{loh2017geometric}.
\end{proof}

Este resultado deja de ser cierto en el contexto que tomamos un conjunto de generadores que no es finito.


Si el grafo de Cayley de un grupo para cierto conjunto de generadores tiene una propiedad $P$ y queremos ver que esta propiedad $P$ es intrínseca al grupo, es decir que no depende del conjunto de generadores que tomemos, nos alcanza con probar que esta propiedad $P$ se preserva por cuasisometrías.
En nuestro caso en particular la propiedad $P$ que nos va a interesar es la de tener treewidth finito.


\begin{prop} \label{treewidth-inv}
	Sean $(\Gamma_{1},d_{1}), (\Gamma_{2}, d_{2})$ grafos no dirigidos de grado uniformemente acotado cuasisométricos entre sí y $\Gamma_{2}$ grafo con treewidth finito.
	Entonces $\Gamma_{1}$ tiene treewidth finito.
\end{prop}
%
%\begin{proof}
%	Sea $\phi:\Gamma_{1} \to \Gamma_{2}$ una cuasisometría entonces existe una constante $A > 0$ de manera que para todo par $v_{1}, v_{2} \in \Gamma_{1}$ tenemos que
%	\[
%	\frac{1}{A} d_X(v_1,v_2) - A \le d_Y(\phi(v_1),\phi(v_2)) \le A d_X(v_1,v_2) + A
%	\]
%	y existe una constante $C \ge 0$ tal que para todo $w \in \Gamma_{2}$ existe $v \in \Gamma_{1}$ de manera que
%	\[
%	d(w,\phi(v)) \le C.
%	\]
%	
%	Sea $(T,f)$ una descomposición en un árbol para $\Gamma_{1}$ tal que tiene bagsize finito.
%	Si consideramos a $g: V(T) \mapsto 2^{V(\Gamma_{1})}$ definida por $g(t) = N^{A(2C+1)}(f(t))$
%	luego $(T,g)$ es una descomposición en un árbol para $\Gamma_{1}$ por el lema \ref{prop-vecinos-desc}.
%	Consideremos entonces la siguiente función 
%	\begin{align*}
%	h: V(T)  &\mapsto 2^{V(\Gamma_{2})} \\
%	h(t) =& N^{C+1}(\phi(g(t)).) 
%	\end{align*}
%	
%	Afirmamos que $(T,h)$ es una descomposición en un árbol para $\Gamma_{2}$ tal que tiene bagsize finito.
%	
%	
%	Para ver esto debemos ver que cumple las tres condiciones que definen a una descomposición en un árbol.
%	\begin{enumerate}[T1.]
%		\item Sea $w \in V(\Gamma_{2})$ luego por la definición de cuasisometría existe $v \in V(\Gamma_{1})$ tal que $d(w, \phi(v)) \le C$.
%		En particular tenemos que si $w \in g(t)$ para $t \in V(T)$ luego  $w \in N^{C+1}(\phi(g(t))) = h(t)$ tal como queríamos ver.
%		
%		\item Sea $\{w,w'\} \in E(\Gamma_{2})$ luego si $v \in V(\Gamma_{1})$ es tal que $d(\phi(v),w) \le C$ entonces $d(\phi(v),w') \le C+1$.
%		Esto nos dice que si $v \in g(t)$ luego $w,w' \in h(t)$ tal como queríamos ver.
%		
%		\item Veamos que el conjunto 
%		\[
%			T_{w} = \{ t \in V(T) \mid w \in h(t)   \}
%		\]
%		es un subárbol de $T$.
%		Para esto vamos a ver que si $w \in h(t)$ y $w \in h(s)$ entonces para todo $r \in V(T)$ tal que $r \in [t,s]$ tenemos que $w \in h(r)$.
%		
%		Como $w \in h(t)$ entonces existe $v \in g(t)$ de manera que $d(\phi(v),w) \le C$ y similarmente existe $v' \in g(s)$ de manera que $d(\phi(v'),w) \le C$.
%		Esto nos dice que 
%		$ d(\phi(v), \phi(v')) \le 2C$
%		y así usando que $\phi$ es una cuasisometría tenemos la siguiente cota
%		\[
%			d(v,v') \le A(2C)+A
%		\]
%		por lo que $v,v' \in g(t) \cap g(s)$.
%		\todo{Esto está mal----}
%		Entonces como $(T,g)$ es una descomposición en un árbol para $\Gamma_{1}$ tenemos que para todo $r \in [t,s]$ vale que $v,v' \in g(r)$.
%		Esto implica que $w \in h(r)$ tal como queríamos ver.
%	\end{enumerate}
%	
%	Por el resultado \ref{coro_vecinos_bagsize_grado} tenemos que $(T,g)$ tiene bagsize finito dado que $(T,f)$ lo tiene. 
%	De esta manera también concluimos que $(T,h)$ tiene bagsize finito por lo que $ \Gamma_{2} $ tiene treewidth finito tal como queríamos ver.
%\end{proof}

\begin{proof}
	Sea $\phi:\Gamma_1 \to \Gamma_2$ cuasisometría y sea $k \in \NN$ el treewidth de $\Gamma_2$.
	Al ser $\phi$ una cuasisometría tenemos que existe $A > 0$ tal que para todo par $v,v' \in V(\Gamma_{1})$ vale que
	\[
	\frac{1}{A} d_{1}(v,v') - A \le d_{2}(\phi(v),\phi(v')) \le A d_{1}(v,v') + A.
	\]
	
	Probemos que $\Gamma_1$ tiene treewidth finito también.
	
	Consideremos $l$ tal que $d(\phi(v),\phi(w)) \le l$ para vértices $v,w \in V(\Gamma_1)$ que estén conectados por una arista.
	Esto lo podemos tomar porque al ser una cuasisometría 
	\[
	d_{2}(\phi(v),\phi(v')) \le A d_{1}(v,v') + A  \le 2A
	\]
	entonces basta con tomar $l \ge 2A$.
	
	Sea $(T,f)$ descomposición en un árbol para $\Gamma_{2}$ tal que tiene treewidth finito.
	Por \ref{prop-vecinos-desc} si tomamos $N^l(f(t))$ los vecinos del bolsón $f(t)$ que están a distancia no mayor a $l$ seguimos teniendo una descomposición en un árbol para $\Gamma_{2}$.  
	Consideraremos los bolsones $g(t) = \phi^{-1}(N^l(f(t)))$ de vértices en $\Gamma_1$. 
	Probemos que $(T,g)$ es una descomposición en un árbol para $\Gamma_{1}$.
	
	Debemos ver que cumplen las tres propiedades.
	
	\begin{enumerate}
		\item[\textbf{T1.}] La primera se cumple puesto que los bolsones $f(t)$ cubren $V(\Gamma_2)$. 
		De esta manera $\bigcup_{t \in T} N^l(f(t)) = V(\Gamma_2)$ y por lo tanto tomando preimagen tenemos que
		\[
		\bigcup_{t \in V(T)} \phi^{-1} (N^l (f(t))) = \bigcup_{t \in V(T)} g(t) = \phi^{-1} (V(\Gamma_2)) = V(\Gamma_1)
		\] 
		donde usamos que la preimagen de la unión es la unión de las preimágenes.
		\item[\textbf{T2.}] La segunda condición usamos que si hay una arista $\{x,y\} \in E(\Gamma_2)$ luego debe ser que $d(\phi(x),\phi(y)) \le l$ por como tomamos a $l$.
		De esta manera como $\phi(x) \in f(t)$ para algún $t \in V(T)$, notemos que $\phi(y) \in N^l(f(t))$ también. 
		Tomando preimagen tenemos que $x,y \in \phi^{-1}(N^l(f(t)))$ y esto es que justamente $x,y \in g(t)$ para un mismo $t \in V(T)$ tal como queríamos ver.		
		\item[\textbf{T3.}] Para la tercera condición si $x \in g(t) \cap g(s)$ queremos ver que $x \in g(r)$ para todo $r \in V(T)$ que aparezca en la geodésica de $s$ a $t$.
		Como la preimagen de una intersección es lo mismo que la intersección de las preimágenes entonces 
		\[
		x \in \phi^{-1}(N^l(f(t))) \cap \phi^{-1}(N^l(f(s))) = \phi^{-1}(N^l(f(t)) \cap N^l (f(s))
		\]
		de esta manera debe existir $v \in V(\Gamma_2)$ tal que $v \in N^l(f(s)) \cap N^l(f(t))$.
		Ahora usamos que esta es una descomposición sobre $\Gamma_2$ para notar que $v \in N^l(f(r))$.
		Tomando preimagen tenemos que $x \in g(r)$ tal como queríamos ver.
	\end{enumerate}
	
	Finalmente debemos ver que el tamaño de los bolsones está acotado uniformemente para probar que $\Gamma_{1}$ tiene treewidth finito.
	Dado que $\Gamma_2$ tiene treewidth finito entonces existe $M \in \NN$ de manera que  $|f(t)| \le M$ uniformemente para todo $t \in V(T)$. 
	Como el grado de los grafos está acotado uniformemente tenemos que por el resultado \ref{coro_vecinos_bagsize_grado} que esta descomposición tiene el bagsize finito por lo que
	\[
	\exists N \in \NN, \forall t \in V(T). \  |N^l(f(t))| \le N.
	\]
	Finalmente notemos que al ser $\phi$ una cuasisometría tenemos que existe $B > 0$ tal que para todo $v \in V(\Gamma_2)$ vale que $|\phi^{-1}(v)| \le B$.
	Esto lo podemos ver porque si $\phi(x) = v = \phi(y)$ entonces
	\[
	\frac{1}{A}d_{1}(x,y) - A \le d_{2}(\phi(x), \phi(y) ) = 0 \implies d_{1}(x,y) \le A^2 < \infty
	\]
	si tomamos $B > A^2$ obtenemos una cota uniforme para todo $v \in \Gamma_2$. 
	Así vemos que:
	\[
	|g(t)| = |\phi^{-1}(N^l(X(t)))| \le BN < \infty
	\]
	Concluímos así que la descomposición que nos armamos para $\Gamma_1$ tiene treewidth finito.
\end{proof}

De esta manera podemos definir el treewidth finito como un invariante para grupos.

\begin{deff}
	Un \emph{grupo $G$ finitamente generado tiene treewidth finito} si para algún conjunto de generadores $A$ vale que $\text{Cay}(G,A)$ tiene treewidth finito.
\end{deff}

%A partir de este resultado podemos ver que la otra manera que teníamos de pensar a los grafos que se parecen a árboles resulta ser más débil. 
%El siguiente resultado lo demostramos en el caso general de un grafo tal que los grados de sus vértices están acotados uniformemente. 
%Como caso particular tenemos los grafos de Cayley de grupos finitamente generados.

\section{Grupos \ic tienen treewidth finito.}\label{secc_MuSch}

En esta sección probaremos un resultado que fue originalmente probado por Muller--Schupp en \cite{muller1985theory}.
En ese trabajo usan una definición que resultó ser equivalente al treewidth que es la de ser $k$-triangulable.
La demostración que presentamos sigue la exposición del trabajo \cite{diekert2017context}.
Primero unos lemas sobre el lenguaje del problema de la palabra de un grupo independiente de contexto.


\begin{lema}\label{palabras-wp}
	Sea $G$ un grupo \ic con gramática $\gramatica$ de manera que $L(\cG) = WP(G,\Sigma)$.
	Sea $A \in V$ una variable de esta gramática y consideremos el lenguaje
	\[
	L_A = \{ w \in \Sigma^*  \ | \ A \deriva w  \}.
	\]
	Sea $\pi:\Sigma^* \to G$ la proyección del monoide libre al grupo.
	Entonces vale el siguiente resultado:
	dadas palabras $v,v' \in L_{A}$ luego $\pi(v) = \pi(v')$.
\end{lema}

\begin{proof}
	Veamos que si $v,v' \in L_A$ entonces $\pi(v){=} \pi(v')$. 
	Es decir son el mismo elemento vistos en el grupo $G$. 
	Para eso si tenemos una derivación que en algún momento llega a $S \deriva \beta A \gamma \deriva uvw$ también tenemos otra derivación que deriva en $S \deriva \beta A \gamma  \deriva uv'w$. 
	Es decir que $uvw, u'v'w' \in \WP$ por lo tanto 
	\begin{equation*}
		\pi(uvw) = 1 = \pi(uv'w) \implies \pi(v) = \pi( v')
	\end{equation*}
	tal como queríamos ver.
\end{proof}

\begin{lema}\label{lema_tw_diam_bagsize}
	Sea $\Gamma$ un grafo de grado acotado uniformemente y 
	$(T,f)$ una descomposición en un árbol para $\Gamma$ de manera que existe $M \in \NN$ tal que para todo $t \in V(T)$ tenemos la siguiente cota
	\[
		\text{diam}( f(t)) < M
	\]   
	entonces la descomposición $(T,f)$ tiene bagsize finito.
\end{lema}
\begin{proof}
	Buscamos $k \in \NN$ tal que nos permita acotar $|f(t)| \le k$ para todo $t \in V(T)$. 
	Como exista $M \in \NN$ tal que 
	\[
	\text{diam}(f(t)) =  \sup_{g,h \in f(t)} d(g,h) \le M
	\] 
	para todo $t$ 
	entonces al ser $\Gamma$ de grado acotado uniformemente por $n \in \NN$ luego 
	\[
		|f(t)| \le n^{M} < \infty
	\]
	para todo $t \in V(T)$ así probando que $(T,f)$ es una descomposición con bagsize finito.
\end{proof}



\begin{teo} \label{teo_ic_implica_tw}\cite{muller1985theory}
	Sea $G$ un grupo \ic entonces $G$ tiene treewidth finito.
\end{teo}
\begin{proof}
	Sea $\Sigma$ un conjunto finito de generadores de $G$.
	La descomposición en un árbol $(T,f)$ que hicimos en \ref{desc-grafo-cayley} es válida para todo grafo de Cayley en particular resulta serlo para $\text{Cay}(G,\Sigma)$.
	Veamos que esta descomposición para $G$ tiene treewidth finito. 
	Para esto vamos a usar el lema \ref{lema_tw_diam_bagsize} para probar que todo $\beta C \in V(T)$ cumple que $\text{diam}(\beta C) < \infty$.
	
	
	Dado que $G$ es un grupo independiente de contexto entonces el lenguaje del problema de la palabra para estos generadores $\WP$ tiene una gramática $ \gramatica$ independiente de contexto que lo genera. 
	Por la proposición \ref{prop_fn_Chomsky} consideremos que la gramática está en forma normal de Chomsky.
	
	Para cada variable $A$ de nuestra gramática podemos considerar el siguiente lenguaje:
	\[
	L_A = \{ w \in \Sigma^* : A \deriva w  \}.
	\]
	Para este lenguaje introduzcamos un número natural $k_A \in \NN$ definido por $k_A = {\min}_{w \in L_A} |w|$. 
	Como tenemos finitas variables en nuestra gramática $\cal G$ podemos considerar $k = \max_{A \in V} k_A$. 
	Veamos que $\text{diam}(\beta C) \le 3k$ para todo $\beta C \in V(T)$.
	
	Sea $C$ componente conexa de $V_{n}$ para algún valor de $n \in \NN$ genérico.
	Sean $g,h \in \beta C$, vamos a acotar $d(g,h)$. 
	Para eso consideremos una geodésica $\alpha$ que una $1$ con $g$ y análogamente otra geodésica $\gamma$ que una $1$ con $h$. 
	Como $C \cup \beta C$ es conexo podemos tomar un camino $\tau$ que una $g$ con $h$ dentro de $C$ exceptuando sus extremos. 
	Sin pérdida de generalidad supongamos que $|\tau| \ge 2$ caso contrario tendríamos que $g$ y $h$ son adyacentes y no tendríamos nada que probar porque en este caso $d(g,h) =1$.
	De esta manera tenemos un triángulo tal que tiene los lados tienen las siguientes etiquetas:
	$l(\alpha) = u $, $l(\tau) =v$ y $l(\gamma) = w$.
	Como $\alpha \tau \gamma$ es un ciclo en el grafo de Cayley entonces $l(\alpha \tau \gamma) = uvw \in  \WP$ por el resultado \ref{obs_grafo_Cayley_palabras} y por lo tanto tenemos alguna derivación $S \deriva uvw$.
	
	Ya que tenemos esta derivación $S \deriva uvw$ consideramos la última variable que deriva a $v$ como subpalabra. 
	Esto es que para la subpalabra $v$ sabemos que existe alguna variable $A$ y palabras posiblemente vacías $v',v''$ tal que $A \deriva v'v''$ donde $v$ es una subpalabra de $v'v''$. 
	Tomamos la última variable que aparece en la derivación antes de que aparezca $v$ como subpalabra en la derivación.
	Ésta variable debe existir porque en particular $S$ cumple que deriva a una palabra que contiene a $v$ como subpalabra.
	
	Como la gramática $ \cG $ está en forma normal de Chomsky sabemos que al suponer que $|v| \ge 2$ entonces la derivación tiene la siguiente pinta
	\begin{equation*}
		S \deriva u'Aw' \to_{\cal G} u'BC w' \deriva u'v'v''w'
	\end{equation*}
	donde $B,C$ son otras variables. 
	En particular notamos que $A \deriva v'v''$, $B \deriva v'$ y $C \deriva v''$.
	
	
	La palabra $u'$ es un prefijo de $u$ luego si consideramos la geodésica $\alpha$ tenemos que al haber leído la etiqueta $u'$ llegamos a un vértice $x$ y por estar sobre la geodésica cumple la siguiente igualdad:
	\begin{equation*}
		d(x,g) = d(1,g) - d(1,x).
	\end{equation*}
	Análogamente $w'$ es un prefijo de $w$ luego si consideramos la geodésica $\gamma$ en la instancia que ya leímos $w'$ comenzando desde $h$ llegamos a cierto vértice $z$ y por la misma razón que en el caso anterior obtenemos
	\begin{equation*}
		d(z,h) = d(1,h) - d(1,z).
	\end{equation*}

	Consideremos el vértice $y$ al que llegamos después de leer $u'v'$.
	Este vértice está en el camino $\tau$ dado que $v$ es subpalabra de $v'v''$.
	Usando que $y \in \tau \subseteq C $ tenemos que $d(1,y) \ge n+1 = d(1,g)$ por ser $C$ una componente conexa de $V_n$, entonces vale la siguiente desigualdad
	\begin{equation*}
		d(x,g) = d(1,g) - d(1,x) \le d(1,y) - d(1,x) = d(x,y)
	\end{equation*}
	donde usamos que $x,y$ están en una misma geodésica.
	Análogamente tenemos que $d(z,h) \le d(z,y)$.
	
	
	Por el lema \ref{palabras-wp} notemos que si reemplazamos $v'$ por la palabra de menor tamaño del lenguaje $L_B$ seguimos teniendo un ciclo pero de longitud idéntica o más chica. 
	La palabra $v'$ la leemos justamente cuando vamos del vértice $x$ al vértice $y$, así la distancia  $d(x,y)$ está acotada por la palabras que puede derivarse de $B$ de menor longitud. 
	Idénticamente hacemos esto para las variables $A$ y $C$.
	Por como definimos a $k$ tenemos las siguientes cotas: 
	$d(x,y), d(y,z), d(x,z) \le k$.

	
	Terminamos de probar que $d(g,h) \le 3k$. 
	Usamos la desigualdad triangular tres veces,
	\begin{align*}
		d(g,h) & \le d(g,x) + d(x,z) + d(h,z) \\
		& \le d(x,y) + d(x,z) + d(y,z) \le 3k
	\end{align*}
	tal como queríamos ver.
	
\end{proof}	

\begin{figure}
\centering
\begin{tikzpicture}[font=\sffamily]
		\path (-1,2) coordinate (A) (8,3) coordinate (B) (8,-8) coordinate (C);
		\draw[thick,path picture={
			\foreach \X in {A,B,C}
			{\draw[line width=0.4pt] (\X) circle (0);}}] (A) node[left]{$g$} to[bend right=10] node (y) [draw,circle,fill = black,pos=0.82,minimum size=.1cm, inner sep=0pt,label= above:y] {}
		(B) node[above right]{$h$} to[bend right=10] node (z) [draw,circle,fill = black,pos=0.62,minimum size=.1cm, inner sep=0pt,label= right:z] {}
		(C) node[below]{$1$} to[bend right=8] node (x) [draw,circle,fill = black,pos=0.72,minimum size=.1cm, inner sep=0pt,label= left:x] {} cycle;
		\draw[line width = 0.8pt] (y) to (z) ;
		\draw[line width = 0.8pt] (z) to (x) ;
		\draw[line width = 0.8pt] (y) to (x) ;
		\node[label=left:B] at (barycentric cs:x=1,y=1) {};
		\node[label=right:C] at (barycentric cs:y=1,z=1) {};
		\node[label=below:A] at (barycentric cs:x=1,z=1) {};
		\node[label=right:{\Large{$\alpha$}}] at (2,-2) {};
		\node[label=right:{\Large{$\tau$}}] at (2.4,2.5) {};
		\node[label=right:{\Large{$\gamma$}}] at (7.5,-1.5) {};
	
\end{tikzpicture} 
	\caption{Diagrama que muestra como acotar la distancia entre $g$ y $h$.}
\end{figure}

	
	
\end{document}