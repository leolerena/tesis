\documentclass[aspectratio=169, 9pt]{beamer}
\usepackage[utf8]{inputenc}
\usepackage[spanish]{babel}

\usepackage{xcolor}
\usepackage{amssymb}
\usepackage{pifont}
\newcommand{\xmark}{\ding{55}}
\usetheme[progressbar=frametitle]{metropolis}
\usepackage{appendixnumberbeamer}
\usepackage{hyperref}
\usepackage{eufrak}
\usepackage{tikz-cd}
\usepackage{tcolorbox}
\usepackage{enumitem}% http://ctan.org/pkg/enumitem
	\definecolor{ao(english)}{rgb}{0.0, 0.5, 0.0}

\setbeamercolor{background canvas}{bg=white}
\usepackage{multicol}
\usepackage{chronology}
%%%%%%%%%%%%%%%%%%%%% ENUMERAR CON COSAS QUE NO SEAN SOLO NÚMEROS %%%%%%%%%%
%\usepackage[shortlabels]{enumitem}
%\setlist[enumerate]{font=\bfseries}

\usepackage{booktabs}
%\usepackage[scale=2]{ccicons}

\usepackage{pgfplots}
%\usepgfplotslibrary{dateplot}
\colorlet{verde}{green!50!black}
\definecolor{amber}{rgb}{1.0, 0.49, 0.0}
\setbeamercolor{progress bar}{fg=amber,bg=alerted text.fg!50!black!10}

\makeatletter
\setlength{\metropolis@titleseparator@linewidth}{2pt}
\setlength{\metropolis@progressonsectionpage@linewidth}{2pt}
\setlength{\metropolis@progressinheadfoot@linewidth}{2pt}
\makeatother

\definecolor{ultramarine}{RGB}{81, 131, 232} 
\setbeamercolor{frametitle}{bg= ultramarine}

\usepackage{framed}
\definecolor{shadecolor}{gray}{0.9}

\renewcommand\qedsymbol{\textcolor{orange}{$\blacksquare$}}

\usepackage[font={footnotesize}]{caption}

\usepackage{xspace}
\newcommand{\themename}{\textbf{\textsc{metropolis}}\xspace}
\newcommand{\fg}{finitamente generado }
\newcommand{\fp}{finitamente presentado }
\newcommand{\rep}{\textsc{rep}}
\newcommand{\coin}{\textsc{coincidencia}}
\newcommand{\mer}{\textsc{merge}}
\newcommand{\scan}{\textsc{scan}}
\newcommand{\definir}{\textsc{definir}}
\newcommand{\scanfill}{\textsc{scanandfill}}
\newcommand{\ded}{\textsc{deducir}}
\newcommand{\recorrer}{\textsc{recorrer}}
\newcommand{\In}{[1 \dots n]}
\newcommand{\ol}{\overline}
\newcommand{\Co}{\cal{C}}



\title{Algoritmo de enumeración de cosets.}
\subtitle{Final de teoría de grupos.}
\date{\today}
\author{Leopoldo Lerena}
\institute{Universidad de Buenos Aires}
% \titlegraphic{\hfill\includegraphics[height=1.5cm]{logo.pdf}}

\begin{document}
\maketitle

\begin{frame}[fragile]{}

El algoritmo de enumeración de cosets nos permite responder, en algunos casos, las siguientes preguntas elementales de un grupo $G$ conociendo una presentación suya.

\begin{itemize}
	\item ¿Cuál es su orden?
	\pause
	\item ¿Cuál es 	el índice de un subgrupo $H$ dado por generadores?
	\pause
	\item ¿Cómo es su grafo de Cayley?
\end{itemize}
\end{frame}

\begin{frame}[fragile]{Algoritmos en teoría de grupos.}
	Muchos preguntas elementales de la teoría de grupos tales como:
	\begin{itemize}
		\item ¿Es un grupo finito?
		\pause
		\item Más aún, ¿es el grupo trivial?
	\end{itemize}
	\pause
	No son \textit{decidibles}. 
	Esto es que no existe un algoritmo que nos pueda responder estas preguntas para cualquier grupo que le preguntemos.
	\medskip
	
	\pause
	
	
	Nos tenemos que conformar con que existan procedimientos que son \textit{semidecidibles}. 
	Esto es que si resulta que este grupo cumple estas propiedades podremos probarlo
	que este procedimiento da la respuesta correcta.
\end{frame}

\begin{frame}[fragile]{Descripción del algoritmo.}
	El procedimiento de enumeración de cosets requiere los siguientes parámetros para la entrada:
	\begin{itemize}
		\item Un grupo $G$ \fp;
		\pause
		\item Una presentación finita de este grupo $\langle X | R \rangle$;
		\pause
		\item Un subgrupo $H \le G$  \fg con generadores $Y$ visto como palabras en $A = X \cup X^{-1}$.
	\end{itemize}
	\pause
	\medskip
	
	En el caso que el procedimiento termine nos va a devolver lo siguiente:
	\begin{itemize}
		\item El índice del subgrupo $H$ en $G$.
		\pause
		\item La representación de $G$ por permutaciones actuando a derecha por multiplicación sobre los cosets a derecha de $H$.
		\pause 
	\end{itemize}  
\end{frame}

\begin{frame}[fragile]{Un poco de historia.}
	El primer algoritmo de esta familia fue construido por Todd--Coxeter en el paper ""
	\alert{Agregar imagen del paper}
	con la idea de poder hacer manualmente este conteo de  
\end{frame}

\begin{frame}[fragile]{}
	\section{Ejemplo a mano del procedimiento.}
	Partamos del grupo \fp $G = \langle x, y | x^2, y^3, xyXY \rangle$.
	\pause
	Tomemos como subgrupo $H = \langle x \rangle$.
	\pause
	El algoritmo entonces lo que nos debería devolver es $|G:H| = 3$.
	\pause
	\medskip
	
	
	Empezamos denotando con \textbf{1} al coset trivial correspondiente a $H$ en $G/H$.
	\pause 
	Como $x \in H$ entonces sabemos que $Hx = H$, denotaremos por $\textbf{1}^x = \textbf{1}$ a la multiplicación de $\textbf{1}$ por $x$.
	Similarmente $\textbf{1}^{X} = \textbf{1}$.
	\pause
	Por otro lado denotemos con \textbf{2} al coset $\textbf{1}^y$ y denotaremos por \textbf{3} al coset $\textbf{1}^{Y}$.
	\pause
	
	Por ahora esto es lo que sabemos de la acción de $G$ sobre los cosets.
	\begin{table}[]
		\begin{tabular}{|l|l|l|l|l|}
			\hline
			Coset     & x          & X          & y          & Y          \\ \hline
			\textbf{1} & \textbf{1} & \textbf{1} & \textbf{2} & \textbf{3} \\ \hline
			\textbf{2} &            &            &            &  \textbf{1}          \\ \hline
			\textbf{3} &            &            &     \textbf{1}       &            \\ \hline
		\end{tabular}
	\end{table}
\end{frame}

\begin{frame}[fragile]{Continuación del ejemplo.}
	
\begin{alertblock}{Pregunta.}
	¿Qué podemos hacer?
\end{alertblock}
\pause
Sabemos que todo coset $\alpha$ es tal que $\alpha^r = \alpha$ para todo $r \in R$.
\pause

Vamos a introducir un proceso que se llama \textit{escanear las relaciones}.

\pause
Por ejemplo tomemos $r = y^3$ y escaneamos a $\textbf{1}$.


\begin{table}[]
	\begin{tabular}{|l|l|l|l|l|}
		\hline
		Coset     & x          & X          & y          & Y          \\ \hline
		\textbf{1} & \textbf{1} & \textbf{1} & \textbf{2} & \textbf{3} \\ \hline
		\textbf{2} &            &            &            &  \textbf{1}          \\ \hline
		\textbf{3} &            &            &     \textbf{1}       &            \\ \hline
	\end{tabular}
\end{table}
\pause

Tenemos que $\textbf{1}^y = \textbf{2}$ por como definimos al coset $\textbf{2}$. 
Por otro lado tenemos que $\textbf{1}^{y^2} = \textbf{2}^y$ todavía no está definido.
\pause


Como $r^{-1} = 1$ visto como elemento en $G$ si escanemos ahora viendo esta relación tenemos que $\textbf{1}^{Y} = \textbf{3}$ porque así lo definimos y por otro lado que $\textbf{3}^{Y}$ no está definido.
\pause

\end{frame}

\begin{frame}[fragile]{Continuación}
	De lo anterior podemos \textit{deducir} que $\textbf{2}^y = \textbf{3}$.
	
	\medskip
	\begin{center}
		\begin{tikzcd}
		\textbf{1} \arrow[r, "y", bend left] & \textbf{2} \arrow[r, "y", dotted, bend left] & \textbf{3} \arrow[r, "y", bend left] & \textbf{1}
		\end{tikzcd}
	\end{center}
	\pause
	
	De esta manera nuestra tabla ahora tiene la siguiente pinta.
	

	
	
	\begin{table}[]
		\begin{tabular}{|l|l|l|l|l|}
			\hline
			Cosets     & x          & X          & y          & Y          \\ \hline
			\textbf{1} & \textbf{1} & \textbf{1} & \textbf{2} & \textbf{3} \\ \hline
			\textbf{2} &            &            & \color{verde}\textbf{3} &     \textbf{1}       \\ \hline
			{\textbf{3}} &            &            &   \textbf{1}         & \color{verde}\textbf{2} \\ \hline
		\end{tabular}
	\end{table}
	
	
\end{frame}

\begin{frame}[fragile]{Continuación}
	Cuando escaneamos la relación $xyXY$ para los cosets \textbf{1} y \textbf{2} 
	llegamos a las siguientes deducciones 
	
	\begin{center}
		\begin{tikzcd}
		1 \arrow[r, "x", bend left] & 1 \arrow[r, "y", bend left]         & 2 \arrow[r, "X", dotted, bend left]            & 2 \arrow[r, "Y", bend left] & 1 \\
		2 \arrow[r, "x", bend left] & 2 \arrow[r, "y", bend left]         & 3 \arrow[r, "X", dotted, bend left]            & 3 \arrow[r, "Y", bend left] & 2
		\end{tikzcd}
	\end{center}

\pause
	Así terminamos de completar la tabla.
	
	\begin{table}[]
		\begin{tabular}{|l|l|l|l|l|}
			\hline
			Cosets     & x          & X          & y          & Y          \\ \hline
			\textbf{1} & \textbf{1} & \textbf{1} & \textbf{2} & \textbf{3} \\ \hline
			\textbf{2} & \color{verde}\textbf{2} & \color{verde}\textbf{2} & \textbf{3} & \textbf{1} \\ \hline
			\textbf{3} & \color{verde}\textbf{3} & \color{verde}\textbf{3} & \textbf{1} & \textbf{2} \\ \hline
		\end{tabular}
	\end{table}
	
	El procedimiento terminó y obtenemos que el índice del subgrupo $H$ en $G$ resulta ser la cantidad de cosets que nos terminaron quedando, en este caso exactamente $3$.
\end{frame}

\begin{frame}[fragile]{Tabla de cosets.}
	\section{Tabla de cosets.}
	Sea $G = \langle X | R \rangle$ un grupo \fp y sea $A = X \cup X^{-1}$ sus generadores y sus inversos.
	\pause
	
	La tabla de cosets $\Co $ va a ser una tupla $(n, \tau, \chi, p)$ donde:
	\pause
	\begin{itemize}
		\item $n$ es la cantidad de cosets que tenemos definidos, esto es el intervalo $\In$ va a representar a todos los cosets definidos;
		\pause
		\item $\tau : \In \to A^*$ es una función que a cada coset $\alpha$ le asigna una palabra en los generadores tal que $\tau(\alpha)$ sea un representante de la clase de este coset.
		\pause
		\item $\chi: \In \times A \to \In$ una función parcial que (cuando está definida) tal que $\chi(\alpha, x) = \beta \iff \alpha^x = \beta$.
		\pause
		\item $p: [1 \dots n] \to [1 \dots n]$ una función tal que $p(\alpha) \le \alpha$ para todo $\alpha \in [1\dots n]$.
	\end{itemize}
	
\end{frame}

\begin{frame}[fragile]{Continuación}
	Definimos los \emph{cosets vivos} como $\Omega = \{ \alpha \in \In  \ | \ p(\alpha) = \alpha \}$.
	Muchas veces nos va suceder que dos valores $\alpha, \beta \in \In$ se corresponden al mismo coset.
	El rol de $p$ es diferenciarnos qué elementos de $\In$ efectivamente se corresponden a distintos cosets.
	\pause
	\medskip
	
	
	
	La tabla está \emph{completa} si la restricción $\chi:\Omega \times A \to \In$ está definida para todos los pares $(\alpha, x) \in \Omega \times A$. 
\end{frame}

\begin{frame}[fragile]{Propiedades de la tabla de cosets.}
	La tabla de cosets vamos a querer que mantenga ciertas propiedades a medida que corramos el algoritmo, esto vendría a ser el invariante de representación.
	\pause
	
	
	
	\begin{enumerate}
		\item $1 \in \Omega$ y $1^{\tau(\alpha)} = \alpha$.
		\pause
		\item $\alpha^x = \beta \iff \beta^{X} = \alpha$.
		\pause
		\item Si $\alpha^x = \beta$ entonces tenemos que $H \tau(\alpha) x = H \tau(\beta)$
		\pause
		\item Para todo $\alpha \in \Omega$ tenemos que $1^{\tau(\alpha)}$ está definido y $1^{\tau(\alpha)} = \alpha$.
	\end{enumerate}
\end{frame}

\begin{frame}[fragile]{Algoritmo para definir.}
	.
\end{frame}

\begin{frame}[fragile]{Algoritmo para recorrer.}
.	
\end{frame}

\begin{frame}[fragile]{Correctitud del algoritmo.}
asdjasjdajsdjasdjasjd
\end{frame}

\begin{frame}[fragile]{Coincidencia.}
	asdasdasd
\end{frame}

\begin{frame}[fragile]{$G$-equivalencias.}
 asdasdasd
\end{frame}

\begin{frame}[fragile]{Correctitud del algoritmo de coincidencia.}
	Idea detallada de la demo. Explicarlo como lo de la $G$-equivalencia.
\end{frame}

\begin{frame}[fragile]{Algoritmo de escanear y completar.}
	asdasdasd
\end{frame}

\begin{frame}[fragile]{Algoritmo de enumeración de cosets.}
asdasdasd
\end{frame}

\begin{frame}[fragile]{Ejemplo del algoritmo en $C_2 \ast C_3$ pt I.}
asdasdasdasd
\end{frame}


\end{document}