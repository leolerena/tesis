\documentclass[aspectratio=169, 9pt]{beamer}
\usepackage[utf8]{inputenc}
\usepackage[spanish]{babel}

\usepackage{xcolor}
\usepackage{amssymb}
\usepackage{pifont}
\newcommand{\xmark}{\ding{55}}
\usetheme[progressbar=frametitle]{metropolis}
\usepackage{appendixnumberbeamer}
\usepackage{hyperref}
\usepackage{eufrak}
\usepackage{tikz-cd}
\usepackage{tcolorbox}
\usepackage{enumitem}% http://ctan.org/pkg/enumitem
	\definecolor{ao(english)}{rgb}{0.0, 0.5, 0.0}

\setbeamercolor{background canvas}{bg=white}
\usepackage{multicol}
\usepackage{chronology}
%%%%%%%%%%%%%%%%%%%%% ENUMERAR CON COSAS QUE NO SEAN SOLO NÚMEROS %%%%%%%%%%
%\usepackage[shortlabels]{enumitem}
%\setlist[enumerate]{font=\bfseries}

\usepackage{booktabs}
%\usepackage[scale=2]{ccicons}

\usepackage{clrscode3e}

%%%%%%%% ALGORITMOS %%%%%%%%
%\usepackage{algorithm}
%\usepackage[noend]{algpseudocode}  % "noend" es para no mostrar los endfor, endif

% PARA PODER HACER EL FLOW CHART
\usetikzlibrary{shapes,arrows}

\usepackage{pgfplots}
%\usepgfplotslibrary{dateplot}
\colorlet{verde}{green!50!black}
\definecolor{amber}{rgb}{1.0, 0.49, 1.15}
\setbeamercolor{progress bar}{fg=amber,bg=alerted text.fg!50!black!10}

\makeatletter
\setlength{\metropolis@titleseparator@linewidth}{2pt}
\setlength{\metropolis@progressonsectionpage@linewidth}{2pt}
\setlength{\metropolis@progressinheadfoot@linewidth}{2pt}
\makeatother

\definecolor{ultramarine}{RGB}{81, 11, 122} 
\setbeamercolor{frametitle}{bg= ultramarine}

\usepackage{framed}
\definecolor{shadecolor}{gray}{0.9}

\renewcommand\qedsymbol{\textcolor{orange}{$\blacksquare$}}

\usepackage[font={footnotesize}]{caption}

% Para agregar columnas.
\usepackage{graphicx}
\usepackage{adjustbox}
\setbeamercovered{invisible}


\usepackage{xspace}
\newcommand{\Z}{\mathbb{Z}}
\newcommand{\themename}{\textbf{\textsc{metropolis}}\xspace}
\newcommand{\fg}{finitamente generado }
\newcommand{\fp}{finitamente presentado }
\newcommand{\rep}{\textsc{rep}}
\newcommand{\coin}{\textsc{coincidencia}}
\newcommand{\mer}{\textsc{merge}}
\newcommand{\edeff}{\textsc{estáDefinido}}
\newcommand{\scan}{\textsc{scan}}
\newcommand{\definir}{\textsc{definir}}
\newcommand{\scanfill}{\textsc{escanearYCompletar}}
\newcommand{\ded}{\textsc{deducir}}
\newcommand{\recorrer}{\textsc{recorrer}}
\newcommand{\In}{[1 \dots n]}
\newcommand{\ol}{\overline}
\newcommand{\Co}{{\cal{C}}}
\newcommand{\Par}{\text{Par}(X)}

\title{Enumeración de cosets.}
\subtitle{Seminario de topología - UBA.}
\date{12 de octubre de 2022.}
\author{Leopoldo Lerena}
\institute{Universidad de Buenos Aires}
% \titlegraphic{\hfill\includegraphics[height=1.5cm]{logo.pdf}}

\begin{document}
\maketitle

\begin{frame}[fragile]{Introducción.}


Muchas preguntas naturales que nos podemos hacer conociendo una presentación de algún grupo tales cómo:

\begin{itemize}
	\item ¿Cuál es su orden?
	\pause
	\item ¿Cuál es 	el índice de un subgrupo?
	\pause
	\item ¿Cómo es su grafo de Cayley?
	\pause
\end{itemize}

Vamos a poder responderlas hoy usando un algoritmo de enumeración de cosets.
\end{frame}

\begin{frame}[fragile]{Algoritmos en teoría de grupos.}
	Tengamos en cuenta que muchas preguntas elementales de la teoría de grupos tales como:
	\begin{itemize}
		\item Dada una palabra en los generadores, ¿es igual a la identidad del grupo?
		\pause
		\item ¿Es un grupo finito?
%		\pause
%		\item Más aún, ¿es el grupo trivial?
	\end{itemize}
	\pause
	No son \textit{decidibles}. 
	No existe un algoritmo que nos pueda responder estas preguntas dada una presentación finita arbitraria.
	\medskip
	
	\pause
	
	
	Nos tenemos que conformar con que existan procedimientos que son \textit{semidecidibles}. 
\end{frame}

\begin{frame}[fragile]{Descripción del algoritmo.}
	El procedimiento de enumeración de cosets requiere los siguientes parámetros para la entrada:
	\begin{itemize}
		\item Un grupo $G$ \fp;
		\pause
		\item Una presentación finita de este grupo $\langle X | R \rangle$.
			Donde $X = \{ x_1, \dots, x_n \}$ y $R = \{ r_1, \dots, r_m \}$ son conjuntos finitos;
 		\pause
		\item Un subgrupo $H \le G$  \fg con generadores $Y$ visto como palabras en $A = X \cup X^{-1}$.
	\end{itemize}
	\pause
	\medskip
	
	En el caso que el procedimiento termine nos va a devolver lo siguiente:
	\pause
	\begin{itemize}
		\item La representación de $G$ por permutaciones actuando a derecha por multiplicación sobre los cosets a derecha de $H$.
	\end{itemize}  
\end{frame}

%\begin{frame}[fragile]{Un poco de historia.}
%	El primer algoritmo de esta familia fue construido por Todd--Coxeter en el paper ""
%	\alert{Agregar imagen del paper}
%	con la idea de poder hacer manualmente este conteo de  
%\end{frame}

\begin{frame}[fragile]{Acciones parciales.}
	
	
	\bigskip
	
	Notación: Dado un conjunto $X$ denotaremos por $\Par = \{ f: X \to X \ | \  f \ \text{es una función parcial} \}$.
	\pause
	\begin{alertblock}{Definición.}
		Dado un conjunto $X$ diremos que una acción parcial de un monoide $M$ sobre $X$ es función $M \to \Par$ tal que cumple las siguientes dos propiedades,
		\begin{itemize}
			\item $x \cdot 1 = x$ para todo $x \in X$;
			\item $x \cdot (g \cdot h) = (x \cdot g) \cdot h$ siempre y cuando estén definidos.
		\end{itemize} 
	\end{alertblock}
	\pause
	\begin{alertblock}{Definición.}
		Dados $X, Y$ conjuntos con acciones parciales de un monoide $M$ diremos que una función $\tau: X \to Y$ es \emph{equivariante} si $\tau (x) \cdot m = \tau (x \cdot m)$ siempre y cuando $x \cdot m$ esté definida.
	\end{alertblock}
	\pause
\end{frame}




\begin{frame}[fragile]{Tabla de cosets.}
Sea $G = \langle X | R \rangle$ un grupo \fp y sea $A = X \cup X^{-1}$ sus generadores y sus inversos.
Sea $H$ subgrupo generado finitamente por $Y$ un conjunto de elementos de $A^*$.
\pause
\begin{alertblock}{Definición.}
	Una tabla de cosets $\Co $ para el grupo $G$ va a ser un conjunto $\In$ con una acción parcial de $A^{*}$ y una función $\tau: \In \to A^*$ tal que cumple las siguientes propiedades:
	\begin{itemize}
		\item $\tau(1)  = 1$;
		\item Para todo $i \in \In$ tenemos que $1\cdot {\tau(i)}$ está definido y $1\cdot {\tau(i)} = i$;
		\item $i \cdot x = j \iff j \cdot x^{-1} = i$ para $x \in A$;
		\item $\ol \tau: \In \to G/H$ definida como $\ol \tau (i) = H \tau (i)$ es equivariante;
	\end{itemize}	
\end{alertblock}
\pause
%Haremos la siguiente interpretación:
%El número $n$ representa la cantidad de cosets enumerados hasta el momento.
%La acción de $A^*$ nos dirá como será la acción de $G$ sobre el conjunto.
%La función $\tau$ nos guardará un representante de cada coset.
%	


\begin{alertblock}{Definición.}
	Una tabla de cosets $\Co$ está \emph{completa} si $A^*$ actúa sobre $\In$.
	%Esto es tenemos una función
%	\[
%		A^* \to \In^{\In}
%	\]
\end{alertblock}

\end{frame}

\begin{frame}[fragile]{Observaciones.}
	Algunas observaciones válidas para cosets.
	\pause
	\begin{itemize}		
		\item Todo coset $Hg \in G/H$ es tal que $(Hg)r = Hg$ para toda $r \in R$.
		
		\item Para todo $y \in Y$ vale que $Hy = H$.
	\end{itemize}	
	\pause
	Similarmente vamos a querer que valgan estas propiedades sobre nuestras tablas de cosets.
	\begin{itemize}
		\item Todo coset $i \in \In$ es tal que $i \cdot r = i$.
		
		\item Para todo $y \in Y$ tenemos que $1 \cdot y = 1$.
	\end{itemize}
\end{frame}


\begin{frame}[fragile]{Correctitud del algoritmo.}
	Nuestro objetivo es ver qué condiciones tiene que cumplir una tabla de cosets $\Co$ para garantizar que enumera correctamente a todos los cosets.
	\pause
	\begin{alertblock}{Proposición.}
		Si $\Co$ está completa y cumple que  $i \cdot r = i$ para todo $i \in \In$ y para toda $r \in R$, entonces tenemos que $G$ actúa sobre $\In$.
	\end{alertblock}
	\medskip
	\pause
	\begin{proof}
		Usando la propiedad que $i \cdot x = j \iff j\cdot x^{-1} = i$ nos da una acción del grupo libre $F(X)$.
		\pause
		Finalmente como $i \cdot r = i$ para todo $i \in \In, r \in R$ tenemos que baja a una acción parcial del grupo $G$. 
	\end{proof}
\end{frame}


\begin{frame}[fragile]{Correctitud del algoritmo.}
	Consideremos la siguiente función,
	\[
	\ol \tau: \In \to G/H
	\]
	tal que $\ol \tau (i) = H \tau (i)$.
	\pause
	Veamos bajo qué condiciones nos queda una biyección. 
	En tal caso habremos enumerado correctamente a los cosets.
	\medskip
	
	
	
	\begin{alertblock}{Proposición.}
		Si la tabla de cosets $\Co$ está completa entonces $\ol \tau$ resulta ser sobreyectiva.
	\end{alertblock}
	\pause
	
	\begin{proof}
		Dado $Hg \in G/H$ consideremos que $g = x_1 \dots x_n$ \pause luego como la tabla está completa tenemos que $1 \cdot x_1 \dots x_n$ está definido y $1 \cdot x_1 \dots x_n  = i$ para algún cierto $i \in \In$.
		\pause
		Usamos que $\ol \tau$ es equivariante para deducir que $\ol \tau (i) = Hg$.
	\end{proof}
	
	
	
	
\end{frame}

\begin{frame}[fragile]{Correctitud del algoritmo.}
	\begin{alertblock}{Proposición}
		Si la tabla de cosets $\Co$ es tal que el coset $1 \cdot y = 1$ para todo $y \in Y$ entonces $\ol \tau$ resulta ser inyectiva.
	\end{alertblock}
	\pause
	\begin{proof}
		Si $\ol \tau (i) = \ol \tau (j)$ entonces $H \tau (i) \tau (j)^{-1} = H$.
		\pause
		Esto nos dice que $\tau (i) \tau(j)^{-1} \in H$ por lo que como $ 1 \cdot {\tau(i)} = i$ y  $1 \cdot y = 1$ para todo $y \in Y$ obtenemos que 
		\[
		1 \cdot \tau(i)\tau(j)^{-1} = 1 \implies i = j.
		\]   
	\end{proof}
\end{frame}



\begin{frame}[fragile]{Correctitud del algoritmo.}
	En definitiva probamos el siguiente resultado sobre la tabla de cosets.
	\pause
	\begin{alertblock}{Teorema}
		Si valen las siguientes condiciones,
		\begin{itemize}
			\item La tabla de cosets está completa;
			\item $1\cdot y = 1$ para todo $y \in Y$;
			\item $i \cdot r  = i$ para toda $r \in R$ y para todo $i \in \In$.
		\end{itemize}
		\pause
		entonces $[G:H] = n$ y tenemos una acción de $G$ sobre $\In$ tal que la función $\ol \tau: \In \to G/H$ resulta ser una biyección equivariante.
	\end{alertblock}
	\medskip
\end{frame}

%\begin{frame}[fragile]{Demostración correctitud de enumeración de cosets.}
%	\textbf{Demostración}.  
%	
%			Se extiende a un morfismo de grupos por la \textbf{P2} y porque escanea todas las relaciones.	
%			\pause
%			% Lo definimos sobre el monoide libre en un pcpio y con la P2 lo tenemos sobre el grupo libre.
%			
%			\medskip
%			Definimos la función 
%			\[
%			\ol \tau: \Omega \to G/H
%			\]
%			tal que $\ol \tau (i) = H \tau (i)$.
%			\pause
%			Probemos que es una biyección.
%			\pause
%			\begin{itemize}
%				\item Sobreyectiva. Usamos que la tabla está completa y que vale \textbf{P4}.
%				\pause
%				\item Inyectiva. Si $\ol \tau (i) = \ol \tau (j)$ entonces $H \tau (i) \tau (j)^{-1} = H$ y acá usamos que escanea correctamente a todos los generadores de $H$.
%				\pause
%			\end{itemize}
%			Con esto concluimos que $[G:H] = |C| = |\Omega|$ y que la acción sobre $\Omega$ coincide con la acción sobre los cosets a derecha.	
%			\qedsymbol
%		
%\end{frame}


\begin{frame}[fragile]{Ejemplo (1/4)}
	\section{Ejemplo a mano del procedimiento.}
	Partamos del grupo \fp $G = \langle x, y \ | \ x^2, y^3, [x,y] \rangle$ tal que $G \simeq \Z / 2\Z \times \Z / 3\Z$.
	\pause
	Tomemos como subgrupo $H = \langle x \rangle$.
	\pause
	El algoritmo va enumerando los cosets de $H$ en $G$ que sabemos que son exactamente 3.
	\pause
	\medskip
	

	Empezamos asignando el número {1} al coset trivial correspondiente a $H$ en $G/H$.
	\pause 
	Como $x \in H$ entonces sabemos que $Hx = H$, por lo tanto $1 \cdot x = 1$.
	\pause
	Similarmente $1 \cdot x^{-1} = 1$.
	\pause
	Por otro lado denotemos con 2 al coset $1 \cdot y$ y así también $2 \cdot y^{-1} = 1$ \pause Denotaremos por 3 al coset $1 \cdot y^{-1}$ y así $3 \cdot y = 1$.
	

	\onslide<4->
	\begin{table}[]
		\begin{tabular}{|c | c | c | c | c |} 
			\hline
			Coset     & x          & $x^{-1}$          & y          & $y^{-1}$          \\ 	\hline 
			{1} & \onslide<5-> {1} & \onslide<6-> {1} & \onslide<7-> {2} & \onslide<8-> {3}  \\   \hline 
			\onslide<7-> {2} &            &            &            &  \onslide<7-> {1}         \\ \hline 
			\onslide<8-> {3} &            &            &    \onslide<8-> {1}       &            \\ \hline
		\end{tabular}
	\end{table}
	

\end{frame}

\begin{frame}[fragile]{Ejemplo (2/4).}
% ¿Qué podemos hacer?

%\pause
%Queremos que valga esto mismo para los que estamos enumerando, esto es que $i^r = i$ para todo $i \in \{ 1,2,3 \}$ y toda $r \in \{  x^2, y^3, xyXY  \}$.

%Vamos a introducir un proceso que se llama \textit{escanear las relaciones}.

Veamos de garantizar que todo coset $i \in \In$ cumpla que $i \cdot r = i$.
Partamos usando $r = y^3$ y consideremos a ${1}$.

\pause

Queremos garantizar que $1\cdot y^3 = 1$. 
\pause

		\begin{align*}
			&1 \cdot {yyy} = 1 \\
			\iff&	1\cdot {yy}	= 1\cdot y^{-1} \\ 
			\iff&	(1 \cdot y)\cdot y = 3 \\ 
			\iff&	2 \cdot y = 3 
		\end{align*}

\end{frame}

\begin{frame}[fragile]{Ejemplo (3/4)}
	De lo anterior podemos \textit{deducir} que ${2}\cdot y = {3}$ y simultáneamente que $3 \cdot y^{-1} = 2$.
	
	\medskip
	\pause
%	\begin{center}
%		\begin{tikzcd}
%		\textbf{1} \arrow[r, "y", bend left] & \textbf{2} \arrow[r, "y", dotted, bend left] & \textbf{3} \arrow[r, "y", bend left] & \textbf{1}
%		\end{tikzcd}
%	\end{center}
	
	
	De esta manera nuestra tabla ahora tiene la siguiente pinta.
	

	
	
	\begin{table}[]
		\begin{tabular}{|l|l|l|l|l|}
			\hline
			Cosets     & x          & $x^{-1}$          & y          & $y^{-1}$          \\ \hline \onslide<2->
			{1} & {1} & {1} & {2} & {3} \\ \hline \onslide<2->
			{2} &            &            & \onslide<2>\color{verde}{3} &     \onslide<2->{1}       \\ \hline \onslide<2->
			{{3}} &       \onslide<2->     &     \onslide<2->       & \onslide<2->  {1}         & \onslide<2>\color{verde}{2} \\ \hline 
		\end{tabular}
	\end{table}
	
	
\end{frame}

\begin{frame}[fragile]{Ejemplo (4/4)}
	Cuando miramos $1 \cdot xyx^{-1}y^{-1}$ llegamos a las siguientes deducción
	\pause
	\begin{center}
		\begin{tikzcd}
		{1} \arrow[r, "x", bend left] & {1} \arrow[r, "y", bend left]         & {2} \arrow[r, "x^{-1}", dotted, bend left]            & {2} \arrow[r, "y^{-1}", bend left] & {1} \\ 
		\end{tikzcd}
	\end{center}
	\pause
	y cuando lo hacemos para el coset 2 obtenemos,
	\begin{center}
		\begin{tikzcd}
			{2} \arrow[r, "x", bend left] & {2} \arrow[r, "y", bend left]         & {3} \arrow[r, "x^{-1}", dotted, bend left]            & {3} \arrow[r, "y^{-1}", bend left] & {2}
		\end{tikzcd}
	\end{center}
	\pause
	Así terminamos de completar la tabla.
	
	\begin{table}[]
		\begin{tabular}{|l|l|l|l|l|}
			\hline
			Cosets     & x          & $x^{-1}$          & y          & $y^{-1}$          \\ \hline
			{1} & {1} & {1} & {2} & {3} \\ \hline
			{2} & {2} & {2} & {3} & {1} \\ \hline
			{3} & {3} & {3} & {1} & {2} \\ \hline
		\end{tabular}
	\end{table}
	
	
	
	
	
	¿El procedimiento terminó? 
	\pause
	Deberíamos verificar que para todo coset $i$ y toda relación $r$ vale que $i \cdot r = i$.

\end{frame}


\begin{frame}[fragile]{Algoritmos de enumeración de coset.}
	Queremos definir un algoritmo que comience con una tabla de cosets vacía y por cada operación que hagamos nos siga dando una tabla de cosets.
	
	\pause
	Si garantizamos que completamos la tabla y que $i \cdot r = i$ para todos los cosets $i \in \In$ y todas las $r \in R$ y $1 \cdot y = 1$ para todo $y \in Y$ entonces por el resultado anterior vamos a tener una enumeración de cosets.
	
\end{frame}

\begin{frame}[fragile]{Método de las relaciones.}
	
	La estrategia para enumerar cosets que vamos a seguir se llama el \alert{método de las relaciones}.
	
	\pause
	
	
	El siguiente grafo nos va a orientar con los algoritmos que vamos a definir:
	\medskip
	%Hacer un tikzcd con el grafo con palabras así como aparecía en el manual.
	
	% Define block styles
	\tikzstyle{block} = [rectangle, draw, fill=blue!20, 
	text width=12em, text centered, rounded corners, minimum height=2.5em]
	\tikzstyle{line} = [draw, -latex']
	
	\begin{tikzpicture}[node distance = 1.5cm, auto]
	% Place nodes
	\node [block] (cosetenum) {\textsc{CosetEnumeration}};
	\node [block, below of=cosetenum] (scanfill) {\scanfill};
	\node [block, below of=scanfill] (def) {\definir};
	\node [block, right of=def, node distance=5cm] (coincidence) {\coin};
	\node [block, left of=def, node distance=5cm] (ded) {\ded};
	
	%\node [block, below of=coincidence] (rep) {\rep};
	%\node [block, left of=rep, node distance=5cm] (merge) {\mer};
	% Draw edges
	\path [line] (cosetenum) -- (scanfill);
	\path [line] (scanfill) -- (def);
	\path [line] (scanfill) -- (ded);
	\path [line] (scanfill) -- (coincidence);
	%\path [line] (coincidence) -- (merge);
	%\path [line] (coincidence) -- (rep);
	\end{tikzpicture}
\end{frame}


\begin{frame}[fragile]{Algoritmo para recorrer una tabla de cosets.}
	Primero definimos un algoritmo que dado un coset y una palabra recorre hasta donde puede y nos devuelve hasta donde llegó y qué es lo que le quedó por leer de la palabra.
	%\begin{codebox}
	%	\Procname{$\proc{Recorrer}(\Co : \text{Tabla de Cosets}, i : \text{Coset}, w : A^*) :  (j : \text{Coset}, \ v : A^*)$}
	%	\li \textbf{Match} $w$, 
	%	\Do 
	%	\li $ | \  \ \emptyset = (i, \emptyset)$ 
	%	\li $ | \ \  xu =$ \textbf{si} $\edeff(\Co, i,x)$ \textbf{entonces}  $\recorrer(\Co, i^x, u) $ \textbf{sino} $(i, w)$
	%	\End
	%\end{codebox}
	
	\pause
	Partimos de una tabla de cosets $\Co$, de un coset $i \in \In$ y de una palabra $w \in A^*$.
		
	\begin{equation*}
		\textsc{recorrer}(i,w) = 
		\begin{cases}
			(i, \emptyset) \ \ &\text{si} \ w = \emptyset \\
			(i, w) \  &\text{si} \ w = xu \  \text{y} \ \ i \cdot x \ \text{no está definido.} \\
			\textsc{recorrer}(i \cdot x, u) \ &\text{si} \ w = xu \ \text{y} \  i \cdot x  \text{ está definido.} \\
		\end{cases}
	\end{equation*}

	
	
	
	%Donde la función \edeff($\Co, i, x$) nos devuelve \textsc{true} si $i^x$ está definido y \textsc{false} si no lo está.
	
	
\end{frame}

\begin{frame}[fragile]{Algoritmo de Scan and Fill.}
	
		Partimos de una tabla de cosets $\Co$, un coset $i \in \In$ y alguna relación $w \in R$. 
		También consideramos el caso que  $i = 1$ y tenemos algún generador $w \in Y$.
		\pause
		
		{Sea} ($j$,$w_1$) := \recorrer$(\Co, i, w)$ y sea $({k}$, $w_2$) :=  \recorrer$(\Co,i, w_1^{-1})$.
		
		\pause
		\begin{equation*}
			\textsc{EscanearYCompletar}(\Co, i , w) = 
			\begin{cases}
				\Co \ &\text{si} \  w_2 = \emptyset \ \text{y} \ j=k  \\
				\coin({\Co, j,k}) & \text{si} \  w_2 = \emptyset \ \text{y} \ j \neq k   \\
				\ded(\Co, j, x) & \text{si} \ w_2 = x   \\
				\scanfill(\definir(\Co,i,x),i,w) & \text{si} \ w_2 = xu
			\end{cases}
		\end{equation*}
		
		\pause
		Obtenemos una tabla de cosets $\Co'$ donde $i \cdot w$ está definido y $i \cdot w = i$.
\end{frame}

\begin{frame}[fragile]{Algoritmo para definir y para deducir.}
	\begin{itemize}
		\item \textbf{Algoritmo para definir.} 
		Partimos de un coset $i$ y de alguna letra $x \in A$ tal que $i \cdot x$ no esté definida.
		Agregamos un coset por lo que nuestro conjunto pasa a ser $[1 \dots n+1]$. 
		Definimos $i \cdot x = n+1$ y $(n+1) \cdot x^{-1} = i$.
		Definimos $\tau(n+1) = \tau (i) x$.
		\pause
		\item \textbf{Algoritmo para deducir.} Partimos de dos cosets $i,j \in \In$ y de una letra $x$ tal que $i \cdot x$ no está definido y $j \cdot x^{-1}$ no está definido.
		Definimos que $i \cdot x = j$ y $j \cdot x^{-1} = i$.
	\end{itemize}
	
	
	\pause
	\begin{alertblock}{Observación.}
		Estos algoritmos nos devuelven tablas de cosets.
	\end{alertblock}
\end{frame}






%\begin{frame}[fragile]{Rep.}
%	
%	Este algoritmo nos busca un representante de la clase de equivalencia de $i$, un coset de una tabla $\Co$, con respecto a la $G-$equivalencia generada por $p$.
%	
%	\pause
%	
%	\begin{codebox}
%		\Procname{$\proc{rep}( \Co : \text{Tabla de cosets}, i : \text{coset}) : (\textbf{k} : \text{coset})  $}
%		\li \textbf{Si} $p(i) = i$ \textbf{entonces} $i$ \textbf{sino} $\rep(\Co, p(i))$
%	\end{codebox}
%\pause
%
%En particular como $p$ es una función decreciente sobre un subconjunto de naturales tenemos que si la aplicamos consecutivamente en algún momento tiene un punto fijo.
%\end{frame}
%
%\begin{frame}[fragile]{Merge.}
%	Este algoritmo une dos clases de equivalencia cambiándole el valor de $p$ del representante de una clase a que valga el de la otra clase y nos devuelve una lista con todos los cosets a los cuales les cambiamos el representante.
%	\pause
%	\begin{codebox}
%		\Procname{$\proc{merge}( \Co : \text{Tabla de cosets}, i : \text{coset}, j : \text{coset}) : (\Co' : \text{Tabla de cosets}, q : \text{lista de cosets})$}
%		\li \textbf{Sean} $\textbf{k} = \rep(i), \textbf{l} = \rep(j)$
%		\li \textbf{Si} $ \textbf{k} \neq \textbf{l}$ \textbf{entonces} 
%		\Do 
%		\li \textbf{Sean} $\mu = \max(\textbf{k}, \textbf{l}), \nu = \min(\textbf{k}, \textbf{l})$
%		\li $p(\mu) = \nu, q = [\mu]$
%	\end{codebox}
%		
%\end{frame}

%\begin{frame}[fragile]{$M$-equivalencias.}
%	%Sabemos que al finalizar el algoritmo nos queda la acción sobre los cosets $G/H$ si nos restringimos a $\Omega$ pero ¿qué sucede en el medio?
%
%	
%	
%	\begin{alertblock}{Definición}
%		Una \alert{M-equivalencia} es una relación de equivalencia sobre un conjunto $X$ con una acción parcial de un monoide $M$ que aparte satisface la siguiente propiedad.
%		Si $x \sim y$ luego para todo $m \in M$  vale que $x \cdot m \sim y \cdot m$ siempre y cuando $x \cdot m$ e $y \cdot m$ estén definidas.
%	\end{alertblock}
%	
%	
%	\begin{alertblock}{Lema.}
%		La acción parcial de $M$ sobre $X$ se restringe a una acción parcial de $M$ sobre $X / \sim.$
%	\end{alertblock}
%
%	\begin{proof}
%		Dado $[x] \in X / \sim$ definimos 
%		\[
%		[x] \cdot m = [x \cdot m]
%		\]
%		que resulta ser una acción parcial porque $M$ actua parcialmente sobre $X$.
%		
%		No depende del representante porque estamos dividiendo por una $M$-equivalencia justamente.
%	\end{proof}
%
%	
%	\begin{alertblock}{Observación.}
%		Esto nos dice que la proyección $\pi:X \to X / \sim$ resulta ser equivariante.
%	\end{alertblock}
%\end{frame}
%
%\begin{frame}[fragile]{$M$-equivalencias generada por un subconjunto.}
%		
%%	Así como tenemos relaciones de equivalencias generadas a partir de algún subconjunto $S \subseteq X \times X$ podemos considerar la menor $M$ equivalencia generada a partir de $S$.
%	\begin{alertblock}{Proposición.}
%		Sea $X$ un conjunto con una acción parcial de un monoide $M$ y sea $S \subset X \times X$ entonces existe una $M$-equivalencia $\sim_S$ que contiene a $S$.
%		
%		Si $\sim_S$ es la menor $M$-equivalencia que contiene a $S$ y $\sim$ es otra $M$ equivalencia que contiene a $S$ entonces cumple la siguiente propiedad universal
%			
%		\begin{center}
%			{\large \begin{tikzcd}
%					{X} \arrow[d, "\pi_S"', two heads] \arrow[r, "\pi", two heads] & {X / \sim} \\
%					{ X  / \sim_S} \arrow[ru, "\exists \ol{ \pi}"', dashed]                       &                       
%			\end{tikzcd}}
%		\end{center}
%	\end{alertblock}
%	\begin{proof}
%		Para ver la existencia consideramos 
%		\[
%			\sim_S = \bigcap_{S \subseteq \sim} \sim.
%		\]
%		La propiedad universal sale de la propiedad universal del cociente y que las proyecciones son equivariantes.
%	\end{proof}
%\end{frame}

%\begin{frame}[fragile]{Cosets vivos.}
%	Cuando enumeramos los cosets muchas veces sucederá que enumeraremos de manera distinta a un mismo coset. 
%	Esto es que para ciertos $i \neq j \in \In$ tendremos que $H \tau (i) = H \tau (j)$.
%	
%	Para distinguirlos introduciremos una función decreciente de los cosets en sí mismo $p: \In \to \In$.
%	
%	
%	\begin{alertblock}{Definición.}
%		Los \emph{cosets vivos} son el subconjunto de cosets enumerados dados por $\Omega = \{ i \in \In  \ | \ p(i) = i \}$.
%	\end{alertblock}
%	
%	Consideremos la relación de equivalencia $\sim$ generada a partir de $p$.
%	Otra manera de decir esto es que 
%	\[
%	\Omega \simeq \In / \sim
%	\]
%	y queremos ver que $\sim$ es un $A^{*}$-equivalencia sobre $\In$ en todo momento del algoritmo.
%	
%\end{frame}

%\begin{frame}[fragile]{$M$-equivalencias en nuestro caso.}
%	En nuestro caso consideramos que estamos en algún momento de nuestro algoritmo entonces:
%	
%	\begin{itemize}
%		\item El conjunto $X = \In$ los cosets definidos.
%		
%		\item La relación $\sim$ es la generada a partir de $p$.
%		
%		\item La acción parcial de $M$ justamente es la multiplicación a derecha de los cosets.
%	\end{itemize}
%	
%	
%	
%	Con esta notación vemos que $\In /\sim \ \simeq \ \Omega$.
%\end{frame}	



%\begin{frame}[fragile]{Coincidencia.}
%
%	Primero unas funciones auxiliares,
%	\pause
%	\begin{codebox}\Procname{$\proc{procesarLetra}(\Co : \text{Tabla de cosets}, i : \In, x : A ) : (\Co' : \text{Tabla de cosets}, q: \text{lista de cosets})$}
%		\li \textbf{Si} \edeff($\Co, i, x$)
%		%No sé si poner una función nueva o dejar esto así, en cierta manera es una auxiliar bastante evidente cómo es su pinta. 
%		\Do 
%		\li \textbf{Sea} $\textbf{l} = \textbf{k}^x, \mu = \rep(\textbf{k}), \nu = \rep(\textbf{l})$
%		\li \textbf{Si} $\edeff(\Co', \mu, x)$ \textbf{entonces} \mer($\Co', \mu^x, \nu$)
%		\li \textbf{Si} $\edeff(\Co', \nu, X)$  \textbf{entonces} \mer($\Co', \nu^X, \mu$)
%		\li \textbf{Sino} $(\definir(\Co, \mu,x), [])$
%	\end{codebox}
%	\pause
%	%Podemos extender esta función para que tome una lista de letras.
%	\begin{codebox}\Procname{$\proc{coincidenciaLista}(\Co : \text{tabla de cosets}, q : \text{lista cosets}) : (\Co' : \text{tabla de cosets} )$}
%		\li \textbf{Match} $(\Co, q)$,
%		\Do
%		\li $(\Co, []) = \Co$
%		\li $(\Co, i::q) = \proc{coincidenceLista}(\Co'', q ++ q')$ 
%		\Do 
%		\li \textbf{Donde} $(\Co'', q') = \proc{procesarLetras} (\Co, i , A)$
%		\li 
%	\end{codebox}
%	\pause
%\end{frame}

\begin{frame}[fragile]{Algoritmo de \coin}
	Este algoritmo parte de una tabla de cosets $\Co$, dos cosets distintos $i,j \in \In$ y nos devuelve el cociente más general $Y$ que cumple lo siguiente,
		\begin{center}
					{\large \begin{tikzcd}
										{\In} \arrow[d, "\pi"', two heads] \arrow[r, "\ol\tau"] & {G/H} \\
										{Y} \arrow[ru, "\exists \tau'"', dashed]                       &                       
								\end{tikzcd}}
		\end{center}
	\pause
		tal que:
		\begin{itemize}
			\item $\pi(i) = \pi(j)$;
			\item $\tau'$ es equivariante y $\tau' ([x]) = \ol\tau(x)$ para cierto $x$ representante de  $[x]$. 
		\end{itemize}
\end{frame}
%
%\begin{frame}[fragile]{Correctitud del algoritmo de coincidencia (1/3).}
%	%Idea detallada de la demo. Explicarlo como lo de la $G$-equivalencia.
%	Sea $\equiv$ la $G$-equivalencia generada a partir de la relación $\sim$ dada por $p$ y del par $(i, j)$.
%	\pause
%	Llamemos $\sim_F$ la relación que nos queda al terminar de hacer \coin.
%	Esto es que $\textbf{k} \sim_F \textbf{l}$ si y solo sí al terminar el algoritmo $\rep (\textbf{k}) = \rep(\textbf{l})$ donde en este caso el representante lo estamos mirando respecto a esta nueva relación de equivalencia.
%	\pause
%	
%	\begin{alertblock}{Lema}
%		Si existen $\textbf{k}, \textbf{l} \in [1,\dots,n], x \in A$ tales que $\textbf{k}^x = \textbf{l}$ antes de comenzar el algoritmo entonces al terminarlo vale que $\rep(\textbf{k})^x = \rep(\textbf{l}).$
%	\end{alertblock}
%	\pause
%	\textbf{Demostración.} 
%	\pause
%	Solo nos interesa los casos que modificamos el valor de $p'$. 
%	Tomemos $\mu = \rep(\textbf{k}) $ y $\nu = \rep(\textbf{l})$.
%	\pause
%	\begin{columns}
%		\begin{column}{0.5\textwidth}
%			\begin{enumerate}
%				\item Sea $\mu^x = \textbf{l}$. 
%				Como $\rep (\textbf{l}) = \rep (\nu)$ por lo tanto $\rep(\textbf{k})^x = \mu^x = \textbf{l} $ y $\textbf{l} \sim \nu $ por lo que $ \rep(\textbf{l}) = \rep (\textbf{l})$.
%				\item Similar al caso anterior tomando $\nu^{x^{-1}} = \textbf{k}$.
%				\item En particular $\rep(\textbf{k})^x = \rep(\textbf{l}).$
%			\end{enumerate}
%		\end{column}
%		\begin{column}{0.5\textwidth}  %%<--- here
%			\begin{enumerate}
%				\item  \textbf{Si} $\edeff(\Co', \mu, x)$ \textbf{entonces}
%				
%				 \mer($\Co', \mu^x, \nu, q$) 
%				\item  \textbf{Si} $\edeff(\Co', \nu, X)$  \textbf{entonces}
%				
%				  \mer($\Co', \nu^X, \mu, q$)
%				\item \textbf{Sino} $\mu^x = \nu$, $\nu^X = \mu$ 
%			\end{enumerate}
%		\end{column}
%	\end{columns}
%\end{frame}
%
%
%\begin{frame}[fragile]{Correctitud del algoritmo de coincidencia (2/3).}
%	\begin{alertblock}{Teorema.}
%		Al terminar del correr el algoritmo \coin($i, j$) obtenemos la siguiente igualdad para todo $\textbf{k}, \textbf{l} \in [1,\dots,n]$,
%		\begin{equation*}
%			\textbf{k} \sim_F \textbf{l} \iff \textbf{k} \equiv \textbf{l}.
%		\end{equation*}
%	\end{alertblock}
%	\pause 
%	
%	\textbf{Demostración.} $(\Rightarrow)$
%	\pause
%	Ni bien comienza el algoritmo sabemos que vale esta implicación porque justamente $\equiv$ es la $G$-congruencia generada a partir de $\sim$ y del par $(i, j)$.
%	\pause
%	Los casos que llamamos a \mer \ son los casos que podría modificarse la relación $\sim$.
%	El primer \mer \ agrega al par $(i, j)$ a la relación $\sim$ y sabemos que este par está por definición en $\equiv$.
%	\pause
%	
%	
%	En los otros llamados de \mer \ si tomamos $\mu = \rep(\textbf{k}) $ y $\nu = \rep(\textbf{l})$.
%	\medskip
%	\pause
%	\begin{columns}
%		\begin{column}{0.1\textwidth}
%			
%		\end{column}
%		\begin{column}{0.6\textwidth}
%			\begin{center}
%				{ Tenemos $\textbf{k}^x = \textbf{l}$ tal que podemos suponer inductivamente que $\textbf{k} \sim \mu$ y que $\textbf{l} \sim \nu $.
%				Si estamos en el caso \textbf{1} entonces está definido $\mu^x \sim_F \nu$ y como es una $G$-equivalencia obtenemos que $\mu^x \equiv \textbf{k}^x \equiv \textbf{l} \equiv \nu$. 
%				Análogamente el caso \textbf{2}.}
%			\end{center} 	
%		\end{column}
%		\begin{column}{0.3\textwidth}
%			{\small \begin{enumerate}
%				\item  \textbf{Si} $\edeff(\Co', \mu, x)$ \textbf{entonces}
%				
%				\mer($\Co', \mu^x, \nu, q$) 
%				\item  \textbf{Si} $\edeff(\Co', \nu, X)$  \textbf{entonces}
%				
%				\mer($\Co', \nu^X, \mu, q$)
%			\end{enumerate}}
%		\end{column}
%	\end{columns}
%\end{frame}
%
%\begin{frame}[fragile]{Correctitud del algoritmo de coincidencia (3/3).}
%	\textbf{Demostración.} $(\Leftarrow)$
%	\pause
%	\medskip
%	\begin{columns}
%		\begin{column}{0.7\textwidth}
%			Usemos que $\equiv$ es la menor $G$-equivalencia que contiene a $\sim$ y al par $(i, j)$. 
%			Veamos que $\sim_F$ es una $G$-equivalencia.
%			\pause
%			
%			
%			Para eso partamos de $\textbf{k} \sim_F \textbf{l}$ y veamos que si $x \in A$ es tal que $\textbf{k}^x$ y $\textbf{l}^x$ estén ambas definidas entonces $\textbf{k}^x \sim_F \textbf{l}^x$.
%			\pause
%			Esto es una consecuencia del lema anterior.
%		\end{column}
%		\begin{column}{0.3\textwidth}
%			\begin{tikzcd}
%			{[1 \dots n]} \arrow[d, "\pi"', two heads] \arrow[r, "\pi_F", two heads] & {[1 \dots n] / \sim_F} \\
%			{[1 \dots n] / \equiv} \arrow[ru, "\ol{ \pi}"', dashed]                       &                       
%			\end{tikzcd}
%			
%			Donde $\ol {\pi}$ es sobreyectiva. 
%		\end{column}
%	\end{columns}
%\qedsymbol
%\end{frame}




\begin{frame}[fragile]{Algoritmo de enumeración de cosets.}

Veamos finalmente un algoritmo de enumeración de cosets.


\begin{alertblock}{Input.}
	Un grupo $G$ dado por una presentación finita $\langle X | R \rangle$ y $H$ un subgrupo \fg dado por generadores $Y$ donde cada generador es un elemento de $A^*$.
\end{alertblock}
\pause
\begin{codebox}
	\Procname{
		$\proc{CosetEnumeration}(\langle X | R \rangle, Y )$
	}
	\li  \Comment Inicializamos la tabla de cosets.
	\li $n=1, \tau(1)=1$,
	\li \For  $y \in Y$ 
		\Do \li \scanfill$(\sim \Co, 1, y)$  
	\End
	\li \For $i \in \In$
	 \Do \li \For $r \in R$
	 \Do \li \For \scanfill$(\sim \Co, i, r)$
	 \End
	  \End
	\End
\end{codebox}
%Acá hay un leve abuso de notación porque en realidad estamos llamando a \scanfill sobre tabla de cosets distintas...


\end{frame}





\begin{frame}[fragile]{Cuándo termina el procedimiento.}
	¿En qué caso termina el algoritmo de enumeración de cosets?
	\pause
	Termina si el subgrupo que tomamos $H$ es tal que $[G:H] < \infty$.
	\pause

	Consideremos el (posiblemente infinito) conjunto $\ol \Omega$ de los cosets que no son eliminados por ningún llamado de \coin.

	
	\pause
	\textbf{Propiedad $\ol \Omega$.} El conjunto $\ol \Omega$ es tal que:
	\begin{enumerate}
		\item Para cada $i \in \overline \Omega$ y para cada $x \in A$ en algún momento $i \cdot x$ va a estar definido;
		\item Para cada $w \in Y$ en algún momento llamamos a \scanfill($\Co,{1},w$);
		\item Para cada $i \in \ol \Omega$ y para cada $w \in R$ en algún momento llamamos a \scanfill$(\Co,i, w)$.
	\end{enumerate}
\end{frame}

\begin{frame}[fragile]{Continuación.}
	\begin{alertblock}{Teorema}
		Si la propiedad $\ol \Omega$ vale y $[G:H] < \infty$ entonces el algoritmo eventualmente termina.
	\end{alertblock}
	
	
	\pause
\begin{proof}
	\pause
		Primero veamos el caso que $\ol \Omega$ es finito.
	\begin{itemize}
		\pause
		\item Usamos la propiedad $\ol \Omega$ para ver que en algún momento $i \cdot x$ queda definido para todo $i \in \ol\Omega, x \in A$.
		\pause 
		Podemos tomar el momento para que todo estos cosets estén estabilizados.
		\pause
		En ese momento tendríamos que $\Omega = \ol \Omega$ y nuestra tabla está completa y así nuestro algoritmo termina.
	\end{itemize}
	\pause
	Veamos el caso que $\ol \Omega$ es infinito.
	\begin{itemize}
		\pause
		\item Como vale la propiedad $\ol \Omega$ estamos en condiciones de usar el teorema anterior de manera que el índice de $H$ resulta ser infinito.
		Esto contradice las hipótesis.
		
	\end{itemize}
		
\end{proof}
	
		%Agregar la propiedad y la proposición anterior al costadito.
	
\end{frame}

\end{document}