\documentclass[tesis.tex]{subfiles}
\newcommand{\aut}{\text{Aut}}


\begin{document}
\section{Cortes de grafos y árboles de estructura.}

\begin{deff}
	Sea $\Gamma$ un grafo conexo y localmente finito. 
	Sea $C \subseteq V(\Gamma)$ luego definiremos la frontera de este conjunto como:	
	\[
	\delta C = \{  (x,y) | x \in C, y \in \ol C    \}
	\]
\end{deff}


\begin{deff}
	Dado $C \subset V(\Gamma)$ diremos que es un \emph{corte} si cumple las siguientes condiciones,
	\begin{itemize}
		\item $C$ y $\ol C$ son conexos y no vacíos.
		\item $|\delta C| \le \infty$.
	\end{itemize}
	Si $|\delta C| = k$ diremos que $C$ es un \emph{$k$-corte}.
\end{deff}	

Idealmente queremos que nuestros cortes nos separen en partes infinitas al grafo pero esto no siempre es posible.
Un ejemplo claro es tomar el grafo de Cayley de $\ZZ^2$.
	
\begin{lema}\label{lema_finitos_kcortes}
	Sea $\Gamma$ un grafo conexo y localmente finito.
	Sea $S \subset V$ un conjunto finito y $k\ge 1$.
	Entonces existen finitos $k$-cortes $C$ con $\beta C \cap S \neq \emptyset$.
\end{lema}	

\begin{proof}
	Primero notemos que nos alcanza con ver los cortes $C$ que tienen una arista fija $(x,y) \in \delta C$.
	Si vemos que estos son finitos entonces como tenemos finitos $x \in S$ y cada vértice tiene finitas aristas dado que el grafo es localmente finito entonces tenemos que la cantidad de $k$ cortes sería finita.
	Consideremos entonces que $S = \{ x \}$ porque los otros casos se siguen de este.
	
	Vamos a probarlo por inducción en $k$.
	
	En el caso base tenemos que $\delta C = (x,y)$.
	En este caso tenemos que solo puede haber finitos cortes (al menos tantos como aristas salen del vértice $x$).
	
	Ahora veamos el paso inductivo.
	Como suponemos que es un $k$-corte con $k \ge 2$ al menos tenemos una arista sea esta $(x,y)$.
	Notemos que si $\Gamma \setminus (x,y)$ no está conectado entonces no puede haber un corte $C'$ tal que $(x,y) \in \delta C'$ y que cumpla que $|\delta C'| \ge 2$.
	Esto nos dice que nos queda ver el caso que $\Gamma \setminus (x,y)$ sigue siendo conexo. 
	Al seguir siendo conexo debe existir un camino de vértices $\gamma$  en $ \Gamma \setminus (x,y) $ tal que $\gamma = x_0 x_1 \dots x_{n}$ con $x_0 = x$ y con $x_{n} = y$.
	Todos estos $k$-cortes $C$ que estamos considerando tienen que cumplir que alguno de las aristas $(x_i,x_{i+1}) \in  \gamma \cap \delta C$.
	Caso contrario tendríamos que para todo $x_i \in \gamma$ vale que $x_{i} \in C$ o bien $x_i \in \ol C$ contradiciendo que $(x,y) \in \delta C$.
	Finalmente como todo $k$-corte que contiene a $(x,y)$ restringido a $\gamma$ resulta ser un $(k-1)$-corte tenemos que al ser estos finitos por la hipótesis inductiva que la cantidad de $k$-cortes es finita tal como queríamos ver.
	
\end{proof}
	
\begin{deff}
	Dado $\Gamma$ un grafo conexo diremos que un camino simple $\alpha$ es un \emph{camino infinito }si
	\[
		\alpha = v_0 \dots v_{n} \dots
	\]	
	Diremos que es un \emph{camino bi-infinito } si
	\[
		\alpha =  \dots v_{-n} \dots v_0 \dots v_{n} \dots 
	\]
\end{deff}	
	

Una primera observación es que dado un corte $C$ tal que $C$ es infinito y $\ol C$ también lo es, podemos armarnos un camino bi infinito $\alpha$ tal que $|\alpha \cap C| = \infty = |\alpha \cap \ol C|$.

No necesariamente vale la vuelta.
Esto es que si tenemos un camino bi-infinito $\alpha$ y un corte $C$ entonces el corte cumple que $|\alpha \cap C| = \infty = |\alpha \cap \ol C|$.

Esto nos lleva a dar la siguiente definición.

\begin{deff}
	Sea $\Gamma$ un grafo conexo y sea $\alpha$ un camino bi-infinito.
	Definimos el conjuntos de cortes del camino como 
	\[
		{\cal C}(\alpha) = \{ C \subset V(\Gamma) \mid  C \ \text{es un corte y} \ |\alpha \cap C| = \infty = |\alpha \cap \ol C| \}
	\] 
\end{deff}

\begin{lema}
Tenemos que ${\cal C}(\alpha) \neq \emptyset$ si y solo sí existe un corte $C$ de manera que $\alpha \setminus \delta C$ tiene dos componentes infinitas.
\end{lema}
\begin{proof}
	...
\end{proof}	
	
\begin{deff}
	Dado un grafo $\Gamma$ y $S \subset V(\Gamma)$ diremos que $\Gamma$ tiene más de un end si $\Gamma \setminus S$ tiene más de una componente conexa.
	Caso contrario diremos que tiene al menos un end.
\end{deff}

%Esta definición de ends para grafos de Cayley resulta ser equivalente a la definición de ends para grupos.
%En el trabajo \cite{} está probada la equivalencia entre estos resultados...	

\begin{deff}
	Un grafo $\Gamma$ es \emph{accesible} si existe $k \in \NN$ de manera que todo camino bi-infinito $\alpha$ cumple que ${\cal C}(\alpha) = \emptyset$ o bien ${\cal C}(\alpha)$ contiene un $k$-corte.
\end{deff}

Veamos que los grafos con treewidth finito son accesibles.
Para eso primero veamos un lema un poco más técnico.


\begin{lema}\label{lema_corte_treewidth}
	Sea $\Gamma$ un grafo de treewidth finito y grado uniformemente acotado.
	Entonces existe $k \in \NN$ tal que:
	Para todo $\gamma$ camino infinito simple, todo $v_0 \in V(\Gamma)$ y todo $n \in \NN$ debe existir un $k-$corte $D$ que cumple las siguientes propiedades, $d(v_0,\ol D) \ge n, v_0 \in D, |\ol D \cap \gamma| = \infty$. 
\end{lema}

\begin{proof}
	Sea $d$ una cota para el grado de los vértices del grafo $\Gamma$ y sea $k$ el treewidth del grafo.
	Propondremos $k = dm$ como la constante que buscamos.
	
	Consideremos $t_0 \in V(T)$ de manera que $v_0 \in X_t$.
	Si tenemos dos vértices $u_1,u_2 \in \ol X_t$ tales que $u_1$ y $u_2$ están en bolsones $X_{t_1}, X_{t_2}$ de manera que $t_1$ y $t_2$ están en componentes conexas distintas de $T \setminus t_0$, luego como los árboles son únicamente geodésicos y usando la proposición \ref{} obtenemos que todo camino $\gamma$ que pase por $u_1$ y por $u_2$ debe pasar en algún momento por $X_{{t}_0}$.
	Como estos caminos que consideramos son simples y los bolsones $|X_{t_0}| < \infty$ obtenemos que debe haber una única componente conexa de $\ol X_{t_0}$ tal que interseca infinitas veces al camino $\gamma$.
	Nombraremos a esta componente conexa $C_{t_0,\gamma}$.
	
	Repetimos este procedimiento por cada vértice de la componente conexa de $\ol X_{t_0}$.
	Esto nos da un camino infinito en el árbol $t_0,t_1, \dots t_n \dots$ y una sucesión de componentes conexas correspondientes $C_{t_0,\gamma} C_{t_1,\gamma} \dots C_{t_n, \gamma} \dots$.
	
	Si elegimos $l \in \NN$ suficientemente grande podemos garantizar que
	\[
		X_{t_l} \cap B_n(v_0) = \emptyset
	\]
	Esto porque por la proposición \ref{} tenemos que la descomposición se puede tomar de manera que cada vértice aparezca en finitos bolsones y que el grafo $\Gamma$ al ser localmente finito nos dice que la bola $|B_n(v_0)| < \infty$.
	
	Ahora vamos a buscar el corte $D$. 
	Para eso vamos a necesitar un conjunto conexo con complemento conexo y no vacío tal que su borde sea finito.
	Consideremos $D$ la componente conexa de $v_0$ en $\ol C_{t_l, \gamma}$.
	Primero veamos que $\ol D$ es conexo.
	Para eso si tenemos un elemento $u_1 \in \ol D$ tiene que estar conectados con alguien del conexo $C_{t_l, \gamma}$.
	Más aún todo camino con algún elemento de $D$ tiene que pasar por $C_{t_l, \gamma}$.
	Caso contrario como tenemos un camino que lo une con $v_0$, dado que el grafo es conexo, esto resultaría en que $u_1 \in D$.
	De esta manera vimos que todo elemento de $\ol D$ está conectado con un conexo $C_{t_l, \gamma}$ evitando pasar por $D$.
	
	Ambos conjuntos son no vacíos porque $C_{t_l, \gamma}$ es infinito y está contenido en $\ol D$.
	Por otro lado tenemos que $v_0 \in D$.
	
	Finalmente veamos que es un $k-$corte.
	Notemos que si $v \in \partial D$ luego tenemos que existe $u \in \ol D$ tal que $(u,v) \in E(\Gamma)$.
	Por lo visto anteriormente todo camino que va de $\ol D$ a $D$ tiene que pasar necesariamente por $C_{t_l, \gamma}$. 
	Entonces nos alcanza con tomar $u \in C_{t_l, \gamma}$.
	En este caso notemos que $v \in X_{t_l}$ caso contrario tendríamos que $v \in C_{t_l, \gamma}$ dado que esta es una de las componentes conexas de $X \setminus X_{t_l}$.
	De esta manera vimos que $\partial D \subseteq X_{t_l}$.
	Como el grafo es localmente finito y tiene treewidth finito concluímos finalmente que $|\delta D| \le dm = k$ tal como queríamos ver.
		
\end{proof}


\begin{teo}
	Sea $\Gamma$ un grafo de treewidth finito y de grado uniformemente acotado.
	Entonces $\Gamma$ es accesible.
\end{teo}
\begin{proof}
	Tomemos $\alpha$ un camino bi-infinito tal que ${\cal C}(\alpha) \neq \emptyset$.
	Sea $C \in {\cal C}(\alpha)$.
	Veamos de construirnos un $k$-corte que $D$ de manera que $D \in {\cal C}(\alpha)$.
	
	Fijemos $v_0  \in \partial C$ y consideremos el siguiente número natural,
	\[
		n = \max \{ d(v_0,w) : w \in \partial C  \}.
	\]
	Por el lema anterior \ref{lema_corte_treewidth} podemos ver que existe un $k$-corte $D$ de manera que $d(v_0, \ol D) \le n$ y que cumple que $|\ol D \cap \gamma| < \infty$.
	
	Este corte $D$ nos sirve porque por la distancia elegida tenemos entonces que $C \subseteq D$ o bien $\ol C \subseteq D$.
	Caso contrario existiría $c_1 \in C \cap \ol D$ y $c_2 \in \ol C \cap \ol D$ tal que al ser $\ol D$ conexo podríamos armarnos un camino dentro de esta componente que una $c_1$ con $c_2$.
	Esto contradice que $\beta C \subseteq D$ por como elegimos a $D$.	
\end{proof}


\subsection{Cortes mínimos.}
En esta subsección vamos a construir un árbol a partir de ciertos cortes de un grafo de manera que este árbol nos de una descomposición.

\begin{deff}
	Dado $\alpha$ camino bi-infinito definimos el conjunto de sus cortes mínimos como
	\[
		\cam = \{  C \in \ca : |\delta C| \ \text{es mínimo}  \}
	\]
	
	Dado un grafo $\Gamma$ definimos el conjunto de sus cortes mínimos como 
	\[
		{\cal C}_{\text{min}} = \bigcup \{ \cam : \alpha \ \text{es un camino bi-infinito}  \}
	\]
\end{deff}

\begin{deff}
	Dos cortes $C,D \in V(\Gamma)$ están anidados si vale alguna de las cuatro inclusiones $C \subseteq D, C \subseteq \ol D, \ol C \subseteq \ol D, \ol C \subseteq D$.
	Los conjuntos $C \cap D, C \cap \ol D, \ol C \cap D, \ol C \cap \ol D$ los llamamos las esquinas.
\end{deff}

\begin{lema}
	Dos cortes $C,D$ están anidados si y solo si alguna de las cuatro esquinas es vacía.
\end{lema}
\begin{proof}
	Para la ida supongamos que $C \subseteq D$ luego tiene que valer que $\ol C \cap \ol D = \emptyset$.
	Para la vuelta si por ejemplo $C \cap D = \emptyset$ como vale que $C = C \cap D \cup C \cap \ol D$ esto nos dice que $C \subseteq \ol D$.
\end{proof}

\begin{lema}
	Dado $C$ corte y $k \in \NN$ tenemos que 
	\[
		| \{  D : C, D \ \text{no están anidados y} \ D \ \text{es un k-corte}   \} | < \infty
	\]
\end{lema}
\begin{proof}
	Por el lema anterior \ref{lema_finitos_kcortes} tenemos finitos $k$-cortes tales que intersecan a $\beta C$ dado que este es un conjunto finito.
	
	Esto nos dice que debemos mirar los casos que $\beta C \subseteq D$.
	Si $C,D$ no estuvieran anidados tendríamos que $C \cap \ol D \neq \emptyset \neq \ol C \cap \ol D$.
	Esto nos diría que podríamos tomarnos un camino entre $c_0 \in C \cap \ol D$ y $c_1 \in \ol C \cap \ol D$ tal que esté contenido en $\ol D$.
	Esto es una contradicción porque este camino tendría que pasar por $\beta C$ y sabemos que $\beta C \subseteq D$.

	En conclusión tenemos que todos los $k-$cortes no anidados tienen que intersecar a $\beta C$ y estos son finitos.			
\end{proof}

Esto nos dice que la siguiente definición es correcta.

\begin{deff}
	Dado $\Gamma$ grafo conexo y localmente finito, $C$ un corte y $k \in \NN$ constante definimos el siguiente número natural,
	\[
		m_k(C) = | \{  D : C, D \ \text{no están anidados y} \ D \ \text{es un k-corte}   \} |. 
	\]
\end{deff}

Consideremos ahora que el grafo $\Gamma$ es accesible y su constante es $k$.
En este caso notaremos para cada corte $C$ el siguiente valor
\[
	m(C) = m_k(C).
\]

\subsection{Cortes óptimos.}

Así como definimos cortes mínimos ahora vamos a definir cortes óptimos que serán un subconjunto de ellos.

\begin{deff}
	Dado $\alpha$ camino bi-infinito sea
	\[
		m_\alpha = \min \{ m(C) : C \in \cam \}.
	\]
	De esta manera los cortes óptimos del camino $\alpha$ serán
	\[
		\copta = \{ C \in \cam : m(C) = m_\alpha  \}
	\]
	y el conjunto de los cortes óptimos análogamente será el conjunto que contenga a todos ellos
	\[
		\copt = \bigcup \ \{ \copta : \alpha \ \text{es un camini bi-infinito}  \}.
	\]
\end{deff}


\subsection{Árbol de estructura.}

Todos los grafos que vamos a considerar a partir de ahora son accesibles.
Veamos que los caminos óptimos resultan ser un poset discreto con la inclusión en este caso.

\begin{lema}\label{lema_intermedios}
	Si $C,D \in \copt$ luego 
	\[
	|\{ E \in \copt : C \subseteq E \subseteq D \}| < \infty
	\]
\end{lema}
\begin{proof}
	Tomemos un camino $\gamma$ tal que salga de $c \in C$ y termine en $d \in \ol D$.
	
	Si $E$ es un corte luego tiene que separar a $\gamma$ caso contrario tendríamos que $d \in E \cap \ol D$.
	Como el grafo que estamos considerando es accesible sabemos que $E \in \copt$ implica que $E$ es un $k$-corte.
	Si ahora usamos el lema \ref{lema_finitos_kcortes} con el conjunto finito $\gamma$ esto nos dice que solo existen finitos cortes $E$ que cumplen lo pedido.
\end{proof}

Ahora vamos a definir la siguiente relación sobre los cortes óptimos.

\begin{deff}
	Dos cortes $C \sim D \in \copt$ si y solo sí
	\[
		C = D \ \text{o bien} \ \ol C \subsetneq D \ \text{y} \ \forall E \in \copt : \ol C \subsetneq E \subseteq D \implies D = E
	\]
\end{deff}

\begin{prop}
	La relación $\sim$ es de equivalencia.
\end{prop}
\begin{proof}
	Por como la definimos es clara que es reflexiva.
	Para ver que es simétrica notemos que si $C \sim D$ y son cortes distintos entonces  
	\[
		\ol C \subsetneq D \implies \ol D \subsetneq C
	\]
	por lo tanto volviendo a tomar complemento obtenemos que
	\[
	\forall E \in \copt : \ol C \subsetneq E \subseteq D \implies D = E \implies \forall E \in \copt : \ol D \subsetneq E \subseteq C \implies C = E.
	\]
	
	Finalmente nos queda ver que la relación resulta ser transitiva.
	Esto es que para ciertos cortes óptimos tenemos que $C \sim D$, $D \sim E$ y queremos ver que $C \sim E$.
	Una primera observación que podemos hacer es que $\emptyset \neq \ol D = C \cap E$.
	Esto sale de que  
	
	Por la proposición \ref{lema_cortes_anidados} tenemos que $C,E$ están anidados.
	Veamos que sucede en los cuatro posibles casos.
	\begin{itemize}
		\item $C \subseteq E$. 
		Tenemos que $\ol D \subsetneq E$ con $\ol D \subsetneq F \subseteq E$ y esto implica que $F=E$.
		Si tomamos $F = C$ entonces obtenemos lo que queríamos ver.
		\item $E \subseteq C$ análogo al caso anterior.
		\item $E \subseteq \ol C$ contradice que $C \cap E \neq \emptyset$.
		\item $\ol C \subseteq E$.
		Como $C \cap E \neq \emptyset$ esto nos dice que $\ol C \subsetneq E$ por lo tanto tenemos que si $F \in \copt$ luego	$\ol C \subsetneq F \subseteq E$ queremos ver que $F=E$.
		Lo separamos en cuatro casos nuevamente gracias a la proposición \ref{lema_cortes_anidados} dado que $F$ debe estar anidado con $D$.
		\begin{itemize}
			\item $D \subseteq F$.
			Esto nos dice que $D \subseteq E$ pero por lo visto tenemos que $\ol D \subseteq E$ y esto es una contradicción porque al ser $E$ un corte no puede ser todo el conjunto de vértices $V(\Gamma)$.
			\item $D \subseteq \ol F$. 
			Esto nos dice que $F \subseteq \ol D$ y por lo tanto tenemos $\ol C \subsetneq F \subseteq \ol D$ pero esto nos dice que $\ol C \subseteq \ol D$ y como $\ol D \subseteq C$ llegamos a una contradicción.
			\item $\ol D \subseteq F$.
			Usamos que $D \sim E$ por lo tanto concluimos que $F=E$ tal como queríamos ver.			
			\item $\ol D \subseteq \ol F$.
			Esto nos dice que $\ol C \subsetneq F \subseteq D \subseteq E$ por lo tanto obtenemos que $F = D$ o bien contradecimos que $C \sim D$.
		\end{itemize}
	\end{itemize}
\end{proof}

\begin{deff}
	Sea $\Gamma$ un grafo conexo, accesible y localmente finito y sea $\copt$ el conjunto de sus cortes óptimos.
	El árbol de estructura $T(\copt)$ de un grafo $\Gamma$ es el siguiente grafo.
	\begin{align*}
		V(T(\copt)) &= \{ [C] : C \in \copt \} \\
		E(T(\copt)) &= \{ ([C], [\ol C]) : C \in \copt   \}
	\end{align*}
\end{deff}

\begin{obs}
	Por nuestra construcción el árbol de estructura no es localmente finito.
	El grado de cada vértice $[C] \in V(T(\copt))$ está acotado por el órden de la clase de equivalencia de $C$ y éste podría ser infinito.
\end{obs}

\begin{prop}
	El grafo $T(\copt)$ es un árbol.
\end{prop}

\begin{proof}
	Debemos ver que es acíclico y conexo.
	
	Primero veamos que no tiene ciclos.
	Supongamos que $\gamma$	es un ciclo y aparte tomesmolo para que sea simple luego tendría la siguiente forma
	\[
		[C_0], [C_1], [C_2], \dots [C_{n}], [C_0]
	\]
	donde $\ol C_{i} \sim C_{i+1}$ para todo $i=1 \dots n$.
	Por otro lado al ser simple nos dice que $C_{i-1} \nsim C_{i+1}$ caso contrario tendríamos que el camino tiene backtracking.
	Esto nos da una cadena de inclusiones
	\begin{equation}\label{eq:inclusiones}
			C_0 \subsetneq C_1 \subsetneq \dots C_{n-1}
	\end{equation}
	tal que $\ol C_{n-1} \sim C_n \sim \ol C_0$.
	Veamos que no puede haber una arista $(C_{n-1}, C_0)$.
	Si la hubiera tendría que valer alguna de estas dos inclusiones:
	\begin{enumerate}
		\item $\ol C_{n-1} \subsetneq C_0$.
		Usando \ref{eq:inclusiones}, esto nos diría que $\ol C_{n-1} \subseteq C_{n-1}$ que es una contradicción.
		\item $\ol C_{0} \subsetneq C_{n-1}$.
		Nuevamente usando las inclusiones \ref{eq:inclusiones} llegaríamos a que $V(\Gamma) = C_0 \cup \ol C_0 \subseteq C_{n-1} \subsetneq V(\Gamma)$ que también es una contradicción.
	\end{enumerate}
	Por lo tanto el grafo resulta ser acíclico.
	
	Veamos ahora que es conexo.
	Sean $[C], [D] \in V(T(\copt))$, veamos de construir un camino entre ellos.
	Por la proposición \ref{lema_cortes_anidados} tenemos que necesariamente están anidados.
	Sin pérdida de generalidad podemos suponer que $C \subseteq D$ dado que
	\[
	([C], [\ol C]), ([D],[\ol D]) \in E(T(\copt))
	\]
	Por el lema \ref{lema_intermedios} sabemos que hay finitos cortes $E$ intermedios.
	Tomemos entonces una sucesión creciente no refinable de cortes de manear que llegamos 
	\[
		C=C_0 \subsetneq C_1 \subsetneq \dots C_n = D
	\]
	luego por como tomamos estos cortes obtenemos que $\ol C_i \sim C_{i+1}$ por lo tanto obtenemos el siguiente camino $[C],[C_1], \dots, [C_{n-1}],[D]$ en el grafo $T(\copt)$ probando así que es conexo.	
\end{proof}

\subsection{Acciones sobre el árbol de estructura.}

Sea $\Gamma$ un grafo accesible, conexo y localmente finito tal que $\aut(\Gamma)$ actúa con finitas órbitas sobre este grafo.

Una primera observación que podemos hacer es que $\aut(\Gamma)$ actúa sobre $\copt$.
Para eso notemos que si $C$ es un corte luego si $\varphi \in \aut(\Gamma)$ tenemos que $\varphi(C)$ es conexo y no vacío. 
Por otro lado al ser un morfismo del grafo $\Gamma$ obtenemos que $\ol{\varphi(C)} = \varphi(\ol C)$ por lo tanto su complemento sigue siendo conexo.
Al ser un automorfismo tenemos que $\varphi(\beta C) = \beta (\varphi C)$ y por este motivo tenemos que $|\varphi (\beta C)| = |\beta (\varphi C)|$ y esto nos dice que manda $k$-cortes en $k$-cortes.
Veamos que manda un corte minimal en uno minimal. 
Para eso si $C \in \copta$ para cierto camino bi-infinito $\alpha$ luego notemos que $\varphi(\alpha)$ es otro camino bi-infinito y por ser un automorfismo obtenemos que $\varphi(C) {\cal C}_{\varphi(\alpha)}$.
Similarmente tenemos que si $C$, $D$ no están anidados luego necesariamente $\varphi(C)$ y $\varphi(D)$ no están anidados tampoco.
Caso contrario si $\varphi(C) \cap \varphi(D) = \emptyset$ luego tendríamos que al ser $\varphi \in \aut(\Gamma)$ que preserva las intersecciones y tiene una inversa por lo que $C \cap D = \emptyset$ y esta es la contradicción.
Juntando todo nos dice que $\aut(\Gamma)$ actúa sobre los cortes óptimos.

Esta acción no solo es sobre el conjunto $\copt$ sino que podemos ver que actúa por medio de morfismos de grafos sobre el árbol de estructura $T(\copt)$.
Para esto notemos que si $C \sim D \in \copt$ son cortes óptimos relacionados luego tenemos que para todo $\varphi \in \aut(\Gamma)$ vale que $\varphi(C) \sim \varphi(D)$.
Para eso notemos que si $\ol C \subsetneq E \subset D$ implica que $E = D$ luego si tenemos que
\[
	\ol {\varphi(C)} \subsetneq E \subset \varphi(D) \iff C \subsetneq \varphi(E) \subset D
\]  
donde usamos que $\varphi$ es un automorfismo y que actúa sobre los cortes óptimos.
Esto nos dice que la acción dada por $\varphi([C]) = [\varphi (C)]$ está bien definida y por lo tanto actúa sobre los vértices.
Para ver que actúa sobre las aristas notemos que al ser un morfismo de grafos y por actúar sobre los vértices que $[\ol \varphi(C)] = [\varphi (\ol C)]$.

\begin{lema}
	Sea $\Gamma$ un grafo accesible.
	Si $\aut(\Gamma)$ actúa con finitas órbitas sobre $\Gamma$ luego actúa con finitas órbitas sobre $\copt$ y sobre el árbol de estructura $T(\copt)$.
\end{lema} 
\begin{proof}
	Veamos primero que actúa con finitas órbitas sobre $\copt$.
	Para eso tenemos que existe $A \in V(\Gamma)$ dominio fundamental finito de la acción de $\aut(\Gamma)$ sobre $\Gamma$.
	Por lo tanto todo corte se puede tomar alguno que esté en su misma órbita, llamesmolo $C$ de manera que $C \cap A \neq \emptyset$.
	Como el grafo es accesible todo corte óptimo es un $k$-corte para un cierto $k$.
	Por el resultado \ref{lema_finitos_kcortes} tenemos que hay finitos cortes posibles por lo tanto tenemos que hay finitas órbitas posibles para esta acción.
	
	Para ver que son finitas las órbitas sobre el árbol de estructura notemos que al ser $V(T\copt)$ isomorfo a un cociente de $\copt$ luego tenemos que en particular las órbitas sobre el árbol de estructura también tienen que ser finitas.
\end{proof}

Un caso particular que nos interesa es cuando el grafo $\Gamma$ es el grafo de Cayley para cierto grupo \fg $G$.
En este caso tenemos que 

Nuestro objetivo ahora es entender cómo es la acción 

\end{document}