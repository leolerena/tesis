\documentclass[tesis.tex]{subfiles}

\begin{document}
\section{Cortes de grafos y árboles de estructura.}

\begin{deff}
	Sea $\Gamma$ un grafo conexo y localmente finito. 
	Sea $C \subseteq V(\Gamma)$ luego definiremos la frontera de este conjunto como:	
	\[
	\delta C = \{  (x,y) | x \in C, y \in \ol C    \}
	\]
\end{deff}


\begin{deff}
	Dado $C \subset V(\Gamma)$ diremos que es un \emph{corte} si cumple las siguientes condiciones,
	\begin{itemize}
		\item $C$ y $\ol C$ son conexos y no vacíos.
		\item $|\delta C| \le \infty$.
	\end{itemize}
	Si $|\delta C| = k$ diremos que $C$ es un \emph{$k$-corte}.
\end{deff}	

Idealmente queremos que nuestros cortes nos separen en partes infinitas al grafo pero esto no siempre es posible.
Un ejemplo claro es tomar el grafo de Cayley de $\ZZ^2$.
	
\begin{lema}
	Sea $\Gamma$ un grafo conexo y localmente finito.
	Sea $S \subset V$ un conjunto finito y $k\ge 1$.
	Entonces existen finitos $k$-cortes $C$ con $\beta C \cap S \neq \emptyset$.
\end{lema}	

\begin{proof}
	Primero notemos que nos alcanza con ver los cortes $C$ que tienen una arista fija $(x,y) \in \delta C$.
	Si vemos que estos son finitos entonces como tenemos finitos $x \in S$ y cada vértice tiene finitas aristas dado que el grafo es localmente finito entonces tenemos que la cantidad de $k$ cortes sería finita.
	Consideremos entonces que $S = \{ x \}$ porque los otros casos se siguen de este.
	
	Vamos a probarlo por inducción en $k$.
	
	En el caso base tenemos que $\delta C = (x,y)$.
	En este caso tenemos que solo puede haber finitos cortes (al menos tantos como aristas salen del vértice $x$).
	
	Ahora veamos el paso inductivo.
	Como suponemos que es un $k$-corte con $k \ge 2$ al menos tenemos una arista sea esta $(x,y)$.
	Notemos que si $\Gamma \setminus (x,y)$ no está conectado entonces no puede haber un corte $C'$ tal que $(x,y) \in \delta C'$ y que cumpla que $|\delta C'| \ge 2$.
	Esto nos dice que nos queda ver el caso que $\Gamma \setminus (x,y)$ sigue siendo conexo. 
	Al seguir siendo conexo debe existir un camino de vértices $\gamma$  en $ \Gamma \setminus (x,y) $ tal que $\gamma = x_0 x_1 \dots x_{n}$ con $x_0 = x$ y con $x_{n} = y$.
	Todos estos $k$-cortes $C$ que estamos considerando tienen que cumplir que alguno de las aristas $(x_i,x_{i+1}) \in  \gamma \cap \delta C$.
	Caso contrario tendríamos que para todo $x_i \in \gamma$ vale que $x_{i} \in C$ o bien $x_i \in \ol C$ contradiciendo que $(x,y) \in \delta C$.
	Finalmente como todo $k$-corte que contiene a $(x,y)$ restringido a $\gamma$ resulta ser un $(k-1)$-corte tenemos que al ser estos finitos por la hipótesis inductiva que la cantidad de $k$-cortes es finita tal como queríamos ver.
	
\end{proof}
	
\begin{deff}
	Dado $\Gamma$ un grafo conexo diremos que un camino simple $\alpha$ es un \emph{camino infinito }si
	\[
		\alpha = v_0 \dots v_{n} \dots
	\]	
	Diremos que es un \emph{camino bi-infinito } si
	\[
		\alpha =  \dots v_{-n} \dots v_0 \dots v_{n} \dots 
	\]
\end{deff}	
	

Una primera observación es que dado un corte $C$ tal que $C$ es infinito y $\ol C$ también lo es, podemos armarnos un camino bi infinito $\alpha$ tal que $|\alpha \cap C| = \infty = |\alpha \cap \ol C|$.

No necesariamente vale la vuelta.
Esto es que si tenemos un camino bi-infinito $\alpha$ y un corte $C$ entonces el corte cumple que $|\alpha \cap C| = \infty = |\alpha \cap \ol C|$.

Esto nos lleva a dar la siguiente definición.

\begin{deff}
	Sea $\Gamma$ un grafo conexo y sea $\alpha$ un camino bi-infinito.
	Definimos el conjuntos de cortes del camino como 
	\[
		{\cal C}(\alpha) = \{ C \subset V(\Gamma) \mid  C \ \text{es un corte y} \ |\alpha \cap C| = \infty = |\alpha \cap \ol C| \}
	\] 
\end{deff}

\begin{lema}
Tenemos que ${\cal C}(\alpha) \neq \emptyset$ si y solo sí existe un corte $C$ de manera que $\alpha \setminus \delta C$ tiene dos componentes infinitas.
\end{lema}
\begin{proof}
	...
\end{proof}	
	
\begin{deff}
	Dado un grafo $\Gamma$ y $S \subset V(\Gamma)$ diremos que $\Gamma$ tiene más de un end si $\Gamma \setminus S$ tiene más de una componente conexa.
	Caso contrario diremos que tiene al menos un end.
\end{deff}

%Esta definición de ends para grafos de Cayley resulta ser equivalente a la definición de ends para grupos.
%En el trabajo \cite{} está probada la equivalencia entre estos resultados...	

\begin{deff}
	Un grafo $\Gamma$ es \emph{accesible} si existe $k \in \NN$ de manera que todo camino bi-infinito $\alpha$ cumple que ${\cal C}(\alpha) = \emptyset$ o bien ${\cal C}(\alpha)$ contiene un $k$-corte.
\end{deff}

Veamos que los grafos con treewidth finito son accesibles.
Para eso primero veamos un lema un poco más técnico.


\begin{lema}
	Sea $\Gamma$ un grafo de treewidth finito y grado uniformemente acotado.
	Entonces existe $k \in \NN$ tal que:
	Para todo $\gamma$ camino infinito simple, todo $v_0 \in V(\Gamma)$ y todo $n \in \NN$ debe existir un $k-$corte $D$ que cumple las siguientes propiedades, $d(v_0,\ol D) \ge n, v_0 \in D, |\ol D \cap \gamma| = \infty$. 
\end{lema}

\begin{proof}
	Sea $d$ una cota para el grado de los vértices del grafo $\Gamma$ y sea $k$ el treewidth del grafo.
	Propondremos $k = dm$ como la constante que buscamos.
	
	Consideremos $t_0 \in V(T)$ de manera que $v_0 \in X_t$.
	Si tenemos dos vértices $u_1,u_2 \in \ol X_t$ tales que $u_1$ y $u_2$ están en bolsones $X_{t_1}, X_{t_2}$ de manera que $t_1$ y $t_2$ están en componentes conexas distintas de $T \setminus t_0$, luego como los árboles son únicamente geodésicos y usando la prop \ref{} obtenemos que todo camino $\gamma$ que pase por $u_1$ y por $u_2$ debe pasar en algún momento por $X_{{t}_0}$.
	Como estos caminos que consideramos son simples y los bolsones $|X_{t_0}| < \infty$ obtenemos que debe haber una única componente conexa de $\ol X_{t_0}$ tal que interseca infinitas veces al camino $\gamma$.
	Nombraremos a esta componente conexa $C_{t_0,\gamma}$.
	
	Repetimos este procedimiento por cada vértice de la componente conexa de $\ol X_{t_0}$.
	Esto nos da un camino infinito en el árbol $t_0,t_1, \dots t_n \dots$ y una sucesión de componentes conexas correspondientes $C_{t_0,\gamma} C_{t_1,\gamma} \dots C_{t_n, \gamma} \dots$.
	
	Si elegimos $l \in \NN$ suficientemente grande podemos garantizar que
	\[
		X_{t_l} \cap B_n(v_0) = \emptyset
	\]
	Esto porque por la prop \ref{} tenemos que la descomposición se puede tomar de manera que cada vértice aparezca en finitos bolsones y que el grafo $\Gamma$ al ser localmente finito nos dice que la bola $|B_n(v_0)| < \infty$.
	
	Ahora vamos a buscar el corte $D$. 
	Para eso vamos a necesitar un conjunto conexo con complemento conexo y no vacío tal que su borde sea finito.
	Consideremos $D$ la componente conexa de $v_0$ en $\ol C_{t_l, \gamma}$.
	Primero veamos que $\ol D$ es conexo.
	Para eso si tenemos un elemento $u_1 \in \ol D$ tiene que estar conectados con alguien del conexo $C_{t_l, \gamma}$.
	Más aún todo camino con algún elemento de $D$ tiene que pasar por $C_{t_l, \gamma}$.
	Caso contrario como tenemos un camino que lo une con $v_0$, dado que el grafo es conexo, esto resultaría en que $u_1 \in D$.
	De esta manera vimos que todo elemento de $\ol D$ está conectado con un conexo $C_{t_l, \gamma}$ evitando pasar por $D$.
	
	Ambos conjuntos son no vacíos porque $C_{t_l, \gamma}$ es infinito y está contenido en $\ol D$.
	Por otro lado tenemos que $v_0 \in D$.
	
	Finalmente veamos que es un $k-$corte.
	Notemos que si $v \in \partial D$ luego tenemos que existe $u \in \ol D$ tal que $(u,v) \in E(\Gamma)$.
	Por lo visto anteriormente todo camino que va de $\ol D$ a $D$ tiene que pasar necesariamente por $C_{t_l, \gamma}$. 
	Entonces nos alcanza con tomar $u \in C_{t_l, \gamma}$.
	En este caso notemos que $v \in X_{t_l}$ caso contrario tendríamos que $v \in C_{t_l, \gamma}$ dado que esta es una de las componentes conexas de $X \setminus X_{t_l}$.
	De esta manera vimos que $\partial D \subseteq X_{t_l}$.
	Como el grafo es localmente finito y tiene treewidth finito concluímos finalmente que $|\delta D| \le dm = k$ tal como queríamos ver.
		
\end{proof}


\begin{teo}
	Sea $\Gamma$ un grafo de treewidth finito y de grado uniformemente acotado.
	Entonces $\Gamma$ es accesible.
\end{teo}
\begin{proof}
	Tomemos $\alpha$ un camino bi-infinito tal que ${\cal C}(\alpha) \neq \emptyset$.
	Sea $C \in {\cal C}(\alpha)$.
	Veamos que $C$ es un $k-$corte.
\end{proof}


	
\end{document}