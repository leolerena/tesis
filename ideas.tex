\documentclass[12pt]{article}
\usepackage{geometry}\geometry{top=5cm,bottom=2cm,left=3cm,right=3cm}
\usepackage[utf8]{inputenc}
\usepackage[spanish]{babel}
\usepackage{amsmath,amsfonts,amsthm,amssymb,mathtools,sectsty}
\pagenumbering{gobble}
\usepackage{subcaption}
%\usepackage{graphicx}
%\usepackage[pdftex,dvipsnames]{xcolor}
\usepackage{cancel}
\usepackage{graphicx}
\usepackage{marginnote}

% Tikz y las librerías para automátas
\usepackage{tikz-cd}
\usepackage{tikz}
\usetikzlibrary{arrows,automata}
\usetikzlibrary{babel} %para evitar que se jodan los automatas de tikz

%Referencias; me gustaría que backref funcione pero no es importante tampoco.
\usepackage[pagebackref]{hyperref}
% Para modificar el estilo de las referencias
\hypersetup{
	colorlinks,
	linkcolor={astral},
	citecolor={blue!30!black},
	urlcolor={blue!80!black}
}
\definecolor{astral}{RGB}{46,116,181}
\colorlet{chulo}{blue!70!purple}
\colorlet{rojo}{purple!65!black}


\paragraphfont{\color{astral!70!black}}
\chapterfont{\color{astral!40!black}}
\subsectionfont{\color{astral!60!black} }
\sectionfont{\color{astral!50!black} }
\usepackage{mathpazo}
\usepackage{amssymb}
\usepackage{eufrak}
%\usepackage{thmtools}

%Esto sirve para armar grafos de Cayley de una manera más copada.
\usetikzlibrary{lindenmayersystems,arrows.meta}
\newcount\quadrant
\pgfdeclarelindenmayersystem{cayley}{
	\rule{A -> B [ R [A] [+A] [-A] ]}
	\symbol{R}{ \pgflsystemstep=0.5\pgflsystemstep } 
	\symbol{-}{
		\pgfmathsetcount\quadrant{Mod(\quadrant+1,4)}
		\tikzset{rotate=90}
	}
	\symbol{+}{
		\pgfmathsetcount\quadrant{Mod(\quadrant-1,4)}
		\tikzset{rotate=-90}
	}
	\symbol{B}{
		\draw [dot-cayley] (0,0) -- (\pgflsystemstep,0) 
		node [font=\footnotesize, midway, 
		anchor={270-mod(\the\quadrant,2)*90}, inner sep=.5ex] 
		{\ifcase\quadrant$a$\or$b$\or$a^{-1}$\or$b^{-1}$\fi};
		\tikzset{xshift=\pgflsystemstep}
	}
}
\tikzset{
	dot/.tip={Circle[sep=-1.5pt,length=3pt]}, cayley/.tip={Stealth[]dot[]}
}

%%%%%%%%%%%%%  TEOREMAS  %%%%%%%%%%%%%%%%%
\theoremstyle{plain} %% el estilo clásico
\newtheorem{teo}{\color{rojo} Teorema}[section]
\newtheorem{prop}[teo]{\color{rojo} Proposición}
\newtheorem{lema}[teo]{\color{rojo} Lema}
\newtheorem{coro}[teo]{\color{rojo} Corolario}
\newtheorem{preg}[teo]{\color{rojo} Pregunta}
% Si pongo [theorem] siguen la numeración de los teoremas. 
% e.j. Teo 1, Lema 2, Teo 3, Teo 4 ...
\theoremstyle{definition}
\newtheorem{deff}{\color{orange!80!black} Definición}[section] 
\newtheorem{ej}{\color{orange!80!black} Ejemplo}[section]

% Remarks
\theoremstyle{remark}
\newtheorem{obs}{\color{orange!85!white} Observación}[section]



%Comandos útiles.
\newcommand\RP{\mathbb{RP}}
\newcommand{\norm}[1]{\left\lVert#1\right\rVert}
\newcommand{\ti}{\tilde}
\newcommand{\RR}{\mathbb R}
\newcommand{\CC}{\mathbb C}
\newcommand{\NN}{\mathbb N}
\newcommand{\ZZ}{\mathbb Z}
\newcommand{\Om}{\Omega}
\newcommand{\A}{\mathcal A}
\newcommand\ol{\overline}
\newcommand{\blue}{\textcolor{chulo}}
\newcommand{\red}{\textcolor{rojo}}
\newcommand{\Gg}{\mathfrak g}
\newcommand{\SL}{SL_2(\mathbb Z)}
\newcommand{\stab}{\text{Stab}}

%%%%%%%%%%%%%%  PAGE SETUP %%%%%%%%%%%%%%%%%


%opening
\usepackage{fancyhdr}
\pagestyle{fancy}
\lhead{Tesis de licenciatura.} % Left Header
\rhead{\thepage} % Right Header

\usepackage{subfiles} % mejor ponerlos al final
\title{Ideas sueltas.}
\date{}

\begin{document}
	
	
\maketitle
	
\section{Dowker y treewidth.}	

Estaría bueno que el teorema de Dowker nos diga algo sobre el treewidth de un grafo.

Si consideremos los mapas
\[
X: V(T) \to V(X)
\]
y 
\[
T: V(X) \to V(T)
\]
entonces podemos definirnos una relación $R \subset V(T) \times V(X)$ por medio de $xRt$ si $x \in X_t$.

A partir del teorema de Dowker nos podemos construir complejos simpliciales $K_X$ y $K_T$ tales que resultan homotópicos entre sí.


Una observación de esto es que 
\begin{obs}
	Si $X$ es un árbol entonces es isomorfo como grafo a su complejo de Dowker. 
	Esto es que $X \simeq K_X$.
\end{obs}

En particular 

\begin{obs}
	$T$ siempre resulta ser un spanning tree del complejo $K_T$.
\end{obs}
	
\section{Otra construcción para que el treewidth sea conexo.}	

Me gustaría probar el siguiente resultado o al menos ver dónde se rompe.

\begin{preg}
	Todo grupo fundamental de un grafo de grupos finito con grupos finitos tiene treewidth conexo finito.
\end{preg}

Lo bueno de esto sería dar una descomposición de un árbol distinta a la que aparece en \cite{diekert_contextfree_2017} tal que sea conexa.

Acá estoy usando el siguiente resultado de \cite{karrass1973finite}.
\begin{teo}
	Un grupo es virtualmente libre sii es el grupo fundamental de un grafo de grupos finito con grupos finitos.
\end{teo}

Primero pensé en el caso de un grupo $G = G \ast H$ que en particular es el grupo fundamental del grafo que tiene dos vértices con grupos $G$ y $H$ que asumo finitos. El grupo de la arista es $1$.

Para este caso quería tomar la siguiente descomposición.
\begin{align*}
V(T) &= \{ v \in G_1 \ast G_2 : v = wa_ih  \} & h \in H, w \in G_1 \ast G_2 \\
 & = \{ v \in G_1 \ast G_2 : v = wb_jg  \} &  g \in G, w \in G_1 \ast G_2 \\
 & = \{ v \in G_1 \ast G_2 : v = wb_ja_i  \} & w \in G_1 \ast G_2 \\
 & = \{ v \in G_1 \ast G_2 : v = wa_ib_j  \} & w \in G_1 \ast G_2 \\
 & = \{ v \in G_1 \ast G_2 : v = a_i, b_j  \} & w \in G_1 \ast G_2 \\
\end{align*}
	
Por como está construído es claro que $T$ es conexo.

Habría que probar que no tiene ciclos... 
\red{HACER}

Para ver que es una descomposición habría que ver las tres propiedades necesarias.

\begin{itemize}
	\item Cubren todo.
	\item Todas las aristas están en algún bolsón por construcción.
	\item Todo vértice está a lo sumo en dos bolsones que resultan conexos por construcción.
\end{itemize}	
	
Finalmente los bolsones son conexos porque básicamente agregamos todas aristas para que nos queden conexas. 
Aparte los bolsones tienen dos vértices conectados por una arista o bien son una copia del grafo de Cayley de $G$ o de $H$.
	
	
Esta misma idea creo que se puede extender fácilmente para obtener una descomposición para un producto libre arbitrario de grupos finitos.	
	
	
Quedaría ver entonces el siguiente resultado.

\begin{preg}
	Un grupo amalgamado $G \ast_{K} H$	de grupos finitos tiene treewidth conexo finito.
\end{preg}	
	
No me queda claro que la idea de antes pueda andar en este caso. 

	
	
\begin{tikzpicture}[
	vertex/.style={draw,circle,fill=white,inner sep=0pt,minimum size=6mm},
	edge/.style={->,shorten >=1pt,>=stealth',semithick}
	]
	
	% Define the vertices
	\node[vertex] (1) at (0,0) {$1$};
	\node[vertex] (2) at (2,0) {$2$};
	
	% Draw the edges
	\draw[edge] (1) to[bend left] (2);
	\draw[edge,color=red] (1) to[bend right] (2);
	
\end{tikzpicture}
	
	
	
	
	
	
	
	
	
	
	
	
\bibliography{bibliografia}
\bibliographystyle{alpha}	
\end{document}