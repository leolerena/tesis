\usepackage{xcolor}
\usepackage{amssymb}
\usepackage{pifont}
\newcommand{\xmark}{\ding{55}}
\usetheme[progressbar=frametitle]{metropolis}
\usepackage{appendixnumberbeamer}
\usepackage{hyperref}
\usepackage{eufrak}
\usepackage{tikz-cd}
\usepackage{tikz}
\usetikzlibrary{arrows,automata}
\usetikzlibrary{babel} %para evitar que se jodan los automatas de tikz
\usetikzlibrary{graphs} 
\usetikzlibrary{calc}
\usetikzlibrary{positioning}
\usepackage{tcolorbox}
\usepackage{enumitem}% http://ctan.org/pkg/enumitem
\definecolor{ao(english)}{rgb}{0.0, 0.5, 0.0}

\setbeamercolor{background canvas}{bg=white}
\usepackage{multicol}
\usepackage{chronology}
%%%%%%%%%%%%%%%%%%%%% ENUMERAR CON COSAS QUE NO SEAN SOLO NÚMEROS %%%%%%%%%%
%\usepackage[shortlabels]{enumitem}
%\setlist[enumerate]{font=\bfseries}

\usepackage{booktabs}
%\usepackage[scale=2]{ccicons}

\usepackage{clrscode3e}

%%%%%%%% ALGORITMOS %%%%%%%%
%\usepackage{algorithm}
%\usepackage[noend]{algpseudocode}  % "noend" es para no mostrar los endfor, endif

% PARA PODER HACER EL FLOW CHART
\usetikzlibrary{shapes}

\usepackage{pgfplots}
%\usepgfplotslibrary{dateplot}
\colorlet{verde}{green!50!black}
\definecolor{amber}{rgb}{1.0, 0.49, 1.15}
\setbeamercolor{progress bar}{fg=applegreen!70!black,bg=alerted text.fg!50!black!10}

\makeatletter
\setlength{\metropolis@titleseparator@linewidth}{2pt}
\setlength{\metropolis@progressonsectionpage@linewidth}{2pt}
\setlength{\metropolis@progressinheadfoot@linewidth}{2pt}
\makeatother

\definecolor{ultramarine}{RGB}{81, 11, 122} 
\setbeamercolor{frametitle}{bg= astral!60!black}

\usepackage{framed}
\definecolor{shadecolor}{gray}{0.9}

\renewcommand\qedsymbol{\textcolor{orange}{$\blacksquare$}}

\usepackage[font={footnotesize}]{caption}

% Para agregar columnas.
\usepackage{graphicx}
\usepackage{adjustbox}
\setbeamercovered{invisible}



\usepackage[framemethod=tikz]{mdframed}

%%%%%%%%%%%%%  TEOREMAS  %%%%%%%%%%%%%%%%%
\theoremstyle{plain} %% el estilo clásico
\newtheorem{teo}{\color{rojo}{ \textsc{ Teorema}}}[section]
\newtheorem{prop}[teo]{\color{rojo} \textsc{Proposición}}
\newtheorem{lema}[teo]{\color{rojo} \textsc{Lema}}
\newtheorem{coro}[teo]{\color{rojo} \textsc{Corolario}}
% Si pongo [theorem] siguen la numeración de los teoremas. 
% e.j. Teo 1, Lema 2, Teo 3, Teo 4 ...
\theoremstyle{definition}
\newtheorem{deff}[teo]{\textsc{ Definición}}{\smallskip}
\newtheorem{ej}[teo]{\textsc{Ejemplo}}{\smallskip}

% Remarks
\theoremstyle{remark}
\newtheorem{obs}[teo]{ \textsc{Observación}}{\smallskip}

% DEMOS EN NEGRITA.
\renewenvironment{proof}{{\textbf{Demostración.}}}{ \hfill $\blacksquare$ \medskip}

%COLORES

\definecolor{astral}{RGB}{46,116,181}
\colorlet{chulo}{blue!70!purple}
\colorlet{rojo}{purple!45!black}
\definecolor{carrotorange}{rgb}{0.93, 0.57, 0.13}
\definecolor{brightcerulean}{rgb}{0.11, 0.67, 0.84}
\definecolor{brightube}{rgb}{0.82, 0.62, 0.91}
\definecolor{cadmiumred}{rgb}{0.89, 0.0, 0.13}
\definecolor{applegreen}{rgb}{0.55, 0.71, 0.0}
\definecolor{aurometalsaurus}{rgb}{0.43, 0.5, 0.5}

%%%%%%%%%%%%%%%%%%%%% ENUMERAR CON COSAS QUE NO SEAN SOLO NÚMEROS %%%%%%%%%%
%\usepackage[shortlabels]{enumitem}
%\setlist[enumerate]{font=\bfseries}


\newcommand\RP{\mathbb{RP}}
\newcommand{\norm}[1]{\left\lVert#1\right\rVert}
\newcommand{\RR}{\mathbb{R}}
\newcommand{\CC}{\mathbb{C}}
\newcommand{\NN}{\mathbb{N}}
\newcommand{\ZZ}{\mathbb{Z}}
\newcommand{\deri}{{\cal D}}

\newcommand{\Om}{\Omega}
\newcommand{\A}{\mathcal A}
\newcommand\ol{\overline}
\newcommand{\blue}{\textcolor{chulo}}
\newcommand{\red}{\textcolor{rojo}}
\newcommand{\Gg}{\mathfrak g}
\newcommand{\SL}{SL_2(\mathbb Z)}
\newcommand{\stab}{\text{Stab}}
\newcommand{\ic}{independiente de contexto}
\newcommand{\APND}{automáta de pila no determinístico}
\newcommand{\APD}{automáta de pila determinístico }
\newcommand{\gramatica}{{\cal G} = (V, \Sigma, P, S)}
\newcommand{\deriva}{\overset{*}{\to_{\cal G}}}
\newcommand{\tto}{\overset{*}{\to}}
\newcommand{\lengderivado}{L({\cal G})}
\newcommand{\fg}{finitamente generado }
%\newcommand{\ol}{\overline{}}

\newcommand*{\cay}[2]{\text{Cay}(#1, #2)}
\newcommand*{\partes}[1]{{\cal{P}}(#1)}

\newcommand{\fp}{finitamente presentado }
\newcommand{\vl}{virtualmente libre}
\newcommand{\vls}{virtualmente libres}
\newcommand{\WP}[2]{\text{WP}(#1, #2)}

\newcommand*{\diam}[1]{\text{diam}\ (#1)}

\newcommand{\cG}{ {\cal G} }
\newcommand{\cGg}{{\cal G} = (V, \Sigma, P, S)}
\newcommand{\cH}{ {\cal H} }
\newcommand{\Xm}{\widetilde X}
%\newcommand{\ol}{\overline{}}
\newcommand*{\TODO}[1]{\textcolor{red}{TODO:#1}}


\newcommand{\ca}[1]{{\cal C}(#1)}
\newcommand{\cmin}[1]{{\cal C}_{\text{min}}(#1)}
%\newcommand{\cam}{{\cal C}_{\text{min}}(\alpha)}
\newcommand{\copt}[1]{{\cal C}_{\text{opt}}(#1)}
%\newcommand{\copt}{{\cal C}_{\text{opt}}}

\newcommand*{\tcajita}[2]{\begin{tcolorbox}[colback=green!5!white,colframe=green!35!black,title=#1]
    #2
  \end{tcolorbox}}

\newcommand*{\teocajita}[1]{\begin{tcolorbox}[colback=rojo!5!white,colframe=rojo!75!black,title=Teorema]
#1
\end{tcolorbox}}

\newcommand*{\lemacajita}[1]{\begin{tcolorbox}[colback=rojo!5!white,colframe=rojo!75!black,title=Lema]
    #1
    \end{tcolorbox}}

\newcommand*{\defcajita}[1]{\begin{tcolorbox}[colback=green!5!white,colframe=green!35!black,title=Definición]
    #1
    \end{tcolorbox}}    