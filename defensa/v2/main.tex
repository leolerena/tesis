\documentclass[aspectratio=169, 11pt]{beamer}
\usepackage[utf8]{inputenc}
\usepackage[spanish]{babel}
\usepackage{xcolor}
\usepackage{amssymb}
\usepackage{pifont}
\usepackage{amsmath,amsfonts,amsthm,amssymb,mathtools,sectsty}
\pagenumbering{gobble}
\usepackage{subcaption}
%\usepackage{graphicx}
%\usepackage[pdftex,dvipsnames]{xcolor}
\usepackage{cancel}
\usepackage{graphicx}
\usepackage{marginnote}

% Tikz y las librerías para automátas
\usepackage{tikz-cd}
\usepackage{tikz}
\usetikzlibrary{arrows,automata}
\usetikzlibrary{babel} %para evitar que se jodan los automatas de tikz

%Referencias; me gustaría que backref funcione pero no es importante tampoco.
\usepackage[pagebackref]{hyperref}
% Para modificar el estilo de las referencias
\hypersetup{
	colorlinks,
	linkcolor={astral},
	citecolor={blue!30!black},
	urlcolor={blue!80!black}
}
\definecolor{astral}{RGB}{46,116,181}
\colorlet{chulo}{blue!70!purple}
\colorlet{rojo}{purple!65!black}


\paragraphfont{\color{astral!70!black}}
\chapterfont{\color{astral!40!black}}
\subsectionfont{\color{astral!60!black} }
\sectionfont{\color{astral!50!black} }
\usepackage{mathpazo}
\usepackage{amssymb}
\usepackage{eufrak}
%\usepackage{thmtools}




\usepackage[framemethod=tikz]{mdframed}

%%%%%%%%%%%%%  TEOREMAS  %%%%%%%%%%%%%%%%%
\theoremstyle{plain} %% el estilo clásico
\newtheorem{teo}{\color{rojo}{ \textsc{ Teorema}}}[section]
\newtheorem{prop}[teo]{\color{rojo} \textsc{Proposición}}
\newtheorem{lema}[teo]{\color{rojo} \textsc{Lema}}
\newtheorem{coro}[teo]{\color{rojo} \textsc{Corolario}}
% Si pongo [theorem] siguen la numeración de los teoremas. 
% e.j. Teo 1, Lema 2, Teo 3, Teo 4 ...
\theoremstyle{definition}
\newtheorem{deff}[teo]{\textsc{ Definición}}{\smallskip}
\newtheorem{ej}[teo]{\textsc{Ejemplo}}{\smallskip}

% Remarks
\theoremstyle{remark}
\newtheorem{obs}[teo]{ \textsc{Observación}}{\smallskip}
\newtheorem{afirmacion}[teo]{ \textsc{Afirmación}}{\smallskip}

% DEMOS EN NEGRITA.
\renewenvironment{proof}{{\textbf{Demostración.}}}{ \hfill $\blacksquare$ \medskip}

%COLORES

\definecolor{astral}{RGB}{46,116,181}
\colorlet{chulo}{blue!70!purple}
\colorlet{rojo}{purple!45!black}
\definecolor{carrotorange}{rgb}{0.93, 0.57, 0.13}
\definecolor{brightcerulean}{rgb}{0.11, 0.67, 0.84}
\definecolor{brightube}{rgb}{0.82, 0.62, 0.91}
\definecolor{cadmiumred}{rgb}{0.89, 0.0, 0.13}
\definecolor{applegreen}{rgb}{0.55, 0.71, 0.0}
\definecolor{aurometalsaurus}{rgb}{0.43, 0.5, 0.5}

%%%%%%%%%%%%%%%%%%%%% ENUMERAR CON COSAS QUE NO SEAN SOLO NÚMEROS %%%%%%%%%%
%\usepackage[shortlabels]{enumitem}
%\setlist[enumerate]{font=\bfseries}


\newcommand\RP{\mathbb{RP}}
\newcommand{\norm}[1]{\left\lVert#1\right\rVert}
\newcommand{\RR}{\mathbb{R}}
\newcommand{\CC}{\mathbb{C}}
\newcommand{\NN}{\mathbb{N}}
\newcommand{\ZZ}{\mathbb{Z}}
\newcommand{\Om}{\Omega}
\newcommand{\A}{\mathcal A}
\newcommand\ol{\overline}
\newcommand{\blue}{\textcolor{chulo}}
\newcommand{\red}{\textcolor{rojo}}
\newcommand{\Gg}{\mathfrak g}
\newcommand{\SL}{SL_2(\mathbb Z)}
\newcommand{\stab}{\text{Stab}}
\newcommand{\ic}{independiente de contexto}
\newcommand{\APND}{automáta de pila no determinístico}
\newcommand{\APD}{automáta de pila determinístico }
\newcommand{\gramatica}{{\cal G} = (V, \Sigma, P, S)}
\newcommand{\deriva}{\overset{*}{\to_{\cal G}}}
\newcommand{\lengderivado}{L({\cal G})}
\newcommand{\fg}{finitamente generado }
%\newcommand{\ol}{\overline{}}

\newcommand{\fp}{finitamente presentado }
\newcommand{\vl}{virtualmente libre}
\newcommand{\vls}{virtualmente libres}
\newcommand{\WP}[2]{\text{WP}(#1, #2)}

\newcommand{\cG}{ {\cal G} }
\newcommand{\cGg}{{\cal G} = (V, \Sigma, P, S)}
\newcommand{\cH}{ {\cal H} }
\newcommand{\Xm}{\widetilde X}
%\newcommand{\ol}{\overline{}}

\newcommand{\vectres}[3]{(#1,#2,#3)}
\newcommand{\confgin}{(q_{0}, w, \$)}
\newcommand{\confguno}{(q, w, \gamma)}
\newcommand{\confgdos}{(p,u,\theta)}

\newcommand{\cay}[2]{\text{Cay}(#1,#2)}

\newcommand{\ca}[1]{{\cal C}(#1)}
\newcommand{\cmin}[1]{{\cal C}_{\text{min}}(#1)}
%\newcommand{\cam}{{\cal C}_{\text{min}}(\alpha)}
\newcommand{\copt}[1]{{\cal C}_{\text{opt}}(#1)}
%\newcommand{\copt}{{\cal C}_{\text{opt}}}
\newcommand*{\rows}{6}

\renewcommand{\indent}{\hspace*{2em}}
\makeatletter
\addtobeamertemplate{block begin}{}{\setlength{\parindent}{1cm}\@afterindentfalse\@afterheading}
\makeatother

\title{El problema de la palabra de los grupos virtualmente libres.}
\subtitle{\textbf{Leopoldo Lerena} \\
		Defensa de tesis de licenciatura.}
\date{Fecha: $\le 2050$.}
\author{Director: Iván Sadofschi Costa}
\institute{Universidad de Buenos Aires}
% \titlegraphic{\hfill\includegraphics[height=1.5cm]{logo.pdf}}

\begin{document}
	\maketitle

	
	
	
	\begin{frame}[fragile]{Problema de la palabra.}
		Sea $G$ un grupo \fg por un conjunto $A$; 
		tal que $G$ es isomorfo a $\langle A \mid R \rangle$ para un conjunto de relaciones $R$.
		
		El \emph{problema de la palabra} consiste en el siguiente problema:
	
		\begin{itemize}
					\item 
						\textbf{Entrada}: Una palabra $w$ en $A \cup A^{-1}$.
					
					\item 
						\textbf{Pregunta}: Decidir si vale $w=1$ en $G$.
		\end{itemize}

		El problema de la palabra no es un problema \alert{decidible}.
		\begin{itemize}
			\item 
				Existen grupos tales que resulta imposible construir un algoritmo que decida si una palabra representa la identidad o no.
		\end{itemize}
		

	\end{frame}

\begin{frame}[fragile]{Grafo de Cayley.}
	Dado un grupo $G$ finitamente generado por $A$ podemos considerar un grafo $\Gamma =\cay{G}{A}$ que es el grafo de Cayley.

	Está definido de manera que 
	\begin{align*}
		V(\Gamma) &= G   \\ 
		& \\
		E(\Gamma) &= \{ \{ g,ga \}  \mid g \in G, a \in A \cup A^{-1}  \}. 
	\end{align*}

	% \begin{columns}
	% 	\begin{column}{0.5\textwidth}
	% 		\tcajita{Ejemplo.}{	Sea $G = \ZZ/2\ZZ \times \ZZ/2\ZZ$ presentado por 
	% 		$G = \langle a,b \mid aba^{-1}b^{-1}, a^{2}, b^{2}\rangle$. Entonces $\cay{G}{\{a,b\}}$ es el \\ siguiente grafo.}
	% 	\end{column}
	% 	\begin{column}{0.5\textwidth}
	% 		\[
	% 			\begin{tikzpicture}
	% 			[scale=0.60,V/.style = {circle, draw,align= center, minimum size=0.5cm,
	% 				minimum size=2em,inner sep=2,
	% 				 fill=astral!15,font=\scriptsize	},fill fraction/.style={path picture={
	% 						\fill[#1] 
	% 						(path picture bounding box.south) rectangle
	% 						(path picture bounding box.north west);
	% 				}},
	% 				fill fraction/.default=astral!90
	% 				]
	% 		\node[align=center] at (2, -6) {};
	% 			\begin{scope}[nodes=V,xshift=4.5cm, yshift=-4cm]
	% 					\node (1) at (0,0) {$1$};
	% 					\node (a) at (2,2)  {$a$};
	% 					\node (b) at (2,-2)     {$b$};
	% 					\node (ab) at (4,0)    {$ab$};
	% 			\end{scope}
				
	% 			\draw   (1)  edge[-] (a);
	% 			\draw   (1)  edge[-] (b);
	% 			\draw   (a)  edge[-] (ab);
	% 			\draw   (b)  edge[-] (ab);
	% 	\end{tikzpicture}
	% 	\]
	% 	\end{column}
	% \end{columns}
\end{frame}

\begin{frame}[fragile]{Ejemplo de grafo de Cayley.}
	\tcajita{Ejemplo.}{	Sea $G = \ZZ/2\ZZ \times \ZZ/2\ZZ$ presentado por 
			$G = \langle a,b \mid aba^{-1}b^{-1}, a^{2}, b^{2}\rangle$. Entonces $\cay{G}{\{a,b\}}$ es el siguiente grafo: \\  \\ 
			\centering
			\begin{tikzpicture}
				[scale=0.60,V/.style = {circle, draw,align= center, minimum size=0.5cm,
					minimum size=2em,inner sep=2,
					 fill=astral!15,font=\scriptsize	},fill fraction/.style={path picture={
							\fill[#1] 
							(path picture bounding box.south) rectangle
							(path picture bounding box.north west);
					}},
					fill fraction/.default=astral!90]
			\node[align=center] at (2, -6) {};
				\begin{scope}[nodes=V,xshift=4.5cm, yshift=-4cm]
						\node (1) at (0,0) {$1$};
						\node (a) at (2,2)  {$a$};
						\node (b) at (2,-2)     {$b$};
						\node (ab) at (4,0)    {$ab$};
				\end{scope}
				\draw   (1)  edge[-] (a);
				\draw   (1)  edge[-] (b);
				\draw   (a)  edge[-] (ab);
				\draw   (b)  edge[-] (ab);
		\end{tikzpicture}
		}
\end{frame}

\begin{frame}[fragile]{Grupos libres.}
	Dado un conjunto $A$ notamos $F_{A}$ al \emph{grupo libre} generado por los elementos de $A$. 

	% \onslide<2->
	El grupo $F_{A}$ viene con una función $\iota: A \to F_{A}$ que denominamos la inclusión de los generadores en el grupo libre y queda caracterizado por la siguiente propiedad universal: 
	Para todo grupo $H$ y toda función $f:A \to H$ existe un único morfismo de grupos $\ol f: F_{A} \to H$ tal que $\ol f \circ \iota = f$.
	\begin{center}
		\begin{tikzcd}
			F_{A}  \arrow[rr, "\ol f", dashed]          &  & H \\
			&  &   \\
			A \arrow[uu, "\iota"] \arrow[rruu, "f", swap] &  &  
		\end{tikzcd}
	\end{center}

	% \onslide<3->
	Decimos que $A$ genera libremente a $F_{A}$ y que $A$ es una base de $F_{A}$.

	% \onslide<4->
	Podemos identificar los elementos de $F_{A}$ con las palabras reducidas en $A \cup A^{-1}$.
\end{frame}

	\begin{frame}[fragile]{Grupos virtualmente libres.}
		Un grupo $G$ es \emph{virtualmente libre} si es finitamente generado y si
		tiene un subgrupo libre $F$ tal que $[G:F] < \infty$.

		\begin{alertblock}{Ejemplos.}
			\begin{itemize}
				\item Los grupos finitos.
				\item Los grupos libres.
				\item El producto semidirecto de un grupo libre con un grupo finito.
				\item El producto libre de dos grupos finitos.
			\end{itemize}
		\end{alertblock}
	\end{frame}

	\begin{frame}[fragile]{Teoría de lenguajes.}
		Dado un conjunto finito $\Sigma$ notaremos por $\Sigma^*$ al monoide libre sobre $\Sigma$.
		
		Un lenguaje $L$ sobre un alfabeto $\Sigma$ es un subconjunto de $\Sigma^*$.
		\begin{alertblock}{Ejemplos.}
			\begin{itemize}
				\item 
					Dado $\Sigma = \{a,b\}$ consideramos el lenguaje de los palíndromos
					\[
						L = \{ w \in \Sigma^{*} \mid w = w^{R}  \}.
					\]
				\item 
					Dado un grupo $G$ finitamente generado por $A$ consideramos el lenguaje
					\[
						\WP{G}{A} = \{ w \in (A \cup A^{-1})^{*} \mid w = 1 \ \text{en $G$} \}.	
					\]
			\end{itemize}
		\end{alertblock}
	\end{frame}
	
	\begin{frame}[fragile]{Ejemplo de gramática.}
		Consideremos las variables $V =\{ S,A \}$ y el alfabeto $\Sigma = \{ a,b \}$.

		Consideramos las siguientes reglas.
		\begin{align*}
			S  & \to aAS  \\
			S  & \to b    \\
			A  & \to baAb \\
			aA & \to bbb
		\end{align*}

		El lenguaje generado por esta gramática es el conjunto de las palabras $w \in \Sigma^{*}$ que podemos obtener a partir de $S$ aplicando las reglas.

		Ejemplo: podemos \emph{derivar} la palabra $abbbbbb$ de la siguiente manera.  
		\[
			S \to aAS \to abaAbS \to abaAbb \to abbbbbb	
		\]
	\end{frame}

	\begin{frame}[fragile]{Gramáticas (pt.1).}
		\tcajita{Definición}{Una \emph{gramática} es una tupla ${\cal G} = (V, \Sigma, P, S)$ donde:
		\begin{itemize}
			\item $V$ es un conjunto finito denominado las \emph{variables};
			\item $\Sigma$ es un conjunto finito disjunto de $V$ que denominamos \emph{símbolos terminales};
			\item $P \subseteq ((V \cup \Sigma)^{*} - \Sigma^{*}) \times (V \cup \Sigma)^*$ es un conjunto finito de \emph{producciones}.
			\item $S \in V$ es el \emph{símbolo inicial};
		\end{itemize}
			Es usual escribir las producciones $(\gamma, \nu) \in P$ como $\gamma \to \nu$.}
		
	\end{frame}
	
	\begin{frame}[fragile]{Gramáticas (pt.2).}
		Una sucesión de producciones $\gamma_{1} \to \gamma_{2} \to \dots \to \gamma_{n}$ la denotamos $\gamma_{1} \tto \gamma_{n}$ y diremos que es un \emph{derivación}.
		\medskip 
		\tcajita{Definición.}{Dada una gramática $\cGg$  definimos el \emph{lenguaje generado por la gramática} como
		\[
		L({\cal G}) = \{ w \in \Sigma^* \ | \ S \tto w   \}.
		\]}

	\end{frame}

	\begin{frame}{Gramáticas independiente de contexto.}
		\begin{deff}
			Una gramática $\gramatica $ es \emph{independiente de contexto} si las producciones tienen la siguiente forma:
			\begin{equation*}
				A \to w
			\end{equation*}
			donde $A \in V, w \in (\Sigma \cup V)^*$.  
			
			Si $L=\lengderivado$ para alguna gramática independiente de contexto $\cal G$ entonces diremos que $L$ es un \emph{lenguaje independiente de contexto}.

		\end{deff}
	\end{frame}

	\begin{frame}[fragile]{Forma normal de Chomsky.}
		\begin{deff}
			Una gramática $\gramatica$ independiente de contexto está en \emph{forma normal de Chomsky} si las producciones son de este tipo:
			\begin{enumerate}
				\item $A \to BC$ donde $A\in V$ y $B,C \in V \setminus \{ S \}$.
				\item $A \to a$ donde $A \in V, a \in \Sigma$.
				\item $S \to \epsilon$ 
			\end{enumerate}
		\end{deff}
		
		\teocajita{Sea $\cG$ una gramática \ic{} entonces existe una gramática \ic{} en forma normal de Chosmky $\cG'$ tal que $L(\cG) = L(\cG')$.}
	\end{frame}

	\begin{frame}[fragile]{Árboles de derivación.}
		Dada $\cG$ una gramática en f.n de Chomsky y una derivación en esta gramática luego le podemos asociar un árbol binario $T(\deri)$ llamado el \emph{árbol de la derivación.}
		
		\TODO{Agregar ejemplito por arriba}.

		Observar que las hojas se corresponden a letras del alfabeto y los vértices que no son hojas se corresponden a variables de la gramática.
	\end{frame}

	\begin{frame}[fragile]{Clasificación del problema de la palabra.}
		\tcajita{Pregunta razonable.}{Si el lenguaje asociado al problema de la palabra de un grupo $G$ es \ic entonces ¿qué nos dice esto del grupo?}

		\teocajita{Un grupo $G$ es \vl{} si y solo si para algún conjunto de generadores $A$ de $G$ vale que el lenguaje $\WP{G}{A}$ es \ic.}
	\end{frame}
	

	\begin{frame}[fragile]{Teorema de Muller-Schupp.}
		\[	
			\begin{tikzpicture}{scale = 0.75}
				\path 
				(0,0) node(a) [rectangle,draw] {$G$ es isomorfo a $\pi_{1}(\cG,P)$ para $\cG$ un grafo de grupos finito con grupos finitos.}
				(5,-3) node(b) [rectangle,draw] {$\cay{G}{A}$ tiene treewidth finito}
				(0,-6) node(c) [rectangle,draw] {$\WP{G}{A}$ es un lenguaje independiente de contexto}
				(-5,-3) node (d) [rectangle,draw] {$G$ es virtualmente libre};
				\draw   
				(d) edge[<-,line width=1.0pt,"Capítulo 2"] (a) 
				(c) edge[<-,line width=1.0pt,"Capítulo 3"] (d)
				(b) edge[<-,line width=1.0pt,"Capítulo 4"] (c)
				(a)  edge[<-,line width=1.0pt,"Capítulo 5"] (b);
			\end{tikzpicture}
		\]
		\TODO{Cambiar el orden del diagrama.}
	\end{frame}

	

	\begin{frame}{Descomposición en un árbol y treewidth de un grafo.}
	Sea $\Gamma$ un grafo no dirigido.
	Una \textbf{descomposición en un árbol} de $\Gamma$ es un par $(T,f)$ donde
	$T$ es un árbol y $f$ una función 
	\[
	f: V(T) \to \partes{V(\Gamma)}
	\]
	Que cumple las siguientes condiciones:
	\begin{enumerate}
		\item Para todo vértice $v \in V(\Gamma)$ debe existir $t \in V(T)$ tal que $v \in f(t)$. 
		\item Para toda arista $\{v,w\} \in E(\Gamma)$ 
		debe existir $t \in V(T)$ tal que $v,w \in f(t)$.
		\item Si $v \in V(\Gamma)$ es tal que $v \in f(t) \cap f(s)$ luego $v \in f(r)$ para todo $r \in V(T)$ en la geodésica que va desde $s$ a $t$.  
	\end{enumerate}
	El \textbf{bagsize} de esta descomposición es el siguiente valor:
	\begin{equation*}
		bs(\Gamma,T,f) = \sup_{t \in V(T)} |f(t)| - 1 \in \  \NN \cup \{ +\infty \}.
	\end{equation*}
	Un grafo $\Gamma$ tiene \textbf{treewidth finito} si existe $(T,f)$ tal que $bs(\Gamma,T,f) < \infty$.
	\end{frame}
	
	\begin{frame}[fragile]{Descomponiendo el grafo de Cayley en un árbol.}
		Dado $\Gamma = \cay{G}{A}$ donde $G$ es finitamente generado por $A$.

		Sea $C \subseteq V(\Gamma)$ definimos:
		\begin{itemize}
			\item  
				La \textbf{vecindad} de $C$ es el siguiente conjunto:

				$N(C) = C \cup \{ v \in V(\Gamma) \mid \exists w \in C, \ \{v,w \} \in E(\Gamma) \}.$

			\item 
				El \textbf{borde} de $C$ es el siguiente conjunto: 

				$\beta C =  N(C) \cap N(C^{c})$.
		\end{itemize} 
	\end{frame}

	\begin{frame}[fragile]{Descomponiendo el grafo de Cayley en un árbol pt.2}
		Para cada $n \in \NN_{0}$ consideramos:
		\[
			V_{n} = V(\Gamma \setminus B_{n}(1))	
		\]

		Consideramos el siguiente árbol $T$ con raíz $1$:
		\begin{align*}
			V(T)  & = \{ \beta C \mid C \in V_{n} \ \text{componente conexa} \}  \cup \{ \{1\} \} \\
			E(T)  =  & \{ \{ \beta C, \beta D \}  \mid C \subseteq D \\ 
			& \text{$C$ es componente conexa de $V_{n+1}$ y $D$ componente conexa de $V_{n}$} \} \\
			& \cup \{ \{\{1\}, \beta C\} \mid C \ \text{es componente conexa de $V_{0}$} \}
		\end{align*}
		Definimos $f: V(T) \to \partes{V(\Gamma)}$ por medio de $f(\beta C) = \beta C$.
		\begin{lema}
			El par $(T,f)$ es una descomposición en un árbol para el grafo $\Gamma$.
		\end{lema}
	\end{frame}

	\begin{frame}[fragile]{Ejemplo de la descomposición en un árbol de un grafo de Cayley (pt. 1).}
	Sea $G = \ZZ / 2\ZZ \ast \ZZ / 3\ZZ $ presentado por 
	$G \simeq \langle a,b \mid a^2, b^3 \rangle$.

	El grafo $\cay{G}{\{ a,b,b^{-1} \}}$ lo representamos de la siguiente manera:
	
		\[
			\begin{tikzpicture}
			[scale=0.60,V/.style = {circle, draw,align= center, minimum size=0.5cm,
				minimum size=2em,inner sep=2,
				 fill=astral!15,font=\scriptsize	},fill fraction/.style={path picture={
						\fill[#1] 
						(path picture bounding box.south) rectangle
						(path picture bounding box.north west);
				}},
				fill fraction/.default=astral!90
				]
		\node[align=center] at (2, -6) {};
			\begin{scope}[nodes=V,xshift=4.5cm, yshift=-4cm]
					\node (1) at (0,0) {$1$};
					\node (a) at (2,0)  {$a$};
					\node (b) at (-2,2)     {$b$};
					\node (bb) at (-2,-2)    {$b^2$};
					\node (ab) at (4,2)      {$ab$};
					\node (abb) at (4,-2)     {$ab^2$};
					\node (ba) at (-4,2)     {$ba$};
					\node (bba) at (-4,-2)     {$b^2a$};
					\node (aba) at (6,2)    {$aba$};
					\node (abba) at (6,-2)    {$ab^2a$};
					\node (bab) at (-6,3)    {$bab$};
					\node (babb) at (-6,1)     {$bab^2$};
					\node (bbab) at (-6,-3)     {$b^2ab$};
					\node (bbabb) at (-6,-1)    {$b^2ab^2$};
					\node (abab) at (8,3)    {$abab$};
					\node (ababb) at (8,1)    {$abab^2$};
					\node (abbab) at (8,-1)     {$ab^2ab$};
					\node (abbabb) at (8,-3)  {$ab^2ab^2$};
			\end{scope}
			
			\draw   (1)  edge[-] (a);
			\draw   (1)  edge[-] (b);
			\draw   (1)  edge[-] (bb);
			\draw   (b)  edge[-] (bb);
			\draw   (b)  edge[-] (ba);
			\draw   (bb)  edge[-] (bba);
			\draw   (a)  edge[-] (ab);
			\draw   (abb)  edge[-] (a);
			\draw   (ab)  edge[-] (abb);
			\draw   (ab)  edge[-] (aba);
			\draw   (aba)  edge[-] (abab);
			\draw   (ababb)  edge[-] (aba);
			\draw   (ababb)  edge[-] (abab);
			\draw   (abb)  edge[-] (abba);
			\draw   (abba)  edge[-] (abbab);
			\draw   (abba)  edge[-] (abbabb);
			\draw   (bba)  edge[-] (bbabb);
			\draw   (bba)  edge[-] (bbab);
			\draw   (ba)  edge[-] (bab);
			\draw   (babb)  edge[-] (bab);
			\draw   (abbab)  edge[-] (abbabb);
			\draw   (ba)  edge[-] (babb);
			\draw   (bbabb)  edge[-] (bbab);
			\draw   (abbab)  edge[-] (abbabb);
	\end{tikzpicture}
	\]
	\end{frame}
	
	\begin{frame}[fragile]{Ejemplo de la descomposición en un árbol de un grafo de Cayley (pt. 2).}
		El árbol $T$ de la descomposición lo representamos de la siguiente manera:

	\[
			\begin{tikzpicture}[scale=0.8, V/.style = {circle, draw,align= center, minimum size=0.5cm,
			minimum size=3em,inner sep=2,
			fill=applegreen!15,font=\scriptsize	},fill fraction/.style={path picture={
				\fill[#1] 
				(path picture bounding box.south) rectangle
				(path picture bounding box.north west);
		}},
		fill fraction/.default=gray!50
		]
		\node[align=center] at (2, -6) {};
		\begin{scope}[nodes=V,xshift=4.5cm, yshift=-4cm]
			\node (1) at (0,0)  {$\{ b,b^2,1 \}$};
			\node (2) at (2,0)  {$1$};
			\node (3) at (4,0)  {$\{ 1,a \}$};
			\node (4) at (-2,3)     {$\{ b,ba \}$};
			\node (5) at (-2,-3)   {$\{b^2,b^2a\}$};
			\node (6) at (6,0)  {$\{ a,ab,ab^2 \}$};
			
			\node (9) at (8,3)  {$\{ab,aba\}$};
			\node (10) at (8,-3)  {$\{ ab^2, ab^2a \}$};
		\end{scope}
		
		
		\draw (1) edge[-] (2);
		\draw (2) edge[-] (3);
		\draw (1) edge[-] (5);
		\draw (1) edge[-] (4);
		\draw (3) edge[-] (6);
		\draw (6) edge[-] (9);
		\draw (6) edge[-] (10);
	\end{tikzpicture}
	\]
	\end{frame}

	\begin{frame}[fragile]{Un lema para forma normal de Chomsky (pt. 3).}
		\lemacajita{Sea $\gramatica$ una gramática independiente de contexto en forma normal de Chomsky.
		Sea $w \in L(\cG)$ tal que $w = tuv$ con $t,u,v \in \Sigma^{*}$. 
		Si fijamos una derivación $S \tto w$ entonces existe un único vértice en el árbol de derivación con la propiedad de ser el más bajo entre aquellos de los que deriva una subpalabra que da lugar a $u$.}
		
		\TODO{}
	\end{frame}
	
	\begin{frame}[fragile]{Independiente de contexto implica treewidth finito (lemas previos).}
		\begin{lema}[1]
			Sea $G$ un grupo \ic{} y $\gramatica$ gramática \ic{}  de manera que $L(\cG) = WP(G,\Sigma)$.
			Sea $A \in V$ una variable de esta gramática y consideremos el lenguaje
			\[
			L_A = \{ w \in \Sigma^*  \ | \ A \tto w  \}.
			\]
			Entonces vale el siguiente resultado:
			dadas palabras $v,v' \in L_{A}$ luego $v = v'$ en $G$.
		\end{lema}

		\begin{lema}[2]
			Sea $\Gamma$ un grafo de grado acotado uniformemente y 
			$(T,f)$ una descomposición en un árbol para $\Gamma$ de manera que existe $M \in \NN$ tal que para todo $t \in V(T)$ tenemos la siguiente cota
			\[
				  \diam{f(t)} < M
			\]   
			entonces la descomposición $(T,f)$ tiene bagsize finito.
		\end{lema}
	\end{frame}

	% \begin{frame}[fragile]{Independiente de contexto implica treewidth finito (pt.1).}
	% 	\begin{teo}
	% 		Sea $G$ un grupo tal que $\WP{G}{\Sigma}$ es un lenguaje \ic{} entonces $\cay{G}{\Sigma}$ tiene treewidth finito.
	% 	\end{teo}
	% 	\textbf{Demostración.}

	% 		\begin{itemize}
	% 			\item 
	% 				Consideramos $(T,f)$  descomposición en un árbol anteriormente construida para $\cay{G}{\Sigma}$.
	% 			\item 
	% 				Por el \textbf{lema  1} nos alcanza con ver que existe $M \in \NN$ tal que para todo $t \in V(T)$ vale que $\diam{f(t)} \le M$.
	% 			\item 
	% 				Sea $\gramatica$ grámatica \ic{} en forma normal de Chomsky tal que $L(\cG) = \WP{G}{\Sigma}$.
	% 		\end{itemize}
	% \end{frame}

	% \begin{frame}[fragile]{Independiente de contexto implica treewidth finito (pt.2).}
		
	% 	\begin{itemize}
	% 		\item 
	% 			Para cada $A \in V$ consideramos 
	% 			\[
	% 				L_{A} = \{ w \in \Sigma^{*} \mid A \tto w \}
	% 			\]
	% 			Definimos $k_{A} = \min \{ |w| \mid w \in L_{A} \}$.
				
	% 		\item 
	% 			Definimos 
	% 			\[
	% 				k = \max_{A \in V} k_{A}	
	% 			\]
	% 			Vamos a probar que para todo $t \in V(T)$ tenemos que $\diam{f(t)} \le 3k$.
	% 	\end{itemize}
	% \end{frame}

	% \begin{frame}[fragile]{Independiente de contexto implica treewidth finito (pt.3).}

	% 	\begin{itemize}
	% 		\item
	% 			Dado que $f(t) = \beta C$ para $C$ componente conexa de $V(\Gamma \setminus B_{n}(1))$ queremos acotar la distancia 
	% 			\[
	% 				d(g,h) \le 3k	
	% 			\]
	% 			para $g,h \in \beta C$ componente conexa de $V(\Gamma \setminus B_{n}(1))$.

	% 		\item 
	% 			Consideramos: 
	% 			\begin{itemize}
	% 				\item $\alpha$ geodésica de $1$ a $g$ con etiqueta $u$.
	% 				\item $\tau$ camino de $g$ a $h$ contenido en $C \cup \beta C$ con etiqueta $v$.
	% 				\item $\beta$ geodésica de $h$ a $1$ con etiqueta $w$.
	% 			\end{itemize}
	% 		\item 
	% 			Notemos que la palabra 	$uvw$ es la etiqueta de un ciclo en el grafo de Cayley por lo tanto $uvw \in \WP{G}{\Sigma}$.
				
	% 			Entonces existe derivación
	% 			\[
	% 				S \tto uvw	
	% 			\]
	% 	\end{itemize}
	% \end{frame}

	% \begin{frame}[fragile]{Independiente de contexto implica treewidth finito (pt.4).}
	% 	\begin{itemize}
	% 		\item 
	% 			Por el \textbf{lema de la forma normal de Chomsky} tenemos que existe una variable $A \in V$ de manera tal que tiene la propiedad de ser la última variable que deriva a $v$ como subpalabra.

	% 		\item 
	% 			Como la gramática está en forma normal de Chomsky y podemos asumir que $|v| \ge 2$ entonces tenemos que la derivación tiene la siguiente pinta:
	% 			\begin{equation*}
	% 				S \tto u'Aw' \to_{\cal G} u'BC w' \tto u'v'v''w'
	% 			\end{equation*}
	% 			donde $B,C$ son otras variables 
	% 			\begin{itemize}
	% 				\item $v$ es una subpalabra de $v'v''$;
	% 				\item $u'$ es un prefijo de $u$;
	% 				\item $w'$ es un posfijo de $w$;
	% 			\end{itemize}
	% 	\end{itemize}
	% \end{frame}

	% \begin{frame}[fragile]{Independiente de contexto implica treewidth finito (pt.5).}
	% 	\begin{itemize}
	% 		\item 
	% 			Si consideramos la geodésica $\alpha$ tenemos que al haber leído la etiqueta $u'$ llegamos a un vértice $x$ y por estar sobre la geodésica cumple la siguiente igualdad:
	% 			\begin{equation*}
	% 				d(x,g) = d(1,g) - d(1,x).
	% 			\end{equation*}
			
	% 		\item 
	% 			Si consideramos la geodésica $\beta$ tenemos que al haber leído la etiqueta $(w')^{-1}$ llegamos a un vértice $z$ y por estar sobre la geodésica cumple la siguiente igualdad:
	% 			\begin{equation*}
	% 				d(z,h)  = d(1,h) - d(1,z).
	% 			\end{equation*}

	% 		\item 
	% 			Consideremos el vértice $y$ al que llegamos después de leer $u'v'$.
	% 			Este vértice está en el camino $\tau$ y usando que $y \in \tau \subseteq C \cup \beta C $ tenemos que
	% 				\[
	% 					d(1,y) \ge n+1 = d(1,g)	
	% 				\]
	% 			por ser $C$ una componente conexa de $V(\Gamma \setminus B_{n}(1))$.
			
	% 	\end{itemize}
	% \end{frame}

	% \begin{frame}[fragile]{Independiente de contexto implica treewidth finito (pt.6).}
	% 	\begin{itemize}
	% 		\item 
	% 			Por el \textbf{lema 2} notemos que si reemplazamos $v'$ por una palabra de menor tamaño del lenguaje $L_B$ seguimos teniendo un ciclo pero de longitud idéntica o más chica.
	% 			Similarmente si reemplazamos $v''$ por una palabra de menor tamaño de $L_{C}$ y $v'v''$ por una de menor tamaño de $L_{A}$. 
				
	% 		\item
	% 			La palabra $v'$ la leemos cuando vamos del vértice $x$ al vértice $y$.

	% 			Como $v' \in L_{B}$ luego 
	% 			\[
	% 				d(x,y) \le k_{B} \le k.	
	% 			\]

	% 		\item 
	% 			La palabra $v'v''$ la leemos cuando vamos del vértice $x$ al vértice $z$.

	% 			Como $v'v'' \in L_{A}$ luego 
	% 			\[
	% 				d(x,z) \le k_{A} \le k	
	% 			\]

	% 		\item 
	% 			La palabra $v''$ la leemos cuando vamos del vértice $y$ al vértice $z$.

	% 			Como $v'' \in L_{C}$ luego 
	% 			\[
	% 				d(y,z) \le k_{C} \le k	
	% 			\] 
	% 	\end{itemize}
	% \end{frame}
	% \begin{frame}[fragile]{Independiente de contexto implica treewidth finito (pt.7).}
	% 	\begin{itemize}
	% 		\item 
	% 			Dado que $d(1,y) \ge n+1 \ge d(1,g)$ 
	% 			\begin{align*}
	% 				d(x,g) &= d(1,g) - d(1,x)  \le d(1,y) - d(1,x) = d(x,y) \\
	% 				d(z,h) &= d(1,h) - d(1,z)  \le d(1,y) - d(1,z) = d(z,y)
	% 			\end{align*}
	% 		\item 
	% 			Terminamos de probar que $d(g,h) \le 3k$. 
				
	% 			Usamos la desigualdad triangular tres veces:
	% 			\begin{align*}
	% 				d(g,h) & \le d(g,x) + d(x,z) + d(h,z) \\
	% 				& \le d(x,y) + d(x,z) + d(y,z) \le 3k
	% 			\end{align*}
	% 	\end{itemize}
	% 	\qed
	% \end{frame}

\end{document}