\documentclass[aspectratio=169, 11pt]{beamer}
\usepackage[utf8]{inputenc}
\usepackage[spanish]{babel}
\usepackage{xcolor}
\usepackage{amssymb}
\usepackage{pifont}
\usepackage{amsmath,amsfonts,amsthm,amssymb,mathtools,sectsty}
\pagenumbering{gobble}
\usepackage{subcaption}
%\usepackage{graphicx}
%\usepackage[pdftex,dvipsnames]{xcolor}
\usepackage{cancel}
\usepackage{graphicx}
\usepackage{marginnote}

% Tikz y las librerías para automátas
\usepackage{tikz-cd}
\usepackage{tikz}
\usetikzlibrary{arrows,automata}
\usetikzlibrary{babel} %para evitar que se jodan los automatas de tikz

%Referencias; me gustaría que backref funcione pero no es importante tampoco.
\usepackage[pagebackref]{hyperref}
% Para modificar el estilo de las referencias
\hypersetup{
	colorlinks,
	linkcolor={astral},
	citecolor={blue!30!black},
	urlcolor={blue!80!black}
}
\definecolor{astral}{RGB}{46,116,181}
\colorlet{chulo}{blue!70!purple}
\colorlet{rojo}{purple!65!black}


\paragraphfont{\color{astral!70!black}}
\chapterfont{\color{astral!40!black}}
\subsectionfont{\color{astral!60!black} }
\sectionfont{\color{astral!50!black} }
\usepackage{mathpazo}
\usepackage{amssymb}
\usepackage{eufrak}
%\usepackage{thmtools}




\usepackage[framemethod=tikz]{mdframed}

%%%%%%%%%%%%%  TEOREMAS  %%%%%%%%%%%%%%%%%
\theoremstyle{plain} %% el estilo clásico
\newtheorem{teo}{\color{rojo}{ \textsc{ Teorema}}}[section]
\newtheorem{prop}[teo]{\color{rojo} \textsc{Proposición}}
\newtheorem{lema}[teo]{\color{rojo} \textsc{Lema}}
\newtheorem{coro}[teo]{\color{rojo} \textsc{Corolario}}
% Si pongo [theorem] siguen la numeración de los teoremas. 
% e.j. Teo 1, Lema 2, Teo 3, Teo 4 ...
\theoremstyle{definition}
\newtheorem{deff}[teo]{\textsc{ Definición}}{\smallskip}
\newtheorem{ej}[teo]{\textsc{Ejemplo}}{\smallskip}

% Remarks
\theoremstyle{remark}
\newtheorem{obs}[teo]{ \textsc{Observación}}{\smallskip}
\newtheorem{afirmacion}[teo]{ \textsc{Afirmación}}{\smallskip}

% DEMOS EN NEGRITA.
\renewenvironment{proof}{{\textbf{Demostración.}}}{ \hfill $\blacksquare$ \medskip}

%COLORES

\definecolor{astral}{RGB}{46,116,181}
\colorlet{chulo}{blue!70!purple}
\colorlet{rojo}{purple!45!black}
\definecolor{carrotorange}{rgb}{0.93, 0.57, 0.13}
\definecolor{brightcerulean}{rgb}{0.11, 0.67, 0.84}
\definecolor{brightube}{rgb}{0.82, 0.62, 0.91}
\definecolor{cadmiumred}{rgb}{0.89, 0.0, 0.13}
\definecolor{applegreen}{rgb}{0.55, 0.71, 0.0}
\definecolor{aurometalsaurus}{rgb}{0.43, 0.5, 0.5}

%%%%%%%%%%%%%%%%%%%%% ENUMERAR CON COSAS QUE NO SEAN SOLO NÚMEROS %%%%%%%%%%
%\usepackage[shortlabels]{enumitem}
%\setlist[enumerate]{font=\bfseries}


\newcommand\RP{\mathbb{RP}}
\newcommand{\norm}[1]{\left\lVert#1\right\rVert}
\newcommand{\RR}{\mathbb{R}}
\newcommand{\CC}{\mathbb{C}}
\newcommand{\NN}{\mathbb{N}}
\newcommand{\ZZ}{\mathbb{Z}}
\newcommand{\Om}{\Omega}
\newcommand{\A}{\mathcal A}
\newcommand\ol{\overline}
\newcommand{\blue}{\textcolor{chulo}}
\newcommand{\red}{\textcolor{rojo}}
\newcommand{\Gg}{\mathfrak g}
\newcommand{\SL}{SL_2(\mathbb Z)}
\newcommand{\stab}{\text{Stab}}
\newcommand{\ic}{independiente de contexto}
\newcommand{\APND}{automáta de pila no determinístico}
\newcommand{\APD}{automáta de pila determinístico }
\newcommand{\gramatica}{{\cal G} = (V, \Sigma, P, S)}
\newcommand{\deriva}{\overset{*}{\to_{\cal G}}}
\newcommand{\lengderivado}{L({\cal G})}
\newcommand{\fg}{finitamente generado }
%\newcommand{\ol}{\overline{}}

\newcommand{\fp}{finitamente presentado }
\newcommand{\vl}{virtualmente libre}
\newcommand{\vls}{virtualmente libres}
\newcommand{\WP}[2]{\text{WP}(#1, #2)}

\newcommand{\cG}{ {\cal G} }
\newcommand{\cGg}{{\cal G} = (V, \Sigma, P, S)}
\newcommand{\cH}{ {\cal H} }
\newcommand{\Xm}{\widetilde X}
%\newcommand{\ol}{\overline{}}

\newcommand{\vectres}[3]{(#1,#2,#3)}
\newcommand{\confgin}{(q_{0}, w, \$)}
\newcommand{\confguno}{(q, w, \gamma)}
\newcommand{\confgdos}{(p,u,\theta)}

\newcommand{\cay}[2]{\text{Cay}(#1,#2)}

\newcommand{\ca}[1]{{\cal C}(#1)}
\newcommand{\cmin}[1]{{\cal C}_{\text{min}}(#1)}
%\newcommand{\cam}{{\cal C}_{\text{min}}(\alpha)}
\newcommand{\copt}[1]{{\cal C}_{\text{opt}}(#1)}
%\newcommand{\copt}{{\cal C}_{\text{opt}}}
\newcommand*{\rows}{6}


\title{El problema de la palabra de los grupos virtualmente libres.}
\subtitle{Defensa de tesis de licenciatura.}
\date{??.}
\author{Leopoldo Lerena \\
		Director: Iván Sadofschi Costa				
		}
\institute{Universidad de Buenos Aires}
% \titlegraphic{\hfill\includegraphics[height=1.5cm]{logo.pdf}}

\begin{document}
	\maketitle

	
	
	
	\begin{frame}[fragile]{Problema de la palabra.}
		Sea $G$ un grupo \fg por un conjunto finito $A$; 
		tal que $G$ es isomorfo a $\langle A \mid R \rangle$ para un conjunto de relaciones $R \subseteq (A \cup A^{-1})^*$.
		
		El \emph{problema de la palabra} consiste en el siguiente problema:
		\begin{itemize}
			\item 
				\textbf{Entrada}: Una palabra $w \in (A \cup A^{-1})^*$.
			
			\item 
				\textbf{Pregunta}: Decidir si vale $w=1$ en $G$.
		\end{itemize}
		
		\alert{El problema de la palabra no es decidible.}
	\end{frame}

	\begin{frame}[fragile]{Grafo de Cayley.}
		Dado un grupo $G$ finitamente generado por $A$ podemos considerar un grafo $\Gamma =\cay{G}{A}$ que es el grafo de Cayley.

		Está definido de manera que 
		\[
			V(\Gamma) = G,   \ \ \ E(\Gamma) = \{ \{ g,ga \}  \mid g \in G, a \in A \cup A^{-1}  \}. 	
		\]
	\end{frame}
	
	\begin{frame}[fragile]{Grupos libres.}
		Dado un conjunto $A$ podemos definir a $F_{A}$ el \emph{grupo libre} generado por los elementos de $A$ como un grupo que tiene una función  $\iota: A \to F_{A}$ que denominamos la inclusión de los generadores en el grupo libre y que está definido por la siguiente propiedad universal: 
	Para todo grupo $H$ y toda función $f:A \to H$ existe un único morfismo de grupos $\ol f: F_{A} \to H$ tal que $\ol f \circ \iota = f$.
	Equivalentemente el siguiente diagrama conmuta,
	\begin{center}
		\begin{tikzcd}
			F_{A}  \arrow[rr, "\ol f", dashed]          &  & H \\
			&  &   \\
			A \arrow[uu, "\iota"] \arrow[rruu, "f", swap] &  &  
		\end{tikzcd}
	\end{center}
	En este diremos que $A$ genera libremente a $F_{A}$ y que $A$ es una base de $F_{A}$.
	\end{frame}

	\begin{frame}[fragile]{Grupos virtualmente libres.}
		Un grupo $G$ es \emph{virtualmente libre} si es finitamente generado y si
	tiene un subgrupo libre $F$ tal que $[G:F] < \infty$.
	\end{frame}

	\begin{frame}[fragile]{Teoría de lenguajes.}
		Dado un conjunto finito $A$ notaremos por $A^*$ al monoide libre sobre $A$.
		
		\begin{deff}
			Un lenguaje $L$ sobre un alfabeto $A$ es un subconjunto de $A^*$.
		\end{deff}	
		
		\alert{Ejemplo.}
			Dado un grupo $G$ finitamente generado por $A$ consideramos el lenguaje $\WP{G}{A} = \{ w \in (A \cup A^{-1})^{*} \mid w = 1 \ \text{en $G$} \}$.

	\end{frame}
	
	\begin{frame}[fragile]{Gramáticas.}
		\begin{deff}
			Una \emph{gramática} es una tupla ${\cal G} = (V, \Sigma, P, S)$ donde:
			\begin{itemize}
				\item $V$ es un conjunto finito denominado las \emph{variables};
				\item $S \in V$ es el \emph{símbolo inicial};
				\item $\Sigma$ es un conjunto finito disjunto de $V$ que denominamos \emph{símbolos terminales};
				\item $P \subseteq (V \cup \Sigma)^*V(V \cup \Sigma)^* \times (V \cup \Sigma)^*$ es un conjunto finito de \emph{producciones}.
			\end{itemize}
		\end{deff}

		Las producciones $(\gamma, \nu) \in P$, las vamos a denotar por medio de la siguiente notación $\gamma \to \nu$.


		\begin{deff}
			Dada una gramática $\cGg$  definimos el \emph{lenguaje generado por la gramática} como
			\[
			L({\cal G}) = \{ w \in \Sigma^* \ | \ S \deriva w   \}.
			\]
		\end{deff}
	\end{frame}
	
	\begin{frame}{Gramáticas regulares.}
		\begin{deff}
			Decimos que una gramática $\gramatica$ es \emph{regular} si las producciones son del estilo
	\begin{enumerate}
		\item $A \to \epsilon$
		\item $A \to a$
		\item $A \to a B$
	\end{enumerate}
	donde $A, B \in V$, $a \in \Sigma$ y $\epsilon$ es la palabra vacía. 
	Si $L=\lengderivado$ para alguna gramática regular $\cal G$ entonces diremos que $L$ es un \emph{lenguaje regular}.
		\end{deff}
	\end{frame}

	\begin{frame}{Gramáticas independiente de contexto.}
		\begin{deff}
			Una gramática $\gramatica $ es \emph{independiente de contexto} si las producciones tienen la siguiente forma:
			\begin{equation*}
				A \to w
			\end{equation*}
			donde $A \in V, w \in (\Sigma \cup V)^*$.  
			Si $L=\lengderivado$ para alguna gramática independiente de contexto $\cal G$ entonces diremos que $L$ es un \emph{lenguaje independiente de contexto}.
		\end{deff}
		
	\end{frame}

	\begin{frame}{Jerarquía de Chomsky.}
		Tenemos la siguiente relación entre los lenguajes independiente de contexto y lenguajes regulares.


			\centering
			\includegraphics[scale = 0.65]{Chomsky-hierarchy.png}
		
	\end{frame}

	\begin{frame}[fragile]{Clasificación del problema de la palabra.}
		
		\begin{teo}[Animisov--1971]
			Sea $G$ un grupo entonces $\WP{G}{A}$ es regular si y solo si $G$ es finito.
		\end{teo}

		\begin{alertblock}{Pregunta natural.}
			¿Cómo podemos caracterizar a los grupos cuyo lenguaje del problema de la palabra es independiente de contexto?
		\end{alertblock}
	\end{frame}
	
	\begin{frame}[fragile]{Teorema de Muller-Schupp.}
		
		\[	
			\begin{tikzpicture}{scale = 0.75}
				\path 
				(0,0) node(a) [rectangle,draw] {Grupo fundamental de un grafo de grupos finito}
				(5,-3) node(b) [rectangle,draw] {Treewidth finito}
				(0,-6) node(c) [rectangle,draw] {Independiente de contexto}
				(-5,-3) node (d) [rectangle,draw] {Virtualmente libre};
				\draw   
				(d) edge[<-,line width=1.0pt,"Teorema \ref{teo_karrass_solitar}"] (a) 
				(c) edge[<-,line width=1.0pt,"Teorema \ref{teo_Muller_Schupp}"] (d)
				(b) edge[<-,line width=1.0pt,"Teorema \ref{teo_ic_implica_tw}"] (c)
				(a)  edge[<-,line width=1.0pt,"Teorema \ref{coro_tw_finito_implica_pi1}"] (b);
			\end{tikzpicture}
		\]
	\end{frame}

	\begin{frame}[fragile]{Forma normal de Chomsky.}
		\begin{deff}
			Una gramática $\gramatica$ independiente de contexto está en \emph{forma normal de Chomsky} si las producciones son de este tipo:
			\begin{enumerate}
				\item $A \to BC$ donde $A\in V$ y $B,C \in V \setminus \{ S \}$.
				\item $A \to a$ donde $A \in V, a \in \Sigma$.
				\item $S \to \epsilon$ 
			\end{enumerate}
		\end{deff}
	\end{frame}

	\begin{frame}[fragile]{Un lema para forma normal de Chomsky pt 1.}
		
		\begin{columns}
			
			\begin{column}{0.5\textwidth}
				Sea la gramática $\gramatica$ definida por:
				\begin{itemize}
					\item $V = \{ S,A,B,C,D \}$;
					\item $\Sigma = \{ a,b,c \}$;
					\item 
						\[
						P = \begin{cases}
								S \to AB \mid CD \mid a \mid b \mid c \\
								A \to CD \mid a \mid b \mid c	\\
								B \to BB \mid a \mid b \mid c	\\
								C \to AB \mid a \mid b \mid c	\\
								D \to DD \mid a \mid b \mid c
						\end{cases}	
						\]
				\end{itemize}
			\end{column}

			\begin{column}{0.5 \textwidth}
				Sea la siguiente derivación de la palabra $w = caabccab$
				\begin{align*}
					S &\to AB \to CDB \to CDBB \to^{*} CDab \to \\
					ABDab & \to  AbDab   \to AbDDab  \to CDbDDab   \to\\
					CDbccab & \to CDDbccab \to Caabccab \to caabccab \\
				\end{align*}
			\end{column}
		\end{columns}
	\end{frame}

	
	\begin{frame}{Ejemplo forma normal de Chomsky pt.2}
		Consideremos la subpalabra $u  = aabcc$ de $w = caabccab$.
		Si consideramos el árbol de derivación obtenemos que... 
		\TODO{Agregar dibujito}. 
		
	\end{frame}

	\begin{frame}[fragile]{Un lema para forma normal de Chomsky pt 2.}
		En general vale el siguiente enunciado.

		\begin{lemma}
			Sea $\gramatica$ una gramática independiente de contexto en forma normal de Chomsky.
			Sea $w \in L(\cG)$ tal que $w = tuv$ con $t,u,v \in \Sigma^{*}$. 
			Si fijamos una derivación $S \to^{*} w$ entonces existe un único vértice en el árbol de derivación con la propiedad de ser el más bajo entre aquellos de los que deriva una subpalabra que contiene a $u$.
		\end{lemma}

	\end{frame}
	
	\begin{frame}{Descomposición en un árbol y treewidth de un grafo.}
	Una \emph{descomposición en un árbol} de un grafo $\Gamma$ es un par $(T,f)$ donde
	$T$ es un árbol y $f$ una función 
	\[
	f: V(T) \to \partes{V(\Gamma)}
	\]
	Que cumple las siguientes condiciones:
	\begin{enumerate}
		\item Para todo vértice $v \in V(\Gamma)$ debe existir $t \in V(T)$ tal que $v \in f(t)$. 
		\item Para toda arista $\{v,w\} \in E(\Gamma)$ 
		debe existir $t \in V(T)$ tal que $v,w \in f(t)$.
		\item Si $v \in V(\Gamma)$ es tal que $v \in f(t) \cap f(s)$ luego $v \in f(r)$ para todo $r \in V(T)$ en la geodésica que va desde $s$ a $t$.  
	\end{enumerate}
	\end{frame}
	
	\begin{frame}[fragile]{Descomponiendo el grafo de Cayley en un árbol.}
		\TODO{}
	\end{frame}

	\begin{frame}{Descomponiendo el grafo de Cayley en un árbol (ejemplo).}
		\TODO{AAA}
	\end{frame}	
	
	\begin{frame}[fragile]{Independiente de contexto implica treewidth finito (lemas previos).}
		\TODO{}
	\end{frame}

	\begin{frame}[fragile]{Independiente de contexto implica treewidth finito (pt. 1).}
		\TODO{}
	\end{frame}

	\begin{frame}[fragile]{Independiente de contexto implica treewidth finito (pt. 2).}
		\TODO{}
	\end{frame}

\end{document}