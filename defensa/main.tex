\documentclass[aspectratio=169, 11pt]{beamer}
\usepackage[utf8]{inputenc}
\usepackage[spanish]{babel}
\usepackage{amsmath,amsfonts,amsthm,amssymb,mathtools,sectsty}
\pagenumbering{gobble}
\usepackage{subcaption}
%\usepackage{graphicx}
%\usepackage[pdftex,dvipsnames]{xcolor}
\usepackage{cancel}
\usepackage{graphicx}
\usepackage{marginnote}
\usepackage{mathabx}
\usepackage{float}
\setlength{\marginparwidth}{2cm}

% Tikz y las librerías para automátas
\usepackage{tikz-cd}
\usepackage{tikz}
\usetikzlibrary{arrows,automata}
\usetikzlibrary{babel} %para evitar que se jodan los automatas de tikz
\usetikzlibrary{graphs} 
\usetikzlibrary{calc}
\usetikzlibrary{positioning}
%\usetikzlibrary{shapes.geometric}  % for [ellipse], [diamond], etc

%\usepackage[backend=biber,...]{biblatex} 

%Referencias; me gustaría que backref funcione pero no es importante tampoco.
\usepackage[pagebackref]{hyperref}
% Para modificar el estilo de las referencias
\hypersetup{
	colorlinks,
	linkcolor={astral},
	citecolor={red!70!black},
	urlcolor={red!80!black}
}
\definecolor{astral}{RGB}{46,116,181}
\colorlet{chulo}{blue!70!purple}
\colorlet{rojo}{purple!45!black}
\definecolor{carrotorange}{rgb}{0.93, 0.57, 0.13}
\definecolor{brightcerulean}{rgb}{0.11, 0.67, 0.84}
\definecolor{brightube}{rgb}{0.82, 0.62, 0.91}
\definecolor{cadmiumred}{rgb}{0.89, 0.0, 0.13}
\definecolor{applegreen}{rgb}{0.55, 0.71, 0.0}
\definecolor{aurometalsaurus}{rgb}{0.43, 0.5, 0.5}

%%%%%%%%%%%%%%%%%%%%% ENUMERAR CON COSAS QUE NO SEAN SOLO NÚMEROS %%%%%%%%%%
\usepackage[shortlabels]{enumitem}
\setlist[enumerate]{font=\bfseries}
\usepackage{adjustbox}


%%%%%%%%%%%%%%%%%%%%%%% TÍTULOS DE SECCIONES MÁS FANCIES %%%%%%%%%%%%%%
\usepackage{titlesec}
\setcounter{secnumdepth}{3} % Hasta que profundidad quiero numerar, 4 sería los párrafos.
\titleformat{\section}[block]{\color{astral!50!black}\Large\bfseries\filcenter}{\S\thesection.}{1em}{}
\titleformat{\subsection}[hang]{\color{astral!50!black}\large\bfseries\filcenter}{\S\thesubsection.}{1em}{}

%%%%%%%%%%%%%%%%%%%%%% COLORES PÁRRAFOS Y CAPÍTULOS %%%%%%%%%%%%%%%%%%%%%%%%
%\paragraphfont{\color{astral!70!black}}
\chapterfont{\color{astral!40!black}}
%\subsectionfont{\color{astral!60!black} }
%\sectionfont{\color{astral!50!black} }

\usepackage{mathpazo}
\usepackage{amssymb}
%\usepackage{thmtools}

%Esto sirve para armar grafos de Cayley de una manera más copada.
\usetikzlibrary{lindenmayersystems,arrows.meta}
\pgfdeclarelindenmayersystem{cayley}{
	\rule{G->G-G+++G--G}
	\symbol{R}{
		\pgflsystemstep=0.5\pgflsystemstep
	}

}

\usepackage[framemethod=tikz]{mdframed}

%%%%%%%%%%%%%  TEOREMAS  %%%%%%%%%%%%%%%%%
\theoremstyle{plain} %% el estilo clásico
\newtheorem{teo}{\color{rojo}{ { Teorema}}}[section]
\newtheorem{prop}[teo]{\color{rojo} {Proposición}}
\newtheorem{lema}[teo]{\color{rojo} {Lema}}
\newtheorem{coro}[teo]{\color{rojo} {Corolario}}
\newtheorem*{aff}{ {Afirmación}}
% Si pongo [theorem] siguen la numeración de los teoremas. 
% e.j. Teo 1, Lema 2, Teo 3, Teo 4 ...
\theoremstyle{definition}
\newtheorem{deff}[teo]{{ Definición}}{\smallskip}
\newtheorem{ej}[teo]{{Ejemplo}}{\smallskip}

% Remarks
\theoremstyle{remark}
\newtheorem{obs}[teo]{ {Observación}}{\smallskip}

%%%%%%%%%% FRAMES PARA TEOREMAS A LO HATCHER %%%%%%%%%%%%%%%%%%%%%%%%%

\surroundwithmdframed[outerlinewidth=0.4pt,
innerlinewidth=0.4pt,
align=center,
middlelinewidth=1pt,
middlelinecolor=white,
innertopmargin=-4pt,
innerbottommargin=0pt,
innerrightmargin=4pt,
innerleftmargin=4pt,
bottomline=false,topline=false,rightline=false]{teo}
\surroundwithmdframed[outerlinewidth=0.4pt,
innerlinewidth=0.4pt,
align=center,
middlelinewidth=1pt,
middlelinecolor=white,
innertopmargin=-4pt,
innerbottommargin=0pt,
innerrightmargin=4pt,
innerleftmargin=4pt,
bottomline=false,topline=false,rightline=false]{lema}

\surroundwithmdframed[outerlinewidth=0.4pt,
innerlinewidth=0.4pt,
align=center,
middlelinewidth=1pt,
middlelinecolor=white,
innertopmargin=-4pt,
innerbottommargin=0pt,
innerrightmargin=4pt,
innerleftmargin=4pt,
bottomline=false,topline=false,rightline=false]{prop}


\surroundwithmdframed[outerlinewidth=0.4pt,
innerlinewidth=0.4pt,
align=center,
middlelinewidth=1pt,
middlelinecolor=white,
innertopmargin=-4pt,
innerbottommargin=0pt,
innerrightmargin=4pt,
innerleftmargin=4pt,
bottomline=false,topline=false,rightline=false]{coro}

%==================================================================%

% DEMOS EN NEGRITA.
\renewenvironment{proof}{{\textbf{Demostración.}}}{ \hfill $\blacksquare$ \medskip} 

%% ========== Para escribir pseudo ==========
%\usepackage{algorithm}
%\usepackage[noend]{algpseudocode}  % "noend" es para no mostrar los endfor, endif
%%\algrenewcommand\alglinenumber[1]{\tiny #1:}  % Para que los numeros de linea del pseudo sean pequeños
%\renewcommand{\thealgorithm}{}  % Que no aparezca el numero luego de "Algorithm"
%\floatname{algorithm}{ }    % Entre {  } que quiero que aparezca en vez de "Algorithm"
%
%% traducciones
%\algrenewcommand\algorithmicwhile{\textbf{mientras}}
%\algrenewcommand\algorithmicdo{\textbf{hacer}}
%\algrenewcommand\algorithmicreturn{\textbf{devolver}}
%\algrenewcommand\algorithmicif{\textbf{si}}
%\algrenewcommand\algorithmicthen{\textbf{entonces}}
%\algrenewcommand\algorithmicfor{\textbf{para}}
%
%%% indentar dentro de los algoritmos
%\algdef{SE}[SUBALG]{Indent}{EndIndent}{}{\algorithmicend\ }%
%\algtext*{Indent}
%\algtext*{EndIndent}

% =========================================================
\usepackage[colorinlistoftodos,prependcaption,textsize=tiny]{todonotes}



%Comandos útiles.
\newcommand\RP{\mathbb{RP}}
\newcommand{\norm}[1]{\left\lVert#1\right\rVert}
\newcommand{\RR}{\mathbb{R}}
\newcommand{\CC}{\mathbb{C}}
\newcommand{\NN}{\mathbb{N}}
\newcommand{\ZZ}{\mathbb{Z}}
\newcommand{\Om}{\Omega}
\newcommand{\A}{\mathcal A}
\newcommand\ol{\overline}
\newcommand{\blue}{\textcolor{chulo}}
\newcommand{\red}{\textcolor{rojo}}
\newcommand{\Gg}{\mathfrak g}
\newcommand{\SL}{SL_2(\mathbb Z)}
\newcommand{\stab}{\text{Stab}}
\newcommand{\ic}{independiente de contexto }
\newcommand{\APND}{automáta de pila no determinístico }
\newcommand{\APD}{automáta de pila determinístico }
\newcommand{\gramatica}{{\cal G} = (V, \Sigma, P, S)}
\newcommand{\deriva}{\overset{*}{\to_{\cal G}}}
\newcommand{\tto}{\overset{*}{\to}}
\newcommand{\lengderivado}{L({\cal G})}
\newcommand{\fg}{grupo finitamente generado }
%\newcommand{\ol}{\overline{}}
\newcommand{\aut}{\text{Aut}}
\newcommand{\Sy}{\text{Sym}} 

\newcommand{\cay}[2]{\text{Cay}(#1,#2)}

\newcommand{\partes}[1]{{\cal{P}}(#1)} 

\newcommand*{\deri}{{\cal D}}
\newcommand*{\lexorder}{\le_{\textrm{lex}}}


\newcommand{\fp}{grupo finitamente presentado }
\newcommand{\vl}{virtualmente libre }
\newcommand{\vls}{virtualmente libres}
\newcommand{\WP}{\text{WP}(G, \Sigma)}

\newcommand{\cG}{ {\cal G} }
\newcommand{\cGg}{{\cal G} = (V, \Sigma, P, S)}
\newcommand{\cH}{ {\cal H} }
\newcommand{\Xm}{\widetilde X}
%\newcommand{\ol}{\overline{}}

%%% Capítulo 5. Cortes.
\newcommand{\olc}[1]{#1^{c}}
\newcommand{\ca}{{\cal C}(\alpha)}
\newcommand{\cmin}{{\cal C}_{\text{min}}}
\newcommand{\cam}{{\cal C}_{\text{min}}(\alpha)}
\newcommand{\copta}{{\cal C}_{\text{opt}}(\alpha)}
\newcommand{\copt}{{\cal C}_{\text{opt}}}
\newcommand*{\rows}{6}

\newcommand{\TODO}[1]{\textcolor{red}{TODO: #1}}

\newenvironment{leoenv}{\color{brightcerulean}}{\ignorespacesafterend}
%%%%%%%%%%%%%%  SETUP DE LA PÁGINA %%%%%%%%%%%%%%%%%
%\usepackage{fancyhdr} 
\pagestyle{headings} 
\pagenumbering{arabic} 
%\foot[C]{\textbf{\thepage}} % except the center
%\setlength{\headheight}{42pt}% ...at least 51.60004pt
%\renewcommand{\headrulewidth}{0.8pt}
%\head[L]{\thepage} 
%\head[R]{\textsl{\leftmark}} 
%\fancyfoot[C]{\thepage}

\usepackage{float}


\usepackage{subfiles} % mejor ponerlos al final


\title{El problema de la palabra de los grupos virtualmente libres.}
\subtitle{Defensa de tesis de licenciatura.}
\date{? de marzo de 2024.}
\author{Leopoldo Lerena}
\institute{Universidad de Buenos Aires}
% \titlegraphic{\hfill\includegraphics[height=1.5cm]{logo.pdf}}

\begin{document}
	\maketitle

	
	\begin{frame}[fragile]{Teoría de lenguajes.}
		Dado un conjunto finito $A$ notaremos por $A^*$ al monoide libre sobre $A$.
		
		\begin{deff}
			Un lenguaje $L$ sobre un alfabeto $A$ es un subconjunto de $A^*$.
		\end{deff}	
		
		Podemos clasificar los lenguajes de acuerdo a qué tipo de máquina las acepta.
		
		Un lenguaje \ic{} es un lenguaje aceptado por un \APND.
			
	\end{frame}
	
	
	\begin{frame}[fragile]{Problema de la palabra.}
		Sea $G$ un grupo \fg por un conjunto finito $A$; 
		tal que $G$ es isomorfo a $\langle A \mid R \rangle$ para un conjunto de relaciones $R \subseteq (A \cup A^{-1})^*$.
		
		El \emph{problema de la palabra} consiste en el siguiente problema:
		\begin{itemize}
			\item 
				\textbf{Entrada}: Una palabra $w \in (A \cup A^{-1})^*$.
			
			\item 
				\textbf{Pregunta}: Decidir si vale $w=1$ en $G$.
		\end{itemize}
		
		Existe un grupo $G$ \fg tal que su problema de la palabra no es decidible.
		
		
		
	\end{frame}
	
	
	
	
	\begin{frame}[fragile]{Teorema de Muller-Schupp.}
		
		
		El problema de la palabra lo podemos considerar como un lenguaje formal:
		\[
		\WP{G}{A \cup A^{-1}} = \{ w \in (A \cup A^{-1})^* \mid w = 1 \}
		\]
		
		\begin{itemize}
			\item 
				Un grupo $G$ \fg es \emph{\ic} si existe un conjunto finito de generadores $A$ tal que $\WP{G}{A \cup A^{-1}}$ es \ic.
				
			\item 
				Un grupo \fg $G$ es \emph{\vl} si existe $F$ subgrupo libre de índice finito.
				
		\end{itemize}
	\end{frame}

	\begin{frame}[fragile]{Teo Muller--Schupp}
		
		\begin{teo}[Muller--Schupp 1983]
			Un grupo es \ic \ si y solo sí  es virtualmente libre.
		\end{teo}
		
		\begin{center}
			\includegraphics[scale = 0.65]{MSch_1983.jpg}
		\end{center}
	
		
	\end{frame}
	
	
	\begin{frame}[fragile]{Teorema de Muller-Schupp.}
		\begin{itemize}
			\item 
				Esta no es la única caracterización que tienen los grupos \vl. 
				(Araujo, Silva - 2018).
		\end{itemize}
		
		
		\begin{center}
			\includegraphics[scale = 0.35]{equivalencias_vl.jpg}
		\end{center}		
	
	\end{frame}	
	
	\begin{frame}[fragile]{Equivalencias.}
		En este trabajo probamos las siguientes equivalencias.
		
		\[	
		\begin{tikzpicture}[scale=0.8]
			\path 
			(0,0) node(a) [rectangle,draw] {Grupo fundamental de un grafo de grupos finito}
			(5,-3) node(b) [rectangle,draw] {Treewidth finito}
			(0,-6) node(c) [rectangle,draw] {Independiente de contexto}
			(-5,-3) node (d) [rectangle,draw] {Virtualmente libre};
			\draw   
			(d) edge[<-,line width=1.0pt] (a) 
			(c) edge[<-,line width=1.0pt] (d)
			(b) edge[<-,line width=1.0pt] (c)
			(a)  edge[<-,line width=1.0pt] (b);
		\end{tikzpicture}
		\]
	\end{frame}
	
	
	
	\begin{frame}[fragile]{Treewidth}
		
		Dado un grafo $\Gamma$ no dirigido una descomposición en un árbol es un par $(T,f)$ tal que cumple las siguientes propiedades.
		
			\begin{enumerate}
				\item[\textbf{T1.}]
				 	Para todo vértice $v \in V(\Gamma)$ debe existir $t \in V(T)$ tal que $v \in f(t)$. 
				\item[\textbf{T2.}]
					Para toda arista $\{v,w\} \in E(\Gamma)$ 
					debe existir $t \in V(T)$ tal que $v,w \in f(t)$.
				\item[\textbf{T3.}] 
					Si $v \in V(\Gamma)$ es tal que $v \in f(t) \cap f(s)$ luego $v \in f(r)$ para todo $r \in V(T)$ en la geodésica que va desde $s$ a $t$ dentro de $T$.  
			\end{enumerate}
		
		Un grafo $\Gamma$ tienen \emph{treewidth finito} si existe una constante $k \in \NN$ y una descomposición en un árbol $(T,f)$ para $\Gamma$ de manera que:
		\[
			\sup_{t \in V(T)} |f(t)| - 1 \le k.
		\]
		
		
	\end{frame}
	
	\begin{frame}[fragile]{Propiedades del treewidth.}
		\begin{itemize}
			\item 
				Sea $\Gamma$ un grafo no dirigido y $C \subseteq V(\Gamma)$ un conjunto de vértices entonces definimos los \emph{vecinos de $C$} por medio de 
				\[
				N(C) = \{ v \in V(\Gamma) \mid \exists w \in C, \ \{v,w \} \in E(\Gamma) \}.
				\]
			
			\item 
				Dado un grafo no dirigido $\Gamma$ y un conjunto de vértices $C \subseteq V(\Gamma)$ definimos el \emph{borde de vértices} de $C$ como
				\[
				\beta C =  N(C) \cap N(\ol{ C}).
				\] 
		\end{itemize}
		
		
		

	\end{frame}
	
	\begin{frame}[fragile]{Descomposición para un grafo de Cayley pt 1.}
		Sea $G$ un grupo \fg por $A$ y $\Gamma = \cay{G}{A}$ el respectivo grafo de Cayley de $G$.
		
		Sea $V_l = \Gamma \setminus N^l(\{1\}) $ tal que $V_0 = \Gamma \setminus \{1\}$.  	
		
		Definimos un grafo $T$ de la siguiente manera:
		\begin{itemize}
			\item 
				Los \textbf{vértices}.
				\[
				V(T) = \{  \beta C : \exists l \in \NN / C \subseteq V_l \ \text{componente conexa} \} \cup \{ 1 \}.
				\]
			
			\item 
				Las \textbf{aristas}.
				\[
				E(T) = \{ \{ \beta C, \beta D \} : \exists l \in \NN / \ C \subseteq D \subseteq V_l \land \ C \subseteq V_{l+1}  \} \cup \{  \{1,\beta C\} : C \subseteq V_0  \}
				\]
		\end{itemize}
	\end{frame}

	\begin{frame}[fragile]{Descomposición para un grafo de Cayley pt 2.}
		\begin{prop}
			$T$ es un árbol.
		\end{prop}		
	
		\begin{proof}
			\begin{itemize}
				\item 
				\textbf{T es acíclico.}
					\red{Todo}
				\item 
				\textbf{T es conexo.}
				\red{Todo}
			\end{itemize}
		\end{proof}
	\end{frame}

	\begin{frame}[fragile]{Descomposición para un grafo de Cayley pt 3.}
		Sea entonces $f: V(T) \to 2^{V(\Gamma)}$ la función definida por $f(\beta C) = \beta C$	.
		
		\begin{prop}
			$(T,f)$ es una descomposición en un árbol para $\cay{G}{A}$.
		\end{prop}
		
		\begin{proof}
			\red{Todo.}
		\end{proof}
	
	\end{frame}

	\begin{frame}[fragile]{Descomposición para un grafo de Cayley pt 4.}
		
		\red{Agregar dibujo del ejemplo que hice en la tesis.}	
		
	\end{frame}
	
	\begin{frame}[fragile]{IC implica TW finito pt 1.}
		
		\begin{prop}
			El treewidth finito es un invariante por cuasisometría.
		\end{prop}
		
		Un \emph{grupo $G$ \fg tiene treewidth finito} si para algún conjunto de generadores $A$ vale que $\text{Cay}(G,A)$ tiene treewidth finito.
			
	\end{frame}

	\begin{frame}[fragile]{IC implica TW finito pt 2.}
		\begin{teo} 
			Sea $G$ un grupo \fg por $A$ entonces si $G$ es \ic{} luego $G$ tiene treewidth finito.
		\end{teo}
		
		\begin{proof}
			\begin{itemize}
				\item 
					Vamos a probar que la descomposición anteriormente construida para $\cay{G}{A}$ tiene treewidth finito.
					
				\item 
					Veamos que $\text{diam}(\beta C) < \infty$ para todo $\beta C \in V(T) $.
					
				\item 
					Como $G$ es \ic{} entonces el lenguaje $\WP{G}{A \cup A^{-1}}$ es generado por una gramática $\gramatica$ que podemos considerar que está en forma normal de Chomsky.
					
				
			\end{itemize}
		\end{proof}
	\end{frame}

	\begin{frame}[fragile]{IC implica TW finito pt 3.}
		\begin{itemize}
			\item 
				Para cada variable $A \in V$ de la gramática consideramos el lenguaje 
				\[
				L_A = \{ w \in \Sigma^* : A \deriva w  \}.
				\]
				Consideramos $k_A \in \NN$ definido por $k_A = {\min}_{w \in L_A} |w|$ y así definimos $k = \max_{A \in V} k_A$.
				
			\item  
				Vamos a probar que $\text{diam}(\beta C) \le 3k$ para todo $\beta C \in V(T)$.
				
			\item 
				Sean $g,h \in \beta C$, vamos a acotar $d(g,h)$. 
				Consideremos una geodésica $\alpha$ que una $1$ con $g$, $\gamma$ geodésica que una $h$ con $1$ y como $C \cup \beta C$ es conexo podemos tomar un camino $\tau$ que una $g$ con $h$ dentro de $C$ exceptuando sus extremos. 
				Las etiquetas del ciclo $\alpha \tau \gamma$ forman una palabra $ uvw \in \WP{G}{A \cup A^{-1}}$.
			
			\item 
				Tenemos una derivación $S \deriva uvw$.
				
				
				\red{Completar esta demo.}
		\end{itemize}
	\end{frame}
	
	
	
	
	
\end{document}