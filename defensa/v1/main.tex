\documentclass[aspectratio=169, 10pt]{beamer}
\usepackage[utf8]{inputenc}
\usepackage[spanish]{babel}
\usepackage{amsmath,amsfonts,amsthm,amssymb,mathtools,sectsty}
\pagenumbering{gobble}
\usepackage{subcaption}
%\usepackage{graphicx}
%\usepackage[pdftex,dvipsnames]{xcolor}
\usepackage{cancel}
\usepackage{graphicx}
\usepackage{marginnote}
\usepackage{mathabx}
\usepackage{float}
\setlength{\marginparwidth}{2cm}

% Tikz y las librerías para automátas
\usepackage{tikz-cd}
\usepackage{tikz}
\usetikzlibrary{arrows,automata}
\usetikzlibrary{babel} %para evitar que se jodan los automatas de tikz
\usetikzlibrary{graphs} 
\usetikzlibrary{calc}
\usetikzlibrary{positioning}
%\usetikzlibrary{shapes.geometric}  % for [ellipse], [diamond], etc

%\usepackage[backend=biber,...]{biblatex} 

%Referencias; me gustaría que backref funcione pero no es importante tampoco.
\usepackage[pagebackref]{hyperref}
% Para modificar el estilo de las referencias
\hypersetup{
	colorlinks,
	linkcolor={astral},
	citecolor={red!70!black},
	urlcolor={red!80!black}
}
\definecolor{astral}{RGB}{46,116,181}
\colorlet{chulo}{blue!70!purple}
\colorlet{rojo}{purple!45!black}
\definecolor{carrotorange}{rgb}{0.93, 0.57, 0.13}
\definecolor{brightcerulean}{rgb}{0.11, 0.67, 0.84}
\definecolor{brightube}{rgb}{0.82, 0.62, 0.91}
\definecolor{cadmiumred}{rgb}{0.89, 0.0, 0.13}
\definecolor{applegreen}{rgb}{0.55, 0.71, 0.0}
\definecolor{aurometalsaurus}{rgb}{0.43, 0.5, 0.5}

%%%%%%%%%%%%%%%%%%%%% ENUMERAR CON COSAS QUE NO SEAN SOLO NÚMEROS %%%%%%%%%%
\usepackage[shortlabels]{enumitem}
\setlist[enumerate]{font=\bfseries}
\usepackage{adjustbox}


%%%%%%%%%%%%%%%%%%%%%%% TÍTULOS DE SECCIONES MÁS FANCIES %%%%%%%%%%%%%%
\usepackage{titlesec}
\setcounter{secnumdepth}{3} % Hasta que profundidad quiero numerar, 4 sería los párrafos.
\titleformat{\section}[block]{\color{astral!50!black}\Large\bfseries\filcenter}{\S\thesection.}{1em}{}
\titleformat{\subsection}[hang]{\color{astral!50!black}\large\bfseries\filcenter}{\S\thesubsection.}{1em}{}

%%%%%%%%%%%%%%%%%%%%%% COLORES PÁRRAFOS Y CAPÍTULOS %%%%%%%%%%%%%%%%%%%%%%%%
%\paragraphfont{\color{astral!70!black}}
\chapterfont{\color{astral!40!black}}
%\subsectionfont{\color{astral!60!black} }
%\sectionfont{\color{astral!50!black} }

\usepackage{mathpazo}
\usepackage{amssymb}
%\usepackage{thmtools}

%Esto sirve para armar grafos de Cayley de una manera más copada.
\usetikzlibrary{lindenmayersystems,arrows.meta}
\pgfdeclarelindenmayersystem{cayley}{
	\rule{G->G-G+++G--G}
	\symbol{R}{
		\pgflsystemstep=0.5\pgflsystemstep
	}

}

\usepackage[framemethod=tikz]{mdframed}

%%%%%%%%%%%%%  TEOREMAS  %%%%%%%%%%%%%%%%%
\theoremstyle{plain} %% el estilo clásico
\newtheorem{teo}{\color{rojo}{ { Teorema}}}[section]
\newtheorem{prop}[teo]{\color{rojo} {Proposición}}
\newtheorem{lema}[teo]{\color{rojo} {Lema}}
\newtheorem{coro}[teo]{\color{rojo} {Corolario}}
\newtheorem*{aff}{ {Afirmación}}
% Si pongo [theorem] siguen la numeración de los teoremas. 
% e.j. Teo 1, Lema 2, Teo 3, Teo 4 ...
\theoremstyle{definition}
\newtheorem{deff}[teo]{{ Definición}}{\smallskip}
\newtheorem{ej}[teo]{{Ejemplo}}{\smallskip}

% Remarks
\theoremstyle{remark}
\newtheorem{obs}[teo]{ {Observación}}{\smallskip}

%%%%%%%%%% FRAMES PARA TEOREMAS A LO HATCHER %%%%%%%%%%%%%%%%%%%%%%%%%

\surroundwithmdframed[outerlinewidth=0.4pt,
innerlinewidth=0.4pt,
align=center,
middlelinewidth=1pt,
middlelinecolor=white,
innertopmargin=-4pt,
innerbottommargin=0pt,
innerrightmargin=4pt,
innerleftmargin=4pt,
bottomline=false,topline=false,rightline=false]{teo}
\surroundwithmdframed[outerlinewidth=0.4pt,
innerlinewidth=0.4pt,
align=center,
middlelinewidth=1pt,
middlelinecolor=white,
innertopmargin=-4pt,
innerbottommargin=0pt,
innerrightmargin=4pt,
innerleftmargin=4pt,
bottomline=false,topline=false,rightline=false]{lema}

\surroundwithmdframed[outerlinewidth=0.4pt,
innerlinewidth=0.4pt,
align=center,
middlelinewidth=1pt,
middlelinecolor=white,
innertopmargin=-4pt,
innerbottommargin=0pt,
innerrightmargin=4pt,
innerleftmargin=4pt,
bottomline=false,topline=false,rightline=false]{prop}


\surroundwithmdframed[outerlinewidth=0.4pt,
innerlinewidth=0.4pt,
align=center,
middlelinewidth=1pt,
middlelinecolor=white,
innertopmargin=-4pt,
innerbottommargin=0pt,
innerrightmargin=4pt,
innerleftmargin=4pt,
bottomline=false,topline=false,rightline=false]{coro}

%==================================================================%

% DEMOS EN NEGRITA.
\renewenvironment{proof}{{\textbf{Demostración.}}}{ \hfill $\blacksquare$ \medskip} 

%% ========== Para escribir pseudo ==========
%\usepackage{algorithm}
%\usepackage[noend]{algpseudocode}  % "noend" es para no mostrar los endfor, endif
%%\algrenewcommand\alglinenumber[1]{\tiny #1:}  % Para que los numeros de linea del pseudo sean pequeños
%\renewcommand{\thealgorithm}{}  % Que no aparezca el numero luego de "Algorithm"
%\floatname{algorithm}{ }    % Entre {  } que quiero que aparezca en vez de "Algorithm"
%
%% traducciones
%\algrenewcommand\algorithmicwhile{\textbf{mientras}}
%\algrenewcommand\algorithmicdo{\textbf{hacer}}
%\algrenewcommand\algorithmicreturn{\textbf{devolver}}
%\algrenewcommand\algorithmicif{\textbf{si}}
%\algrenewcommand\algorithmicthen{\textbf{entonces}}
%\algrenewcommand\algorithmicfor{\textbf{para}}
%
%%% indentar dentro de los algoritmos
%\algdef{SE}[SUBALG]{Indent}{EndIndent}{}{\algorithmicend\ }%
%\algtext*{Indent}
%\algtext*{EndIndent}

% =========================================================
\usepackage[colorinlistoftodos,prependcaption,textsize=tiny]{todonotes}



%Comandos útiles.
\newcommand\RP{\mathbb{RP}}
\newcommand{\norm}[1]{\left\lVert#1\right\rVert}
\newcommand{\RR}{\mathbb{R}}
\newcommand{\CC}{\mathbb{C}}
\newcommand{\NN}{\mathbb{N}}
\newcommand{\ZZ}{\mathbb{Z}}
\newcommand{\Om}{\Omega}
\newcommand{\A}{\mathcal A}
\newcommand\ol{\overline}
\newcommand{\blue}{\textcolor{chulo}}
\newcommand{\red}{\textcolor{rojo}}
\newcommand{\Gg}{\mathfrak g}
\newcommand{\SL}{SL_2(\mathbb Z)}
\newcommand{\stab}{\text{Stab}}
\newcommand{\ic}{independiente de contexto }
\newcommand{\APND}{automáta de pila no determinístico }
\newcommand{\APD}{automáta de pila determinístico }
\newcommand{\gramatica}{{\cal G} = (V, \Sigma, P, S)}
\newcommand{\deriva}{\overset{*}{\to_{\cal G}}}
\newcommand{\tto}{\overset{*}{\to}}
\newcommand{\lengderivado}{L({\cal G})}
\newcommand{\fg}{grupo finitamente generado }
%\newcommand{\ol}{\overline{}}
\newcommand{\aut}{\text{Aut}}
\newcommand{\Sy}{\text{Sym}} 

\newcommand{\cay}[2]{\text{Cay}(#1,#2)}

\newcommand{\partes}[1]{{\cal{P}}(#1)} 

\newcommand*{\deri}{{\cal D}}
\newcommand*{\lexorder}{\le_{\textrm{lex}}}


\newcommand{\fp}{grupo finitamente presentado }
\newcommand{\vl}{virtualmente libre }
\newcommand{\vls}{virtualmente libres}
\newcommand{\WP}{\text{WP}(G, \Sigma)}

\newcommand{\cG}{ {\cal G} }
\newcommand{\cGg}{{\cal G} = (V, \Sigma, P, S)}
\newcommand{\cH}{ {\cal H} }
\newcommand{\Xm}{\widetilde X}
%\newcommand{\ol}{\overline{}}

%%% Capítulo 5. Cortes.
\newcommand{\olc}[1]{#1^{c}}
\newcommand{\ca}{{\cal C}(\alpha)}
\newcommand{\cmin}{{\cal C}_{\text{min}}}
\newcommand{\cam}{{\cal C}_{\text{min}}(\alpha)}
\newcommand{\copta}{{\cal C}_{\text{opt}}(\alpha)}
\newcommand{\copt}{{\cal C}_{\text{opt}}}
\newcommand*{\rows}{6}

\newcommand{\TODO}[1]{\textcolor{red}{TODO: #1}}

\newenvironment{leoenv}{\color{brightcerulean}}{\ignorespacesafterend}
%%%%%%%%%%%%%%  SETUP DE LA PÁGINA %%%%%%%%%%%%%%%%%
%\usepackage{fancyhdr} 
\pagestyle{headings} 
\pagenumbering{arabic} 
%\foot[C]{\textbf{\thepage}} % except the center
%\setlength{\headheight}{42pt}% ...at least 51.60004pt
%\renewcommand{\headrulewidth}{0.8pt}
%\head[L]{\thepage} 
%\head[R]{\textsl{\leftmark}} 
%\fancyfoot[C]{\thepage}

\usepackage{float}


\usepackage{subfiles} % mejor ponerlos al final

\renewcommand{\indent}{\hspace*{2em}}
\makeatletter
\addtobeamertemplate{block begin}{}{\setlength{\parindent}{1cm}\@afterindentfalse\@afterheading}
\makeatother

\title{El problema de la palabra de los grupos virtualmente libres.}
\subtitle{\textbf{Leopoldo Lerena} \\
		Defensa de tesis de licenciatura.}
\date{Fecha: ---}

\author{Director: Iván Sadofschi Costa}
\institute{Universidad de Buenos Aires}
% \titlegraphic{\hfill\includegraphics[height=1.5cm]{logo.pdf}}

\begin{document}
	\maketitle

	
	
	
	\begin{frame}[fragile]{Problema de la palabra.}
		Sea $G$ un grupo \fg por un conjunto $A$; 
		tal que $G$ es isomorfo a $\langle A \mid R \rangle$ para un conjunto de relaciones $R$.
		
		El \emph{problema de la palabra} consiste en el siguiente problema:
	
		\begin{itemize}
					\item 
						\textbf{Entrada}: Una palabra $w$ en $A \cup A^{-1}$.
					
					\item 
						\textbf{Pregunta}: Decidir si vale $w=1$ en $G$.
		\end{itemize}

		El problema de la palabra no es un problema \alert{decidible}.
		\begin{itemize}
			\item 
				Existen grupos tales que resulta imposible construir un algoritmo que decida si una palabra representa la identidad o no.
		\end{itemize}
		

	\end{frame}

\begin{frame}[fragile]{Grafo de Cayley.}
	Dado un grupo $G$ finitamente generado por $A$ podemos considerar un grafo $\Gamma =\cay{G}{A}$ que es el grafo de Cayley.

	Está definido de manera que 
	\begin{align*}
		V(\Gamma) &= G   \\ 
		& \\
		E(\Gamma) &= \{ \{ g,ga \}  \mid g \in G, a \in A \cup A^{-1}  \}. 
	\end{align*}

	% \begin{columns}
	% 	\begin{column}{0.5\textwidth}
	% 		\tcajita{Ejemplo.}{	Sea $G = \ZZ/2\ZZ \times \ZZ/2\ZZ$ presentado por 
	% 		$G = \langle a,b \mid aba^{-1}b^{-1}, a^{2}, b^{2}\rangle$. Entonces $\cay{G}{\{a,b\}}$ es el \\ siguiente grafo.}
	% 	\end{column}
	% 	\begin{column}{0.5\textwidth}
	% 		\[
	% 			\begin{tikzpicture}
	% 			[scale=0.60,V/.style = {circle, draw,align= center, minimum size=0.5cm,
	% 				minimum size=2em,inner sep=2,
	% 				 fill=astral!15,font=\scriptsize	},fill fraction/.style={path picture={
	% 						\fill[#1] 
	% 						(path picture bounding box.south) rectangle
	% 						(path picture bounding box.north west);
	% 				}},
	% 				fill fraction/.default=astral!90
	% 				]
	% 		\node[align=center] at (2, -6) {};
	% 			\begin{scope}[nodes=V,xshift=4.5cm, yshift=-4cm]
	% 					\node (1) at (0,0) {$1$};
	% 					\node (a) at (2,2)  {$a$};
	% 					\node (b) at (2,-2)     {$b$};
	% 					\node (ab) at (4,0)    {$ab$};
	% 			\end{scope}
				
	% 			\draw   (1)  edge[-] (a);
	% 			\draw   (1)  edge[-] (b);
	% 			\draw   (a)  edge[-] (ab);
	% 			\draw   (b)  edge[-] (ab);
	% 	\end{tikzpicture}
	% 	\]
	% 	\end{column}
	% \end{columns}
\end{frame}

\begin{frame}[fragile]{Ejemplo de grafo de Cayley.}
	\tcajita{Ejemplo.}{	Sea $G = \ZZ/2\ZZ \times \ZZ/2\ZZ$ presentado por 
			$G = \langle a,b \mid aba^{-1}b^{-1}, a^{2}, b^{2}\rangle$. Entonces $\cay{G}{\{a,b\}}$ es el siguiente grafo: \\  \\ 
			\centering
			\begin{tikzpicture}
				[scale=0.60,V/.style = {circle, draw,align= center, minimum size=0.5cm,
					minimum size=2em,inner sep=2,
					 fill=astral!15,font=\scriptsize	},fill fraction/.style={path picture={
							\fill[#1] 
							(path picture bounding box.south) rectangle
							(path picture bounding box.north west);
					}},
					fill fraction/.default=astral!90]
			\node[align=center] at (2, -6) {};
				\begin{scope}[nodes=V,xshift=4.5cm, yshift=-4cm]
						\node (1) at (0,0) {$1$};
						\node (a) at (2,2)  {$a$};
						\node (b) at (2,-2)     {$b$};
						\node (ab) at (4,0)    {$ab$};
				\end{scope}
				\draw   (1)  edge[-] (a);
				\draw   (1)  edge[-] (b);
				\draw   (a)  edge[-] (ab);
				\draw   (b)  edge[-] (ab);
		\end{tikzpicture}
		}
\end{frame}

\begin{frame}[fragile]{Grupos libres.}
	Dado un conjunto $A$ notamos $F_{A}$ al \emph{grupo libre} generado por los elementos de $A$. 

	% \onslide<2->
	El grupo $F_{A}$ viene con una función $\iota: A \to F_{A}$ que denominamos la inclusión de los generadores en el grupo libre y queda caracterizado por la siguiente propiedad universal: 
	Para todo grupo $H$ y toda función $f:A \to H$ existe un único morfismo de grupos $\ol f: F_{A} \to H$ tal que $\ol f \circ \iota = f$.
	\begin{center}
		\begin{tikzcd}
			F_{A}  \arrow[rr, "\ol f", dashed]          &  & H \\
			&  &   \\
			A \arrow[uu, "\iota"] \arrow[rruu, "f", swap] &  &  
		\end{tikzcd}
	\end{center}

	% \onslide<3->
	Decimos que $A$ genera libremente a $F_{A}$ y que $A$ es una base de $F_{A}$.

	% \onslide<4->
	Podemos identificar los elementos de $F_{A}$ con las palabras reducidas en $A \cup A^{-1}$.
\end{frame}

	\begin{frame}[fragile]{Grupos virtualmente libres.}
		Un grupo $G$ es \emph{virtualmente libre} si es finitamente generado y si
		tiene un subgrupo libre $F$ tal que $[G:F] < \infty$.

		\begin{alertblock}{Ejemplos.}
			\begin{itemize}
				\item Los grupos finitos.
				\item Los grupos libres.
				\item El producto semidirecto de un grupo libre con un grupo finito.
				\item El producto libre de dos grupos finitos.
			\end{itemize}
		\end{alertblock}
	\end{frame}

	\begin{frame}[fragile]{Teoría de lenguajes.}
		Dado un conjunto finito $\Sigma$ notaremos por $\Sigma^*$ al monoide libre sobre $\Sigma$.
		
		
			Un lenguaje $L$ sobre un alfabeto $\Sigma$ es un subconjunto de $\Sigma^*$.
			
		
		\begin{alertblock}{Ejemplos.}
			\begin{itemize}
				\item 
					Dado $\Sigma = \{a,b\}$ consideramos el lenguaje de los palíndromos
					\[
						L = \{ w \in \Sigma^{*} \mid w = w^{R}  \}.
					\]
				\item 
					Dado un grupo $G$ finitamente generado por $A$ consideramos el lenguaje
					\[
						\WP{G}{A} = \{ w \in (A \cup A^{-1})^{*} \mid w = 1 \ \text{en $G$} \}.	
					\]
			\end{itemize}
		\end{alertblock}
	\end{frame}
	
	\begin{frame}[fragile]{Ejemplo de gramática.}
		Consideremos las variables $V =\{ S,A \}$ y el alfabeto $\Sigma = \{ a,b \}$.

		Consideramos las siguientes reglas.
		\begin{align*}
			S  & \to aAS  \\
			S  & \to b    \\
			A  & \to baAb \\
			aA & \to bbb
		\end{align*}

		El lenguaje generado por esta gramática es el conjunto de las palabras $w \in \Sigma^{*}$ que podemos obtener a partir de $S$ aplicando las reglas.

		Ejemplo: podemos \emph{derivar} la palabra $abbbbbb$ de la siguiente manera.  
		\[
			S \to aAS \to abaAbS \to abaAbb \to abbbbbb	
		\]
	\end{frame}

	\begin{frame}[fragile]{Gramáticas (pt.1).}
		\tcajita{Definición}{Una \emph{gramática} es una tupla ${\cal G} = (V, \Sigma, P, S)$ donde:
		\begin{itemize}
			\item $V$ es un conjunto finito denominado las \emph{variables};
			\item $\Sigma$ es un conjunto finito disjunto de $V$ que denominamos \emph{símbolos terminales};
			\item $P \subseteq ((V \cup \Sigma)^{*} - \Sigma^{*}) \times (V \cup \Sigma)^*$ es un conjunto finito de \emph{producciones}.
			\item $S \in V$ es el \emph{símbolo inicial};
		\end{itemize}
			Es usual escribir las producciones $(\gamma, \nu) \in P$ como $\gamma \to \nu$.}
		
	\end{frame}
	
	\begin{frame}[fragile]{Gramáticas (pt.2).}
		Una sucesión de producciones $\gamma_{1} \to \gamma_{2} \to \dots \to \gamma_{n}$ la denotamos $\gamma_{1} \tto \gamma_{n}$ y diremos que es un \emph{derivación}.
		\medskip 
		\tcajita{Definición.}{Dada una gramática $\cGg$  definimos el \emph{lenguaje generado por la gramática} como
		\[
		L({\cal G}) = \{ w \in \Sigma^* \ | \ S \tto w   \}.
		\]}

	\end{frame}


	\begin{frame}[fragile]{Gramáticas independiente de contexto.}
		\defcajita{Una gramática $\gramatica $ es \emph{independiente de contexto} si las producciones tienen la siguiente forma:
		\[
			A \to w
		\]
		donde $A \in V, w \in (\Sigma \cup V)^*$.  \\
		Si $L=\lengderivado$ para alguna gramática independiente de contexto $\cal G$ entonces diremos que $L$ es un \emph{lenguaje independiente de contexto}.}
	\end{frame}

	
	\begin{frame}[fragile]{Teorema de Muller--Shupp}
		Si un grupo $G$ es tal que para cierto conjunto de generadores $A$ vale que $\WP{G}{A}$ es independiente de contexto entonces para todo conjunto de generadores $B$ vale que $\WP{G}{B}$ es independiente de contexto.

		\bigskip

		\teocajita{Un grupo $G$ es \vl{} si y solo si para algún conjunto de generadores $A$ de $G$ vale que el lenguaje $\WP{G}{A}$ es \ic.}
	\end{frame}

	\begin{frame}[fragile]{Teorema de Muller-Schupp.}
		
		\[	
			\begin{tikzpicture}{scale = 0.75}
				\path 
				(0,0) node(a) [rectangle,draw] {$G$ es isomorfo a $\pi_{1}(\cG,P)$ para $\cG$ un grafo de grupos finito con grupos finitos.}
				(5,-3) node(b) [rectangle,draw] {$\cay{G}{A}$ tiene treewidth finito}
				(0,-6) node(c) [rectangle,draw] {$\WP{G}{A}$ es un lenguaje independiente de contexto}
				(-5,-3) node (d) [rectangle,draw] {$G$ es virtualmente libre};
				\draw   
				(d) edge[<-,line width=1.0pt,"Capítulo 2"] (a) 
				(c) edge[<-,line width=1.0pt,"Capítulo 3"] (d)
				(b) edge[<-,line width=1.0pt,"Capítulo 4"] (c)
				(a)  edge[<-,line width=1.0pt,"Capítulo 5"] (b);
			\end{tikzpicture}
		\]
		\TODO{Cambiar el orden del diagrama.}
	\end{frame}

	

	\begin{frame}{Descomposición en un árbol y treewidth de un grafo.}
	Sea $\Gamma$ un grafo no dirigido.
	Una \textbf{descomposición en un árbol} de $\Gamma$ es un par $(T,f)$ donde
	$T$ es un árbol y $f$ una función 
	\[
	f: V(T) \to \partes{V(\Gamma)}
	\]
	Que cumple las siguientes condiciones:
	\begin{enumerate}
		\item Para todo vértice $v \in V(\Gamma)$ debe existir $t \in V(T)$ tal que $v \in f(t)$. 
		\item Para toda arista $\{v,w\} \in E(\Gamma)$ 
		debe existir $t \in V(T)$ tal que $v,w \in f(t)$.
		\item Si $v \in V(\Gamma)$ es tal que $v \in f(t) \cap f(s)$ luego $v \in f(r)$ para todo $r \in V(T)$ en la geodésica que va desde $s$ a $t$.  
	\end{enumerate}
	El \textbf{bagsize} de esta descomposición es el siguiente valor:
	\begin{equation*}
		bs(\Gamma,T,f) = \sup_{t \in V(T)} |f(t)| - 1 \in \  \NN \cup \{ +\infty \}.
	\end{equation*}
	Un grafo $\Gamma$ tiene \textbf{treewidth finito} si existe $(T,f)$ tal que $bs(\Gamma,T,f) < \infty$.
	\end{frame}
	
	\begin{frame}[fragile]{Descomponiendo el grafo de Cayley en un árbol.}
		Dado $\Gamma = \cay{G}{A}$ donde $G$ es finitamente generado por $A$.

		Sea $C \subseteq V(\Gamma)$ definimos:
		\begin{itemize}
			\item  
				La \textbf{vecindad} de $C$ es el siguiente conjunto:

				$N(C) = C \cup \{ v \in V(\Gamma) \mid \exists w \in C, \ \{v,w \} \in E(\Gamma) \}.$

			\item 
				El \textbf{borde} de $C$ es el siguiente conjunto: 

				$\beta C =  N(C) \cap N(C^{c})$.
		\end{itemize} 
	\end{frame}

	\begin{frame}[fragile]{Descomponiendo el grafo de Cayley en un árbol pt.2}
		Para cada $n \in \NN_{0}$ consideramos:
		\[
			V_{n} = V(\Gamma \setminus B_{n}(1))	
		\]

		Consideramos el siguiente árbol $T$ con raíz $1$:
		\begin{align*}
			V(T)  & = \{ \beta C \mid C \in V_{n} \ \text{componente conexa} \}  \cup \{ \{1\} \} \\
			E(T)  =  & \{ \{ \beta C, \beta D \}  \mid C \subseteq D \\ 
			& \text{$C$ es componente conexa de $V_{n+1}$ y $D$ componente conexa de $V_{n}$} \} \\
			& \cup \{ \{\{1\}, \beta C\} \mid C \ \text{es componente conexa de $V_{0}$} \}
		\end{align*}
		Definimos $f: V(T) \to \partes{V(\Gamma)}$ por medio de $f(\beta C) = \beta C$.
		
		\lemacajita{El par $(T,f)$ es una descomposición en un árbol para el grafo $\Gamma$.}
		
	\end{frame}

	\begin{frame}[fragile]{Ejemplo de la descomposición en un árbol de un grafo de Cayley (pt. 1).}
	Sea $G = \ZZ / 2\ZZ \ast \ZZ / 3\ZZ $ presentado por 
	$G \simeq \langle a,b \mid a^2, b^3 \rangle$.

	El grafo $\cay{G}{\{ a,b,b^{-1} \}}$ lo representamos de la siguiente manera:
	
		\[
			\begin{tikzpicture}
			[scale=0.60,V/.style = {circle, draw,align= center, minimum size=0.5cm,
				minimum size=2em,inner sep=2,
				 fill=astral!15,font=\scriptsize	},fill fraction/.style={path picture={
						\fill[#1] 
						(path picture bounding box.south) rectangle
						(path picture bounding box.north west);
				}},
				fill fraction/.default=astral!90
				]
		\node[align=center] at (2, -6) {};
			\begin{scope}[nodes=V,xshift=4.5cm, yshift=-4cm]
					\node (1) at (0,0) {$1$};
					\node (a) at (2,0)  {$a$};
					\node (b) at (-2,2)     {$b$};
					\node (bb) at (-2,-2)    {$b^2$};
					\node (ab) at (4,2)      {$ab$};
					\node (abb) at (4,-2)     {$ab^2$};
					\node (ba) at (-4,2)     {$ba$};
					\node (bba) at (-4,-2)     {$b^2a$};
					\node (aba) at (6,2)    {$aba$};
					\node (abba) at (6,-2)    {$ab^2a$};
					\node (bab) at (-6,3)    {$bab$};
					\node (babb) at (-6,1)     {$bab^2$};
					\node (bbab) at (-6,-3)     {$b^2ab$};
					\node (bbabb) at (-6,-1)    {$b^2ab^2$};
					\node (abab) at (8,3)    {$abab$};
					\node (ababb) at (8,1)    {$abab^2$};
					\node (abbab) at (8,-1)     {$ab^2ab$};
					\node (abbabb) at (8,-3)  {$ab^2ab^2$};
			\end{scope}
			
			\draw   (1)  edge[-] (a);
			\draw   (1)  edge[-] (b);
			\draw   (1)  edge[-] (bb);
			\draw   (b)  edge[-] (bb);
			\draw   (b)  edge[-] (ba);
			\draw   (bb)  edge[-] (bba);
			\draw   (a)  edge[-] (ab);
			\draw   (abb)  edge[-] (a);
			\draw   (ab)  edge[-] (abb);
			\draw   (ab)  edge[-] (aba);
			\draw   (aba)  edge[-] (abab);
			\draw   (ababb)  edge[-] (aba);
			\draw   (ababb)  edge[-] (abab);
			\draw   (abb)  edge[-] (abba);
			\draw   (abba)  edge[-] (abbab);
			\draw   (abba)  edge[-] (abbabb);
			\draw   (bba)  edge[-] (bbabb);
			\draw   (bba)  edge[-] (bbab);
			\draw   (ba)  edge[-] (bab);
			\draw   (babb)  edge[-] (bab);
			\draw   (abbab)  edge[-] (abbabb);
			\draw   (ba)  edge[-] (babb);
			\draw   (bbabb)  edge[-] (bbab);
			\draw   (abbab)  edge[-] (abbabb);
	\end{tikzpicture}
	\]
	\end{frame}
	
	\begin{frame}[fragile]{Ejemplo de la descomposición en un árbol de un grafo de Cayley (pt. 2).}
		El árbol $T$ de la descomposición lo representamos de la siguiente manera:

	\[
			\begin{tikzpicture}[scale=0.8, V/.style = {circle, draw,align= center, minimum size=0.5cm,
			minimum size=3em,inner sep=2,
			fill=applegreen!15,font=\scriptsize	},fill fraction/.style={path picture={
				\fill[#1] 
				(path picture bounding box.south) rectangle
				(path picture bounding box.north west);
		}},
		fill fraction/.default=gray!50
		]
		\node[align=center] at (2, -6) {};
		\begin{scope}[nodes=V,xshift=4.5cm, yshift=-4cm]
			\node (1) at (0,0)  {$\{ b,b^2,1 \}$};
			\node (2) at (2,0)  {$\{1\}$};
			\node (3) at (4,0)  {$\{ 1,a \}$};
			\node (4) at (-2,3)     {$\{ b,ba \}$};
			\node (5) at (-2,-3)   {$\{b^2,b^2a\}$};
			\node (6) at (6,0)  {$\{ a,ab,ab^2 \}$};
			
			\node (9) at (8,3)  {$\{ab,aba\}$};
			\node (10) at (8,-3)  {$\{ ab^2, ab^2a \}$};
		\end{scope}
		
		
		\draw (1) edge[-] (2);
		\draw (2) edge[-] (3);
		\draw (1) edge[-] (5);
		\draw (1) edge[-] (4);
		\draw (3) edge[-] (6);
		\draw (6) edge[-] (9);
		\draw (6) edge[-] (10);
	\end{tikzpicture}
	\]
	\end{frame}

	\begin{frame}[fragile]{Lema descomposición en un árbol treewidth finito.}
		\lemacajita{Sea $G$ grupo finitamente generado por $X$,
		$\Gamma = \cay{G}{X}$ el grafo de Cayley y $(T,f)$ una descomposición en un árbol para $\Gamma$.
		Entonces si existe $M \in \NN$ tal que para todo $t \in V(T)$ vale que:
		\[
			\diam{f(t)} \le M 
		\]
		luego $\Gamma$ tiene treewidth finito.}

		\textbf{Idea de la demostración.}

		Basta notar que el grafo de Cayley tiene grado acotado por $2|X|$ y así para todo $t \in V(T)$ vale que:
		\[
			|f(t)| \le 2|X|^{M} < \infty
		\]
		por lo que $\Gamma$ tiene treewidth finito.
		\qed 

	\end{frame}

	\begin{frame}[fragile]{Un lema para el lenguaje del problema de la palabra.}
		\lemacajita{Sea $G$ un grupo generado por $\Sigma$ tal que $\WP{G}{\Sigma}$ es independiente de contexto. 
		Sea $\gramatica$ una gramática tal que $L(\cG) = \WP{G}{\Sigma}$. 
		Sea $A \in V$ luego si definimos: 
		\[
			L_{A} = \{  w \in \Sigma^{*} \mid  A \deriva w \}
		\]
		Entonces para todas $w_{1},w_{2} \in L_{A}$ vale que:
		$w_{1} = w_{2}$ en $G$}

		\textbf{Idea de la demostración.}

		Sea una derivación 
		\[
			S \deriva uAv
		\]
		luego usando que $w_{1},w_{2} \in L_{A}$ obtenemos que $S \deriva uAv \deriva uw_{1}v $ y que $S \deriva uAv \deriva uw_{2}v$ lo que implica que $w_{1} = w_{2}$ en $G$
		\qed
	\end{frame}

	\begin{frame}[fragile]{Forma normal de Chomsky.}
		\defcajita{
			Una gramática $\gramatica$ independiente de contexto está en \emph{forma normal de Chomsky} si las producciones son de este tipo:
			\begin{enumerate}
				\item $A \to BC$ donde $A\in V$ y $B,C \in V \setminus \{ S \}$.
				\item $A \to a$ donde $A \in V, a \in \Sigma$.
				\item $S \to \epsilon$ 
			\end{enumerate}}
		\bigskip 
		\teocajita{
			Sea $\cG$ una gramática \ic{} entonces existe una gramática \ic{} en forma normal de Chosmky $\cG'$ tal que $L(\cG) = L(\cG')$.}
	\end{frame}

	\begin{frame}[fragile]{Árbol de derivación.}
		Sea $\cG$ una gramática en forma normal de Chomsky luego si $w \in L(\cG)$ y $\deri = S \deriva w$ es una derivación le asociamos un árbol binario $T(\deri)$ que denominamos \emph{el árbol de derivación}.

		\alert{Ejemplo}.

		Sea $\gramatica$ tal que $V = \{ S,A,B \}$, $\Sigma = \{ a, b\}$ y las reglas son:
		$P = \begin{cases}
								S \to AB \mid   \\
								A \to  BA \mid a	\\
								B \to AB \mid BB \mid b 	\\
		\end{cases}$

		Luego tenemos la siguiente derivación para la palabra $bbbabb \in L(\cG)$.
		\[
			\deri = S \to AB \to BAB \to bAB \to bBAB \to bbAB \to bbaB \to bbaBB \to bbabB \to bbbabb.
		\]
	\end{frame}

	\begin{frame}[fragile]{Árbol de derivación.}
		El árbol de la derivación $\deri = S \to AB \to BAB \to bAB \to bBAB \to bbAB \to bbaB \to bbaBB \to bbabB \to bbbabb$
		\[
			\begin{tikzpicture}[node distance=0.4cm]
				\node (s) {$S$};
				\node[below left=0.3cm and 3.2cm of  s] (0) {$A$};
				\node[below right=0.3cm and 3.2cm of  s] (1) {$B$};
				\node[below left=0.3cm and 1.2cm of  0] (00) {$B$};
				\node[below right=0.3cm and 1.2cm of  0] (01) {$A$};
				\node[below=1.2cm of  00] (000) {$b$};
				\node[below left=0.3cm and 1.2cm of  01] (010) {$B$};
				\node[below right=0.3cm and 1.2cm of  01] (011) {$A$};
				\node[below= 1.2cm of  010] (0100) {$b$};
				\node[below= 1.2cm of  011] (0110) {$a$};
				\node[below left=0.3cm and 1.2cm of  1] (10) {$B$};
				\node[below right=0.3cm and 1.2cm of  1] (11) {$B$};
				\node[below=1.2cm of  10] (100) {$b$};
				\node[below=1.2cm of  11] (110) {$b$};

				\draw (s) -- (0);
				\draw (s) -- (1);
				\draw (0) -- (00);
				\draw (0) -- (01);
				\draw (1) -- (10);
				\draw (1) -- (11);
				\draw (00) -- (000);
				\draw (01) -- (010);
				\draw (01) -- (011);
				\draw (010) -- (0100);
				\draw (011) -- (0110);
				\draw (10) -- (100);
				\draw (11) -- (110);

			\end{tikzpicture}
		\]

		\begin{itemize}
			\item Las etiquetas de las hojas del árbol forman la palabra que derivamos.
			\item Los vértices que no son hojas tienen de etiqueta una variable de la gramática.
		\end{itemize}
	\end{frame}

	\begin{frame}[fragile]{Árbol de derivación.}
		El árbol de la derivación $\deri = S \to AB \to BAB \to bAB \to bBAB \to bbAB \to bbaB \to bbaBB \to bbabB \to bbabb$
		\[
			\begin{tikzpicture}[node distance=0.4cm]
				\node (s) {$S$};
				\node[below left=0.3cm and 3.2cm of  s] (0) {\alert{$A$}};
				\node[below right=0.3cm and 3.2cm of  s] (1) {$B$};
				\node[below left=0.3cm and 1.2cm of  0] (00) {$B$};
				\node[below right=0.3cm and 1.2cm of  0] (01) {$A$};
				\node[below=1.2cm of  00] (000) {\alert{$b$}};
				\node[below left=0.3cm and 1.2cm of  01] (010) {$B$};
				\node[below right=0.3cm and 1.2cm of  01] (011) {$A$};
				\node[below= 1.2cm of  010] (0100) {\alert{$b$}};
				\node[below= 1.2cm of  011] (0110) {\alert{$a$}};
				\node[below left=0.3cm and 1.2cm of  1] (10) {$B$};
				\node[below right=0.3cm and 1.2cm of  1] (11) {$B$};
				\node[below=1.2cm of  10] (100) {$b$};
				\node[below=1.2cm of  11] (110) {$b$};

				\draw (s) -- (0);
				\draw (s) -- (1);
				\draw (0) -- (00);
				\draw (0) -- (01);
				\draw (1) -- (10);
				\draw (1) -- (11);
				\draw (00) -- (000);
				\draw (01) -- (010);
				\draw (01) -- (011);
				\draw (010) -- (0100);
				\draw (011) -- (0110);
				\draw (10) -- (100);
				\draw (11) -- (110);

			\end{tikzpicture}
		\]

		\begin{itemize}
			\item Las etiquetas de las hojas del árbol forman la palabra que derivamos.
			\item Los vértices que no son hojas tienen de etiqueta una variable de la gramática.
			\item Diremos que el vértice \alert{A} produce la subpalabra \alert{bba}.
		\end{itemize}
	\end{frame}
	

	\begin{frame}[fragile]{Un lema para forma normal de Chomsky.}
		\lemacajita{Sea $\gramatica$ una gramática independiente de contexto en forma normal de Chomsky.
		Sea $uvw \in L(\cG)$. 
		Si fijamos una derivación $\deri = S \tto uvw$ entonces entonces existe un vértice que es el más grande entre aquellos con la propiedad de producir una palabra que contiene a $v$.}

		\textbf{Ejemplo:}
		Si consideramos la subpalabra $v = \alert{bb}$ luego el vértice $\alert{A}$ es el vértice que cumple esta propiedad y produce la subpalabra $v' = {bba}$.
		
			
			
			\begin{center}
				\begin{adjustbox}{width = 0.5 \textwidth}
					\begin{tikzpicture}[node distance=0.2cm]
						\node (s) {$S$};
						\node[below left=0.15cm and 2.2cm of  s] (0) {\alert{$A$}};
						\node[below right=0.15cm and 2.2cm of  s] (1) {$B$};
						\node[below left=0.15cm and 1.0cm of  0] (00) {$B$};
						\node[below right=0.15cm and 1.0cm of  0] (01) {$A$};
						\node[below=1cm of  00] (000) {\alert{\bf$b$}};
						\node[below left=0.15cm and 1.0cm of  01] (010) {$B$};
						\node[below right=0.15cm and 1.0cm of  01] (011) {$A$};
						\node[below=1cm of  010] (0100) {\alert{$b$}};
						\node[below=1cm of  011] (0110) {{$a$}};
						\node[below left=0.15cm and 1.0cm of  1] (10) {$B$};
						\node[below right=0.15cm and 1.0cm of  1] (11) {$B$};
						\node[below=1cm of  10] (100) {$b$};
						\node[below=1cm of  11] (110) {$b$};
		
						\draw (s) -- (0);
						\draw (s) -- (1);
						\draw (0) -- (00);
						\draw (0) -- (01);
						\draw (1) -- (10);
						\draw (1) -- (11);
						\draw (00) -- (000);
						\draw (01) -- (010);
						\draw (01) -- (011);
						\draw (010) -- (0100);
						\draw (011) -- (0110);
						\draw (10) -- (100);
						\draw (11) -- (110);
		
					\end{tikzpicture}
				\end{adjustbox}
			\end{center}
		
	\end{frame}


	\begin{frame}[fragile]{Teorema de Muller--Schupp.}
		\teocajita{Sea $G$ un grupo finitamente generado por $X$ luego si $\WP{G}{X}$ es un lenguaje independiente de contexto entonces el grafo  $\cay{G}{X}$ tiene treewidth finito.}

		\textbf{Idea de la demostración.}
		\begin{itemize}
			\item Vamos a probar que la descomposición en un árbol $(T,f)$ anteriormente construida para un grafo de Cayley arbitrario tiene bagsize finito.
			
			\item Por el \textbf{lema 1} nos alcanza con ver que existe $M \in \NN$ tal que:
			\[
				\diam{f(t)} \le M 
			\]  

		\end{itemize}
	\end{frame}

	\begin{frame}[fragile]{Teorema de Muller--Schupp.}
		\begin{itemize}
			\item Queremos encontrar $M \in \NN$ tal que para todo $n \in \NN$, $C$ componente conexa de $\Gamma \setminus B_{n}(1)$
			y para todos $g,h \in \beta C$ se verifique:
			\[
				d(g,h) \le M.
			\]

			\item Por ser $\WP{G}{X}$ un lenguaje independiente de contexto tenemos una gramática $\cG = (V,X \cup X^{-1}, P, S)$ tal que $L(\cG) = \WP{G}{X}$.
			Por el \textbf{lema 2} sabemos que para cada variable $A \in V$ vale que si $w_{1},w_{2} \in L_{A}$ luego:
			\[
				w_{1} = w_{2} \ \text{en } G
			\]
			Para cada $A \in V$ definimos 
			\[
				k_{A} = \underset{w \in L_{A}}{\min} |w|
			\]
			luego sea $k = \underset{A \in V}{\max} \ k_{A}$.

			\item Proponemos $\alert{M = 3k}$.
		\end{itemize}
	\end{frame}

	\begin{frame}[fragile]{Teorema de Muller--Schupp.}
		
		\begin{columns}
			\begin{column}{0.65 \textwidth}
				\begin{center}
					\begin{adjustbox}{width = 
						0.65\textwidth}
						\begin{tikzpicture}[font=\sffamily]
								\path (-1,2) coordinate (A) (8,3) coordinate (B) (8,-8) coordinate (C);
								\draw[thick,path picture={
									\foreach \X in {A,B,C}
									{\draw[line width=0.4pt] (\X) circle (0);}}] (A) node[label={[font=\large]},left]{{\huge $g$}} to[bend left=22] node (B) {}
								(B) node[above right]{\huge $h$} to[bend right=10] node (C){}
								(C) node[below]{\huge $1$} to[bend right=8] node (A) {} cycle;
								% \draw[line width = 0.8pt] (y) to (z) ;
								% \draw[line width = 0.8pt] (z) to (x) ;
								% \draw[line width = 0.8pt] (y) to (x) ;
								% \node[label=left:B] at (barycentric cs:x=1,y=1) {};
								% \node[label=right:C] at (barycentric cs:y=1,z=1) {};
								% \node[label=below:A] at (barycentric cs:x=1,z=1) {};
								\node[label=right:{\huge{$\alpha$}}] at (2,-2) {};
								\node[label=right:{\huge{$\tau$}}] at (2.4,3.67) {};
								\node[label=right:{\huge{$\gamma$}}] at (7.5,-1.5) {};

								\draw [very thick, cadmiumred, label= $B_{n}(1)$] (-3,2.25)[bend left=10]  to (10,2.5);
								\node[label=right:{\huge{\textcolor{cadmiumred}{$B_{n}(1)$}}}] at (-3,3) {};
						\end{tikzpicture}
					\end{adjustbox}
				\end{center}
			\end{column}
			\begin{column}{0.35 \textwidth}
				Donde:
				\begin{itemize}
					\item $\alpha$ es una geodésica de $1$ a $g$;
					\item $\gamma$ es una geodésica de $h$ a $1$.
					\item $\tau$ es un camino de $g$ a $h$ dentro de $C \cup \beta C$;
				\end{itemize}
			\end{column}
		\end{columns}
	\end{frame}

	\begin{frame}[fragile]{Teorema de Muller--Schupp.}
		\begin{itemize}
			\item Si consideramos el \emph{ciclo} $\alpha \tau \gamma$ luego las etiquetas del camino nos da una palabra $uvw \in \WP{G}{X}$.
			Por lo tanto tenemos una derivación
			\[
				\deri = S \deriva uvw.
			\]

			\item Por el \textbf{lema 3} existe una variable $A \in V$ tal que
			\[
				\deri = S \deriva \delta A \eta \to \delta BC \eta \deriva u'v'v''w' = uvw
			\]
			donde:
			\begin{itemize}
				\item La palabra $v'v''$ es la palabra más chica que contiene a $v$ como subpalabra que deriva de una variable.
				\item $A \deriva v'v''$
				\item $\delta \deriva u'$ y $\eta \deriva w'$
			\end{itemize} 
			
		\end{itemize}
	\end{frame}
	
\end{document}