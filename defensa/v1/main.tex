\documentclass[aspectratio=169, 10pt]{beamer}
\usepackage[utf8]{inputenc}
\usepackage[spanish]{babel}
\usepackage{amsmath,amsfonts,amsthm,amssymb,mathtools,sectsty}
\pagenumbering{gobble}
\usepackage{subcaption}
%\usepackage{graphicx}
%\usepackage[pdftex,dvipsnames]{xcolor}
\usepackage{cancel}
\usepackage{graphicx}
\usepackage{marginnote}
\usepackage{mathabx}
\usepackage{float}
\setlength{\marginparwidth}{2cm}

% Tikz y las librerías para automátas
\usepackage{tikz-cd}
\usepackage{tikz}
\usetikzlibrary{arrows,automata}
\usetikzlibrary{babel} %para evitar que se jodan los automatas de tikz
\usetikzlibrary{graphs} 
\usetikzlibrary{calc}
\usetikzlibrary{positioning}
%\usetikzlibrary{shapes.geometric}  % for [ellipse], [diamond], etc

%\usepackage[backend=biber,...]{biblatex} 

%Referencias; me gustaría que backref funcione pero no es importante tampoco.
\usepackage[pagebackref]{hyperref}
% Para modificar el estilo de las referencias
\hypersetup{
	colorlinks,
	linkcolor={astral},
	citecolor={red!70!black},
	urlcolor={red!80!black}
}
\definecolor{astral}{RGB}{46,116,181}
\colorlet{chulo}{blue!70!purple}
\colorlet{rojo}{purple!45!black}
\definecolor{carrotorange}{rgb}{0.93, 0.57, 0.13}
\definecolor{brightcerulean}{rgb}{0.11, 0.67, 0.84}
\definecolor{brightube}{rgb}{0.82, 0.62, 0.91}
\definecolor{cadmiumred}{rgb}{0.89, 0.0, 0.13}
\definecolor{applegreen}{rgb}{0.55, 0.71, 0.0}
\definecolor{aurometalsaurus}{rgb}{0.43, 0.5, 0.5}

%%%%%%%%%%%%%%%%%%%%% ENUMERAR CON COSAS QUE NO SEAN SOLO NÚMEROS %%%%%%%%%%
\usepackage[shortlabels]{enumitem}
\setlist[enumerate]{font=\bfseries}
\usepackage{adjustbox}


%%%%%%%%%%%%%%%%%%%%%%% TÍTULOS DE SECCIONES MÁS FANCIES %%%%%%%%%%%%%%
\usepackage{titlesec}
\setcounter{secnumdepth}{3} % Hasta que profundidad quiero numerar, 4 sería los párrafos.
\titleformat{\section}[block]{\color{astral!50!black}\Large\bfseries\filcenter}{\S\thesection.}{1em}{}
\titleformat{\subsection}[hang]{\color{astral!50!black}\large\bfseries\filcenter}{\S\thesubsection.}{1em}{}

%%%%%%%%%%%%%%%%%%%%%% COLORES PÁRRAFOS Y CAPÍTULOS %%%%%%%%%%%%%%%%%%%%%%%%
%\paragraphfont{\color{astral!70!black}}
\chapterfont{\color{astral!40!black}}
%\subsectionfont{\color{astral!60!black} }
%\sectionfont{\color{astral!50!black} }

\usepackage{mathpazo}
\usepackage{amssymb}
%\usepackage{thmtools}

%Esto sirve para armar grafos de Cayley de una manera más copada.
\usetikzlibrary{lindenmayersystems,arrows.meta}
\pgfdeclarelindenmayersystem{cayley}{
	\rule{G->G-G+++G--G}
	\symbol{R}{
		\pgflsystemstep=0.5\pgflsystemstep
	}

}

\usepackage[framemethod=tikz]{mdframed}

%%%%%%%%%%%%%  TEOREMAS  %%%%%%%%%%%%%%%%%
\theoremstyle{plain} %% el estilo clásico
\newtheorem{teo}{\color{rojo}{ { Teorema}}}[section]
\newtheorem{prop}[teo]{\color{rojo} {Proposición}}
\newtheorem{lema}[teo]{\color{rojo} {Lema}}
\newtheorem{coro}[teo]{\color{rojo} {Corolario}}
\newtheorem*{aff}{ {Afirmación}}
% Si pongo [theorem] siguen la numeración de los teoremas. 
% e.j. Teo 1, Lema 2, Teo 3, Teo 4 ...
\theoremstyle{definition}
\newtheorem{deff}[teo]{{ Definición}}{\smallskip}
\newtheorem{ej}[teo]{{Ejemplo}}{\smallskip}

% Remarks
\theoremstyle{remark}
\newtheorem{obs}[teo]{ {Observación}}{\smallskip}

%%%%%%%%%% FRAMES PARA TEOREMAS A LO HATCHER %%%%%%%%%%%%%%%%%%%%%%%%%

\surroundwithmdframed[outerlinewidth=0.4pt,
innerlinewidth=0.4pt,
align=center,
middlelinewidth=1pt,
middlelinecolor=white,
innertopmargin=-4pt,
innerbottommargin=0pt,
innerrightmargin=4pt,
innerleftmargin=4pt,
bottomline=false,topline=false,rightline=false]{teo}
\surroundwithmdframed[outerlinewidth=0.4pt,
innerlinewidth=0.4pt,
align=center,
middlelinewidth=1pt,
middlelinecolor=white,
innertopmargin=-4pt,
innerbottommargin=0pt,
innerrightmargin=4pt,
innerleftmargin=4pt,
bottomline=false,topline=false,rightline=false]{lema}

\surroundwithmdframed[outerlinewidth=0.4pt,
innerlinewidth=0.4pt,
align=center,
middlelinewidth=1pt,
middlelinecolor=white,
innertopmargin=-4pt,
innerbottommargin=0pt,
innerrightmargin=4pt,
innerleftmargin=4pt,
bottomline=false,topline=false,rightline=false]{prop}


\surroundwithmdframed[outerlinewidth=0.4pt,
innerlinewidth=0.4pt,
align=center,
middlelinewidth=1pt,
middlelinecolor=white,
innertopmargin=-4pt,
innerbottommargin=0pt,
innerrightmargin=4pt,
innerleftmargin=4pt,
bottomline=false,topline=false,rightline=false]{coro}

%==================================================================%

% DEMOS EN NEGRITA.
\renewenvironment{proof}{{\textbf{Demostración.}}}{ \hfill $\blacksquare$ \medskip} 

%% ========== Para escribir pseudo ==========
%\usepackage{algorithm}
%\usepackage[noend]{algpseudocode}  % "noend" es para no mostrar los endfor, endif
%%\algrenewcommand\alglinenumber[1]{\tiny #1:}  % Para que los numeros de linea del pseudo sean pequeños
%\renewcommand{\thealgorithm}{}  % Que no aparezca el numero luego de "Algorithm"
%\floatname{algorithm}{ }    % Entre {  } que quiero que aparezca en vez de "Algorithm"
%
%% traducciones
%\algrenewcommand\algorithmicwhile{\textbf{mientras}}
%\algrenewcommand\algorithmicdo{\textbf{hacer}}
%\algrenewcommand\algorithmicreturn{\textbf{devolver}}
%\algrenewcommand\algorithmicif{\textbf{si}}
%\algrenewcommand\algorithmicthen{\textbf{entonces}}
%\algrenewcommand\algorithmicfor{\textbf{para}}
%
%%% indentar dentro de los algoritmos
%\algdef{SE}[SUBALG]{Indent}{EndIndent}{}{\algorithmicend\ }%
%\algtext*{Indent}
%\algtext*{EndIndent}

% =========================================================
\usepackage[colorinlistoftodos,prependcaption,textsize=tiny]{todonotes}



%Comandos útiles.
\newcommand\RP{\mathbb{RP}}
\newcommand{\norm}[1]{\left\lVert#1\right\rVert}
\newcommand{\RR}{\mathbb{R}}
\newcommand{\CC}{\mathbb{C}}
\newcommand{\NN}{\mathbb{N}}
\newcommand{\ZZ}{\mathbb{Z}}
\newcommand{\Om}{\Omega}
\newcommand{\A}{\mathcal A}
\newcommand\ol{\overline}
\newcommand{\blue}{\textcolor{chulo}}
\newcommand{\red}{\textcolor{rojo}}
\newcommand{\Gg}{\mathfrak g}
\newcommand{\SL}{SL_2(\mathbb Z)}
\newcommand{\stab}{\text{Stab}}
\newcommand{\ic}{independiente de contexto }
\newcommand{\APND}{automáta de pila no determinístico }
\newcommand{\APD}{automáta de pila determinístico }
\newcommand{\gramatica}{{\cal G} = (V, \Sigma, P, S)}
\newcommand{\deriva}{\overset{*}{\to_{\cal G}}}
\newcommand{\tto}{\overset{*}{\to}}
\newcommand{\lengderivado}{L({\cal G})}
\newcommand{\fg}{grupo finitamente generado }
%\newcommand{\ol}{\overline{}}
\newcommand{\aut}{\text{Aut}}
\newcommand{\Sy}{\text{Sym}} 

\newcommand{\cay}[2]{\text{Cay}(#1,#2)}

\newcommand{\partes}[1]{{\cal{P}}(#1)} 

\newcommand*{\deri}{{\cal D}}
\newcommand*{\lexorder}{\le_{\textrm{lex}}}


\newcommand{\fp}{grupo finitamente presentado }
\newcommand{\vl}{virtualmente libre }
\newcommand{\vls}{virtualmente libres}
\newcommand{\WP}{\text{WP}(G, \Sigma)}

\newcommand{\cG}{ {\cal G} }
\newcommand{\cGg}{{\cal G} = (V, \Sigma, P, S)}
\newcommand{\cH}{ {\cal H} }
\newcommand{\Xm}{\widetilde X}
%\newcommand{\ol}{\overline{}}

%%% Capítulo 5. Cortes.
\newcommand{\olc}[1]{#1^{c}}
\newcommand{\ca}{{\cal C}(\alpha)}
\newcommand{\cmin}{{\cal C}_{\text{min}}}
\newcommand{\cam}{{\cal C}_{\text{min}}(\alpha)}
\newcommand{\copta}{{\cal C}_{\text{opt}}(\alpha)}
\newcommand{\copt}{{\cal C}_{\text{opt}}}
\newcommand*{\rows}{6}

\newcommand{\TODO}[1]{\textcolor{red}{TODO: #1}}

\newenvironment{leoenv}{\color{brightcerulean}}{\ignorespacesafterend}
%%%%%%%%%%%%%%  SETUP DE LA PÁGINA %%%%%%%%%%%%%%%%%
%\usepackage{fancyhdr} 
\pagestyle{headings} 
\pagenumbering{arabic} 
%\foot[C]{\textbf{\thepage}} % except the center
%\setlength{\headheight}{42pt}% ...at least 51.60004pt
%\renewcommand{\headrulewidth}{0.8pt}
%\head[L]{\thepage} 
%\head[R]{\textsl{\leftmark}} 
%\fancyfoot[C]{\thepage}

\usepackage{float}


\usepackage{subfiles} % mejor ponerlos al final

\renewcommand{\indent}{\hspace*{2em}}
\makeatletter
\addtobeamertemplate{block begin}{}{\setlength{\parindent}{1cm}\@afterindentfalse\@afterheading}
\makeatother

\title{El problema de la palabra para los grupos virtualmente libres.}
\subtitle{\textbf{Leopoldo Lerena} \\
		Defensa de tesis de licenciatura.}
\date{Fecha: ---}

\author{Director: Iván Sadofschi Costa}
\institute{Universidad de Buenos Aires}
% \titlegraphic{\hfill\includegraphics[height=1.5cm]{logo.pdf}}

\begin{document}
	\maketitle

	
	
	
	\begin{frame}[fragile]{Problema de la palabra.}
		Sea $ G = \langle X \mid R \rangle$ grupo finitamente presentado.
		
		\pause 

		El \textbf{problema de la palabra} consiste en:
		\pause 
		\begin{itemize}
					\item 
						\textbf{Entrada}: Una palabra $w$ en $X \cup X^{-1}$.
					\pause 
					\item 
						\textbf{Pregunta}: ¿Vale $w=1$ en $G$?
		\end{itemize}
		\pause 

		El problema de la palabra no es \emph{decidible}.
		\pause 
		\begin{itemize}
			\item 
				Existen grupos que no admiten un algoritmo que decida si una palabra representa la identidad o no.
		\end{itemize}
		
		
	\end{frame}

	\begin{frame}[fragile]{Grafo de Cayley.}
		\defcajita{Sea $G$ grupo finitamente generado por $X$. 
		El \textbf{grafo de Cayley} $\Gamma$ es el siguiente grafo no dirigido:
		\vspace{-0.15in}	
		\begin{align*}
			V &= G   \\ 
			E &= \{ \{ g,gx \}  \mid g \in G, x \in X  \}. 
		\end{align*}
		\vspace*{-0.3 in}
		}
				\onslide<2->{
				\textbf{Ejemplo.}
				Sea $G = \ZZ/2\ZZ \times \ZZ/2\ZZ$ presentado por $G = \langle a,b \mid aba^{-1}b^{-1}, a^{2}, b^{2}\rangle$.}


				\onslide<3->{
				\vspace*{-0.1in}
				
				\begin{center}
					\begin{adjustbox}{width = 0.45 \textwidth}
						\begin{tikzpicture}
									[scale=0.60,V/.style = {circle, draw,align= center, minimum size=0.5cm,
										minimum size=2em,inner sep=2,
										fill=astral!15,font=\scriptsize	},fill fraction/.style={path picture={
												\fill[#1] 
												(path picture bounding box.south) rectangle
												(path picture bounding box.north west);
										}},
										fill fraction/.default=astral!90]
								\node[align=center] at (1.5, -4.5) {\Large{$\Gamma \ \ =$}};
									\begin{scope}[nodes=V,xshift=4.5cm, yshift=-4cm]
											\node (1) at (0,0) {$1$};
											\node (a) at (2,2)  {$a$};
											\node (b) at (2,-2)     {$b$};
											\node (ab) at (4,0)    {$ab$};
									\end{scope}
									\draw   (1)  edge[-] (a);
									\draw   (1)  edge[-] (b);
									\draw   (a)  edge[-] (ab);
									\draw   (b)  edge[-] (ab);
						\end{tikzpicture}
					\end{adjustbox}
				\end{center}}			
	\end{frame}


	\begin{frame}[fragile]{Grupos virtualmente libres.}
		\onslide<+->
		Dado un conjunto $X$ notamos $F_{X}$ al \textbf{grupo libre} con base $X$.

		\onslide<+->
		Identificamos los elementos de $F_{X}$ con las palabras reducidas en $X \cup X^{-1}$.
		
		\onslide<+->
		\defcajita{Un grupo $G$ es \textbf{virtualmente libre} si es finitamente generado y
		tiene un subgrupo libre $F$ tal que $[G:F] < \infty$.}
			\onslide<+->
			\textbf{Ejemplos}.
				\begin{itemize}
					\item Los grupos finitos.
					\item \pause Los grupos libres.
					\item \pause El producto semidirecto de un grupo libre con un grupo finito.
					\item \pause El producto libre de dos grupos finitos.
				\end{itemize}
	\end{frame}

	\begin{frame}[fragile]{Lenguajes.}
		Dado un conjunto finito $\Sigma$ notaremos por $\Sigma^*$ al monoide libre sobre $\Sigma$.
		\pause 
		Un \textbf{lenguaje} $L$ sobre un alfabeto $\Sigma$ es un subconjunto de $\Sigma^*$.
		\pause 	

			\textbf{Ejemplos}.


			\begin{itemize}
				\item 
					Dado $\Sigma = \{a,b\}$ consideramos el lenguaje de los palíndromos
					\[
						L = \{ w \in \Sigma^{*} \mid w = w^{R}  \}.
					\]
				\item \pause 
					Dado un grupo $G$ finitamente generado por $X$ consideramos el lenguaje
					\[
						\WP{G}{X} = \{ w \in (X \cup X^{-1})^{*} \mid w = 1 \ \text{en $G$} \}.	
					\]
			\end{itemize}
	\end{frame}
	
	\begin{frame}[fragile]{Ejemplo de gramática.}
		Consideremos las variables $V =\{ S,T \}$ y el alfabeto $\Sigma = \{ a,b \}$.
		\pause 
		Las siguientes reglas determinan una gramática:
		\begin{align*}
			S  & \to aTS  \\
			S  & \to b    \\
			T  & \to baTb \\
			aT & \to bbb
		\end{align*}

		\pause 

		A partir de una gramática podemos \emph{generar} palabras sobre $\Sigma$.

		\pause 
		\textbf{Ejemplo}: la palabra $abbbbbb$ la generamos de la siguiente manera:  
		\[
			S \to aTS \to abaTbS \to abaTbb \to abbbbbb	
		\]
	\end{frame}

	\begin{frame}[fragile]{Gramáticas.}
		\tcajita{Definición}{Una \textbf{gramática} es una tupla ${\cal G} = (V, \Sigma, P, S)$ donde:
		\pause 
		\begin{itemize}
			\item $V$ es un conjunto finito denominado las \textbf{variables};
			\pause 
			\item $\Sigma$ es un conjunto finito disjunto de $V$ que denominamos el \textbf{alfabeto};
			\pause 
			\item $P \subseteq ((V \cup \Sigma)^{*} - \Sigma^{*}) \times (V \cup \Sigma)^*$ es un conjunto finito de \textbf{producciones}; 
			\pause 
			\item $S \in V$ es el \textbf{símbolo inicial}.
			\pause 
		\end{itemize}}
		\onslide<6->{Es usual escribir las producciones $(\gamma, \nu) \in P$ como $\gamma \to \nu$.}
	\end{frame}
	
	\begin{frame}[fragile]{Derivaciones.}
		Podemos considerar la relación sobre 
		$(V \cup \Sigma)^{*}$ dada por 
		\[
		\delta \ \gamma \ \eta \to \delta \  \nu \ \eta 
		\]
		si $\gamma \to \nu \in P$ y $\delta, \eta \in (V \cup \Sigma)^{*}$.
		
		\pause 


		Denotaremos por $\tto$ la clausura reflexiva y transitiva de $\to$.
		\pause 

		Si $\gamma \tto \delta$ diremos que $\gamma$ \textbf{deriva} en $\delta$.
		\pause 
		\tcajita{Definición.}{Dada una gramática $\cGg$  definimos el \textbf{lenguaje generado por la gramática} como
		\[
		L({\cal G}) = \{ w \in \Sigma^* \ | \ S \tto w   \}.
		\]}

	\end{frame}


	\begin{frame}[fragile]{Lenguajes independiente de contexto.}
		\defcajita{Una gramática $\gramatica $ es \textbf{independiente de contexto} si las producciones tienen la siguiente forma:
		\[
			A \to w
		\]
		donde $A \in V, w \in (\Sigma \cup V)^*$.}
		\pause 
		\bigskip 

		% \defcajita{Si $L=\lengderivado$ para alguna gramática independiente de contexto $\cal G$ entonces diremos que $L$ es un \textbf{lenguaje independiente de contexto}.}
		\defcajita{Un lenguaje $L$ es \textbf{ independiente de contexto} si existe $\cG$ gramática \ic{} tal que 
		$L = \lengderivado$.}

	\end{frame}

	
	\begin{frame}[fragile]{Grupos \ic{}.}
		\defcajita{Un grupo $G$ es \textbf{independiente de contexto} si existe un conjunto de generadores $X$ tal que 
		$\WP{G}{X}$ es independiente de contexto. }
		\pause 
		%Si un grupo $G$ es tal que para cierto conjunto de generadores $A$ vale que $\WP{G}{A}$ es independiente de contexto entonces para todo conjunto de generadores $B$ vale que $\WP{G}{B}$ es independiente de contexto.

		\bigskip

		\teocajitados{Muller--Schupp}{Un grupo $G$ es \vl{} si y solo si es \ic.}
	\end{frame}

	\begin{frame}[fragile]{Teorema de Muller-Schupp.}
		
		\[	
			\begin{tikzpicture}{scale = 0.75}
				\path 
				(0,0) node(a) [rectangle,draw] {$G$ es virtualmente libre
				}
				(5,-3) node(b) [rectangle,draw, text width=4.2cm, align=center] {$G$ es isomorfo a $\pi_{1}(\cG,P)$ para $\cG$ un grafo de grupos finito con grupos finitos.}
				(0,-6) node(c) [rectangle,draw] {Grafo de Cayley tiene treewidth finito}
				(-5,-3) node (d) [rectangle,draw] {$G$ es independiente de contexto};
				\draw   
				(d) edge[<-,line width=1.0pt,"Capítulo 3"] (a) 
				(c) edge[<-,line width=1.0pt,"Capítulo 4"] (d)
				(b) edge[<-,line width=1.0pt,"Capítulo 5"] (c)
				(a)  edge[<-,line width=1.0pt,"Capítulo 2"] (b);
			\end{tikzpicture}
		\]
	\end{frame}

	

	\begin{frame}{Descomposición en un árbol.}
	Sea $\Gamma$ un grafo no dirigido.
	Una \textbf{descomposición en un árbol} es un par $(T,f)$ donde
	$T$ es un árbol y $f$ una función 
	\[
	f: V(T) \to \partes{V(\Gamma)}
	\]
	que cumple las siguientes condiciones:
	\pause 
	\begin{enumerate}
		\item Para todo vértice $v \in V(\Gamma)$ el subgrafo inducido por $\{ t \in V(T) \mid v \in f(t)\}$ es un árbol.

		\item Para toda arista $\{v,w\} \in E(\Gamma)$ 
		 existe $t \in V(T)$ tal que $v,w \in f(t)$. 
		
	\end{enumerate}
	\end{frame}
	
	\begin{frame}[fragile]{Ejemplo de descomposición en un árbol.}
		\centering
			\begin{tikzpicture}[scale=1.35,V/.style = {circle, draw, fill=gray!15,font=\footnotesize},fill fraction/.style={path picture={
						\fill[#1] 
						(path picture bounding box.south) rectangle
						(path picture bounding box.north west);
				}},
				fill fraction/.default=gray!50
	,			]
				% \node[align=center] at (4.5, 1) {Grafo \textbf{$T$}};
				% \begin{scope}[nodes=V, xshift=4.5cm, yshift=0cm]
				% 	\node (1) [fill = brightcerulean!80] {$t_1$};
				% 	\node (2) [below=of 1, fill=cadmiumred!75] {$t_{2}$};
				% 	\node (3) [below=of 2, fill=brightube!90] {$t_3$};
				% \end{scope}	
				% \draw   (1)  edge[-,] (2)
				% (2)  edge[-] (3);
				
				\node[align=center] at (0.5, 1) {Grafo \textbf{$\Gamma$}};
				\begin{scope}[nodes=V,xshift=0.5cm, yshift=0cm]
					\node (1)  [] {$v_{1}$};
					\node (2) [below=of 1]{$v_{2}$};
					\node (3) [right=of 2]    {$v_{3}$};
					\node (4) [below=of 2]    {$v_{4}$};
					\node (5) [below=of 3]    {$v_{5}$};
					\node (6) [below=of 4]    {$v_{6}$};
				\end{scope}
				\draw   (1)  edge[-] (2);
				\draw   (1)  edge[-] (3);
				\draw   (2)  edge[-] (3);
				\draw   (3)  edge[-] (4);
				\draw   (4)  edge[-] (5);
				\draw   (2)  edge[-] (5);
				\draw   (4)  edge[-] (6);
			\end{tikzpicture}
	\end{frame}
	\begin{frame}[fragile]{Ejemplo de descomposición en un árbol.}
		
		\centering
			\begin{tikzpicture}[scale=1.35,V/.style = {circle, draw, fill=gray!15,font=\footnotesize},fill fraction/.style={path picture={
						\fill[#1] 
						(path picture bounding box.south) rectangle
						(path picture bounding box.north west);
				}},
				fill fraction/.default=gray!50]
				\node[align=center] at (4.5, 1) {Grafo \textbf{$T$}};
				\begin{scope}[nodes=V, xshift=4.5cm, yshift=0cm]
					\node (1) [fill = brightcerulean!80] {$t_1$};
					\node (2) [below=of 1, fill=cadmiumred!75] {$t_{2}$};
					\node (3) [below=of 2, fill=brightube!90] {$t_3$};
					\node (4) [below=of 3, fill=orange!90] {$t_4$};

				\end{scope}	
				\draw   (1)  edge[-,] (2)
				(2)  edge[-] (3)
				(4)  edge[-] (3);
				
				\node[align=center] at (-1.5, 1) {Grafo \textbf{$\Gamma$}};
				\begin{scope}[nodes=V,xshift=-1.5cm, yshift=0cm]
					\node (1)  [] {$v_{1}$};
					\node (2) [below=of 1]{$v_{2}$};
					\node (3) [right=of 2]    {$v_{3}$};
					\node (4) [below=of 2]    {$v_{4}$};
					\node (5) [below=of 3]    {$v_{5}$};
					\node (6) [below=of 4]    {$v_{6}$};
				\end{scope}
				\draw   (1)  edge[-] (2);
				\draw   (1)  edge[-] (3);
				\draw   (2)  edge[-] (3);
				\draw   (3)  edge[-] (4);
				\draw   (4)  edge[-] (5);
				\draw   (2)  edge[-] (5);
				\draw   (4)  edge[-] (6);
			\end{tikzpicture}
	\end{frame}
	\begin{frame}[fragile]{Ejemplo de descomposición en un árbol.}
		\centering
		\begin{tikzpicture}[scale=0.8,V/.style = {circle, draw, fill=gray!15,font=\footnotesize},fill fraction/.style={path picture={
				\fill[#1] 
				(path picture bounding box.south) rectangle
				(path picture bounding box.north west);
		}},
		fill fraction/.default=gray!50]
			\node[align=center] at (6.5, 1) {Grafo \textbf{$T$}};
			\begin{scope}[nodes=V, xshift=6.5cm, yshift=0cm]
				\node (1) [fill = brightcerulean!80] {$t_1$};
				\node (2) [below=of 1, fill=cadmiumred!75] {$t_{2}$};
				\node (3) [below=of 2, fill=brightube!90] {$t_3$};
				\node (4) [below=of 3, fill=orange!90] {$t_4$};
			\end{scope}	
			\draw   (1)  edge[-] (2)
			(2)  edge[-] (3)
			(3)  edge[-] (4);
			
			\node[align=center] at (-3.5, 1) {Grafo \textbf{$\Gamma$}};
			\begin{scope}[nodes=V,xshift=-3.5cm, yshift=0cm]
				\node (1)  [fill = brightcerulean!80] {$v_{1}$};
				\node (2) [below=of 1, fill = brightcerulean!80, fill fraction=cadmiumred!75]{$v_{2}$};
				\node (3) [right=of 2, fill = brightcerulean!80, fill fraction=cadmiumred!75	]    {$v_{3}$};
				\node (4) [below=of 2,fill = cadmiumred!75, fill fraction=brightube!90]    {$v_{4}$};
				\node (5) [below=of 3, fill = cadmiumred!75]    {$v_{5}$};
				\node (6) [below=of 4,fill = brightube!90]    {$v_{6}$};
			\end{scope}
			\draw   (1)  edge[-] (2);
			\draw   (1)  edge[-] (3);
			\draw   (2)  edge[-] (3);
			\draw   (3)  edge[-] (4);
			\draw   (4)  edge[-] (5);
			\draw   (2)  edge[-] (5);
			\draw   (4)  edge[-] (6);
		\end{tikzpicture}
	\end{frame}

	\begin{frame}[fragile]{Treewidth.}
		\defcajita{Sea $\Gamma$ un grafo y $(T,f)$ una descomposición en un árbol.
			La \textbf{anchura} de la descomposición es el siguiente valor:
			\begin{equation*}
				a(T,f) = \sup_{t \in V(T)} |f(t)| - 1 \in \  \NN \cup \{ +\infty \}.
			\end{equation*}}
		\pause 

		\defcajita{El \textbf{treewidth} de un grafo $\Gamma$ es 
		la mínima anchura posible de una descomposición en un árbol.
		% \begin{equation*}
			% tw(\Gamma) = \min \{  a(T,f)  \mid (T,f) \ \text{descomposición en un árbol} \}
		% \end{equation*}
		}

	\end{frame}

	\begin{frame}[fragile]{Grafos como espacios métricos.}
		Sea $\Gamma$ un grafo conexo luego podemos considerar el espacio métrico dado por $(V(\Gamma), d)$ donde:
		\[
			d(x,y) = \min \{ |\gamma| : \gamma \ \text{camino de} \ x  \ \text{a} \ y  \}	
		\]
		\pause 
		si $\gamma$ es un camino que realiza la distancia entre $x$ e $y$ diremos que $\gamma$ es una \textbf{geodésica}.
		\pause 
		\defcajita{Sea $\Gamma$ un grafo conexo luego dado un conjunto $C$ definimos el \textbf{borde de vértices} de $C$ como
		\[
			\beta(C) = B_{1}\left[C\right] \cap B_{1}\left[C^{c}\right]	
		\]}
	\end{frame}

	\begin{frame}[fragile]{Descomposición en un árbol para el grafo de Cayley.}
		Sea $G$ grupo finitamente generado por $X$ y sea $\Gamma$ su grafo de Cayley.
		
		\pause 
		Para cada $n \in \NN_{0}$ consideramos:
		\[
			V_{n} = \{C : C \ \text{es componente conexa de} \ \Gamma \setminus B_{n}[1]\}	
		\]
		\pause 
		Consideramos el siguiente árbol $T$ con raíz $\{1\}$:
		\pause 
		\[
			V(T) = \{ \{1\}\} \cup \bigcup_{n \in \NN_{0}}  \{ \beta C : C \in V_{n} \}
		\]
		\pause 
		\[
			E(T) = \{ \{\beta C, \beta D\} : C \in V_{n}, D \in V_{n+1}, D \subseteq C\} \cup 
			\{ \{1, \beta C\} : C \in V_{0} \}
		\]
		\pause 
		
		Definimos $f: V(T) \to \partes{V(\Gamma)}$ por medio de $f(Y) = Y$ para todo $Y \in V(T)$.
		\pause 
		
		\lemacajita{$(T,f)$ es una descomposición en un árbol.}
		
	\end{frame}

	\begin{frame}[fragile]{Ejemplo.}
	Sea $G = \ZZ / 2\ZZ \ast \ZZ / 3\ZZ $ presentado por 
	$G \simeq \langle a,b \mid a^2, b^3 \rangle$.
	El grafo de Cayley $\Gamma$ lo representamos de la siguiente manera:
	
		\[
			\begin{tikzpicture}
			[scale=0.60,V/.style = {circle, draw,align= center, minimum size=0.5cm,
				minimum size=2em,inner sep=2,
				 fill=astral!15,font=\scriptsize	},fill fraction/.style={path picture={
						\fill[#1] 
						(path picture bounding box.south) rectangle
						(path picture bounding box.north west);
				}},
				fill fraction/.default=astral!90
				]
		\node[align=center] at (2, -6) {};
			\begin{scope}[nodes=V,xshift=4.5cm, yshift=-4cm]
					\node (1) at (0,0) {$1$};
					\node (a) at (2,0)  {$a$};
					\node (b) at (-2,2)     {$b$};
					\node (bb) at (-2,-2)    {$b^2$};
					\node (ab) at (4,2)      {$ab$};
					\node (abb) at (4,-2)     {$ab^2$};
					\node (ba) at (-4,2.75)     {$ba$};
					\node (bba) at (-4,-2.75)     {$b^2a$};
					\node (aba) at (6,2.75)    {$aba$};
					\node (abba) at (6,-2.75)    {$ab^2a$};
					\node (bab) at (-6,3.75)    {$bab$};
					\node (babb) at (-6,1.75)     {$bab^2$};
					\node (bbab) at (-6,-3.75)     {$b^2ab$};
					\node (bbabb) at (-6,-1.75)    {$b^2ab^2$};
					\node (abab) at (8,3.75)    {$abab$};
					\node (ababb) at (8,1.75)    {$abab^2$};
					\node (abbab) at (8,-1.75)     {$ab^2ab$};
					\node (abbabb) at (8,-3.75)  {$ab^2ab^2$};
			\end{scope}
			
			
			\node[above right=.3cm of abbab] (p_abbab) {$\dots$};
			\node[below right=.3cm of abbabb] (p_abbabb) {$\dots$};
			\node[above left=.3cm of bab] (p_bab) {$\dots$};
			\node[below left=.3cm of babb] (p_babb) {$\dots$};
			\node[above left=.3cm of bbabb] (p_bbabb) {$\dots$};
			\node[below left=.3cm of bbab] (p_bbab) {$\dots$};
			\node[above right=.3cm of abab] (p_abab) {$\dots$};
			\node[below right=.3cm of ababb] (p_ababb) {$\dots$};
	
			\foreach \x in {abbab,abbabb,bab,babb,bbabb,bbab,abab,ababb}
				\draw (\x) edge [-] (p_\x);
			
			\draw   (1)  edge[-] (a);
			\draw   (1)  edge[-] (b);
			\draw   (1)  edge[-] (bb);
			\draw   (b)  edge[-] (bb);
			\draw   (b)  edge[-] (ba);
			\draw   (bb)  edge[-] (bba);
			\draw   (a)  edge[-] (ab);
			\draw   (abb)  edge[-] (a);
			\draw   (ab)  edge[-] (abb);
			\draw   (ab)  edge[-] (aba);
			\draw   (aba)  edge[-] (abab);
			\draw   (ababb)  edge[-] (aba);
			\draw   (ababb)  edge[-] (abab);
			\draw   (abb)  edge[-] (abba);
			\draw   (abba)  edge[-] (abbab);
			\draw   (abba)  edge[-] (abbabb);
			\draw   (bba)  edge[-] (bbabb);
			\draw   (bba)  edge[-] (bbab);
			\draw   (ba)  edge[-] (bab);
			\draw   (babb)  edge[-] (bab);
			\draw   (abbab)  edge[-] (abbabb);
			\draw   (ba)  edge[-] (babb);
			\draw   (bbabb)  edge[-] (bbab);
			\draw   (abbab)  edge[-] (abbabb);
	\end{tikzpicture}
	\]
	\end{frame}
	
	\begin{frame}[fragile]{Ejemplo.}
		La descomposición en un árbol $(T,f)$ la representamos de la siguiente manera:
	\[
			\begin{tikzpicture}[scale=0.8, V/.style = {circle, draw,align= center, minimum size=0.5cm,
			minimum size=3em,inner sep=2,
			fill=applegreen!15,font=\scriptsize	},fill fraction/.style={path picture={
				\fill[#1] 
				(path picture bounding box.south) rectangle
				(path picture bounding box.north west);
		}},
		fill fraction/.default=gray!50
		]
		\node[align=center] at (2, -6) {};
		\begin{scope}[nodes=V,xshift=4.5cm, yshift=-4cm]
			\node (1) at (0,0)  {$\{ b,b^2,1 \}$};
			\node (2) at (2,0)  {$\{1\}$};
			\node (3) at (4,0)  {$\{ 1,a \}$};
			\node (4) at (-2,3)     {$\{ b,ba \}$};
			\node (5) at (-2,-3)   {$\{b^2,b^2a\}$};
			\node (6) at (6,0)  {$\{ a,ab,ab^2 \}$};
			
			\node (9) at (8,3)  {$\{ab,aba\}$};
			\node (10) at (8,-3)  {$\{ ab^2, ab^2a \}$};
		\end{scope}
		
		\node[above right=.45cm of 9] (p_9) {$\dots$};
		\node[below right=.45cm of 10] (p_10) {$\dots$};
		\node[above left=.45cm of 4] (p_4) {$\dots$};
		\node[below left=.45cm of 5] (p_5) {$\dots$};

		\foreach \x in {4,5,9,10}
			\draw (\x) edge[-] (p_\x);


		\draw (1) edge[-] (2);
		\draw (2) edge[-] (3);
		\draw (1) edge[-] (5);
		\draw (1) edge[-] (4);
		\draw (3) edge[-] (6);
		\draw (6) edge[-] (9);
		\draw (6) edge[-] (10);
	\end{tikzpicture}
	\]
	\end{frame}

	\begin{frame}[fragile]{Lema 1.}
		\lemacajitados{1}{Sea $G$ grupo finitamente generado por $X$,
		$\Gamma$ el grafo de Cayley y $(T,f)$ una descomposición en un árbol.
		Entonces si existe $M \in \NN$ tal que para todo $t \in V(T)$ vale que:
		\[
			\diam{f(t)} \le M 
		\]
		luego $\Gamma$ tiene treewidth finito.}

		\textbf{Idea de la demostración:}
		\pause 
		Basta notar que el grafo de Cayley tiene grado acotado por $2|X|$ y así para todo $t \in V(T)$ vale que:
		\[
			|f(t)| \le (2|X|)^{M} < \infty
		\]
		por lo que $\Gamma$ tiene treewidth finito.
		\qed 

	\end{frame}

	\begin{frame}[fragile]{Lema 2.}
		\lemacajitados{2}{Sea $G$ un grupo generado por $X$ tal que $\WP{G}{X}$ es independiente de contexto. 
		Sea $\cG = (V,X \cup X^{-1},P,S)$ una gramática tal que $L(\cG) = \WP{G}{X}$. 
		Sea $A \in V$ luego si definimos: 
		\[
			L_{A} = \{  w \in (X \cup X^{-1})^{*} \mid  A \deriva w \}
		\]
		Entonces para todas $w_{1},w_{2} \in L_{A}$ vale que
		$w_{1} = w_{2}$ en $G$}
		\textbf{Idea de la demostración:}
		\pause 
		Sea una derivación 
		\[
			S \deriva uAv
		\]
		
		luego usando que $w_{1},w_{2} \in L_{A}$ obtenemos que $S \deriva uAv \deriva uw_{1}v $ y que $S \deriva uAv \deriva uw_{2}v$ lo que implica que $w_{1} = w_{2}$ en $G$
		\qed
	\end{frame}

	\begin{frame}[fragile]{Forma normal de Chomsky.}
		\defcajita{
			Una gramática $\gramatica$ independiente de contexto está en \textbf{forma normal de Chomsky} si las producciones son de este tipo: 
			\begin{enumerate}
				\item $A \to BC$ donde $A\in V$ y $B,C \in V \setminus \{ S \}$. 
				\item $A \to a$ donde $A \in V, a \in \Sigma$. 
				\item $S \to \epsilon$  
			\end{enumerate}}
		\bigskip 
	\end{frame}

	\begin{frame}[fragile]{Forma normal de Chomsky.}
		\defcajita{
			Una gramática $\gramatica$ independiente de contexto está en \textbf{forma normal de Chomsky} si las producciones son de este tipo:  
			\begin{enumerate}
				\item $A \to BC$ donde $A\in V$ y $B,C \in V \setminus \{ S \}$.  
				\item $A \to a$ donde $A \in V, a \in \Sigma$.  
				\item $S \to \epsilon$  
			\end{enumerate}}

		\teocajita{
			Sea $\cG$ una gramática \ic{} entonces existe una gramática \ic{} en forma normal de Chosmky $\cG'$ tal que $L(\cG) = L(\cG')$.}
	\end{frame}

	\begin{frame}[fragile]{Árbol de derivación.}
		Sea $\cG$ una gramática en forma normal de Chomsky luego si $w \in L(\cG)$ y $\deri = S \deriva w$ es una derivación le asociamos un árbol binario $T(\deri)$ que denominamos el \textbf{árbol de derivación}.
		\pause 


		\textbf{Ejemplo}.

		Sea $\gramatica$ tal que $V = \{ S,T,U \}$, $\Sigma = \{ a, b\}$ y las reglas son:
		$P = \begin{cases}
								S \to TU   \\
								A \to  UT \mid a	\\
								B \to TU \mid UU \mid b 	\\
		\end{cases}$
		\pause 


		Luego tenemos la siguiente derivación para la palabra $bbabb \in L(\cG)$.
		\[
			\deri = S \to TU \to UTU \to bTU \to bUTB \to bbTU \to bbaU \to bbaUU \to bbabU \to bbabb.
		\]
	\end{frame}

	\begin{frame}[fragile]{Árbol de derivación.}
		El árbol de la derivación $\deri = S \to TU \to UTU \to bTU \to bUTB \to bbTU \to bbaU \to bbaUU \to bbabU \to bbabb.$
		\pause 
		\[
			\begin{tikzpicture}[node distance=0.4cm]
				\node (s) {$S$};
				\node[below left=0.3cm and 3.2cm of  s] (0) {$T$};
				\node[below right=0.3cm and 3.2cm of  s] (1) {$U$};
				\node[below left=0.3cm and 1.2cm of  0] (00) {$U$};
				\node[below right=0.3cm and 1.2cm of  0] (01) {$T$};
				\node[below=1.2cm of  00] (000) {$b$};
				\node[below left=0.3cm and 1.2cm of  01] (010) {$U$};
				\node[below right=0.3cm and 1.2cm of  01] (011) {$T$};
				\node[below= 1.2cm of  010] (0100) {$b$};
				\node[below= 1.2cm of  011] (0110) {$a$};
				\node[below left=0.3cm and 1.2cm of  1] (10) {$U$};
				\node[below right=0.3cm and 1.2cm of  1] (11) {$U$};
				\node[below=1.2cm of  10] (100) {$b$};
				\node[below=1.2cm of  11] (110) {$b$};

				\draw (s) -- (0);
				\draw (s) -- (1);
				\draw (0) -- (00);
				\draw (0) -- (01);
				\draw (1) -- (10);
				\draw (1) -- (11);
				\draw (00) -- (000);
				\draw (01) -- (010);
				\draw (01) -- (011);
				\draw (010) -- (0100);
				\draw (011) -- (0110);
				\draw (10) -- (100);
				\draw (11) -- (110);

			\end{tikzpicture}
		\]
		\pause 
		\begin{itemize}
			\item Las etiquetas de las hojas del árbol forman la palabra que derivamos. \pause 
			\item Los vértices que no son hojas tienen de etiqueta una variable de la gramática.
		\end{itemize}
	\end{frame}

	\begin{frame}[fragile]{Árbol de derivación.}
		El árbol de la derivación $\deri = S \to TU \to UTU \to bTU \to bUTB \to bbTU \to bbaU \to bbaUU \to bbabU \to bbabb.$
		\[
			\begin{tikzpicture}[node distance=0.4cm]
				\node (s) {$S$};
				\node[below left=0.3cm and 3.2cm of  s] (0) {\alert{$T$}};
				\node[below right=0.3cm and 3.2cm of  s] (1) {$U$};
				\node[below left=0.3cm and 1.2cm of  0] (00) {$U$};
				\node[below right=0.3cm and 1.2cm of  0] (01) {$T$};
				\node[below=1.2cm of  00] (000) {\alert{$b$}};
				\node[below left=0.3cm and 1.2cm of  01] (010) {$U$};
				\node[below right=0.3cm and 1.2cm of  01] (011) {$T$};
				\node[below= 1.2cm of  010] (0100) {\alert{$b$}};
				\node[below= 1.2cm of  011] (0110) {\alert{$a$}};
				\node[below left=0.3cm and 1.2cm of  1] (10) {$U$};
				\node[below right=0.3cm and 1.2cm of  1] (11) {$U$};
				\node[below=1.2cm of  10] (100) {$b$};
				\node[below=1.2cm of  11] (110) {$b$};

				\draw (s) -- (0);
				\draw (s) -- (1);
				\draw (0) -- (00);
				\draw (0) -- (01);
				\draw (1) -- (10);
				\draw (1) -- (11);
				\draw (00) -- (000);
				\draw (01) -- (010);
				\draw (01) -- (011);
				\draw (010) -- (0100);
				\draw (011) -- (0110);
				\draw (10) -- (100);
				\draw (11) -- (110);

			\end{tikzpicture}
		\]

		\begin{itemize}
			\item Las etiquetas de las hojas del árbol forman la palabra que derivamos.
			\item Los vértices que no son hojas tienen de etiqueta una variable de la gramática.
			\item Diremos que el vértice  con etiqueta \alert{T} produce la subpalabra \alert{bba}.
		\end{itemize}
	\end{frame}
	

	\begin{frame}[fragile]{Un lema para forma normal de Chomsky.}
		\lemacajita{
		Sea $\gramatica$ una gramática independiente de contexto en forma normal de Chomsky.
		Sea $uvw \in L(\cG)$ y una derivación $\deri = S \tto uvw$.
		Luego existe un vértice que es el más grande entre aquellos con la propiedad de producir una palabra que contiene a $v$.
		}
		\pause 
		\hspace{0.18 in}Continuando el ejemplo anterior, sean 
				$u = 1, v = bb, w=abb$.
		\vspace{-0.25in}
		\begin{columns}
			\begin{column}{0.03 \textwidth}
				
			\end{column}
			\begin{column}{0.4 \textwidth}
				
				Entonces el vértice con etiqueta \alert{T} es el vértice más grande que produce una palabra que contiene a \blue{$v$}.		
			\end{column}

			\begin{column}{0.57 \textwidth}
				\begin{center}
					\begin{adjustbox}{width = 1.05 \textwidth}
						\begin{tikzpicture}[node distance=0.2cm]
							\node (s) {$S$};
							\node[below left=0.3cm and 3.2cm of  s] (0) {\alert{$T$}};
							\node[below right=0.3cm and 3.2cm of  s] (1) {$U$};
							\node[below left=0.3cm and 1.2cm of  0] (00) {$U$};
							\node[below right=0.3cm and 1.2cm of  0] (01) {$T$};
							\node[below=1.2cm of  00] (000) {\blue{$b$}};
							\node[below left=0.3cm and 1.2cm of  01] (010) {$U$};
							\node[below right=0.3cm and 1.2cm of  01] (011) {$T$};
							\node[below= 1.2cm of  010] (0100) {\blue{$b$}};
							\node[below= 1.2cm of  011] (0110) {$a$};
							\node[below left=0.3cm and 1.2cm of  1] (10) {$U$};
							\node[below right=0.3cm and 1.2cm of  1] (11) {$U$};
							\node[below=1.2cm of  10] (100) {$b$};
							\node[below=1.2cm of  11] (110) {$b$};
			
							\draw (s) -- (0);
							\draw (s) -- (1);
							\draw (0) -- (00);
							\draw (0) -- (01);
							\draw (1) -- (10);
							\draw (1) -- (11);
							\draw (00) -- (000);
							\draw (01) -- (010);
							\draw (01) -- (011);
							\draw (010) -- (0100);
							\draw (011) -- (0110);
							\draw (10) -- (100);
							\draw (11) -- (110);
			
						\end{tikzpicture}
					\end{adjustbox}
				\end{center}	
			\end{column}
		\end{columns}
	\end{frame}

	\begin{frame}[fragile]{Lema 3.}
		\lemacajitados{3}{Sea $\gramatica$ una gramática independiente de contexto en forma normal de Chomsky.
		Sea $uvw \in L(\cG)$ de modo que $|v| \ge 2$ y una derivación $\deri = S \tto uvw$. 
		Entonces existe una variable $A$ tal que:
		\[
				\deri = S \tto \delta A \eta \to \delta BC \eta \tto u'v'v''w'
		\]
			donde:
			\begin{itemize}
				\item $u'v'v''w' = uvw$. 
				% \item La palabra $v'v''$ es la palabra más chica que contiene a $v$ como subpalabra que deriva de una variable.
				\item $A \to BC \tto v'v''$, $B \tto v'$ y $C \tto v''$. 
				\item $\delta \tto u'$ y $\eta \tto w'$. 
				\item $v$ está contenida en $v'v''$ como subpalabra y $v$ no está contenido en $v'$ ni en $v''$ como subpalabra. 
				\item $|u| \ge |u'|$ y $|w| \ge |w'|$. 
			\end{itemize} 
		}
	\end{frame}

	\begin{frame}[fragile]{Lema 3.}
		\begin{center}
			\begin{adjustbox}{width = 0.65\textwidth}
				\begin{tabular}{|c|c|c|c|c|c|}
					\hline 
					\multicolumn{6}{|c|}{$S$} \\
					\hline
					$\delta$ &  \multicolumn{4}{|c|}{$A$} & $\eta$ \\
					\hline 
					$\delta$ & \multicolumn{1}{|c}{} & \multicolumn{1}{c|}{$B$} & \multicolumn{1}{|c}{$C$} & \multicolumn{1}{c|}{} & $\eta$ \\
					\hline
					$u'$ & \multicolumn{2}{|c|}{$v'$} & \multicolumn{2}{|c|}{$v''$} & $w'$ \\
					\hline 
					\multicolumn{2}{|c|}{$u$} & \multicolumn{2}{|c|}{$v$} &  \multicolumn{2}{|c|}{$w$}\\
					\hline 
				\end{tabular}
			\end{adjustbox}
		\end{center}
	\end{frame}

	\begin{frame}[fragile]{Teorema de Muller-Schupp.}
		
		\[	
			\begin{tikzpicture}{scale = 0.75}
				\path 
				(0,0) node(a) [rectangle,draw] {$G$ es virtualmente libre
				}
				(5,-3) node(b) [rectangle,draw, text width=4.2cm, align=center] {$G$ es isomorfo a $\pi_{1}(\cG,P)$ para $\cG$ un grafo de grupos finito con grupos finitos.}
				(0,-6) node(c) [rectangle,draw] {El grafo de Cayley $\Gamma$ tiene treewidth finito}
				(-5,-3) node (d) [rectangle,draw] {$G$ es independiente de contexto};
				\draw   
				(d) edge[<-,line width=1.0pt,""] (a) 
				(c) edge[<-,line width=1.0pt, cadmiumred, ""] (d)
				(b) edge[<-,line width=1.0pt,""] (c)
				(a)  edge[<-,line width=1.0pt,""] (b);
			\end{tikzpicture}
		\]
	\end{frame}


	\begin{frame}[fragile]{Teorema de Muller--Schupp.}
		\teocajita{Sea $G$ un grupo finitamente generado por $X$ luego $G$ es un grupo independiente de contexto entonces su  grafo de Cayley $\Gamma$ tiene treewidth finito.}
		\pause 
		\textbf{Idea de la demostración.}
		\begin{itemize}
			\item Vamos a probar que la descomposición en un árbol $(T,f)$ anteriormente construida para un grafo de Cayley arbitrario tiene anchura finito.
			\pause 
			\item Por el \textbf{lema 1} nos alcanza con ver que existe $M \in \NN$ tal que:
			\[
				\diam{f(t)} \le M 
			\]  

		\end{itemize}
	\end{frame}

	\begin{frame}[fragile]{Teorema de Muller--Schupp.}
		\begin{itemize}
			\item Queremos encontrar $M \in \NN$ tal que para todo $n \in \NN$, $C \in V_{n}$
			y para todos $g,h \in \beta C$ se verifique:
			\[
				d(g,h) \le M.
			\]
			\pause 
			\item Por ser $\WP{G}{X}$ un lenguaje independiente de contexto tenemos una gramática $\cG = (V,X \cup X^{-1}, P, S)$ en forma normal de Chomsky 
			tal que $L(\cG) = \WP{G}{X}$.
			\pause 
			Para cada $A \in V$ definimos 
			\[
				k_{A} = \underset{w \in L_{A}}{\min} |w|.
			\]
			\pause 
			Sea $k = \underset{A \in V}{\max} \ k_{A}$.
			\pause 
			\item Proponemos $\alert{M = 3k}$.
		\end{itemize}
	\end{frame}

	\begin{frame}[fragile]{Teorema de Muller--Schupp.}
		\begin{columns}
			\begin{column}{0.65 \textwidth}
				\onslide<1->
				\begin{center}
					\begin{adjustbox}{width = 
						0.65\textwidth}
						\begin{tikzpicture}[font=\sffamily]
								\path[] (-1,2) coordinate[cadmiumred] (A) (8,3) coordinate (B) (8,-8) coordinate (C);
								\draw[
									thick,path picture={
									\foreach \X in {A,B,C}
									{\draw[line width=0.4pt] (\X) circle (0);}}] (A) node[label={[font=\large]},left]{{\Huge $g$}} to[bend left=22] node (B) {}
								(B) node[above right]{\Huge $h$} to[bend right=10] node (C){}
								(C) node[below]{\Huge $1$} to[bend right=8] node (A) {} cycle;
								% \draw[line width = 0.8pt] (y) to (z) ;
								% \draw[line width = 0.8pt] (z) to (x) ;
								% \draw[line width = 0.8pt] (y) to (x) ;
								% \node[label=left:B] at (barycentric cs:x=1,y=1) {};
								% \node[label=right:C] at (barycentric cs:y=1,z=1) {};
								% \node[label=below:A] at (barycentric cs:x=1,z=1) {};
								\onslide<2->
								{\node[label=right:{\Huge{$\alpha$}}] at (2,-2) {};}
								\onslide<4->
								{\node[label=right:{\Huge{$\tau$}}] at (2.4,3.67) {};}
								\onslide<3->{
								\node[label=right:{\Huge{$\gamma$}}] at (7.5,-1.5) {};}

								\draw [very thick, cadmiumred, label= $B_{n}(1)$] (-3,2.25)[bend left=10]  to (10,2.5);
								\node[label=right:{\huge{\textcolor{cadmiumred}{$B_{n}[1]$}}}] at (-3,3) {};
						\end{tikzpicture}
					\end{adjustbox}
				\end{center}
			\end{column}
			\begin{column}{0.35 \textwidth}
				\begin{itemize}
					\item \onslide<2->{$\alpha$ es una geodésica de $1$ a $g$ con etiqueta $u$}
					\item \onslide<3->{$\gamma$ es una geodésica de $h$ a $1$ con etiqueta $w$}
					\item \onslide<4->{$\tau$ es un camino de $g$ a $h$ dentro de $C \cup \beta C$ con etiqueta $v$. 
					}
					\only<5->{\footnotetext{Podemos tomar $\tau$ de manera que exceptuando sus extremos el camino está contenido en $C$.}}
				\end{itemize}
			\end{column}
		\end{columns}
	\end{frame}

	\begin{frame}[fragile]{Teorema de Muller--Schupp.}
		\begin{columns}
			\begin{column}{0.65 \textwidth}
				\begin{center}
					\begin{adjustbox}{width = 
						0.65\textwidth}
						\begin{tikzpicture}[font=\sffamily]
								\path (-1,2) coordinate (A) (8,3) coordinate (B) (8,-8) coordinate (C);
								\draw[thick,path picture={
									\foreach \X in {A,B,C}
									{\draw[line width=0.4pt] (\X) circle (0);}}] (A) node[label={[font=\large]},left]{{\Huge $g$}} to[bend left=22] node (B) {}
								(B) node[above right]{\Huge $h$} to[bend right=10] node (C){}
								(C) node[below]{\Huge $1$} to[bend right=8] node (A) {} cycle;
								% \draw[line width = 0.8pt] (y) to (z) ;
								% \draw[line width = 0.8pt] (z) to (x) ;
								% \draw[line width = 0.8pt] (y) to (x) ;
								% \node[label=left:B] at (barycentric cs:x=1,y=1) {};
								% \node[label=right:C] at (barycentric cs:y=1,z=1) {};
								% \node[label=below:A] at (barycentric cs:x=1,z=1) {};
								\node[label=right:{\Huge{$\alpha$}}] at (2,-2) {};
								\node[label=right:{\Huge{$\tau$}}] at (2.4,3.67) {};
								\node[label=right:{\Huge{$\gamma$}}] at (7.5,-1.5) {};

						\end{tikzpicture}
					\end{adjustbox}
				\end{center}
			\end{column}
			\begin{column}{0.35 \textwidth}
				Si consideramos el \textbf{ciclo} $\alpha \tau \gamma$ luego las etiquetas del camino forman la palabra $uvw \in \WP{G}{X}$.
				\pause 
				Por lo tanto tenemos una derivación
				\[
				\deri = S \deriva uvw.
				\]
			\end{column}
		\end{columns}
	\end{frame}

	\begin{frame}[fragile]{Teorema de Muller--Schupp.}
		\begin{columns}
			\begin{column}{0.65 \textwidth}
				\begin{center}
					\begin{adjustbox}{width = 
						0.65\textwidth}
						\begin{tikzpicture}[font=\sffamily]
								\path (-1,2) coordinate (A) (8,3) coordinate (B) (8,-8) coordinate (C);
								\draw[thick,path picture={
									\foreach \X in {A,B,C}
									{\draw[line width=0.4pt] (\X) circle (0);}}] (A) node[label={[font=\large]},left]{{\Huge $g$}} to[bend left=22] node (B) {}
								(B) node[above right]{\Huge $h$} to[bend right=10] node (C){}
								(C) node[below]{\Huge $1$} to[bend right=8] node (A) {} cycle;
								% \draw[line width = 0.8pt] (y) to (z) ;
								% \draw[line width = 0.8pt] (z) to (x) ;
								% \draw[line width = 0.8pt] (y) to (x) ;
								% \node[label=left:B] at (barycentric cs:x=1,y=1) {};
								% \node[label=right:C] at (barycentric cs:y=1,z=1) {};
								% \node[label=below:A] at (barycentric cs:x=1,z=1) {};
								 \node[label=right:{\Huge{$\alpha$}}] at (2,-2) {};
								 \node[label=right:{\Huge{$\tau$}}] at (2.4,3.67) {};
								 \node[label=right:{\Huge{$\gamma$}}] at (7.5,-1.5) {};
						\end{tikzpicture}
					\end{adjustbox}
				\end{center}
			\end{column}
			\begin{column}{0.35 \textwidth}
				Por el \textbf{lema 3} podemos escribir 
				\[
					uvw = u'v'v''w'
				\]
				de manera que:
			\end{column}
		\end{columns}
	\end{frame}

	\begin{frame}[fragile]{Teorema de Muller--Schupp.}
		\begin{columns}
			\begin{column}{0.65 \textwidth}
					\begin{center}
						\begin{adjustbox}{width = 
							0.65\textwidth}
							\begin{tikzpicture}[font=\sffamily]
									\path (-1,2) coordinate (A) (8,3) coordinate (B) (8,-8) coordinate (C);
									\draw[thick,path picture={
										\foreach \X in {A,B,C}
										{\draw[line width=0.4pt] (\X) circle (0);}}] (A) node[left]{\huge $g$} to[bend left=20] node (y) [draw,circle,fill = black,pos=0.82,minimum size=.1cm, inner sep=0pt,label= above:\huge y] {}
									(B) node[above right]{\huge $h$} to[bend right=10] node (z) [draw,circle,fill = black,pos=0.62,minimum size=.1cm, inner sep=0pt,label= right:\huge z] {}
									(C) node[below]{\huge $1$} to[bend right=8] node (x) [draw,circle,fill = black,pos=0.72,minimum size=.1cm, inner sep=0pt,label= left:\huge x] {} cycle;
									% \draw[line width = 0.8pt] (y) to (z) ;
									% \draw[line width = 0.8pt] (z) to (x) ;
									% \draw[line width = 0.8pt] (y) to (x) ;
									% \node[label=left:B] at (barycentric cs:x=1,y=1) {};
									% \node[label=right:C] at (barycentric cs:y=1,z=1) {};
									% \node[label=below:A] at (barycentric cs:x=1,z=1) {};
									% \node[label=right:{\huge{$\alpha$}}] at (2,-2) {};
									% \node[label=right:{\huge{$\tau$}}] at (2.4,3.67) {};
									% \node[label=right:{\huge{$\gamma$}}] at (7.5,-1.5) {};
	
									%\draw [very thick, cadmiumred, label= $B_{n}(1)$] (-3,2.25)[bend left=10]  to (10,2.5);
									%\node[label=right:{\huge{\textcolor{cadmiumred}{$B_{n}(1)$}}}] at (-3,3) {};
									\draw[ultra thick,cadmiumred] (C) to[bend right = 6.25] (x);
							\end{tikzpicture}
						\end{adjustbox}
					\end{center}
				\end{column}
			\begin{column}{0.35 \textwidth}
				Por el \textbf{lema 3} podemos escribir 
				\[
					uvw = u'v'v''w'
				\]
				de manera que:
				\begin{itemize}
					\item Al leer la etiqueta $u'$  desde $1$ llegamos a un vértice $x$.
				\end{itemize}
			\end{column}
		\end{columns}
	\end{frame}

	\begin{frame}[fragile]{Teorema de Muller--Schupp.}
		\begin{columns}
			\begin{column}{0.65 \textwidth}
					\begin{center}
						\begin{adjustbox}{width = 
							0.65\textwidth}
							\begin{tikzpicture}[font=\sffamily]
									\path (-1,2) coordinate (A) (8,3) coordinate (B) (8,-8) coordinate (C);
									\draw[thick,path picture={
										\foreach \X in {A,B,C}
										{\draw[line width=0.4pt] (\X) circle (0);}}] (A) node[left]{\huge $g$} to[bend left=20] node (y) [draw,circle,fill = black,pos=0.82,minimum size=.1cm, inner sep=0pt,label= above:\huge y] {}
									(B) node[above right]{\huge $h$} to[bend right=10] node (z) [draw,circle,fill = black,pos=0.62,minimum size=.1cm, inner sep=0pt,label= right:\huge z] {}
									(C) node[below]{\huge $1$} to[bend right=8] node (x) [draw,circle,fill = black,pos=0.72,minimum size=.1cm, inner sep=0pt,label= left:\huge x] {} cycle;
									% \draw[line width = 0.8pt] (y) to (z) ;
									% \draw[line width = 0.8pt] (z) to (x) ;
									% \draw[line width = 0.8pt] (y) to (x) ;
									% \node[label=left:B] at (barycentric cs:x=1,y=1) {};
									% \node[label=right:C] at (barycentric cs:y=1,z=1) {};
									% \node[label=below:A] at (barycentric cs:x=1,z=1) {};
									% \node[label=right:{\huge{$\alpha$}}] at (2,-2) {};
									% \node[label=right:{\huge{$\tau$}}] at (2.4,3.67) {};
									% \node[label=right:{\huge{$\gamma$}}] at (7.5,-1.5) {};
	
									%\draw [very thick, cadmiumred, label= $B_{n}(1)$] (-3,2.25)[bend left=10]  to (10,2.5);
									%\node[label=right:{\huge{\textcolor{cadmiumred}{$B_{n}(1)$}}}] at (-3,3) {};
									\draw[ultra thick,cadmiumred] (x) to[bend right = 3.25] (A);
									\draw[ultra thick,cadmiumred] (A) to[bend left = 16.25] (y);
							\end{tikzpicture}
						\end{adjustbox}
					\end{center}
				\end{column}
			\begin{column}{0.35 \textwidth}
				Por el \textbf{lema 3} podemos escribir 
				\[
					uvw = u'v'v''w'
				\]
				de manera que:
				\begin{itemize}
					\item Al leer la etiqueta $u'$  desde $1$ llegamos a un vértice $x$.
					\item Al leer la etiqueta $v'$ desde $x$ llegamos a un vértice $y$. 
				\end{itemize}
			\end{column}
		\end{columns}
	\end{frame}

	\begin{frame}[fragile]{Teorema de Muller--Schupp.}
		\begin{columns}
			\begin{column}{0.65 \textwidth}
					\begin{center}
						\begin{adjustbox}{width = 
							0.65\textwidth}
							\begin{tikzpicture}[font=\sffamily]
									\path (-1,2) coordinate (A) (8,3) coordinate (B) (8,-8) coordinate (C);
									\draw[thick,path picture={
										\foreach \X in {A,B,C}
										{\draw[line width=0.4pt] (\X) circle (0);}}] (A) node[left]{\huge $g$} to[bend left=20] node (y) [draw,circle,fill = black,pos=0.82,minimum size=.1cm, inner sep=0pt,label= above:\huge y] {}
									(B) node[above right]{\huge $h$} to[bend right=10] node (z) [draw,circle,fill = black,pos=0.62,minimum size=.1cm, inner sep=0pt,label= right:\huge z] {}
									(C) node[below]{\huge $1$} to[bend right=8] node (x) [draw,circle,fill = black,pos=0.72,minimum size=.1cm, inner sep=0pt,label= left:\huge x] {} cycle;
									% \draw[line width = 0.8pt] (y) to (z) ;
									% \draw[line width = 0.8pt] (z) to (x) ;
									% \draw[line width = 0.8pt] (y) to (x) ;
									% \node[label=left:B] at (barycentric cs:x=1,y=1) {};
									% \node[label=right:C] at (barycentric cs:y=1,z=1) {};
									% \node[label=below:A] at (barycentric cs:x=1,z=1) {};
									% \node[label=right:{\huge{$\alpha$}}] at (2,-2) {};
									% \node[label=right:{\huge{$\tau$}}] at (2.4,3.67) {};
									% \node[label=right:{\huge{$\gamma$}}] at (7.5,-1.5) {};
	
									%\draw [very thick, cadmiumred, label= $B_{n}(1)$] (-3,2.25)[bend left=10]  to (10,2.5);
									%\node[label=right:{\huge{\textcolor{cadmiumred}{$B_{n}(1)$}}}] at (-3,3) {};
									\draw[ultra thick,cadmiumred] (B) to[bend right = 4.55] (z);
									\draw[ultra thick,cadmiumred] (y) to[bend left = 6.25] (B);
							\end{tikzpicture}
						\end{adjustbox}
					\end{center}
				\end{column}
			\begin{column}{0.35 \textwidth}
				Por el \textbf{lema 3} podemos escribir 
				\[
					uvw = u'v'v''w'
				\]
				de manera que:
				\begin{itemize}
					\item Al leer la etiqueta $u'$  desde $1$ llegamos a un vértice $x$.
					\item Al leer la etiqueta $v'$ desde $x$ llegamos a un vértice $y$. 
					\item Al leer la etiqueta $v''$ desde $y$ llegamos a un vértice $z$.
				\end{itemize}
			\end{column}
		\end{columns}
	\end{frame}

	\begin{frame}[fragile]{Teorema de Muller--Schupp.}
		\begin{columns}
			\begin{column}{0.65 \textwidth}
					\begin{center}
						\begin{adjustbox}{width = 
							0.65\textwidth}
							\begin{tikzpicture}[font=\sffamily]
									\path (-1,2) coordinate (A) (8,3) coordinate (B) (8,-8) coordinate (C);
									\draw[thick,path picture={
										\foreach \X in {A,B,C}
										{\draw[line width=0.4pt] (\X) circle (0);}}] (A) node[left]{\huge $g$} to[bend left=20] node (y) [draw,circle,fill = black,pos=0.82,minimum size=.1cm, inner sep=0pt,label= above:\huge y] {}
									(B) node[above right]{\huge $h$} to[bend right=10] node (z) [draw,circle,fill = black,pos=0.62,minimum size=.1cm, inner sep=0pt,label= right:\huge z] {}
									(C) node[below]{\huge $1$} to[bend right=8] node (x) [draw,circle,fill = black,pos=0.72,minimum size=.1cm, inner sep=0pt,label= left:\huge x] {} cycle;
									% \draw[line width = 0.8pt] (y) to (z) ;
									% \draw[line width = 0.8pt] (z) to (x) ;
									% \draw[line width = 0.8pt] (y) to (x) ;
									% \node[label=left:B] at (barycentric cs:x=1,y=1) {};
									% \node[label=right:C] at (barycentric cs:y=1,z=1) {};
									% \node[label=below:A] at (barycentric cs:x=1,z=1) {};
									% \node[label=right:{\huge{$\alpha$}}] at (2,-2) {};
									% \node[label=right:{\huge{$\tau$}}] at (2.4,3.67) {};
									% \node[label=right:{\huge{$\gamma$}}] at (7.5,-1.5) {};
	
									%\draw [very thick, cadmiumred, label= $B_{n}(1)$] (-3,2.25)[bend left=10]  to (10,2.5);
									%\node[label=right:{\huge{\textcolor{cadmiumred}{$B_{n}(1)$}}}] at (-3,3) {};
									\draw[ultra thick,cadmiumred] (z) to[bend right = 4.55] (C);
							\end{tikzpicture}
						\end{adjustbox}
					\end{center}
				\end{column}
			\begin{column}{0.35 \textwidth}
				Por el \textbf{lema 3} podemos escribir 
				\[
					uvw = u'v'v''w'
				\]
				de manera que:
				\begin{itemize}
					\item Al leer la etiqueta $u'$  desde $1$ llegamos a un vértice $x$.
					\item Al leer la etiqueta $v'$ desde $x$ llegamos a un vértice $y$. 
					\item Al leer la etiqueta $v''$ desde $y$ llegamos a un vértice $z$.
					\item Al leer la etiqueta $w'$ desde $z$ volvemos a $1$.
				\end{itemize}
			\end{column}
		\end{columns}
	\end{frame}

	\begin{frame}[fragile]{Teorema de Muller--Schupp.}
		\begin{columns}
			\begin{column}{0.65 \textwidth}
					\begin{center}
						\begin{adjustbox}{width = 
							0.65\textwidth}
							\begin{tikzpicture}[font=\sffamily]
									\path (-1,2) coordinate (A) (8,3) coordinate (B) (8,-8) coordinate (C);
									\draw[thick,path picture={
										\foreach \X in {A,B,C}
										{\draw[line width=0.4pt] (\X) circle (0);}}] (A) node[left]{\huge $g$} to[bend left=20] node (y) [draw,circle,fill = black,pos=0.82,minimum size=.1cm, inner sep=0pt,label= above:\huge y] {}
									(B) node[above right]{\huge $h$} to[bend right=10] node (z) [draw,circle,fill = black,pos=0.62,minimum size=.1cm, inner sep=0pt,label= right:\huge z] {}
									(C) node[below]{\huge $1$} to[bend right=8] node (x) [draw,circle,fill = black,pos=0.72,minimum size=.1cm, inner sep=0pt,label= left:\huge x] {} cycle;
									% \draw[line width = 0.8pt] (y) to (z) ;
									% \draw[line width = 0.8pt] (z) to (x) ;
									% \draw[line width = 0.8pt] (y) to (x) ;
									% \node[label=left:B] at (barycentric cs:x=1,y=1) {};
									% \node[label=right:C] at (barycentric cs:y=1,z=1) {};
									% \node[label=below:A] at (barycentric cs:x=1,z=1) {};
									% \node[label=right:{\huge{$\alpha$}}] at (2,-2) {};
									% \node[label=right:{\huge{$\tau$}}] at (2.4,3.67) {};
									% \node[label=right:{\huge{$\gamma$}}] at (7.5,-1.5) {};
	
									%\draw [very thick, cadmiumred, label= $B_{n}(1)$] (-3,2.25)[bend left=10]  to (10,2.5);
									%\node[label=right:{\huge{\textcolor{cadmiumred}{$B_{n}(1)$}}}] at (-3,3) {};
									% \draw[ultra thick,cadmiumred] (z) to[bend right = 4.55] (C);
									\onslide<3>{\draw[ultra thick,blue] (x) to[bend right = 3.25] (A);
									\draw[ultra thick,blue] (A) to[bend left = 16.25] (y);
									\node[label=left: \huge \textcolor{cadmiumred}B] at (barycentric cs:x=1,y=1) {};}
									\onslide<3->{
									\draw[line width = 0.8pt, cadmiumred] (y) to (x) ;
									}

									\onslide<4>{\draw[ultra thick,blue] (B) to[bend right = 4.55] (z);
									\draw[ultra thick,blue] (y) to[bend left = 6.25] (B);
									\node[label=right:\huge \textcolor{cadmiumred}C] at (barycentric cs:y=0.9,z=0.8) {};}

									\onslide<4->{
									\draw[line width = 0.8pt, cadmiumred] (y) to (z) ;
									}

									\onslide<5>{\draw[ultra thick,blue] (x) to[bend right = 3.25] (A);
									\draw[ultra thick,blue] (A) to[bend left = 16.25] (y);
									\draw[ultra thick,blue] (B) to[bend right = 4.55] (z);
									\draw[ultra thick,blue] (y) to[bend left = 6.25] (B);}

									\onslide<5->{
									\draw[line width = 0.8pt, cadmiumred] (z) to (x) ;
									\node[label=below:\huge \textcolor{cadmiumred}A] at (barycentric cs:x=1,z=1) {};}
							\end{tikzpicture}
						\end{adjustbox}
					\end{center}
				\end{column}
			\begin{column}{0.35 \textwidth}
				Existen $A,B,C \in V$ que cumplen que:
				\begin{itemize}
					\item $A \tto v'v''$
					\item $B \tto v'$
					\item $C \tto v''$
				\end{itemize}
				
				\onslide<2->{Por la elección de $k$ y usando el \textbf{lema 2}:}
				\begin{itemize}
					\item \onslide<3->{Tenemos que $d(x,y) \le k$.}
					\item \onslide<4->{Similarmente $d(y,z) \le k$}
					\item \onslide<5->{Similarmente $d(x,z) \le k$}
				\end{itemize}
			\end{column}
		\end{columns}
	\end{frame}

	% \begin{frame}[fragile]{Teorema de Muller--Schupp.}
	% 	\begin{columns}
	% 		\begin{column}{0.65 \textwidth}
	% 			\begin{center}
	% 				\begin{adjustbox}{width = 
	% 					0.65\textwidth}
	% 					\begin{tikzpicture}[font=\sffamily]
	% 							\path (-1,2) coordinate (A) (8,3) coordinate (B) (8,-8) coordinate (C);
	% 							\draw[thick,path picture={
	% 								\foreach \X in {A,B,C}
	% 								{\draw[line width=0.4pt] (\X) circle (0);}}] (A) node[left]{\huge $g$} to[bend left=20] node (y) [draw,circle,fill = black,pos=0.82,minimum size=.1cm, inner sep=0pt,label= above:\huge y] {}
	% 							(B) node[above right]{\huge $h$} to[bend right=10] node (z) [draw,circle,fill = black,pos=0.62,minimum size=.1cm, inner sep=0pt,label= right:\huge z] {}
	% 							(C) node[below]{\huge $1$} to[bend right=8] node (x) [draw,circle,fill = black,pos=0.72,minimum size=.1cm, inner sep=0pt,label= left:\huge x] {} cycle;
	% 							\draw[line width = 0.8pt, cadmiumred] (y) to (z) ;
	% 							\draw[line width = 0.8pt, cadmiumred] (z) to (x) ;
	% 							\draw[line width = 0.8pt, cadmiumred] (y) to (x) ;
	% 							\node[label=left: \huge \textcolor{cadmiumred}B] at (barycentric cs:x=1,y=1) {};
	% 							\node[label=right:\huge \textcolor{cadmiumred}C] at (barycentric cs:y=0.9,z=0.8) {};
	% 							\node[label=below:\huge \textcolor{cadmiumred}A] at (barycentric cs:x=1,z=1) {};
	% 							\node[label=right:{\huge{$\alpha$}}] at (2,-2) {};
	% 							\node[label=right:{\huge{$\tau$}}] at (2.4,3.67) {};
	% 							\node[label=right:{\huge{$\gamma$}}] at (7.5,-1.5) {};

	% 							%\draw [very thick, cadmiumred, label= $B_{n}(1)$] (-3,2.25)[bend left=10]  to (10,2.5);
	% 							%\node[label=right:{\huge{\textcolor{cadmiumred}{$B_{n}(1)$}}}] at (-3,3) {};
	% 					\end{tikzpicture}
	% 				\end{adjustbox}
	% 			\end{center}
	% 		\end{column}
	% 		\begin{column}{0.35 \textwidth}
	% 			\begin{itemize}
	% 			\item Como $A \deriva v'v''$ luego $v'v'' \in L_{A}$  y así $d(x,z) \le k$.
	% 			\item Como $B \deriva v'$ luego $v' \in L_{B}$  y así $d(x,y) \le k$.
	% 			\item Como $C \deriva v''$ luego $v'' \in L_{C}$  y así $d(y,z) \le k$.
	% 			\end{itemize}
	% 		\end{column}
	% 	\end{columns}
	% \end{frame}

	\begin{frame}[fragile]{Teorema de Muller--Schupp.}
		\begin{columns}
			\begin{column}{0.65 \textwidth}
				\begin{center}
					\begin{adjustbox}{width = 
						0.65\textwidth}
						\begin{tikzpicture}[font=\sffamily]
								\path (-1,2) coordinate (A) (8,3) coordinate (B) (8,-8) coordinate (C);
								\draw[thick,path picture={
									\foreach \X in {A,B,C}
									{\draw[line width=0.4pt] (\X) circle (0);}}] (A) node[left]{\huge $g$} to[bend left=20] node (y) [draw,circle,fill = black,pos=0.82,minimum size=.1cm, inner sep=0pt,label= above:\huge y] {}
								(B) node[above right]{\huge $h$} to[bend right=10] node (z) [draw,circle,fill = black,pos=0.62,minimum size=.1cm, inner sep=0pt,label= right:\huge z] {}
								(C) node[below]{\huge $1$} to[bend right=8] node (x) [draw,circle,fill = black,pos=0.72,minimum size=.1cm, inner sep=0pt,label= left:\huge x] {} cycle;
								\draw[line width = 0.8pt, cadmiumred] (y) to (z) ;
								\draw[line width = 0.8pt, cadmiumred] (z) to (x) ;
								\draw[line width = 0.8pt, cadmiumred] (y) to (x) ;
								\node[label=left: \huge \textcolor{cadmiumred}B] at (barycentric cs:x=1,y=1) {};
								\node[label=right:\huge \textcolor{cadmiumred}C] at (barycentric cs:y=0.9,z=0.8) {};
								\node[label=below:\huge \textcolor{cadmiumred}A] at (barycentric cs:x=1,z=1) {};
								\node[label=right:{\huge{$\alpha$}}] at (2,-2) {};
								\node[label=right:{\huge{$\tau$}}] at (2.4,3.67) {};
								\node[label=right:{\huge{$\gamma$}}] at (7.5,-1.5) {};

								%\draw [very thick, cadmiumred, label= $B_{n}(1)$] (-3,2.25)[bend left=10]  to (10,2.5);
								%\node[label=right:{\huge{\textcolor{cadmiumred}{$B_{n}(1)$}}}] at (-3,3) {};
						\end{tikzpicture}
					\end{adjustbox}
				\end{center}
			\end{column}
			\begin{column}{0.35 \textwidth}
				\begin{align*}
					d(g,h) &\le d(g,x) + d(x,z) + d(h,z)
				\end{align*}
			\end{column}
		\end{columns}
	\end{frame}
	\begin{frame}[fragile]{Teorema de Muller--Schupp.}
		\begin{columns}
			\begin{column}{0.65 \textwidth}
				\begin{center}
					\begin{adjustbox}{width = 
						0.65\textwidth}
						\begin{tikzpicture}[font=\sffamily]
								\path (-1,2) coordinate (A) (8,3) coordinate (B) (8,-8) coordinate (C);
								\draw[thick,path picture={
									\foreach \X in {A,B,C}
									{\draw[line width=0.4pt] (\X) circle (0);}}] (A) node[left]{\huge $g$} to[bend left=20] node (y) [draw,circle,fill = black,pos=0.82,minimum size=.1cm, inner sep=0pt,label= above:\huge y] {}
								(B) node[above right]{\huge $h$} to[bend right=10] node (z) [draw,circle,fill = black,pos=0.62,minimum size=.1cm, inner sep=0pt,label= right:\huge z] {}
								(C) node[below]{\huge $1$} to[bend right=8] node (x) [draw,circle,fill = black,pos=0.72,minimum size=.1cm, inner sep=0pt,label= left:\huge x] {} cycle;
								\draw[line width = 0.8pt, cadmiumred] (y) to (z) ;
								\draw[line width = 0.8pt, cadmiumred] (z) to (x) ;
								\draw[line width = 0.8pt, cadmiumred] (y) to (x) ;
								\node[label=left: \huge \textcolor{cadmiumred}B] at (barycentric cs:x=1,y=1) {};
								\node[label=right:\huge \textcolor{cadmiumred}C] at (barycentric cs:y=0.9,z=0.8) {};
								\node[label=below:\huge \textcolor{cadmiumred}A] at (barycentric cs:x=1,z=1) {};
								\node[label=right:{\huge{$\alpha$}}] at (2,-2) {};
								\node[label=right:{\huge{$\tau$}}] at (2.4,3.67) {};
								\node[label=right:{\huge{$\gamma$}}] at (7.5,-1.5) {};

								%\draw [very thick, cadmiumred, label= $B_{n}(1)$] (-3,2.25)[bend left=10]  to (10,2.5);
								%\node[label=right:{\huge{\textcolor{cadmiumred}{$B_{n}(1)$}}}] at (-3,3) {};
						\end{tikzpicture}
					\end{adjustbox}
				\end{center}
			\end{column}
			\begin{column}{0.35 \textwidth}
				\pause {Como $\alpha$ y $\gamma$ son geodésicas 
				\begin{itemize}
					\item $d(g,x) = d(1,g) - d(1,x)$  
					\item $d(h,z) = d(1,h) - d(1,z) $
				\end{itemize}}
				\pause {Como $y$ está en el camino $\tau$ que lo tomamos dentro de $C$ luego
				\begin{itemize}
					\item $d(1,y) \ge d(1,g) $
					\item $d(1,y) \ge d(1,h) $
				\end{itemize}
				}
				\pause 
				Entonces: 
				\begin{align*}
					d(g,x) &\le d(1,y) - d(1,x)  \le d(x,y) \\
					d(h,z) &\le d(1,y) - d(1,z) \le d(z,y) \\ 
				\end{align*}
				
			\end{column}
		\end{columns}
	\end{frame}

	\begin{frame}[fragile]{Teorema de Muller--Schupp.}
		\begin{columns}
			\begin{column}{0.65 \textwidth}
				\begin{center}
					\begin{adjustbox}{width = 
						0.65\textwidth}
						\begin{tikzpicture}[font=\sffamily]
								\path (-1,2) coordinate (A) (8,3) coordinate (B) (8,-8) coordinate (C);
								\draw[thick,path picture={
									\foreach \X in {A,B,C}
									{\draw[line width=0.4pt] (\X) circle (0);}}] (A) node[left]{\huge $g$} to[bend left=20] node (y) [draw,circle,fill = black,pos=0.82,minimum size=.1cm, inner sep=0pt,label= above:\huge y] {}
								(B) node[above right]{\huge $h$} to[bend right=10] node (z) [draw,circle,fill = black,pos=0.62,minimum size=.1cm, inner sep=0pt,label= right:\huge z] {}
								(C) node[below]{\huge $1$} to[bend right=8] node (x) [draw,circle,fill = black,pos=0.72,minimum size=.1cm, inner sep=0pt,label= left:\huge x] {} cycle;
								\draw[line width = 0.8pt, cadmiumred] (y) to (z) ;
								\draw[line width = 0.8pt, cadmiumred] (z) to (x) ;
								\draw[line width = 0.8pt, cadmiumred] (y) to (x) ;
								\node[label=left: \huge \textcolor{cadmiumred}B] at (barycentric cs:x=1,y=1) {};
								\node[label=right:\huge \textcolor{cadmiumred}C] at (barycentric cs:y=0.9,z=0.8) {};
								\node[label=below:\huge \textcolor{cadmiumred}A] at (barycentric cs:x=1,z=1) {};
								\node[label=right:{\huge{$\alpha$}}] at (2,-2) {};
								\node[label=right:{\huge{$\tau$}}] at (2.4,3.67) {};
								\node[label=right:{\huge{$\gamma$}}] at (7.5,-1.5) {};

								%\draw [very thick, cadmiumred, label= $B_{n}(1)$] (-3,2.25)[bend left=10]  to (10,2.5);
								%\node[label=right:{\huge{\textcolor{cadmiumred}{$B_{n}(1)$}}}] at (-3,3) {};
						\end{tikzpicture}
					\end{adjustbox}
				\end{center}
			\end{column}
			\begin{column}{0.35 \textwidth}
				Por lo que concluimos que:

				\begin{align*}
					d(g,h) & \le d(g,x) + d(h,x) + d(x,z) \\
					&\le  d(x,y) + d(x,z) + d(z,y) \le 3k.
				\end{align*}
				\qed 
				
			\end{column}
		\end{columns}
	\end{frame}
\end{document}
