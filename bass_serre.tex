\documentclass[tesis.tex]{subfiles}

%\newcommand{\ol}{\overline{}}
\newcommand{\ic}{independiente de contexto }
\newcommand{\APND}{automáta de pila no determinístico }
\newcommand{\APD}{automáta de pila determinístico }
\newcommand{\fg}{grupo finitamente generado }
\newcommand{\fp}{grupo finitamente presentado }
\newcommand{\vl}{virtualmente libre}
\newcommand{\WP}{\text{WP}(G, \Sigma)}
\newcommand{\deriva}{\overset{*}{\Rightarrow_{\cal G}}}
\newcommand{\cG}{ {\cal G} }


\begin{document}

\chapter{Teoría de Bass Serre.}

Introduciremos los resultados y definiciones más importantes de la teoría de Bass Serre. 
Las referencias clásicas que seguiremos son \cite{serre2002trees} y \cite{}.
En esta sección los grafos serán siempre conexos y dirigidos.
La definición que tomaremos, dada originalmente por Gersten, será levemente diferente a la usada en la sección \ref{}.
Dado un grafo $Y= (V(Y),E(Y))$ por cada arista $y \in E(Y)$ tendremos una designación $s(y) = P, t(y) = Q$ si la arista comienza en $P$ y termina en $Q$.
Dada $y \in E(Y)$ denotaremos $\overline y$ la arista con designación opuesta.

\begin{deff}
	Un \blue{grafo de grupos $\cal G$ sobre $Y$} está definido por lo siguiente:
	\begin{enumerate}
		\item Para cada vértice $P \in V(Y)$ tenemos un grupo $G_P$.
		\item  Para cada arista $y \in E(Y)$ tenemos un grupo $G_y$ tal que $G_y \le G_{s(y)}$.
		\item Para todo $y \in E(Y)$ tenemos un isomorfismo de $G_y$ a $G_{\overline y}$ que denotaremos por $a \mapsto a^{\overline y} $ tal que $(a^{\overline y})^y  = a$ para todo $a \in G_y$.
	\end{enumerate}
\end{deff}

En general para la definición de grafo de grupos no es necesario pedir que $G_y \le G_{s(y)}$ sino que alcanza con tomar un monomorfismo. 
Para facilitar la construcción más en adelante supondremos que es un subgrupo.

Dado un grafo $Y = (V(Y), E(Y))$ consideremos el grupo libre $F_{E(Y)}$ generado por las aristas del grafo.

\begin{deff}
	Dado un grafo de grupos $\cal G$ sobre un grafo $Y = (V(Y), E(Y))$ podemos armarnos el siguiente grupo 
	\begin{equation*}
		F({\cal G}) = \ast_{P \in V(Y)} G_P \ast F_{E(Y)} / \langle \langle  R \rangle \rangle
	\end{equation*}
	donde $R = \{  \overline y = y^{-1}, \ ya^yy^{-1} = a^{\overline y} \}$ para $y \in E(Y), a \in G_y$. 
\end{deff}

\begin{obs}
	Las relaciones $R$ por las que dividimos anteriormente se pueden condensar en 
	\[
	R = \{  ya^y \overline y = a^{\overline y}  \}.
	\]
\end{obs}

\begin{deff} \label{def_pi1_arbol}
	Sea un grafo de grupos $\cal G$ sobre $Y$.
	Consideremos $T$ un árbol de expansión del grafo $Y$.
	\footnote{Como suponemos que $Y$ es conexo existen (posiblemente varios) árboles de expansión.}
	Nos armamos el siguiente \blue{grupo fundamental del grafo de grupos}
	\begin{equation*}
		\pi_1({\cal G}, T) = F({\cal G}) / \{ y \  | \ y \in E(T)  \}.
	\end{equation*}
\end{deff}

Esto es que para todas las aristas que aparecen en el árbol de expansión las identificamos con el $1$. 
Esta definición se puede ver que en el caso particular de un grafo coincide con la definición que tenemos del grupo fundamental de un espacio topológico.

\begin{ej}
	Sea un grafo de grupos $\cal G$ sobre un grafo $Y$ tal que $G_y = \{ 1 \}$ para todo $y \in E(Y)$.
	Consideremos una orientación $A$ para las aristas del grafo esto es que para cada par de vértices unido por una arista $y$ tenemos que $y \in A \iff \overline y \notin A$.
	El grupo fundamental $\pi({\cal G}, T)$ está generado por los elementos $G_P, P \in V(Y)$ y los elementos $g_y \in A \setminus (T \cap A)$.
	Esto es que nos queda el siguiente grupo
	\[
	\pi_1({\cal G}, T) = \ast_{P \in V(Y)} G_P \ast F
	\]
	donde $F$ es el grupo libre con base $A \setminus (T \cap A)$.
	Esto es que nos queda $\pi_1({\cal G}, T) \simeq \pi_1(Y) \ast_{P \in V(Y)} G_P$.
\end{ej}

\begin{ej}
	Si el grafo es un segmento en particular él mismo es el árbol de expansión puesto que es un árbol.
	De esta manera si $Y$ es el siguiente segmento... \todo[inline]{Agregar gráfico}
	entonces
	\[
	\pi_1({\cal G}, Y) = G_P \ast_{G_y} G_Q.
	\]
\end{ej}

Podemos dar otra definición de grupo fundamental pero está vez usando caminos así como lo hacemos para el grupo fundamental usual.

\medskip
Dados dos vértices $P,Q \in V(Y)$ denotaremos por $\Pi(P,Q)$ el conjunto de caminos de $P$ a $Q$.
De esta manera nos queda definido
\[
\Pi(P,Q) = \{  y_1 \dots y_k \ | \ s(y_1)=P, \ t(y_k) = Q, \ t(y_i) = s(y_{i+1})  \ \text{para} \ 1 \le i \le k \}
\]
tal que estos son subconjuntos en $F(\cG)$ si miramos a las aristas como elementos del grupo.
En particular podemos tomar $g_0y_1g_1 \dots g_{k-1}y_kg_k \in F(\cG)$ tales que
\[
g_i \in G_{s(y_i)}, \ \ g_k \in G_Q,
\]
A estos subconjuntos los denotaremos $\pi(\cG, P, Q)$. 
En particular para todo $P \in V(Y)$ tenemos que $\pi(\cG,P,P)$ es un subgrupo de $F(\cG)$ esto porque si hacemos el producto entre dos elementos de este subconjunto tenemos que
\begin{equation*}
	(g_0y_1g_1 \dots g_{k-1}y_kg_k ) \circ ( g'_0y'_1g'_1 \dots g'_{k-1}y'_kg'_k) =  g_0y_1g_1 \dots g_{k-1}y_k(g_kg'_0)y'_1g'_1 \dots g'_{k-1}y'_kg'_k
\end{equation*}
donde $g_kg'_0 \in G_P$ por lo tanto su producto está bien definido y en definitiva nos queda otro elemento de $\pi(\cG, P, P)$ tal como queríamos ver.

\todo[inline]{Para ver que es un subgrupo tendría que ver que cumple las relaciones?}
\begin{deff}
	\blue{El grupo fundamental de $\cG$ respecto a un punto de base $P$} se define como $\pi_1(\cG, P) = \pi(\cG,P,P)$.
\end{deff}

Notemos que en particular para esta construcción obtuvimos un subgrupo del grupo $F(\cal G)$ mientras que en el caso de la definición anterior \ref{def_pi1_arbol} obtuvimos un cociente.
Podemos ver que ambas definiciones son equivalentes porque son isomorfos como grupos.

\begin{teo}
	$\pi_1({\cal G}, P)$ es isomorfo con $\pi_1({\cal G}, T)$.
\end{teo}
\begin{proof}
	Veamos que la composición de la inclusión $\iota: \pi_1( \cG ,P)  \to F(\cG)$ con la proyección $\pi: F(\cG) \to \pi_1(\cG, T)$ es un isomorfismo.
	
	Dados $P,Q \in V(Y)$ consideremos la geodésica $\alpha$ que los une sobre $T$.
	Si leemos la sucesión de aristas que recorre esta geodésica tenemos una palabra en $T(P,Q) \in F_{E(Y)}$ de manera que nos define un elemento en el grupo $F(\cG)$.
	Ahora definamos el siguiente morfismo (que depende del árbol de expansión que tomamos):
	
	\begin{align*}
		\tau: & F(\cG) \to \pi_1(\cG,P)  \\
		& \tau(y) = T[P,s(y)]yT[t(y),P] \ \ \text{para } \ y\in E(Y) \\
		& \tau(g) = T[P,Q] g T(Q,P) \ \  \text{para} \ Q \in V(Y), \ g \in G_Q
	\end{align*}
	
	Lo definimos sobre los generadores del producto libre y por como lo tomamos siempre nos devuelve un elemento de $\pi(\cG, P)$.
	Para ver que está bien definido debemos ver que cumple la relación $\tau(\overline y a^y y) = \tau (a^{\overline y})$ para toda arista $y \in E(Y)$.
	Esto vale porque justamente 
	\begin{align*}
		\tau(\overline y a^y y) & = T[P,s(\overline y)]\ol yT[t(\ol y),P] T[P,s(y)] a^y T[t(y),P] T[P,s(\overline y)]yT[t( y),P] \\
		& = T[P,s(\overline y)]  \ol y a^y y T[t( y),P] \\
		& = \tau (a^{\ol y}).
	\end{align*}
	Es un epimorfismo porque en particular todo elemento de $\pi_1(\cG, P)$ es un camino que termina y comienza en $P$ intercalado con elementos de los estabilizadores de los vértices que aparecen en el camino.
	De esta manera como $\tau(y)=1$ para todo $y \in E(T)$, notemos que pasa al cociente y existe $\ol \tau: \pi_{1}(\cG, T) \to \pi_1 (\cG, P)$ epimorfismo.
	
	Para terminar la demostración probemos que componiendo con $\nu$ nos queda la identidad.
	Si hacemos $\nu \circ \ol \tau$ notemos que 
	\begin{equation*}
		\nu \circ \ol \tau (a^y) = a^y, \ \ \nu \circ \ol \tau (y) = y, 
	\end{equation*}
	para $y \in E(Y) \setminus E(T)$ porque justamente estamos dividiendo por estas aristas. 
	Esto porque $T[P,Q]$ es un camino de aristas en el árbol de expansión por lo tanto sobre $\pi_1(\cG, T)$ son la identidad.
\end{proof}

\todo[inline]{Buscar una manera decente de probar este resultado. No es tan trivial como parece.}

\begin{teo}
	Para todo $P \in V(Y)$ vale que $G_P \le \pi_1(\cG)$. 
\end{teo}
\begin{proof}
	Demo no trivial porque no sabemos bien que al pasar al cociente todo siga andando. 
	Diekert supera este problema usando sistemas de reescrituras y eso lo vuelve más liviano.
\end{proof}

\subsection{Árbol de Bass Serre.}

Vamos a construirnos un revestimiento para un grafo de grupos arbitrario. 
Como el revestimiento universal de todo grafo es un árbol obtendremos un resultado similar en este caso.

Dado $\cG$ un grafo de grupos sobre un grafo $Y$ conexo tomemos un árbol $T$ de expansión de este grafo.
Consideremos también $A$ una orientación de las aristas del grafo $Y$.

Queremos construirnos un grafo $\tilde X$ tal que tenga las siguientes propiedades:

\begin{enumerate}
	\item Una acción de $\pi_1(\cG, T)$ en $\tilde X$.
	\item Un morfismo de grafos sobreyectivo $p:\tilde X \to Y$. 
	\item Unas secciones de los vértices de $Y$ en los de $\tilde X$ e idénticamente secciones en las aristas.
\end{enumerate}

A nuestras secciones las vamos a denotar de manera que para todo vértice $P \in Y$ tendremos un $ \tilde P \in V(\tilde X)$.
Mirando la acción de $\pi_1(\cG, T)$ sobre $\tilde X$ nos gustaría que el estabilizador respecto a $\tilde P$ sea el mismo que el estabilizador de $P$ en el grafo de grupos $\cG$. 
Para esto podemos definir directamente para que $\tilde P \simeq \pi_1(\cG, T) / G_P$.
Análogamente para todas las aristas de nuestro grafo queremos que el estabilizador sea isomorfo al subgrupo $G^y_y$ dentro de $G_{s(y)}$.

Usando estas condiciones tenemos que
\begin{equation*}
	V(\tilde X) = \coprod_{P \in V(Y)} \pi_1(\cG, T) / G_P, \ \  E(\tilde X) = \coprod_{y \in E(Y)} \pi_1(\cG, T) / G_y
\end{equation*}

para obtener un grafo nos queda redefinir para tener las aristas con la otra orientación

\medskip

Por la construcción que hicimos nos queda que $\tilde X / \pi_1(\cG, T) \simeq Y$.

\begin{teo}[Serre]
	El grafo $\tilde X$ anteriormente construido es un árbol.
\end{teo}
\begin{proof}
	Demo no trivial, requiere palabras reducidas.
	Ver \cite{serre2002trees}.
\end{proof}


\begin{deff}
	El grafo $\tilde X$ es el \blue{árbol de Bass Serre} del grafo de grupos $\cG$.
\end{deff}

\begin{ej}
	Hacer un ejemplo así como hice antes.
\end{ej}

Dar conexión con la idea topológica de un revestimiento de un grafo..

\subsection{Acciones de grupos sobre árboles.}

Dar definiciones de los morfismos que salen del árbol y los que van a parar al grupo libre del grafo de grupos.



El siguiente resultado nos da la estructura de un grupo que actúa sobre un grafo conexo sin inversiones de aristas. 
Es el resultado central de la teoría de Bass Serre.

\begin{teo}[\cite{serre2002trees}]
	Sea $G$ un grupo actuando en grafo conexo $X$ sin inversiones de aristas y sea $\cal G$ el grafo de grupos asociado sobre $Y = X / G$.
	Consideremos $T$ un árbol de expansión de $Y$, $\psi: \ol X \to X$ y $\varphi: \pi_1({\cal G}, T) \to G$ luego las siguientes afirmaciones son equivalentes.
	\begin{enumerate}[(a)]
		\item el grafo $\ol X$ es un árbol;
		\item el morfismo $\psi: \ol X \to X$ es un isomorfismo de grafos;
		\item el morfismo $\varphi:\pi_1({\cal G}, T) \to G$ es un isomorfismo de grupos.
	\end{enumerate}
\end{teo}


Como corolario obtenemos una caracterización que usaremos varias veces y es como el resultado es más conocido.

\begin{coro}
	Un grupo $G$ es libre si y solo sí $G$ actúa libremente sobre un árbol sin inversiones de aristas. 
\end{coro}
\begin{proof}
	Se sigue fácilmente.
\end{proof}

\begin{coro}
	Dado $G$ un grupo libre todo subgrupo $H \le G$ resulta ser libre.
\end{coro}
\begin{proof}
	Fácil de lo anterior.
\end{proof}

A partir de este teorema podemos obtener un corolario que nos cuantifique qué tan grande puede ser el índice de un subgrupo de un grupo libre.


\begin{coro}\cite{karrass1973finite}
	Sea $\cal G$ un grafo de grupos finito con grupos sobre los vértices finitos.
	Sea $G = \pi_1 ({\cal G}, T)$ y sea $F$ un subgrupo libre de $G$ de índice finito con rango $r(F)$.
	Luego la siguiente ecuación vale:
	
	\begin{equation*}
		\frac{r(F) - 1}{(G:F)} = \sum_{y \in E(Y)} \frac{1}{2 \cdot |G_y|} - \sum_{P \in V(Y)} \frac{1}{|G_P|}.
	\end{equation*}
\end{coro}
\begin{proof}
	\red{sorry}
\end{proof}

\begin{teo}
	Sea $\cal G$ un grafo de grupos sobre un grafo finito $Y$ con grupos \fg en los vértices.
	Entonces $\pi_1(\cal G)$ es \fg si y solo sí todos los grupos de los vértices lo son.
\end{teo}

\begin{proof}
\end{proof}


\begin{teo}
	Sea $\cal G$ un grafo de grupos sobre un grafo finito $Y$ con grupos finitos en los vértices.
	Entonces $\pi_1(\cal G)$ es un grupo \fg virtualmente libre.
\end{teo}
\begin{proof}
	Sea $X$ un conjunto finito tal que $G_P | |X|$ para todo $P \in V(Y)$, esto lo podemos tomar porque el grafo $Y$ es finito.
	
	
	Para cada $P \in V(Y)$ podemos armarnos una acción libre de $G_P$ en $X$. 
	Esto nos da un monomorfismo $G_P \to S(X)$ donde $S(X)$ es el grupo simétrico de $X$.
	A su vez por cada grupo de una arista $G_y$ obtenemos dos acciones libres sobre $X$.
	Por el lema \ref{} podemos elegirnos para cada $y \in E(Y)$ alguna $\varphi_y \in S(X)$ tal que haga conmutar a ...
	
	
	Usando la definición del grupo $F(\cG)$ tenemos un morfismo de grupos $h: F(\cG) \to S(X)$ tal que la restricción a $G_P$ es inyectiva para todo $P \in V(Y)$.
	Si fijamos $P \in V(Y)$ y consideramos $F = \{  g \in \pi_1(G,P) \ | \ h(g) = 1  \}$  entonces obtenemos que $F \cap G_Q = \{ 1 \}$ para todo $Q \in V(Y)$.
	
	Esto nos dice que el grupo $F$ actúa libremente sobre el árbol de Bass Serre de nuestro grafo de grupos $\cG$.
	
	Como actúa libremente sobre un árbol por el resultado \ref{} tenemos que este grupo debe ser libre. 
	Así obtuvimos un subgrupo libre de $\pi_1(\cG, P)$ y queremos ver que tiene índice finito.
	Para eso notemos que al ser $|X| < \infty$ entonces...
	
	Finalmente notemos que por \ref{} el grupo es \fg tal como queríamos ver.
\end{proof}

\section{Productos semi directos.}

En esta sección vamos a refinar los resultados obtenidos anteriormente para ver que los grupos fundamentales de grafos finitos de grupos finitos y por lo tanto los grupos virtualmente libres se pueden meter dentro de un producto semidirecto $F Q$ donde $F$ es un grupo libre y $Q$ es un cociente finito de $G$.
La referencia de esta sección es el paper \cite{} que a su vez se basó en los métodos del paper de Dahmani-Guirardel.

\begin{prop}
	
\end{prop}
\begin{proof}
	\red{ sorry}
\end{proof}

\begin{ej}
	Ejemplo de $SL_2(\ZZ)$.
\end{ej}

\section{Ends de grupos.}

\begin{deff}
	Ends de grupos
\end{deff}

\begin{obs}
	No dependen de las presentaciones.
\end{obs}

\begin{teo}[Hopf]
	La cantidad de ends de un grupo \fg solo puede ser...
\end{teo}
\begin{proof}
\end{proof}

\begin{deff}
	Grupo accesible.
\end{deff}

\subsection{Accesibilidad en terminos de teoría de grafos.}

\begin{deff}
	Def de VK95 usando grafos.
\end{deff}

\begin{teo}
	Probar que son equivalentes con la otra construcción.
\end{teo}
\begin{proof}
	
\end{proof}


\begin{deff}
	Grafos minor excluded
\end{deff}

\begin{teo}[Khukro]
	Recaracterización de virtualmente libre usando menores de grafos.
\end{teo}
\begin{proof}
\end{proof}

\begin{teo}
	Grupo qi a un árbol $\implies$ minor excluded?
\end{teo}
\begin{proof}
\end{proof}









































\end{document}