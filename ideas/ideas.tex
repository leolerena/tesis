\documentclass[12pt]{article}
\usepackage{geometry}\geometry{top=5cm,bottom=2cm,left=3cm,right=3cm}
\usepackage[utf8]{inputenc}
\usepackage[spanish]{babel}



%opening
\usepackage{fancyhdr}
\pagestyle{fancy}
\lhead{Tesis de licenciatura.} % Left Header
\rhead{\thepage} % Right Header

\usepackage{subfiles} % mejor ponerlos al final
\title{Ideas sueltas.}
\date{}
\usepackage{amsmath,amsfonts,amsthm,amssymb,mathtools,sectsty}
\pagenumbering{gobble}
\usepackage{subcaption}
%\usepackage{graphicx}
%\usepackage[pdftex,dvipsnames]{xcolor}
\usepackage{cancel}
\usepackage{graphicx}
\usepackage{marginnote}
\usepackage{mathabx}
\usepackage{float}
\setlength{\marginparwidth}{2cm}

% Tikz y las librerías para automátas
\usepackage{tikz-cd}
\usepackage{tikz}
\usetikzlibrary{arrows,automata}
\usetikzlibrary{babel} %para evitar que se jodan los automatas de tikz
\usetikzlibrary{graphs} 
\usetikzlibrary{calc}
\usetikzlibrary{positioning}
%\usetikzlibrary{shapes.geometric}  % for [ellipse], [diamond], etc

%\usepackage[backend=biber,...]{biblatex} 

%Referencias; me gustaría que backref funcione pero no es importante tampoco.
\usepackage[pagebackref]{hyperref}
% Para modificar el estilo de las referencias
\hypersetup{
	colorlinks,
	linkcolor={astral},
	citecolor={red!70!black},
	urlcolor={red!80!black}
}
\definecolor{astral}{RGB}{46,116,181}
\colorlet{chulo}{blue!70!purple}
\colorlet{rojo}{purple!45!black}
\definecolor{carrotorange}{rgb}{0.93, 0.57, 0.13}
\definecolor{brightcerulean}{rgb}{0.11, 0.67, 0.84}
\definecolor{brightube}{rgb}{0.82, 0.62, 0.91}
\definecolor{cadmiumred}{rgb}{0.89, 0.0, 0.13}
\definecolor{applegreen}{rgb}{0.55, 0.71, 0.0}
\definecolor{aurometalsaurus}{rgb}{0.43, 0.5, 0.5}

%%%%%%%%%%%%%%%%%%%%% ENUMERAR CON COSAS QUE NO SEAN SOLO NÚMEROS %%%%%%%%%%
\usepackage[shortlabels]{enumitem}
\setlist[enumerate]{font=\bfseries}
\usepackage{adjustbox}


%%%%%%%%%%%%%%%%%%%%%%% TÍTULOS DE SECCIONES MÁS FANCIES %%%%%%%%%%%%%%
\usepackage{titlesec}
\setcounter{secnumdepth}{3} % Hasta que profundidad quiero numerar, 4 sería los párrafos.
\titleformat{\section}[block]{\color{astral!50!black}\Large\bfseries\filcenter}{\S\thesection.}{1em}{}
\titleformat{\subsection}[hang]{\color{astral!50!black}\large\bfseries\filcenter}{\S\thesubsection.}{1em}{}

%%%%%%%%%%%%%%%%%%%%%% COLORES PÁRRAFOS Y CAPÍTULOS %%%%%%%%%%%%%%%%%%%%%%%%
%\paragraphfont{\color{astral!70!black}}
\chapterfont{\color{astral!40!black}}
%\subsectionfont{\color{astral!60!black} }
%\sectionfont{\color{astral!50!black} }

\usepackage{mathpazo}
\usepackage{amssymb}
%\usepackage{thmtools}

%Esto sirve para armar grafos de Cayley de una manera más copada.
\usetikzlibrary{lindenmayersystems,arrows.meta}
\pgfdeclarelindenmayersystem{cayley}{
	\rule{G->G-G+++G--G}
	\symbol{R}{
		\pgflsystemstep=0.5\pgflsystemstep
	}

}

\usepackage[framemethod=tikz]{mdframed}

%%%%%%%%%%%%%  TEOREMAS  %%%%%%%%%%%%%%%%%
\theoremstyle{plain} %% el estilo clásico
\newtheorem{teo}{\color{rojo}{ { Teorema}}}[section]
\newtheorem{prop}[teo]{\color{rojo} {Proposición}}
\newtheorem{lema}[teo]{\color{rojo} {Lema}}
\newtheorem{coro}[teo]{\color{rojo} {Corolario}}
\newtheorem*{aff}{ {Afirmación}}
% Si pongo [theorem] siguen la numeración de los teoremas. 
% e.j. Teo 1, Lema 2, Teo 3, Teo 4 ...
\theoremstyle{definition}
\newtheorem{deff}[teo]{{ Definición}}{\smallskip}
\newtheorem{ej}[teo]{{Ejemplo}}{\smallskip}

% Remarks
\theoremstyle{remark}
\newtheorem{obs}[teo]{ {Observación}}{\smallskip}

%%%%%%%%%% FRAMES PARA TEOREMAS A LO HATCHER %%%%%%%%%%%%%%%%%%%%%%%%%

\surroundwithmdframed[outerlinewidth=0.4pt,
innerlinewidth=0.4pt,
align=center,
middlelinewidth=1pt,
middlelinecolor=white,
innertopmargin=-4pt,
innerbottommargin=0pt,
innerrightmargin=4pt,
innerleftmargin=4pt,
bottomline=false,topline=false,rightline=false]{teo}
\surroundwithmdframed[outerlinewidth=0.4pt,
innerlinewidth=0.4pt,
align=center,
middlelinewidth=1pt,
middlelinecolor=white,
innertopmargin=-4pt,
innerbottommargin=0pt,
innerrightmargin=4pt,
innerleftmargin=4pt,
bottomline=false,topline=false,rightline=false]{lema}

\surroundwithmdframed[outerlinewidth=0.4pt,
innerlinewidth=0.4pt,
align=center,
middlelinewidth=1pt,
middlelinecolor=white,
innertopmargin=-4pt,
innerbottommargin=0pt,
innerrightmargin=4pt,
innerleftmargin=4pt,
bottomline=false,topline=false,rightline=false]{prop}


\surroundwithmdframed[outerlinewidth=0.4pt,
innerlinewidth=0.4pt,
align=center,
middlelinewidth=1pt,
middlelinecolor=white,
innertopmargin=-4pt,
innerbottommargin=0pt,
innerrightmargin=4pt,
innerleftmargin=4pt,
bottomline=false,topline=false,rightline=false]{coro}

%==================================================================%

% DEMOS EN NEGRITA.
\renewenvironment{proof}{{\textbf{Demostración.}}}{ \hfill $\blacksquare$ \medskip} 

%% ========== Para escribir pseudo ==========
%\usepackage{algorithm}
%\usepackage[noend]{algpseudocode}  % "noend" es para no mostrar los endfor, endif
%%\algrenewcommand\alglinenumber[1]{\tiny #1:}  % Para que los numeros de linea del pseudo sean pequeños
%\renewcommand{\thealgorithm}{}  % Que no aparezca el numero luego de "Algorithm"
%\floatname{algorithm}{ }    % Entre {  } que quiero que aparezca en vez de "Algorithm"
%
%% traducciones
%\algrenewcommand\algorithmicwhile{\textbf{mientras}}
%\algrenewcommand\algorithmicdo{\textbf{hacer}}
%\algrenewcommand\algorithmicreturn{\textbf{devolver}}
%\algrenewcommand\algorithmicif{\textbf{si}}
%\algrenewcommand\algorithmicthen{\textbf{entonces}}
%\algrenewcommand\algorithmicfor{\textbf{para}}
%
%%% indentar dentro de los algoritmos
%\algdef{SE}[SUBALG]{Indent}{EndIndent}{}{\algorithmicend\ }%
%\algtext*{Indent}
%\algtext*{EndIndent}

% =========================================================
\usepackage[colorinlistoftodos,prependcaption,textsize=tiny]{todonotes}



%Comandos útiles.
\newcommand\RP{\mathbb{RP}}
\newcommand{\norm}[1]{\left\lVert#1\right\rVert}
\newcommand{\RR}{\mathbb{R}}
\newcommand{\CC}{\mathbb{C}}
\newcommand{\NN}{\mathbb{N}}
\newcommand{\ZZ}{\mathbb{Z}}
\newcommand{\Om}{\Omega}
\newcommand{\A}{\mathcal A}
\newcommand\ol{\overline}
\newcommand{\blue}{\textcolor{chulo}}
\newcommand{\red}{\textcolor{rojo}}
\newcommand{\Gg}{\mathfrak g}
\newcommand{\SL}{SL_2(\mathbb Z)}
\newcommand{\stab}{\text{Stab}}
\newcommand{\ic}{independiente de contexto }
\newcommand{\APND}{automáta de pila no determinístico }
\newcommand{\APD}{automáta de pila determinístico }
\newcommand{\gramatica}{{\cal G} = (V, \Sigma, P, S)}
\newcommand{\deriva}{\overset{*}{\to_{\cal G}}}
\newcommand{\tto}{\overset{*}{\to}}
\newcommand{\lengderivado}{L({\cal G})}
\newcommand{\fg}{grupo finitamente generado }
%\newcommand{\ol}{\overline{}}
\newcommand{\aut}{\text{Aut}}
\newcommand{\Sy}{\text{Sym}} 

\newcommand{\cay}[2]{\text{Cay}(#1,#2)}

\newcommand{\partes}[1]{{\cal{P}}(#1)} 

\newcommand*{\deri}{{\cal D}}
\newcommand*{\lexorder}{\le_{\textrm{lex}}}


\newcommand{\fp}{grupo finitamente presentado }
\newcommand{\vl}{virtualmente libre }
\newcommand{\vls}{virtualmente libres}
\newcommand{\WP}{\text{WP}(G, \Sigma)}

\newcommand{\cG}{ {\cal G} }
\newcommand{\cGg}{{\cal G} = (V, \Sigma, P, S)}
\newcommand{\cH}{ {\cal H} }
\newcommand{\Xm}{\widetilde X}
%\newcommand{\ol}{\overline{}}

%%% Capítulo 5. Cortes.
\newcommand{\olc}[1]{#1^{c}}
\newcommand{\ca}{{\cal C}(\alpha)}
\newcommand{\cmin}{{\cal C}_{\text{min}}}
\newcommand{\cam}{{\cal C}_{\text{min}}(\alpha)}
\newcommand{\copta}{{\cal C}_{\text{opt}}(\alpha)}
\newcommand{\copt}{{\cal C}_{\text{opt}}}
\newcommand*{\rows}{6}

\newcommand{\TODO}[1]{\textcolor{red}{TODO: #1}}

\newenvironment{leoenv}{\color{brightcerulean}}{\ignorespacesafterend}
%%%%%%%%%%%%%%  SETUP DE LA PÁGINA %%%%%%%%%%%%%%%%%
%\usepackage{fancyhdr} 
\pagestyle{headings} 
\pagenumbering{arabic} 
%\foot[C]{\textbf{\thepage}} % except the center
%\setlength{\headheight}{42pt}% ...at least 51.60004pt
%\renewcommand{\headrulewidth}{0.8pt}
%\head[L]{\thepage} 
%\head[R]{\textsl{\leftmark}} 
%\fancyfoot[C]{\thepage}

\usepackage{float}


\usepackage{subfiles} % mejor ponerlos al final
\begin{document}
	
	
\maketitle
	
\section{Dowker y treewidth.}	

Estaría bueno que el teorema de Dowker nos diga algo sobre el treewidth de un grafo.

Si consideremos los mapas
\[
X: V(T) \to V(X)
\]
y 
\[
T: V(X) \to V(T)
\]
entonces podemos definirnos una relación $R \subset V(T) \times V(X)$ por medio de $xRt$ si $x \in X_t$.

A partir del teorema de Dowker nos podemos construir complejos simpliciales $K_X$ y $K_T$ tales que resultan homotópicos entre sí.


Una afirmacionervación de esto es que 
\begin{afirmacion}
	Si $X$ es un árbol entonces es isomorfo como grafo a su complejo de Dowker. 
	Esto es que $X \simeq K_X$.
\end{afirmacion}

En particular 

\begin{afirmacion}
	$T$ siempre resulta ser un spanning tree del complejo $K_T$.
\end{afirmacion}
	
\section{Otra construcción para que el treewidth sea conexo.}

	

Me gustaría probar el siguiente resultado o al menos ver dónde se rompe.

\begin{afirmacion}
	Todo grupo fundamental de un grafo de grupos finito con grupos finitos tiene treewidth conexo finito.
\end{afirmacion}

Lo bueno de esto sería dar una descomposición de un árbol distinta a la que aparece en \cite{diekert_contextfree_2017} tal que sea conexa.

Acá estoy usando el siguiente resultado de \cite{karrass1973finite}.
\begin{teo}
	Un grupo es virtualmente libre sii es el grupo fundamental de un grafo de grupos finito con grupos finitos.
\end{teo}

Primero pensé en el caso de un grupo $G = G \ast H$ que en particular es el grupo fundamental del grafo que tiene dos vértices con grupos $G$ y $H$ que asumo finitos. El grupo de la arista es $1$.

Para este caso quería tomar la siguiente descomposición.
\begin{align*}
V(T) &= \{ v \in G_1 \ast G_2 : v = wa_ih  \} & h \in H, w \in G_1 \ast G_2 \\
 & = \{ v \in G_1 \ast G_2 : v = wb_jg  \} &  g \in G, w \in G_1 \ast G_2 \\
 & = \{ v \in G_1 \ast G_2 : v = wb_ja_i  \} & w \in G_1 \ast G_2 \\
 & = \{ v \in G_1 \ast G_2 : v = wa_ib_j  \} & w \in G_1 \ast G_2 \\
 & = \{ v \in G_1 \ast G_2 : v = a_i, b_j  \} & w \in G_1 \ast G_2 \\
\end{align*}
	
Por como está construído es claro que $T$ es conexo.

Habría que probar que no tiene ciclos... 
\red{HACER}

Para ver que es una descomposición habría que ver las tres propiedades necesarias.

\begin{itemize}
	\item Cubren todo.
	\item Todas las aristas están en algún bolsón por construcción.
	\item Todo vértice está a lo sumo en dos bolsones que resultan conexos por construcción.
\end{itemize}	
	
Finalmente los bolsones son conexos porque básicamente agregamos todas aristas para que nos queden conexas. 
Aparte los bolsones tienen dos vértices conectados por una arista o bien son una copia del grafo de Cayley de $G$ o de $H$.
	
	
Esta misma idea creo que se puede extender fácilmente para obtener una descomposición para un producto libre arbitrario de grupos finitos.	
	
	
Quedaría ver entonces el siguiente resultado.

\begin{afirmacion}
	Un grupo amalgamado $G \ast_{K} H$	de grupos finitos tiene treewidth conexo finito.
\end{afirmacion}	
	
No me queda claro que la idea de antes pueda andar en este caso. 

	





\section{Grupos independiente de contextos actúan sobre un árbol.}

Sea $G$ un grupo \ic{} entonces busco un árbol $T$ de manera que $G$ actúa sobre $T$ sin inversión de aristas.

\subsection{Idea 1.}
Como $G$ es determinístico \ic{} entonces si consideramos el conjunto 
\[
	\text{Config}({\cal M}) = \{ \confguno \in Q \times \Sigma^* \times \Gamma \}.
\]
si definimos las aristas para que 
\[
	\{ \confguno, \confgdos  \} \iff \delta(q,w_{1},\gamma_{1}) = (p, \theta').
\]

\begin{afirmacion}
	$T$ definido como $V(T) = \text{Config}({\cal M})$ y con las aristas tal como hice arriba resulta que no tiene ciclos.
	Como $\cal M$ es determinístico esta razón nos dice que el grafo no debería tener ciclos. 
\end{afirmacion}


\begin{afirmacion}
	El grafo no es conexo.
	De hecho es un bosque porque podemos encontrarnos configuraciones distintas tales que no están conectadas en el grafo.
\end{afirmacion}

No es un árbol pero es un bosque.
¿Cómo sería una acción de $G$ sobre este bosque?

\begin{afirmacion}
	Podríamos definir para que 
	\[
		g \cdot \confguno = \confgdos
	\]
	si $\vectres{q}{g}{\gamma} = \confgdos$.
	
	\red{Todo: Chequear que está bien definida esta acción.}
\end{afirmacion}

\subsection{Idea 2.}
	Consideramos el siguiente grafo que por construcción es isomorfo al grafo de Cayley.
	Sea $\Gamma$ el grafo definido como:
	\begin{equation}
		V(\Gamma) = \{ \confguno \mid \exists u \land w \in \text{Pos}(u), \vectres{q_{0}}{u}{\$} \vdash^{*} \confguno \} 
	\end{equation}
	Esto básicamente nos dice que estamos mirando las configuraciones alcanzables del autómata.
	
	Podría considerar palabras $u$ tales que $u$ es una forma normal.
	
	\begin{afirmacion}
		Tenemos una acción de $G$ en este conjunto dada por 
		\[
			g \cdot \confguno = \confgdos 
		\]
	si $\vectres{q}{g}{\gamma} \vdash^{*} \vectres{p}{\epsilon}{\theta}$.
	\end{afirmacion}


\subsection{Idea 3.}
	Encontrar un árbol a partir del sistema de reescritura que nos surge del \APND.

\subsection{Idea 4.}
	Encontrar un árbol a partir de la gramática del lenguaje del problema de la palabra.


\section{Grupos virtualmente libres tienen treewidth conexo finito.}

\section{}
\end{document}